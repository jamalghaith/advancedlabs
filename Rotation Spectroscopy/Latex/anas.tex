\documentclass{article}

\usepackage{amsmath}

\usepackage{float}

\usepackage{mathrsfs}

\usepackage{graphicx}

\usepackage{hyperref}

\usepackage[utf8]{inputenc}

\usepackage{subcaption}

\usepackage{geometry}

\usepackage{multirow}

\usepackage{siunitx}

\usepackage[square,numbers]{natbib}

\usepackage[english]{babel}
%Includes "References" in the table of contents
\usepackage[nottoc]{tocbibind}

\usepackage[parfill]{parskip}

\usepackage[toc,page]{appendix}

\usepackage{tabularx}

\usepackage{booktabs}

\usepackage{pgfplots}

\usepackage{adjustbox}

\usepackage{lmodern}

\usepackage{subcaption}

\usepackage{tikz}

\newcommand{\pvec}[1]{\vec{#1}\mkern2mu\vphantom{#1}}

\DeclareSIUnit{\angstrom}{\textup{\AA}}

\geometry{legalpaper, portrait, margin=1in}

\title{Lab Report}

\author{Jamal Ghaith}
\author{Anas Roumieh}

\date{01.04.2024}

\begin{document}

\begin{titlepage}
	\centering
	{\scshape\LARGE University of Leipzig \par}
	\vspace{1cm}
	{\scshape\ Advanced Labs\par}
	\vspace{1.5cm}
	{\huge\bfseries Lab report\par}
	\vspace{2cm}
	{\huge\bfseries Rotation-Vibration Spectra of Molecules\par}
	\vspace{2cm}
	{\Large Jamal Ghaith 3792970\par}
    {\Large Anas Roumieh 3766647\par}
	\vfill

    {\Large Conducted on: \par}
	\vfill
\end{titlepage}


\tableofcontents
\pagenumbering{gobble}
\pagebreak{}
\pagenumbering{arabic}

\section{Introduction}

\pagebreak{}

\section{Analysis}

\subsection{Task 4}

In this task, the gap width of a cuvette was determined using the interference method. From Fig. \ref{fig:Cuvette dips}, 10 points of interference dips were extracted and plotted against over the corresponding wavelength. Linear regression was performed on the data points to determine the slope of the line which is twice the gap width of the cuvette. Eq. \ref{cuvette_fit} was used. The gap width was found to be $d = 47.52 \pm 0.15 \ \mu$m.


\begin{figure}[h!]
	\centering
	\begin{subfigure}{0.45\textwidth}
		\centering
		\scalebox{0.50}{%% Creator: Matplotlib, PGF backend
%%
%% To include the figure in your LaTeX document, write
%%   \input{<filename>.pgf}
%%
%% Make sure the required packages are loaded in your preamble
%%   \usepackage{pgf}
%%
%% Also ensure that all the required font packages are loaded; for instance,
%% the lmodern package is sometimes necessary when using math font.
%%   \usepackage{lmodern}
%%
%% Figures using additional raster images can only be included by \input if
%% they are in the same directory as the main LaTeX file. For loading figures
%% from other directories you can use the `import` package
%%   \usepackage{import}
%%
%% and then include the figures with
%%   \import{<path to file>}{<filename>.pgf}
%%
%% Matplotlib used the following preamble
%%   
%%   \usepackage{fontspec}
%%   \makeatletter\@ifpackageloaded{underscore}{}{\usepackage[strings]{underscore}}\makeatother
%%
\begingroup%
\makeatletter%
\begin{pgfpicture}%
\pgfpathrectangle{\pgfpointorigin}{\pgfqpoint{5.924769in}{4.080862in}}%
\pgfusepath{use as bounding box, clip}%
\begin{pgfscope}%
\pgfsetbuttcap%
\pgfsetmiterjoin%
\definecolor{currentfill}{rgb}{1.000000,1.000000,1.000000}%
\pgfsetfillcolor{currentfill}%
\pgfsetlinewidth{0.000000pt}%
\definecolor{currentstroke}{rgb}{1.000000,1.000000,1.000000}%
\pgfsetstrokecolor{currentstroke}%
\pgfsetdash{}{0pt}%
\pgfpathmoveto{\pgfqpoint{0.000000in}{-0.000000in}}%
\pgfpathlineto{\pgfqpoint{5.924769in}{-0.000000in}}%
\pgfpathlineto{\pgfqpoint{5.924769in}{4.080862in}}%
\pgfpathlineto{\pgfqpoint{0.000000in}{4.080862in}}%
\pgfpathlineto{\pgfqpoint{0.000000in}{-0.000000in}}%
\pgfpathclose%
\pgfusepath{fill}%
\end{pgfscope}%
\begin{pgfscope}%
\pgfsetbuttcap%
\pgfsetmiterjoin%
\definecolor{currentfill}{rgb}{1.000000,1.000000,1.000000}%
\pgfsetfillcolor{currentfill}%
\pgfsetlinewidth{0.000000pt}%
\definecolor{currentstroke}{rgb}{0.000000,0.000000,0.000000}%
\pgfsetstrokecolor{currentstroke}%
\pgfsetstrokeopacity{0.000000}%
\pgfsetdash{}{0pt}%
\pgfpathmoveto{\pgfqpoint{0.795366in}{0.646140in}}%
\pgfpathlineto{\pgfqpoint{5.824769in}{0.646140in}}%
\pgfpathlineto{\pgfqpoint{5.824769in}{3.734428in}}%
\pgfpathlineto{\pgfqpoint{0.795366in}{3.734428in}}%
\pgfpathlineto{\pgfqpoint{0.795366in}{0.646140in}}%
\pgfpathclose%
\pgfusepath{fill}%
\end{pgfscope}%
\begin{pgfscope}%
\pgfsetbuttcap%
\pgfsetroundjoin%
\definecolor{currentfill}{rgb}{0.000000,0.000000,0.000000}%
\pgfsetfillcolor{currentfill}%
\pgfsetlinewidth{0.803000pt}%
\definecolor{currentstroke}{rgb}{0.000000,0.000000,0.000000}%
\pgfsetstrokecolor{currentstroke}%
\pgfsetdash{}{0pt}%
\pgfsys@defobject{currentmarker}{\pgfqpoint{0.000000in}{-0.048611in}}{\pgfqpoint{0.000000in}{0.000000in}}{%
\pgfpathmoveto{\pgfqpoint{0.000000in}{0.000000in}}%
\pgfpathlineto{\pgfqpoint{0.000000in}{-0.048611in}}%
\pgfusepath{stroke,fill}%
}%
\begin{pgfscope}%
\pgfsys@transformshift{1.023975in}{0.646140in}%
\pgfsys@useobject{currentmarker}{}%
\end{pgfscope}%
\end{pgfscope}%
\begin{pgfscope}%
\definecolor{textcolor}{rgb}{0.000000,0.000000,0.000000}%
\pgfsetstrokecolor{textcolor}%
\pgfsetfillcolor{textcolor}%
\pgftext[x=1.023975in,y=0.548917in,,top]{\color{textcolor}\rmfamily\fontsize{14.000000}{16.800000}\selectfont \(\displaystyle {2400}\)}%
\end{pgfscope}%
\begin{pgfscope}%
\pgfsetbuttcap%
\pgfsetroundjoin%
\definecolor{currentfill}{rgb}{0.000000,0.000000,0.000000}%
\pgfsetfillcolor{currentfill}%
\pgfsetlinewidth{0.803000pt}%
\definecolor{currentstroke}{rgb}{0.000000,0.000000,0.000000}%
\pgfsetstrokecolor{currentstroke}%
\pgfsetdash{}{0pt}%
\pgfsys@defobject{currentmarker}{\pgfqpoint{0.000000in}{-0.048611in}}{\pgfqpoint{0.000000in}{0.000000in}}{%
\pgfpathmoveto{\pgfqpoint{0.000000in}{0.000000in}}%
\pgfpathlineto{\pgfqpoint{0.000000in}{-0.048611in}}%
\pgfusepath{stroke,fill}%
}%
\begin{pgfscope}%
\pgfsys@transformshift{1.855281in}{0.646140in}%
\pgfsys@useobject{currentmarker}{}%
\end{pgfscope}%
\end{pgfscope}%
\begin{pgfscope}%
\definecolor{textcolor}{rgb}{0.000000,0.000000,0.000000}%
\pgfsetstrokecolor{textcolor}%
\pgfsetfillcolor{textcolor}%
\pgftext[x=1.855281in,y=0.548917in,,top]{\color{textcolor}\rmfamily\fontsize{14.000000}{16.800000}\selectfont \(\displaystyle {2600}\)}%
\end{pgfscope}%
\begin{pgfscope}%
\pgfsetbuttcap%
\pgfsetroundjoin%
\definecolor{currentfill}{rgb}{0.000000,0.000000,0.000000}%
\pgfsetfillcolor{currentfill}%
\pgfsetlinewidth{0.803000pt}%
\definecolor{currentstroke}{rgb}{0.000000,0.000000,0.000000}%
\pgfsetstrokecolor{currentstroke}%
\pgfsetdash{}{0pt}%
\pgfsys@defobject{currentmarker}{\pgfqpoint{0.000000in}{-0.048611in}}{\pgfqpoint{0.000000in}{0.000000in}}{%
\pgfpathmoveto{\pgfqpoint{0.000000in}{0.000000in}}%
\pgfpathlineto{\pgfqpoint{0.000000in}{-0.048611in}}%
\pgfusepath{stroke,fill}%
}%
\begin{pgfscope}%
\pgfsys@transformshift{2.686588in}{0.646140in}%
\pgfsys@useobject{currentmarker}{}%
\end{pgfscope}%
\end{pgfscope}%
\begin{pgfscope}%
\definecolor{textcolor}{rgb}{0.000000,0.000000,0.000000}%
\pgfsetstrokecolor{textcolor}%
\pgfsetfillcolor{textcolor}%
\pgftext[x=2.686588in,y=0.548917in,,top]{\color{textcolor}\rmfamily\fontsize{14.000000}{16.800000}\selectfont \(\displaystyle {2800}\)}%
\end{pgfscope}%
\begin{pgfscope}%
\pgfsetbuttcap%
\pgfsetroundjoin%
\definecolor{currentfill}{rgb}{0.000000,0.000000,0.000000}%
\pgfsetfillcolor{currentfill}%
\pgfsetlinewidth{0.803000pt}%
\definecolor{currentstroke}{rgb}{0.000000,0.000000,0.000000}%
\pgfsetstrokecolor{currentstroke}%
\pgfsetdash{}{0pt}%
\pgfsys@defobject{currentmarker}{\pgfqpoint{0.000000in}{-0.048611in}}{\pgfqpoint{0.000000in}{0.000000in}}{%
\pgfpathmoveto{\pgfqpoint{0.000000in}{0.000000in}}%
\pgfpathlineto{\pgfqpoint{0.000000in}{-0.048611in}}%
\pgfusepath{stroke,fill}%
}%
\begin{pgfscope}%
\pgfsys@transformshift{3.517894in}{0.646140in}%
\pgfsys@useobject{currentmarker}{}%
\end{pgfscope}%
\end{pgfscope}%
\begin{pgfscope}%
\definecolor{textcolor}{rgb}{0.000000,0.000000,0.000000}%
\pgfsetstrokecolor{textcolor}%
\pgfsetfillcolor{textcolor}%
\pgftext[x=3.517894in,y=0.548917in,,top]{\color{textcolor}\rmfamily\fontsize{14.000000}{16.800000}\selectfont \(\displaystyle {3000}\)}%
\end{pgfscope}%
\begin{pgfscope}%
\pgfsetbuttcap%
\pgfsetroundjoin%
\definecolor{currentfill}{rgb}{0.000000,0.000000,0.000000}%
\pgfsetfillcolor{currentfill}%
\pgfsetlinewidth{0.803000pt}%
\definecolor{currentstroke}{rgb}{0.000000,0.000000,0.000000}%
\pgfsetstrokecolor{currentstroke}%
\pgfsetdash{}{0pt}%
\pgfsys@defobject{currentmarker}{\pgfqpoint{0.000000in}{-0.048611in}}{\pgfqpoint{0.000000in}{0.000000in}}{%
\pgfpathmoveto{\pgfqpoint{0.000000in}{0.000000in}}%
\pgfpathlineto{\pgfqpoint{0.000000in}{-0.048611in}}%
\pgfusepath{stroke,fill}%
}%
\begin{pgfscope}%
\pgfsys@transformshift{4.349200in}{0.646140in}%
\pgfsys@useobject{currentmarker}{}%
\end{pgfscope}%
\end{pgfscope}%
\begin{pgfscope}%
\definecolor{textcolor}{rgb}{0.000000,0.000000,0.000000}%
\pgfsetstrokecolor{textcolor}%
\pgfsetfillcolor{textcolor}%
\pgftext[x=4.349200in,y=0.548917in,,top]{\color{textcolor}\rmfamily\fontsize{14.000000}{16.800000}\selectfont \(\displaystyle {3200}\)}%
\end{pgfscope}%
\begin{pgfscope}%
\pgfsetbuttcap%
\pgfsetroundjoin%
\definecolor{currentfill}{rgb}{0.000000,0.000000,0.000000}%
\pgfsetfillcolor{currentfill}%
\pgfsetlinewidth{0.803000pt}%
\definecolor{currentstroke}{rgb}{0.000000,0.000000,0.000000}%
\pgfsetstrokecolor{currentstroke}%
\pgfsetdash{}{0pt}%
\pgfsys@defobject{currentmarker}{\pgfqpoint{0.000000in}{-0.048611in}}{\pgfqpoint{0.000000in}{0.000000in}}{%
\pgfpathmoveto{\pgfqpoint{0.000000in}{0.000000in}}%
\pgfpathlineto{\pgfqpoint{0.000000in}{-0.048611in}}%
\pgfusepath{stroke,fill}%
}%
\begin{pgfscope}%
\pgfsys@transformshift{5.180507in}{0.646140in}%
\pgfsys@useobject{currentmarker}{}%
\end{pgfscope}%
\end{pgfscope}%
\begin{pgfscope}%
\definecolor{textcolor}{rgb}{0.000000,0.000000,0.000000}%
\pgfsetstrokecolor{textcolor}%
\pgfsetfillcolor{textcolor}%
\pgftext[x=5.180507in,y=0.548917in,,top]{\color{textcolor}\rmfamily\fontsize{14.000000}{16.800000}\selectfont \(\displaystyle {3400}\)}%
\end{pgfscope}%
\begin{pgfscope}%
\definecolor{textcolor}{rgb}{0.000000,0.000000,0.000000}%
\pgfsetstrokecolor{textcolor}%
\pgfsetfillcolor{textcolor}%
\pgftext[x=3.310068in,y=0.320695in,,top]{\color{textcolor}\rmfamily\fontsize{14.000000}{16.800000}\selectfont Wavenumber [cm\(\displaystyle ^{-1}\)]}%
\end{pgfscope}%
\begin{pgfscope}%
\pgfsetbuttcap%
\pgfsetroundjoin%
\definecolor{currentfill}{rgb}{0.000000,0.000000,0.000000}%
\pgfsetfillcolor{currentfill}%
\pgfsetlinewidth{0.803000pt}%
\definecolor{currentstroke}{rgb}{0.000000,0.000000,0.000000}%
\pgfsetstrokecolor{currentstroke}%
\pgfsetdash{}{0pt}%
\pgfsys@defobject{currentmarker}{\pgfqpoint{-0.048611in}{0.000000in}}{\pgfqpoint{-0.000000in}{0.000000in}}{%
\pgfpathmoveto{\pgfqpoint{-0.000000in}{0.000000in}}%
\pgfpathlineto{\pgfqpoint{-0.048611in}{0.000000in}}%
\pgfusepath{stroke,fill}%
}%
\begin{pgfscope}%
\pgfsys@transformshift{0.795366in}{0.726946in}%
\pgfsys@useobject{currentmarker}{}%
\end{pgfscope}%
\end{pgfscope}%
\begin{pgfscope}%
\definecolor{textcolor}{rgb}{0.000000,0.000000,0.000000}%
\pgfsetstrokecolor{textcolor}%
\pgfsetfillcolor{textcolor}%
\pgftext[x=0.350000in, y=0.659474in, left, base]{\color{textcolor}\rmfamily\fontsize{14.000000}{16.800000}\selectfont \(\displaystyle {0.78}\)}%
\end{pgfscope}%
\begin{pgfscope}%
\pgfsetbuttcap%
\pgfsetroundjoin%
\definecolor{currentfill}{rgb}{0.000000,0.000000,0.000000}%
\pgfsetfillcolor{currentfill}%
\pgfsetlinewidth{0.803000pt}%
\definecolor{currentstroke}{rgb}{0.000000,0.000000,0.000000}%
\pgfsetstrokecolor{currentstroke}%
\pgfsetdash{}{0pt}%
\pgfsys@defobject{currentmarker}{\pgfqpoint{-0.048611in}{0.000000in}}{\pgfqpoint{-0.000000in}{0.000000in}}{%
\pgfpathmoveto{\pgfqpoint{-0.000000in}{0.000000in}}%
\pgfpathlineto{\pgfqpoint{-0.048611in}{0.000000in}}%
\pgfusepath{stroke,fill}%
}%
\begin{pgfscope}%
\pgfsys@transformshift{0.795366in}{1.304809in}%
\pgfsys@useobject{currentmarker}{}%
\end{pgfscope}%
\end{pgfscope}%
\begin{pgfscope}%
\definecolor{textcolor}{rgb}{0.000000,0.000000,0.000000}%
\pgfsetstrokecolor{textcolor}%
\pgfsetfillcolor{textcolor}%
\pgftext[x=0.350000in, y=1.237336in, left, base]{\color{textcolor}\rmfamily\fontsize{14.000000}{16.800000}\selectfont \(\displaystyle {0.80}\)}%
\end{pgfscope}%
\begin{pgfscope}%
\pgfsetbuttcap%
\pgfsetroundjoin%
\definecolor{currentfill}{rgb}{0.000000,0.000000,0.000000}%
\pgfsetfillcolor{currentfill}%
\pgfsetlinewidth{0.803000pt}%
\definecolor{currentstroke}{rgb}{0.000000,0.000000,0.000000}%
\pgfsetstrokecolor{currentstroke}%
\pgfsetdash{}{0pt}%
\pgfsys@defobject{currentmarker}{\pgfqpoint{-0.048611in}{0.000000in}}{\pgfqpoint{-0.000000in}{0.000000in}}{%
\pgfpathmoveto{\pgfqpoint{-0.000000in}{0.000000in}}%
\pgfpathlineto{\pgfqpoint{-0.048611in}{0.000000in}}%
\pgfusepath{stroke,fill}%
}%
\begin{pgfscope}%
\pgfsys@transformshift{0.795366in}{1.882671in}%
\pgfsys@useobject{currentmarker}{}%
\end{pgfscope}%
\end{pgfscope}%
\begin{pgfscope}%
\definecolor{textcolor}{rgb}{0.000000,0.000000,0.000000}%
\pgfsetstrokecolor{textcolor}%
\pgfsetfillcolor{textcolor}%
\pgftext[x=0.350000in, y=1.815199in, left, base]{\color{textcolor}\rmfamily\fontsize{14.000000}{16.800000}\selectfont \(\displaystyle {0.82}\)}%
\end{pgfscope}%
\begin{pgfscope}%
\pgfsetbuttcap%
\pgfsetroundjoin%
\definecolor{currentfill}{rgb}{0.000000,0.000000,0.000000}%
\pgfsetfillcolor{currentfill}%
\pgfsetlinewidth{0.803000pt}%
\definecolor{currentstroke}{rgb}{0.000000,0.000000,0.000000}%
\pgfsetstrokecolor{currentstroke}%
\pgfsetdash{}{0pt}%
\pgfsys@defobject{currentmarker}{\pgfqpoint{-0.048611in}{0.000000in}}{\pgfqpoint{-0.000000in}{0.000000in}}{%
\pgfpathmoveto{\pgfqpoint{-0.000000in}{0.000000in}}%
\pgfpathlineto{\pgfqpoint{-0.048611in}{0.000000in}}%
\pgfusepath{stroke,fill}%
}%
\begin{pgfscope}%
\pgfsys@transformshift{0.795366in}{2.460534in}%
\pgfsys@useobject{currentmarker}{}%
\end{pgfscope}%
\end{pgfscope}%
\begin{pgfscope}%
\definecolor{textcolor}{rgb}{0.000000,0.000000,0.000000}%
\pgfsetstrokecolor{textcolor}%
\pgfsetfillcolor{textcolor}%
\pgftext[x=0.350000in, y=2.393061in, left, base]{\color{textcolor}\rmfamily\fontsize{14.000000}{16.800000}\selectfont \(\displaystyle {0.84}\)}%
\end{pgfscope}%
\begin{pgfscope}%
\pgfsetbuttcap%
\pgfsetroundjoin%
\definecolor{currentfill}{rgb}{0.000000,0.000000,0.000000}%
\pgfsetfillcolor{currentfill}%
\pgfsetlinewidth{0.803000pt}%
\definecolor{currentstroke}{rgb}{0.000000,0.000000,0.000000}%
\pgfsetstrokecolor{currentstroke}%
\pgfsetdash{}{0pt}%
\pgfsys@defobject{currentmarker}{\pgfqpoint{-0.048611in}{0.000000in}}{\pgfqpoint{-0.000000in}{0.000000in}}{%
\pgfpathmoveto{\pgfqpoint{-0.000000in}{0.000000in}}%
\pgfpathlineto{\pgfqpoint{-0.048611in}{0.000000in}}%
\pgfusepath{stroke,fill}%
}%
\begin{pgfscope}%
\pgfsys@transformshift{0.795366in}{3.038396in}%
\pgfsys@useobject{currentmarker}{}%
\end{pgfscope}%
\end{pgfscope}%
\begin{pgfscope}%
\definecolor{textcolor}{rgb}{0.000000,0.000000,0.000000}%
\pgfsetstrokecolor{textcolor}%
\pgfsetfillcolor{textcolor}%
\pgftext[x=0.350000in, y=2.970924in, left, base]{\color{textcolor}\rmfamily\fontsize{14.000000}{16.800000}\selectfont \(\displaystyle {0.86}\)}%
\end{pgfscope}%
\begin{pgfscope}%
\pgfsetbuttcap%
\pgfsetroundjoin%
\definecolor{currentfill}{rgb}{0.000000,0.000000,0.000000}%
\pgfsetfillcolor{currentfill}%
\pgfsetlinewidth{0.803000pt}%
\definecolor{currentstroke}{rgb}{0.000000,0.000000,0.000000}%
\pgfsetstrokecolor{currentstroke}%
\pgfsetdash{}{0pt}%
\pgfsys@defobject{currentmarker}{\pgfqpoint{-0.048611in}{0.000000in}}{\pgfqpoint{-0.000000in}{0.000000in}}{%
\pgfpathmoveto{\pgfqpoint{-0.000000in}{0.000000in}}%
\pgfpathlineto{\pgfqpoint{-0.048611in}{0.000000in}}%
\pgfusepath{stroke,fill}%
}%
\begin{pgfscope}%
\pgfsys@transformshift{0.795366in}{3.616259in}%
\pgfsys@useobject{currentmarker}{}%
\end{pgfscope}%
\end{pgfscope}%
\begin{pgfscope}%
\definecolor{textcolor}{rgb}{0.000000,0.000000,0.000000}%
\pgfsetstrokecolor{textcolor}%
\pgfsetfillcolor{textcolor}%
\pgftext[x=0.350000in, y=3.548787in, left, base]{\color{textcolor}\rmfamily\fontsize{14.000000}{16.800000}\selectfont \(\displaystyle {0.88}\)}%
\end{pgfscope}%
\begin{pgfscope}%
\definecolor{textcolor}{rgb}{0.000000,0.000000,0.000000}%
\pgfsetstrokecolor{textcolor}%
\pgfsetfillcolor{textcolor}%
\pgftext[x=0.294444in,y=2.190284in,,bottom,rotate=90.000000]{\color{textcolor}\rmfamily\fontsize{14.000000}{16.800000}\selectfont Transmittance [\(\displaystyle \%\)]}%
\end{pgfscope}%
\begin{pgfscope}%
\pgfpathrectangle{\pgfqpoint{0.795366in}{0.646140in}}{\pgfqpoint{5.029404in}{3.088289in}}%
\pgfusepath{clip}%
\pgfsetrectcap%
\pgfsetroundjoin%
\pgfsetlinewidth{0.501875pt}%
\definecolor{currentstroke}{rgb}{0.000000,0.000000,0.000000}%
\pgfsetstrokecolor{currentstroke}%
\pgfsetdash{}{0pt}%
\pgfpathmoveto{\pgfqpoint{5.596160in}{1.140938in}}%
\pgfpathlineto{\pgfqpoint{5.595640in}{1.139129in}}%
\pgfpathlineto{\pgfqpoint{5.594601in}{1.164272in}}%
\pgfpathlineto{\pgfqpoint{5.594082in}{1.176956in}}%
\pgfpathlineto{\pgfqpoint{5.593562in}{1.173703in}}%
\pgfpathlineto{\pgfqpoint{5.593043in}{1.167063in}}%
\pgfpathlineto{\pgfqpoint{5.592004in}{1.209465in}}%
\pgfpathlineto{\pgfqpoint{5.590964in}{1.259762in}}%
\pgfpathlineto{\pgfqpoint{5.590445in}{1.246934in}}%
\pgfpathlineto{\pgfqpoint{5.587327in}{1.126573in}}%
\pgfpathlineto{\pgfqpoint{5.586288in}{1.195850in}}%
\pgfpathlineto{\pgfqpoint{5.584730in}{1.297700in}}%
\pgfpathlineto{\pgfqpoint{5.583690in}{1.259150in}}%
\pgfpathlineto{\pgfqpoint{5.583171in}{1.259570in}}%
\pgfpathlineto{\pgfqpoint{5.582651in}{1.270211in}}%
\pgfpathlineto{\pgfqpoint{5.582132in}{1.268810in}}%
\pgfpathlineto{\pgfqpoint{5.580573in}{1.184810in}}%
\pgfpathlineto{\pgfqpoint{5.580054in}{1.187571in}}%
\pgfpathlineto{\pgfqpoint{5.578495in}{1.346399in}}%
\pgfpathlineto{\pgfqpoint{5.577456in}{1.397418in}}%
\pgfpathlineto{\pgfqpoint{5.574338in}{1.188112in}}%
\pgfpathlineto{\pgfqpoint{5.573299in}{1.261628in}}%
\pgfpathlineto{\pgfqpoint{5.572260in}{1.375195in}}%
\pgfpathlineto{\pgfqpoint{5.571740in}{1.372339in}}%
\pgfpathlineto{\pgfqpoint{5.570182in}{1.295631in}}%
\pgfpathlineto{\pgfqpoint{5.569662in}{1.302497in}}%
\pgfpathlineto{\pgfqpoint{5.568103in}{1.325890in}}%
\pgfpathlineto{\pgfqpoint{5.566025in}{1.475496in}}%
\pgfpathlineto{\pgfqpoint{5.565506in}{1.463436in}}%
\pgfpathlineto{\pgfqpoint{5.563947in}{1.395948in}}%
\pgfpathlineto{\pgfqpoint{5.563427in}{1.405081in}}%
\pgfpathlineto{\pgfqpoint{5.562908in}{1.405293in}}%
\pgfpathlineto{\pgfqpoint{5.561349in}{1.289899in}}%
\pgfpathlineto{\pgfqpoint{5.560830in}{1.313673in}}%
\pgfpathlineto{\pgfqpoint{5.558751in}{1.476339in}}%
\pgfpathlineto{\pgfqpoint{5.558232in}{1.476620in}}%
\pgfpathlineto{\pgfqpoint{5.557193in}{1.511019in}}%
\pgfpathlineto{\pgfqpoint{5.554595in}{1.591015in}}%
\pgfpathlineto{\pgfqpoint{5.554075in}{1.601600in}}%
\pgfpathlineto{\pgfqpoint{5.553036in}{1.531534in}}%
\pgfpathlineto{\pgfqpoint{5.551997in}{1.457228in}}%
\pgfpathlineto{\pgfqpoint{5.551477in}{1.482911in}}%
\pgfpathlineto{\pgfqpoint{5.547840in}{1.733928in}}%
\pgfpathlineto{\pgfqpoint{5.546801in}{1.781192in}}%
\pgfpathlineto{\pgfqpoint{5.546282in}{1.772438in}}%
\pgfpathlineto{\pgfqpoint{5.542645in}{1.665278in}}%
\pgfpathlineto{\pgfqpoint{5.540047in}{1.806263in}}%
\pgfpathlineto{\pgfqpoint{5.539008in}{1.780282in}}%
\pgfpathlineto{\pgfqpoint{5.535890in}{1.563247in}}%
\pgfpathlineto{\pgfqpoint{5.535371in}{1.593192in}}%
\pgfpathlineto{\pgfqpoint{5.533293in}{1.726074in}}%
\pgfpathlineto{\pgfqpoint{5.532773in}{1.731652in}}%
\pgfpathlineto{\pgfqpoint{5.530695in}{1.827513in}}%
\pgfpathlineto{\pgfqpoint{5.528616in}{1.976224in}}%
\pgfpathlineto{\pgfqpoint{5.528097in}{1.971602in}}%
\pgfpathlineto{\pgfqpoint{5.526538in}{1.904077in}}%
\pgfpathlineto{\pgfqpoint{5.526019in}{1.907915in}}%
\pgfpathlineto{\pgfqpoint{5.525499in}{1.916647in}}%
\pgfpathlineto{\pgfqpoint{5.524979in}{1.910246in}}%
\pgfpathlineto{\pgfqpoint{5.523940in}{1.890313in}}%
\pgfpathlineto{\pgfqpoint{5.522382in}{1.965103in}}%
\pgfpathlineto{\pgfqpoint{5.521862in}{1.956859in}}%
\pgfpathlineto{\pgfqpoint{5.521342in}{1.950494in}}%
\pgfpathlineto{\pgfqpoint{5.520303in}{2.000050in}}%
\pgfpathlineto{\pgfqpoint{5.518745in}{2.093864in}}%
\pgfpathlineto{\pgfqpoint{5.518225in}{2.088165in}}%
\pgfpathlineto{\pgfqpoint{5.517186in}{2.060996in}}%
\pgfpathlineto{\pgfqpoint{5.515627in}{2.141680in}}%
\pgfpathlineto{\pgfqpoint{5.515108in}{2.123468in}}%
\pgfpathlineto{\pgfqpoint{5.514069in}{2.068796in}}%
\pgfpathlineto{\pgfqpoint{5.513549in}{2.076043in}}%
\pgfpathlineto{\pgfqpoint{5.513029in}{2.085451in}}%
\pgfpathlineto{\pgfqpoint{5.512510in}{2.075004in}}%
\pgfpathlineto{\pgfqpoint{5.511471in}{2.045663in}}%
\pgfpathlineto{\pgfqpoint{5.508873in}{2.124299in}}%
\pgfpathlineto{\pgfqpoint{5.507314in}{2.213013in}}%
\pgfpathlineto{\pgfqpoint{5.503677in}{2.098461in}}%
\pgfpathlineto{\pgfqpoint{5.503158in}{2.105794in}}%
\pgfpathlineto{\pgfqpoint{5.502119in}{2.146843in}}%
\pgfpathlineto{\pgfqpoint{5.501599in}{2.143231in}}%
\pgfpathlineto{\pgfqpoint{5.499521in}{2.040178in}}%
\pgfpathlineto{\pgfqpoint{5.498482in}{2.098537in}}%
\pgfpathlineto{\pgfqpoint{5.495884in}{2.249168in}}%
\pgfpathlineto{\pgfqpoint{5.495364in}{2.249742in}}%
\pgfpathlineto{\pgfqpoint{5.494325in}{2.289725in}}%
\pgfpathlineto{\pgfqpoint{5.492247in}{2.362804in}}%
\pgfpathlineto{\pgfqpoint{5.491208in}{2.376808in}}%
\pgfpathlineto{\pgfqpoint{5.489649in}{2.289010in}}%
\pgfpathlineto{\pgfqpoint{5.488610in}{2.316454in}}%
\pgfpathlineto{\pgfqpoint{5.488090in}{2.340532in}}%
\pgfpathlineto{\pgfqpoint{5.487571in}{2.340110in}}%
\pgfpathlineto{\pgfqpoint{5.486532in}{2.325435in}}%
\pgfpathlineto{\pgfqpoint{5.484453in}{2.419737in}}%
\pgfpathlineto{\pgfqpoint{5.483934in}{2.412087in}}%
\pgfpathlineto{\pgfqpoint{5.482375in}{2.375326in}}%
\pgfpathlineto{\pgfqpoint{5.478218in}{2.455625in}}%
\pgfpathlineto{\pgfqpoint{5.476140in}{2.515918in}}%
\pgfpathlineto{\pgfqpoint{5.475101in}{2.449800in}}%
\pgfpathlineto{\pgfqpoint{5.474062in}{2.357087in}}%
\pgfpathlineto{\pgfqpoint{5.473542in}{2.362686in}}%
\pgfpathlineto{\pgfqpoint{5.470425in}{2.537424in}}%
\pgfpathlineto{\pgfqpoint{5.469905in}{2.543534in}}%
\pgfpathlineto{\pgfqpoint{5.468866in}{2.517286in}}%
\pgfpathlineto{\pgfqpoint{5.467308in}{2.455670in}}%
\pgfpathlineto{\pgfqpoint{5.466788in}{2.457655in}}%
\pgfpathlineto{\pgfqpoint{5.464710in}{2.494923in}}%
\pgfpathlineto{\pgfqpoint{5.463151in}{2.584925in}}%
\pgfpathlineto{\pgfqpoint{5.462631in}{2.577992in}}%
\pgfpathlineto{\pgfqpoint{5.461592in}{2.561008in}}%
\pgfpathlineto{\pgfqpoint{5.460034in}{2.576661in}}%
\pgfpathlineto{\pgfqpoint{5.456916in}{2.480193in}}%
\pgfpathlineto{\pgfqpoint{5.456397in}{2.485458in}}%
\pgfpathlineto{\pgfqpoint{5.454318in}{2.688870in}}%
\pgfpathlineto{\pgfqpoint{5.453279in}{2.628937in}}%
\pgfpathlineto{\pgfqpoint{5.452760in}{2.602759in}}%
\pgfpathlineto{\pgfqpoint{5.452240in}{2.614734in}}%
\pgfpathlineto{\pgfqpoint{5.450162in}{2.718671in}}%
\pgfpathlineto{\pgfqpoint{5.449642in}{2.718277in}}%
\pgfpathlineto{\pgfqpoint{5.448084in}{2.611269in}}%
\pgfpathlineto{\pgfqpoint{5.446525in}{2.575166in}}%
\pgfpathlineto{\pgfqpoint{5.446005in}{2.571728in}}%
\pgfpathlineto{\pgfqpoint{5.445486in}{2.575224in}}%
\pgfpathlineto{\pgfqpoint{5.443927in}{2.673415in}}%
\pgfpathlineto{\pgfqpoint{5.442888in}{2.730047in}}%
\pgfpathlineto{\pgfqpoint{5.442368in}{2.729220in}}%
\pgfpathlineto{\pgfqpoint{5.437173in}{2.549616in}}%
\pgfpathlineto{\pgfqpoint{5.436653in}{2.552627in}}%
\pgfpathlineto{\pgfqpoint{5.432497in}{2.707705in}}%
\pgfpathlineto{\pgfqpoint{5.430938in}{2.740012in}}%
\pgfpathlineto{\pgfqpoint{5.429899in}{2.742936in}}%
\pgfpathlineto{\pgfqpoint{5.428860in}{2.713076in}}%
\pgfpathlineto{\pgfqpoint{5.426781in}{2.671776in}}%
\pgfpathlineto{\pgfqpoint{5.426262in}{2.673288in}}%
\pgfpathlineto{\pgfqpoint{5.424703in}{2.697555in}}%
\pgfpathlineto{\pgfqpoint{5.422105in}{2.680420in}}%
\pgfpathlineto{\pgfqpoint{5.421066in}{2.696802in}}%
\pgfpathlineto{\pgfqpoint{5.420547in}{2.689258in}}%
\pgfpathlineto{\pgfqpoint{5.419507in}{2.668060in}}%
\pgfpathlineto{\pgfqpoint{5.418988in}{2.671425in}}%
\pgfpathlineto{\pgfqpoint{5.417949in}{2.631069in}}%
\pgfpathlineto{\pgfqpoint{5.417429in}{2.611111in}}%
\pgfpathlineto{\pgfqpoint{5.416910in}{2.622909in}}%
\pgfpathlineto{\pgfqpoint{5.415351in}{2.688597in}}%
\pgfpathlineto{\pgfqpoint{5.414831in}{2.683775in}}%
\pgfpathlineto{\pgfqpoint{5.413273in}{2.655551in}}%
\pgfpathlineto{\pgfqpoint{5.412234in}{2.685823in}}%
\pgfpathlineto{\pgfqpoint{5.411714in}{2.679998in}}%
\pgfpathlineto{\pgfqpoint{5.410675in}{2.641770in}}%
\pgfpathlineto{\pgfqpoint{5.410155in}{2.647461in}}%
\pgfpathlineto{\pgfqpoint{5.409116in}{2.667960in}}%
\pgfpathlineto{\pgfqpoint{5.408077in}{2.653211in}}%
\pgfpathlineto{\pgfqpoint{5.407038in}{2.712192in}}%
\pgfpathlineto{\pgfqpoint{5.405479in}{2.813447in}}%
\pgfpathlineto{\pgfqpoint{5.404440in}{2.741275in}}%
\pgfpathlineto{\pgfqpoint{5.402362in}{2.513780in}}%
\pgfpathlineto{\pgfqpoint{5.401842in}{2.528433in}}%
\pgfpathlineto{\pgfqpoint{5.399244in}{2.714061in}}%
\pgfpathlineto{\pgfqpoint{5.398725in}{2.692375in}}%
\pgfpathlineto{\pgfqpoint{5.397166in}{2.625616in}}%
\pgfpathlineto{\pgfqpoint{5.396647in}{2.630764in}}%
\pgfpathlineto{\pgfqpoint{5.395088in}{2.684667in}}%
\pgfpathlineto{\pgfqpoint{5.394568in}{2.682793in}}%
\pgfpathlineto{\pgfqpoint{5.393529in}{2.642692in}}%
\pgfpathlineto{\pgfqpoint{5.390931in}{2.437735in}}%
\pgfpathlineto{\pgfqpoint{5.390412in}{2.462838in}}%
\pgfpathlineto{\pgfqpoint{5.388333in}{2.692745in}}%
\pgfpathlineto{\pgfqpoint{5.387814in}{2.658725in}}%
\pgfpathlineto{\pgfqpoint{5.385736in}{2.483585in}}%
\pgfpathlineto{\pgfqpoint{5.385216in}{2.485009in}}%
\pgfpathlineto{\pgfqpoint{5.384177in}{2.568456in}}%
\pgfpathlineto{\pgfqpoint{5.382618in}{2.698134in}}%
\pgfpathlineto{\pgfqpoint{5.381579in}{2.684871in}}%
\pgfpathlineto{\pgfqpoint{5.381060in}{2.696013in}}%
\pgfpathlineto{\pgfqpoint{5.380540in}{2.692371in}}%
\pgfpathlineto{\pgfqpoint{5.376903in}{2.433481in}}%
\pgfpathlineto{\pgfqpoint{5.375864in}{2.385254in}}%
\pgfpathlineto{\pgfqpoint{5.374305in}{2.489057in}}%
\pgfpathlineto{\pgfqpoint{5.373786in}{2.484357in}}%
\pgfpathlineto{\pgfqpoint{5.372227in}{2.381475in}}%
\pgfpathlineto{\pgfqpoint{5.371707in}{2.387453in}}%
\pgfpathlineto{\pgfqpoint{5.369110in}{2.450468in}}%
\pgfpathlineto{\pgfqpoint{5.368590in}{2.454067in}}%
\pgfpathlineto{\pgfqpoint{5.367551in}{2.420673in}}%
\pgfpathlineto{\pgfqpoint{5.367031in}{2.424501in}}%
\pgfpathlineto{\pgfqpoint{5.365473in}{2.509875in}}%
\pgfpathlineto{\pgfqpoint{5.364953in}{2.504489in}}%
\pgfpathlineto{\pgfqpoint{5.362355in}{2.374057in}}%
\pgfpathlineto{\pgfqpoint{5.361836in}{2.379065in}}%
\pgfpathlineto{\pgfqpoint{5.360796in}{2.384524in}}%
\pgfpathlineto{\pgfqpoint{5.359757in}{2.408762in}}%
\pgfpathlineto{\pgfqpoint{5.359238in}{2.401446in}}%
\pgfpathlineto{\pgfqpoint{5.358199in}{2.355824in}}%
\pgfpathlineto{\pgfqpoint{5.357679in}{2.360687in}}%
\pgfpathlineto{\pgfqpoint{5.355081in}{2.443376in}}%
\pgfpathlineto{\pgfqpoint{5.352483in}{2.354591in}}%
\pgfpathlineto{\pgfqpoint{5.350405in}{2.274622in}}%
\pgfpathlineto{\pgfqpoint{5.348846in}{2.331657in}}%
\pgfpathlineto{\pgfqpoint{5.348327in}{2.317549in}}%
\pgfpathlineto{\pgfqpoint{5.344690in}{2.062528in}}%
\pgfpathlineto{\pgfqpoint{5.343651in}{2.108459in}}%
\pgfpathlineto{\pgfqpoint{5.341053in}{2.270222in}}%
\pgfpathlineto{\pgfqpoint{5.338975in}{2.223531in}}%
\pgfpathlineto{\pgfqpoint{5.336377in}{2.035645in}}%
\pgfpathlineto{\pgfqpoint{5.335857in}{2.040520in}}%
\pgfpathlineto{\pgfqpoint{5.334818in}{2.081567in}}%
\pgfpathlineto{\pgfqpoint{5.334299in}{2.069585in}}%
\pgfpathlineto{\pgfqpoint{5.332740in}{1.985270in}}%
\pgfpathlineto{\pgfqpoint{5.332220in}{1.997576in}}%
\pgfpathlineto{\pgfqpoint{5.330142in}{2.090562in}}%
\pgfpathlineto{\pgfqpoint{5.329622in}{2.086973in}}%
\pgfpathlineto{\pgfqpoint{5.329103in}{2.077772in}}%
\pgfpathlineto{\pgfqpoint{5.328583in}{2.079810in}}%
\pgfpathlineto{\pgfqpoint{5.328064in}{2.087937in}}%
\pgfpathlineto{\pgfqpoint{5.327544in}{2.076576in}}%
\pgfpathlineto{\pgfqpoint{5.324946in}{1.931803in}}%
\pgfpathlineto{\pgfqpoint{5.322868in}{1.901774in}}%
\pgfpathlineto{\pgfqpoint{5.321829in}{1.837642in}}%
\pgfpathlineto{\pgfqpoint{5.321309in}{1.838243in}}%
\pgfpathlineto{\pgfqpoint{5.319231in}{1.934586in}}%
\pgfpathlineto{\pgfqpoint{5.318712in}{1.930079in}}%
\pgfpathlineto{\pgfqpoint{5.315594in}{1.828840in}}%
\pgfpathlineto{\pgfqpoint{5.315075in}{1.816370in}}%
\pgfpathlineto{\pgfqpoint{5.313516in}{1.901488in}}%
\pgfpathlineto{\pgfqpoint{5.312996in}{1.878448in}}%
\pgfpathlineto{\pgfqpoint{5.310918in}{1.753697in}}%
\pgfpathlineto{\pgfqpoint{5.309359in}{1.813113in}}%
\pgfpathlineto{\pgfqpoint{5.307801in}{1.731627in}}%
\pgfpathlineto{\pgfqpoint{5.306762in}{1.741968in}}%
\pgfpathlineto{\pgfqpoint{5.305203in}{1.686475in}}%
\pgfpathlineto{\pgfqpoint{5.304683in}{1.711131in}}%
\pgfpathlineto{\pgfqpoint{5.303644in}{1.771348in}}%
\pgfpathlineto{\pgfqpoint{5.302085in}{1.679189in}}%
\pgfpathlineto{\pgfqpoint{5.301566in}{1.697349in}}%
\pgfpathlineto{\pgfqpoint{5.300527in}{1.737755in}}%
\pgfpathlineto{\pgfqpoint{5.298448in}{1.621423in}}%
\pgfpathlineto{\pgfqpoint{5.297929in}{1.650783in}}%
\pgfpathlineto{\pgfqpoint{5.296890in}{1.707800in}}%
\pgfpathlineto{\pgfqpoint{5.294811in}{1.510705in}}%
\pgfpathlineto{\pgfqpoint{5.293772in}{1.544552in}}%
\pgfpathlineto{\pgfqpoint{5.293253in}{1.541842in}}%
\pgfpathlineto{\pgfqpoint{5.292733in}{1.528984in}}%
\pgfpathlineto{\pgfqpoint{5.292214in}{1.536184in}}%
\pgfpathlineto{\pgfqpoint{5.290655in}{1.644637in}}%
\pgfpathlineto{\pgfqpoint{5.290135in}{1.627085in}}%
\pgfpathlineto{\pgfqpoint{5.288057in}{1.513800in}}%
\pgfpathlineto{\pgfqpoint{5.287538in}{1.511832in}}%
\pgfpathlineto{\pgfqpoint{5.286498in}{1.481075in}}%
\pgfpathlineto{\pgfqpoint{5.285459in}{1.447321in}}%
\pgfpathlineto{\pgfqpoint{5.284420in}{1.495189in}}%
\pgfpathlineto{\pgfqpoint{5.283381in}{1.544838in}}%
\pgfpathlineto{\pgfqpoint{5.282861in}{1.541534in}}%
\pgfpathlineto{\pgfqpoint{5.277146in}{1.240038in}}%
\pgfpathlineto{\pgfqpoint{5.276627in}{1.273029in}}%
\pgfpathlineto{\pgfqpoint{5.275588in}{1.360766in}}%
\pgfpathlineto{\pgfqpoint{5.275068in}{1.345441in}}%
\pgfpathlineto{\pgfqpoint{5.274548in}{1.317591in}}%
\pgfpathlineto{\pgfqpoint{5.274029in}{1.325653in}}%
\pgfpathlineto{\pgfqpoint{5.272990in}{1.413732in}}%
\pgfpathlineto{\pgfqpoint{5.272470in}{1.404675in}}%
\pgfpathlineto{\pgfqpoint{5.270911in}{1.281517in}}%
\pgfpathlineto{\pgfqpoint{5.269353in}{1.367023in}}%
\pgfpathlineto{\pgfqpoint{5.267274in}{1.223933in}}%
\pgfpathlineto{\pgfqpoint{5.266755in}{1.250445in}}%
\pgfpathlineto{\pgfqpoint{5.264677in}{1.361427in}}%
\pgfpathlineto{\pgfqpoint{5.263118in}{1.272382in}}%
\pgfpathlineto{\pgfqpoint{5.262079in}{1.244472in}}%
\pgfpathlineto{\pgfqpoint{5.261559in}{1.248209in}}%
\pgfpathlineto{\pgfqpoint{5.261040in}{1.253876in}}%
\pgfpathlineto{\pgfqpoint{5.260520in}{1.252464in}}%
\pgfpathlineto{\pgfqpoint{5.256364in}{1.172085in}}%
\pgfpathlineto{\pgfqpoint{5.254285in}{1.252219in}}%
\pgfpathlineto{\pgfqpoint{5.253766in}{1.248128in}}%
\pgfpathlineto{\pgfqpoint{5.251687in}{1.184775in}}%
\pgfpathlineto{\pgfqpoint{5.251168in}{1.194568in}}%
\pgfpathlineto{\pgfqpoint{5.250648in}{1.186610in}}%
\pgfpathlineto{\pgfqpoint{5.248570in}{0.976256in}}%
\pgfpathlineto{\pgfqpoint{5.248051in}{0.997037in}}%
\pgfpathlineto{\pgfqpoint{5.247011in}{1.037211in}}%
\pgfpathlineto{\pgfqpoint{5.246492in}{1.027404in}}%
\pgfpathlineto{\pgfqpoint{5.245972in}{1.016331in}}%
\pgfpathlineto{\pgfqpoint{5.245453in}{1.026105in}}%
\pgfpathlineto{\pgfqpoint{5.242335in}{1.183484in}}%
\pgfpathlineto{\pgfqpoint{5.241816in}{1.177689in}}%
\pgfpathlineto{\pgfqpoint{5.240257in}{1.131615in}}%
\pgfpathlineto{\pgfqpoint{5.239737in}{1.133725in}}%
\pgfpathlineto{\pgfqpoint{5.235061in}{0.986940in}}%
\pgfpathlineto{\pgfqpoint{5.233503in}{0.941022in}}%
\pgfpathlineto{\pgfqpoint{5.232983in}{0.945844in}}%
\pgfpathlineto{\pgfqpoint{5.229866in}{1.050734in}}%
\pgfpathlineto{\pgfqpoint{5.229346in}{1.032888in}}%
\pgfpathlineto{\pgfqpoint{5.226748in}{0.894593in}}%
\pgfpathlineto{\pgfqpoint{5.225709in}{0.833722in}}%
\pgfpathlineto{\pgfqpoint{5.225190in}{0.844837in}}%
\pgfpathlineto{\pgfqpoint{5.222592in}{0.947989in}}%
\pgfpathlineto{\pgfqpoint{5.222072in}{0.937348in}}%
\pgfpathlineto{\pgfqpoint{5.221033in}{0.888352in}}%
\pgfpathlineto{\pgfqpoint{5.220513in}{0.900915in}}%
\pgfpathlineto{\pgfqpoint{5.219474in}{0.944870in}}%
\pgfpathlineto{\pgfqpoint{5.217916in}{0.794888in}}%
\pgfpathlineto{\pgfqpoint{5.217396in}{0.799924in}}%
\pgfpathlineto{\pgfqpoint{5.215318in}{1.019228in}}%
\pgfpathlineto{\pgfqpoint{5.214798in}{1.011001in}}%
\pgfpathlineto{\pgfqpoint{5.214279in}{1.004298in}}%
\pgfpathlineto{\pgfqpoint{5.213759in}{1.006889in}}%
\pgfpathlineto{\pgfqpoint{5.213240in}{1.008901in}}%
\pgfpathlineto{\pgfqpoint{5.212200in}{0.992444in}}%
\pgfpathlineto{\pgfqpoint{5.211681in}{0.993978in}}%
\pgfpathlineto{\pgfqpoint{5.210122in}{1.071848in}}%
\pgfpathlineto{\pgfqpoint{5.209603in}{1.071126in}}%
\pgfpathlineto{\pgfqpoint{5.206485in}{0.934896in}}%
\pgfpathlineto{\pgfqpoint{5.205446in}{0.947275in}}%
\pgfpathlineto{\pgfqpoint{5.204407in}{0.981121in}}%
\pgfpathlineto{\pgfqpoint{5.202848in}{1.055348in}}%
\pgfpathlineto{\pgfqpoint{5.200770in}{0.864756in}}%
\pgfpathlineto{\pgfqpoint{5.200250in}{0.871813in}}%
\pgfpathlineto{\pgfqpoint{5.199731in}{0.879853in}}%
\pgfpathlineto{\pgfqpoint{5.199211in}{0.879561in}}%
\pgfpathlineto{\pgfqpoint{5.198172in}{0.850278in}}%
\pgfpathlineto{\pgfqpoint{5.197133in}{0.812471in}}%
\pgfpathlineto{\pgfqpoint{5.196613in}{0.819812in}}%
\pgfpathlineto{\pgfqpoint{5.196094in}{0.832450in}}%
\pgfpathlineto{\pgfqpoint{5.195574in}{0.826660in}}%
\pgfpathlineto{\pgfqpoint{5.194535in}{0.786516in}}%
\pgfpathlineto{\pgfqpoint{5.194016in}{0.803821in}}%
\pgfpathlineto{\pgfqpoint{5.190379in}{1.035792in}}%
\pgfpathlineto{\pgfqpoint{5.189859in}{1.055076in}}%
\pgfpathlineto{\pgfqpoint{5.189340in}{1.049289in}}%
\pgfpathlineto{\pgfqpoint{5.188300in}{0.993861in}}%
\pgfpathlineto{\pgfqpoint{5.187781in}{0.997784in}}%
\pgfpathlineto{\pgfqpoint{5.186742in}{1.040360in}}%
\pgfpathlineto{\pgfqpoint{5.186222in}{1.024448in}}%
\pgfpathlineto{\pgfqpoint{5.184663in}{0.959968in}}%
\pgfpathlineto{\pgfqpoint{5.182585in}{0.981701in}}%
\pgfpathlineto{\pgfqpoint{5.181546in}{0.987013in}}%
\pgfpathlineto{\pgfqpoint{5.179468in}{0.920473in}}%
\pgfpathlineto{\pgfqpoint{5.178948in}{0.930711in}}%
\pgfpathlineto{\pgfqpoint{5.177389in}{0.998293in}}%
\pgfpathlineto{\pgfqpoint{5.176350in}{1.066612in}}%
\pgfpathlineto{\pgfqpoint{5.175831in}{1.043550in}}%
\pgfpathlineto{\pgfqpoint{5.174272in}{0.873041in}}%
\pgfpathlineto{\pgfqpoint{5.173753in}{0.888016in}}%
\pgfpathlineto{\pgfqpoint{5.171674in}{1.027614in}}%
\pgfpathlineto{\pgfqpoint{5.170635in}{1.001267in}}%
\pgfpathlineto{\pgfqpoint{5.170116in}{1.001898in}}%
\pgfpathlineto{\pgfqpoint{5.168037in}{1.085555in}}%
\pgfpathlineto{\pgfqpoint{5.167518in}{1.080801in}}%
\pgfpathlineto{\pgfqpoint{5.165959in}{1.141641in}}%
\pgfpathlineto{\pgfqpoint{5.165439in}{1.117865in}}%
\pgfpathlineto{\pgfqpoint{5.164400in}{1.039256in}}%
\pgfpathlineto{\pgfqpoint{5.163881in}{1.043520in}}%
\pgfpathlineto{\pgfqpoint{5.162842in}{1.069742in}}%
\pgfpathlineto{\pgfqpoint{5.162322in}{1.063869in}}%
\pgfpathlineto{\pgfqpoint{5.160244in}{1.033654in}}%
\pgfpathlineto{\pgfqpoint{5.159724in}{1.034724in}}%
\pgfpathlineto{\pgfqpoint{5.158685in}{1.058141in}}%
\pgfpathlineto{\pgfqpoint{5.156087in}{1.235653in}}%
\pgfpathlineto{\pgfqpoint{5.155568in}{1.219202in}}%
\pgfpathlineto{\pgfqpoint{5.153489in}{1.071923in}}%
\pgfpathlineto{\pgfqpoint{5.152970in}{1.086914in}}%
\pgfpathlineto{\pgfqpoint{5.151411in}{1.154808in}}%
\pgfpathlineto{\pgfqpoint{5.150892in}{1.150275in}}%
\pgfpathlineto{\pgfqpoint{5.150372in}{1.155013in}}%
\pgfpathlineto{\pgfqpoint{5.148294in}{1.395096in}}%
\pgfpathlineto{\pgfqpoint{5.147255in}{1.337458in}}%
\pgfpathlineto{\pgfqpoint{5.144657in}{1.195526in}}%
\pgfpathlineto{\pgfqpoint{5.144137in}{1.196112in}}%
\pgfpathlineto{\pgfqpoint{5.143098in}{1.195066in}}%
\pgfpathlineto{\pgfqpoint{5.142059in}{1.201459in}}%
\pgfpathlineto{\pgfqpoint{5.141020in}{1.245295in}}%
\pgfpathlineto{\pgfqpoint{5.139461in}{1.329920in}}%
\pgfpathlineto{\pgfqpoint{5.138422in}{1.294512in}}%
\pgfpathlineto{\pgfqpoint{5.137902in}{1.304091in}}%
\pgfpathlineto{\pgfqpoint{5.135824in}{1.434780in}}%
\pgfpathlineto{\pgfqpoint{5.133746in}{1.289162in}}%
\pgfpathlineto{\pgfqpoint{5.133226in}{1.313970in}}%
\pgfpathlineto{\pgfqpoint{5.132187in}{1.363426in}}%
\pgfpathlineto{\pgfqpoint{5.131668in}{1.347990in}}%
\pgfpathlineto{\pgfqpoint{5.130628in}{1.300328in}}%
\pgfpathlineto{\pgfqpoint{5.130109in}{1.309637in}}%
\pgfpathlineto{\pgfqpoint{5.127511in}{1.419521in}}%
\pgfpathlineto{\pgfqpoint{5.126992in}{1.411596in}}%
\pgfpathlineto{\pgfqpoint{5.126472in}{1.403702in}}%
\pgfpathlineto{\pgfqpoint{5.125952in}{1.405175in}}%
\pgfpathlineto{\pgfqpoint{5.125433in}{1.408066in}}%
\pgfpathlineto{\pgfqpoint{5.124394in}{1.400858in}}%
\pgfpathlineto{\pgfqpoint{5.121276in}{1.470456in}}%
\pgfpathlineto{\pgfqpoint{5.120757in}{1.463994in}}%
\pgfpathlineto{\pgfqpoint{5.119718in}{1.449984in}}%
\pgfpathlineto{\pgfqpoint{5.119198in}{1.452806in}}%
\pgfpathlineto{\pgfqpoint{5.118159in}{1.485940in}}%
\pgfpathlineto{\pgfqpoint{5.115561in}{1.680756in}}%
\pgfpathlineto{\pgfqpoint{5.114522in}{1.659326in}}%
\pgfpathlineto{\pgfqpoint{5.113483in}{1.615241in}}%
\pgfpathlineto{\pgfqpoint{5.112963in}{1.618466in}}%
\pgfpathlineto{\pgfqpoint{5.111924in}{1.646767in}}%
\pgfpathlineto{\pgfqpoint{5.111405in}{1.637999in}}%
\pgfpathlineto{\pgfqpoint{5.109846in}{1.593516in}}%
\pgfpathlineto{\pgfqpoint{5.109326in}{1.594683in}}%
\pgfpathlineto{\pgfqpoint{5.107768in}{1.632270in}}%
\pgfpathlineto{\pgfqpoint{5.106209in}{1.667747in}}%
\pgfpathlineto{\pgfqpoint{5.105170in}{1.648288in}}%
\pgfpathlineto{\pgfqpoint{5.104131in}{1.630931in}}%
\pgfpathlineto{\pgfqpoint{5.103091in}{1.650668in}}%
\pgfpathlineto{\pgfqpoint{5.102572in}{1.650492in}}%
\pgfpathlineto{\pgfqpoint{5.101533in}{1.628394in}}%
\pgfpathlineto{\pgfqpoint{5.101013in}{1.635217in}}%
\pgfpathlineto{\pgfqpoint{5.096337in}{1.842465in}}%
\pgfpathlineto{\pgfqpoint{5.094778in}{1.925318in}}%
\pgfpathlineto{\pgfqpoint{5.093739in}{1.901885in}}%
\pgfpathlineto{\pgfqpoint{5.092700in}{1.851419in}}%
\pgfpathlineto{\pgfqpoint{5.092181in}{1.855125in}}%
\pgfpathlineto{\pgfqpoint{5.089583in}{2.010097in}}%
\pgfpathlineto{\pgfqpoint{5.088024in}{1.906217in}}%
\pgfpathlineto{\pgfqpoint{5.087504in}{1.907448in}}%
\pgfpathlineto{\pgfqpoint{5.085946in}{1.981011in}}%
\pgfpathlineto{\pgfqpoint{5.085426in}{1.973093in}}%
\pgfpathlineto{\pgfqpoint{5.084387in}{1.938405in}}%
\pgfpathlineto{\pgfqpoint{5.083868in}{1.939077in}}%
\pgfpathlineto{\pgfqpoint{5.081789in}{2.077170in}}%
\pgfpathlineto{\pgfqpoint{5.081270in}{2.064635in}}%
\pgfpathlineto{\pgfqpoint{5.079711in}{1.969084in}}%
\pgfpathlineto{\pgfqpoint{5.079191in}{1.976431in}}%
\pgfpathlineto{\pgfqpoint{5.077633in}{2.100659in}}%
\pgfpathlineto{\pgfqpoint{5.076594in}{2.181178in}}%
\pgfpathlineto{\pgfqpoint{5.076074in}{2.166843in}}%
\pgfpathlineto{\pgfqpoint{5.073996in}{1.935709in}}%
\pgfpathlineto{\pgfqpoint{5.069839in}{2.271922in}}%
\pgfpathlineto{\pgfqpoint{5.069320in}{2.264394in}}%
\pgfpathlineto{\pgfqpoint{5.067241in}{2.135243in}}%
\pgfpathlineto{\pgfqpoint{5.066722in}{2.141568in}}%
\pgfpathlineto{\pgfqpoint{5.065683in}{2.168694in}}%
\pgfpathlineto{\pgfqpoint{5.064124in}{2.122936in}}%
\pgfpathlineto{\pgfqpoint{5.062046in}{2.273233in}}%
\pgfpathlineto{\pgfqpoint{5.061526in}{2.251040in}}%
\pgfpathlineto{\pgfqpoint{5.060487in}{2.207393in}}%
\pgfpathlineto{\pgfqpoint{5.059967in}{2.214140in}}%
\pgfpathlineto{\pgfqpoint{5.059448in}{2.220929in}}%
\pgfpathlineto{\pgfqpoint{5.058928in}{2.217074in}}%
\pgfpathlineto{\pgfqpoint{5.058409in}{2.214980in}}%
\pgfpathlineto{\pgfqpoint{5.057370in}{2.263206in}}%
\pgfpathlineto{\pgfqpoint{5.055811in}{2.322603in}}%
\pgfpathlineto{\pgfqpoint{5.053733in}{2.232558in}}%
\pgfpathlineto{\pgfqpoint{5.053213in}{2.246858in}}%
\pgfpathlineto{\pgfqpoint{5.050615in}{2.448532in}}%
\pgfpathlineto{\pgfqpoint{5.049576in}{2.401914in}}%
\pgfpathlineto{\pgfqpoint{5.048017in}{2.330133in}}%
\pgfpathlineto{\pgfqpoint{5.047498in}{2.332257in}}%
\pgfpathlineto{\pgfqpoint{5.044900in}{2.440770in}}%
\pgfpathlineto{\pgfqpoint{5.044380in}{2.438362in}}%
\pgfpathlineto{\pgfqpoint{5.043861in}{2.440258in}}%
\pgfpathlineto{\pgfqpoint{5.040224in}{2.542388in}}%
\pgfpathlineto{\pgfqpoint{5.039704in}{2.530339in}}%
\pgfpathlineto{\pgfqpoint{5.038146in}{2.451324in}}%
\pgfpathlineto{\pgfqpoint{5.033470in}{2.599899in}}%
\pgfpathlineto{\pgfqpoint{5.032950in}{2.596286in}}%
\pgfpathlineto{\pgfqpoint{5.030352in}{2.503435in}}%
\pgfpathlineto{\pgfqpoint{5.029833in}{2.524910in}}%
\pgfpathlineto{\pgfqpoint{5.029313in}{2.552962in}}%
\pgfpathlineto{\pgfqpoint{5.028793in}{2.551092in}}%
\pgfpathlineto{\pgfqpoint{5.027235in}{2.435361in}}%
\pgfpathlineto{\pgfqpoint{5.026715in}{2.466975in}}%
\pgfpathlineto{\pgfqpoint{5.025156in}{2.567824in}}%
\pgfpathlineto{\pgfqpoint{5.023598in}{2.532161in}}%
\pgfpathlineto{\pgfqpoint{5.019961in}{2.634792in}}%
\pgfpathlineto{\pgfqpoint{5.018922in}{2.671241in}}%
\pgfpathlineto{\pgfqpoint{5.018402in}{2.667984in}}%
\pgfpathlineto{\pgfqpoint{5.017883in}{2.670243in}}%
\pgfpathlineto{\pgfqpoint{5.016843in}{2.709694in}}%
\pgfpathlineto{\pgfqpoint{5.016324in}{2.707392in}}%
\pgfpathlineto{\pgfqpoint{5.014765in}{2.612145in}}%
\pgfpathlineto{\pgfqpoint{5.014246in}{2.632724in}}%
\pgfpathlineto{\pgfqpoint{5.012167in}{2.703041in}}%
\pgfpathlineto{\pgfqpoint{5.011128in}{2.723921in}}%
\pgfpathlineto{\pgfqpoint{5.010609in}{2.716424in}}%
\pgfpathlineto{\pgfqpoint{5.009050in}{2.645651in}}%
\pgfpathlineto{\pgfqpoint{5.008530in}{2.660013in}}%
\pgfpathlineto{\pgfqpoint{5.007491in}{2.720979in}}%
\pgfpathlineto{\pgfqpoint{5.006972in}{2.718840in}}%
\pgfpathlineto{\pgfqpoint{5.004893in}{2.597127in}}%
\pgfpathlineto{\pgfqpoint{5.004374in}{2.611503in}}%
\pgfpathlineto{\pgfqpoint{5.002296in}{2.715289in}}%
\pgfpathlineto{\pgfqpoint{5.001776in}{2.709351in}}%
\pgfpathlineto{\pgfqpoint{5.001256in}{2.708658in}}%
\pgfpathlineto{\pgfqpoint{4.997100in}{2.824374in}}%
\pgfpathlineto{\pgfqpoint{4.996580in}{2.814853in}}%
\pgfpathlineto{\pgfqpoint{4.995541in}{2.758632in}}%
\pgfpathlineto{\pgfqpoint{4.995022in}{2.764799in}}%
\pgfpathlineto{\pgfqpoint{4.993463in}{2.833396in}}%
\pgfpathlineto{\pgfqpoint{4.992943in}{2.826188in}}%
\pgfpathlineto{\pgfqpoint{4.990865in}{2.783123in}}%
\pgfpathlineto{\pgfqpoint{4.989826in}{2.802274in}}%
\pgfpathlineto{\pgfqpoint{4.989306in}{2.792499in}}%
\pgfpathlineto{\pgfqpoint{4.988267in}{2.759710in}}%
\pgfpathlineto{\pgfqpoint{4.986189in}{2.826229in}}%
\pgfpathlineto{\pgfqpoint{4.985669in}{2.818515in}}%
\pgfpathlineto{\pgfqpoint{4.984111in}{2.798490in}}%
\pgfpathlineto{\pgfqpoint{4.982552in}{2.723555in}}%
\pgfpathlineto{\pgfqpoint{4.982032in}{2.735120in}}%
\pgfpathlineto{\pgfqpoint{4.980474in}{2.776702in}}%
\pgfpathlineto{\pgfqpoint{4.979954in}{2.774389in}}%
\pgfpathlineto{\pgfqpoint{4.978915in}{2.795035in}}%
\pgfpathlineto{\pgfqpoint{4.978396in}{2.792959in}}%
\pgfpathlineto{\pgfqpoint{4.976837in}{2.684866in}}%
\pgfpathlineto{\pgfqpoint{4.976317in}{2.688266in}}%
\pgfpathlineto{\pgfqpoint{4.975278in}{2.743963in}}%
\pgfpathlineto{\pgfqpoint{4.974759in}{2.742053in}}%
\pgfpathlineto{\pgfqpoint{4.973200in}{2.681584in}}%
\pgfpathlineto{\pgfqpoint{4.972680in}{2.690862in}}%
\pgfpathlineto{\pgfqpoint{4.970602in}{2.793233in}}%
\pgfpathlineto{\pgfqpoint{4.969563in}{2.824025in}}%
\pgfpathlineto{\pgfqpoint{4.966445in}{2.705034in}}%
\pgfpathlineto{\pgfqpoint{4.962809in}{2.771848in}}%
\pgfpathlineto{\pgfqpoint{4.961769in}{2.830196in}}%
\pgfpathlineto{\pgfqpoint{4.961250in}{2.817627in}}%
\pgfpathlineto{\pgfqpoint{4.958132in}{2.594850in}}%
\pgfpathlineto{\pgfqpoint{4.957093in}{2.590842in}}%
\pgfpathlineto{\pgfqpoint{4.956574in}{2.591105in}}%
\pgfpathlineto{\pgfqpoint{4.956054in}{2.589522in}}%
\pgfpathlineto{\pgfqpoint{4.955535in}{2.595336in}}%
\pgfpathlineto{\pgfqpoint{4.953976in}{2.670329in}}%
\pgfpathlineto{\pgfqpoint{4.953456in}{2.663002in}}%
\pgfpathlineto{\pgfqpoint{4.952417in}{2.619744in}}%
\pgfpathlineto{\pgfqpoint{4.951898in}{2.621042in}}%
\pgfpathlineto{\pgfqpoint{4.949819in}{2.710391in}}%
\pgfpathlineto{\pgfqpoint{4.947222in}{2.497217in}}%
\pgfpathlineto{\pgfqpoint{4.946702in}{2.530073in}}%
\pgfpathlineto{\pgfqpoint{4.945663in}{2.612068in}}%
\pgfpathlineto{\pgfqpoint{4.945143in}{2.610840in}}%
\pgfpathlineto{\pgfqpoint{4.943585in}{2.489337in}}%
\pgfpathlineto{\pgfqpoint{4.943065in}{2.500084in}}%
\pgfpathlineto{\pgfqpoint{4.941506in}{2.643704in}}%
\pgfpathlineto{\pgfqpoint{4.940987in}{2.631907in}}%
\pgfpathlineto{\pgfqpoint{4.939948in}{2.589940in}}%
\pgfpathlineto{\pgfqpoint{4.939428in}{2.592473in}}%
\pgfpathlineto{\pgfqpoint{4.938908in}{2.593673in}}%
\pgfpathlineto{\pgfqpoint{4.936830in}{2.507718in}}%
\pgfpathlineto{\pgfqpoint{4.936311in}{2.530693in}}%
\pgfpathlineto{\pgfqpoint{4.934752in}{2.644954in}}%
\pgfpathlineto{\pgfqpoint{4.934232in}{2.638594in}}%
\pgfpathlineto{\pgfqpoint{4.929037in}{2.461215in}}%
\pgfpathlineto{\pgfqpoint{4.928517in}{2.460506in}}%
\pgfpathlineto{\pgfqpoint{4.927998in}{2.453575in}}%
\pgfpathlineto{\pgfqpoint{4.925919in}{2.256565in}}%
\pgfpathlineto{\pgfqpoint{4.924880in}{2.284327in}}%
\pgfpathlineto{\pgfqpoint{4.923841in}{2.330456in}}%
\pgfpathlineto{\pgfqpoint{4.923321in}{2.326597in}}%
\pgfpathlineto{\pgfqpoint{4.922802in}{2.326232in}}%
\pgfpathlineto{\pgfqpoint{4.921243in}{2.379162in}}%
\pgfpathlineto{\pgfqpoint{4.920724in}{2.369505in}}%
\pgfpathlineto{\pgfqpoint{4.918126in}{2.311443in}}%
\pgfpathlineto{\pgfqpoint{4.916048in}{2.221309in}}%
\pgfpathlineto{\pgfqpoint{4.913969in}{2.338843in}}%
\pgfpathlineto{\pgfqpoint{4.912930in}{2.396568in}}%
\pgfpathlineto{\pgfqpoint{4.912411in}{2.390128in}}%
\pgfpathlineto{\pgfqpoint{4.910852in}{2.316281in}}%
\pgfpathlineto{\pgfqpoint{4.910332in}{2.327714in}}%
\pgfpathlineto{\pgfqpoint{4.909813in}{2.340229in}}%
\pgfpathlineto{\pgfqpoint{4.909293in}{2.329774in}}%
\pgfpathlineto{\pgfqpoint{4.907215in}{2.122520in}}%
\pgfpathlineto{\pgfqpoint{4.906695in}{2.131099in}}%
\pgfpathlineto{\pgfqpoint{4.905137in}{2.171704in}}%
\pgfpathlineto{\pgfqpoint{4.904097in}{2.161828in}}%
\pgfpathlineto{\pgfqpoint{4.903578in}{2.166356in}}%
\pgfpathlineto{\pgfqpoint{4.901500in}{2.253507in}}%
\pgfpathlineto{\pgfqpoint{4.900980in}{2.233479in}}%
\pgfpathlineto{\pgfqpoint{4.899941in}{2.171402in}}%
\pgfpathlineto{\pgfqpoint{4.899421in}{2.173619in}}%
\pgfpathlineto{\pgfqpoint{4.898382in}{2.203616in}}%
\pgfpathlineto{\pgfqpoint{4.897343in}{2.113662in}}%
\pgfpathlineto{\pgfqpoint{4.895784in}{1.961488in}}%
\pgfpathlineto{\pgfqpoint{4.894745in}{2.047047in}}%
\pgfpathlineto{\pgfqpoint{4.893187in}{2.212052in}}%
\pgfpathlineto{\pgfqpoint{4.892147in}{2.102693in}}%
\pgfpathlineto{\pgfqpoint{4.891108in}{1.981943in}}%
\pgfpathlineto{\pgfqpoint{4.890589in}{1.982104in}}%
\pgfpathlineto{\pgfqpoint{4.889550in}{2.015288in}}%
\pgfpathlineto{\pgfqpoint{4.889030in}{2.013517in}}%
\pgfpathlineto{\pgfqpoint{4.886432in}{1.975738in}}%
\pgfpathlineto{\pgfqpoint{4.884354in}{1.930046in}}%
\pgfpathlineto{\pgfqpoint{4.881237in}{1.851995in}}%
\pgfpathlineto{\pgfqpoint{4.880717in}{1.857723in}}%
\pgfpathlineto{\pgfqpoint{4.879158in}{1.965702in}}%
\pgfpathlineto{\pgfqpoint{4.878639in}{1.964393in}}%
\pgfpathlineto{\pgfqpoint{4.877080in}{1.851342in}}%
\pgfpathlineto{\pgfqpoint{4.875521in}{1.775136in}}%
\pgfpathlineto{\pgfqpoint{4.875002in}{1.770004in}}%
\pgfpathlineto{\pgfqpoint{4.872924in}{1.714891in}}%
\pgfpathlineto{\pgfqpoint{4.869806in}{1.794313in}}%
\pgfpathlineto{\pgfqpoint{4.869287in}{1.790421in}}%
\pgfpathlineto{\pgfqpoint{4.868767in}{1.784511in}}%
\pgfpathlineto{\pgfqpoint{4.868247in}{1.788142in}}%
\pgfpathlineto{\pgfqpoint{4.867208in}{1.810610in}}%
\pgfpathlineto{\pgfqpoint{4.866169in}{1.759147in}}%
\pgfpathlineto{\pgfqpoint{4.864610in}{1.669707in}}%
\pgfpathlineto{\pgfqpoint{4.863052in}{1.725135in}}%
\pgfpathlineto{\pgfqpoint{4.862532in}{1.723085in}}%
\pgfpathlineto{\pgfqpoint{4.861493in}{1.684845in}}%
\pgfpathlineto{\pgfqpoint{4.859934in}{1.542586in}}%
\pgfpathlineto{\pgfqpoint{4.859415in}{1.544818in}}%
\pgfpathlineto{\pgfqpoint{4.857856in}{1.608391in}}%
\pgfpathlineto{\pgfqpoint{4.856297in}{1.582996in}}%
\pgfpathlineto{\pgfqpoint{4.854219in}{1.460801in}}%
\pgfpathlineto{\pgfqpoint{4.853700in}{1.479468in}}%
\pgfpathlineto{\pgfqpoint{4.851621in}{1.579445in}}%
\pgfpathlineto{\pgfqpoint{4.850582in}{1.545598in}}%
\pgfpathlineto{\pgfqpoint{4.849023in}{1.473719in}}%
\pgfpathlineto{\pgfqpoint{4.846945in}{1.594825in}}%
\pgfpathlineto{\pgfqpoint{4.846426in}{1.571165in}}%
\pgfpathlineto{\pgfqpoint{4.844867in}{1.414568in}}%
\pgfpathlineto{\pgfqpoint{4.844347in}{1.418592in}}%
\pgfpathlineto{\pgfqpoint{4.843828in}{1.431361in}}%
\pgfpathlineto{\pgfqpoint{4.843308in}{1.426478in}}%
\pgfpathlineto{\pgfqpoint{4.839671in}{1.298842in}}%
\pgfpathlineto{\pgfqpoint{4.837073in}{1.448010in}}%
\pgfpathlineto{\pgfqpoint{4.836554in}{1.447773in}}%
\pgfpathlineto{\pgfqpoint{4.832917in}{1.255784in}}%
\pgfpathlineto{\pgfqpoint{4.832397in}{1.262176in}}%
\pgfpathlineto{\pgfqpoint{4.830839in}{1.323419in}}%
\pgfpathlineto{\pgfqpoint{4.827202in}{1.184682in}}%
\pgfpathlineto{\pgfqpoint{4.826163in}{1.238106in}}%
\pgfpathlineto{\pgfqpoint{4.825643in}{1.270774in}}%
\pgfpathlineto{\pgfqpoint{4.825123in}{1.270708in}}%
\pgfpathlineto{\pgfqpoint{4.823045in}{1.106315in}}%
\pgfpathlineto{\pgfqpoint{4.822526in}{1.139762in}}%
\pgfpathlineto{\pgfqpoint{4.821486in}{1.199582in}}%
\pgfpathlineto{\pgfqpoint{4.820967in}{1.198885in}}%
\pgfpathlineto{\pgfqpoint{4.820447in}{1.193814in}}%
\pgfpathlineto{\pgfqpoint{4.819928in}{1.195098in}}%
\pgfpathlineto{\pgfqpoint{4.818369in}{1.222612in}}%
\pgfpathlineto{\pgfqpoint{4.817330in}{1.150489in}}%
\pgfpathlineto{\pgfqpoint{4.816291in}{1.050202in}}%
\pgfpathlineto{\pgfqpoint{4.815771in}{1.054474in}}%
\pgfpathlineto{\pgfqpoint{4.814732in}{1.076305in}}%
\pgfpathlineto{\pgfqpoint{4.813173in}{1.016578in}}%
\pgfpathlineto{\pgfqpoint{4.812654in}{1.031758in}}%
\pgfpathlineto{\pgfqpoint{4.811095in}{1.119708in}}%
\pgfpathlineto{\pgfqpoint{4.809017in}{1.015970in}}%
\pgfpathlineto{\pgfqpoint{4.808497in}{1.024148in}}%
\pgfpathlineto{\pgfqpoint{4.806939in}{1.082975in}}%
\pgfpathlineto{\pgfqpoint{4.805380in}{0.993239in}}%
\pgfpathlineto{\pgfqpoint{4.804860in}{1.015486in}}%
\pgfpathlineto{\pgfqpoint{4.803302in}{1.103003in}}%
\pgfpathlineto{\pgfqpoint{4.802782in}{1.100619in}}%
\pgfpathlineto{\pgfqpoint{4.801743in}{1.046931in}}%
\pgfpathlineto{\pgfqpoint{4.799665in}{0.896398in}}%
\pgfpathlineto{\pgfqpoint{4.797586in}{1.039138in}}%
\pgfpathlineto{\pgfqpoint{4.797067in}{1.004761in}}%
\pgfpathlineto{\pgfqpoint{4.794989in}{0.879831in}}%
\pgfpathlineto{\pgfqpoint{4.794469in}{0.876332in}}%
\pgfpathlineto{\pgfqpoint{4.793949in}{0.881775in}}%
\pgfpathlineto{\pgfqpoint{4.791352in}{1.006518in}}%
\pgfpathlineto{\pgfqpoint{4.790312in}{0.989178in}}%
\pgfpathlineto{\pgfqpoint{4.788754in}{0.928980in}}%
\pgfpathlineto{\pgfqpoint{4.788234in}{0.945597in}}%
\pgfpathlineto{\pgfqpoint{4.787195in}{1.026779in}}%
\pgfpathlineto{\pgfqpoint{4.786675in}{1.014270in}}%
\pgfpathlineto{\pgfqpoint{4.785117in}{0.850169in}}%
\pgfpathlineto{\pgfqpoint{4.784597in}{0.860896in}}%
\pgfpathlineto{\pgfqpoint{4.783558in}{0.929319in}}%
\pgfpathlineto{\pgfqpoint{4.783039in}{0.922349in}}%
\pgfpathlineto{\pgfqpoint{4.781480in}{0.805679in}}%
\pgfpathlineto{\pgfqpoint{4.780960in}{0.824451in}}%
\pgfpathlineto{\pgfqpoint{4.778882in}{0.910082in}}%
\pgfpathlineto{\pgfqpoint{4.776804in}{1.019721in}}%
\pgfpathlineto{\pgfqpoint{4.776284in}{1.018876in}}%
\pgfpathlineto{\pgfqpoint{4.774206in}{0.888681in}}%
\pgfpathlineto{\pgfqpoint{4.773167in}{0.897064in}}%
\pgfpathlineto{\pgfqpoint{4.771608in}{0.868559in}}%
\pgfpathlineto{\pgfqpoint{4.771088in}{0.874229in}}%
\pgfpathlineto{\pgfqpoint{4.769530in}{0.947451in}}%
\pgfpathlineto{\pgfqpoint{4.769010in}{0.946749in}}%
\pgfpathlineto{\pgfqpoint{4.766412in}{0.827090in}}%
\pgfpathlineto{\pgfqpoint{4.765893in}{0.836623in}}%
\pgfpathlineto{\pgfqpoint{4.764334in}{0.976628in}}%
\pgfpathlineto{\pgfqpoint{4.763815in}{0.969363in}}%
\pgfpathlineto{\pgfqpoint{4.762256in}{0.923145in}}%
\pgfpathlineto{\pgfqpoint{4.761217in}{0.943147in}}%
\pgfpathlineto{\pgfqpoint{4.757580in}{0.867496in}}%
\pgfpathlineto{\pgfqpoint{4.757060in}{0.872886in}}%
\pgfpathlineto{\pgfqpoint{4.755501in}{0.957652in}}%
\pgfpathlineto{\pgfqpoint{4.754982in}{0.957258in}}%
\pgfpathlineto{\pgfqpoint{4.754462in}{0.959627in}}%
\pgfpathlineto{\pgfqpoint{4.752904in}{1.007901in}}%
\pgfpathlineto{\pgfqpoint{4.751345in}{0.928712in}}%
\pgfpathlineto{\pgfqpoint{4.750825in}{0.934709in}}%
\pgfpathlineto{\pgfqpoint{4.748747in}{0.994696in}}%
\pgfpathlineto{\pgfqpoint{4.748228in}{0.997450in}}%
\pgfpathlineto{\pgfqpoint{4.747708in}{0.994451in}}%
\pgfpathlineto{\pgfqpoint{4.746149in}{0.969291in}}%
\pgfpathlineto{\pgfqpoint{4.745630in}{0.969593in}}%
\pgfpathlineto{\pgfqpoint{4.745110in}{0.969173in}}%
\pgfpathlineto{\pgfqpoint{4.744591in}{0.970601in}}%
\pgfpathlineto{\pgfqpoint{4.743551in}{0.981324in}}%
\pgfpathlineto{\pgfqpoint{4.743032in}{0.981071in}}%
\pgfpathlineto{\pgfqpoint{4.742512in}{0.981054in}}%
\pgfpathlineto{\pgfqpoint{4.740954in}{1.010546in}}%
\pgfpathlineto{\pgfqpoint{4.740434in}{1.000093in}}%
\pgfpathlineto{\pgfqpoint{4.738356in}{0.932196in}}%
\pgfpathlineto{\pgfqpoint{4.737836in}{0.932935in}}%
\pgfpathlineto{\pgfqpoint{4.736797in}{0.972433in}}%
\pgfpathlineto{\pgfqpoint{4.735238in}{1.101313in}}%
\pgfpathlineto{\pgfqpoint{4.734719in}{1.087173in}}%
\pgfpathlineto{\pgfqpoint{4.733680in}{1.034862in}}%
\pgfpathlineto{\pgfqpoint{4.733160in}{1.038595in}}%
\pgfpathlineto{\pgfqpoint{4.732641in}{1.046722in}}%
\pgfpathlineto{\pgfqpoint{4.732121in}{1.044510in}}%
\pgfpathlineto{\pgfqpoint{4.730562in}{1.035880in}}%
\pgfpathlineto{\pgfqpoint{4.729004in}{0.918256in}}%
\pgfpathlineto{\pgfqpoint{4.728484in}{0.940528in}}%
\pgfpathlineto{\pgfqpoint{4.726406in}{1.059533in}}%
\pgfpathlineto{\pgfqpoint{4.725367in}{1.101298in}}%
\pgfpathlineto{\pgfqpoint{4.723808in}{1.147563in}}%
\pgfpathlineto{\pgfqpoint{4.722769in}{1.167572in}}%
\pgfpathlineto{\pgfqpoint{4.722249in}{1.178983in}}%
\pgfpathlineto{\pgfqpoint{4.721730in}{1.172822in}}%
\pgfpathlineto{\pgfqpoint{4.720691in}{1.143578in}}%
\pgfpathlineto{\pgfqpoint{4.719132in}{1.197670in}}%
\pgfpathlineto{\pgfqpoint{4.718612in}{1.188596in}}%
\pgfpathlineto{\pgfqpoint{4.717573in}{1.166784in}}%
\pgfpathlineto{\pgfqpoint{4.715495in}{1.235220in}}%
\pgfpathlineto{\pgfqpoint{4.714975in}{1.215535in}}%
\pgfpathlineto{\pgfqpoint{4.713936in}{1.162269in}}%
\pgfpathlineto{\pgfqpoint{4.713417in}{1.170674in}}%
\pgfpathlineto{\pgfqpoint{4.711338in}{1.292882in}}%
\pgfpathlineto{\pgfqpoint{4.710819in}{1.281430in}}%
\pgfpathlineto{\pgfqpoint{4.708740in}{1.223213in}}%
\pgfpathlineto{\pgfqpoint{4.708221in}{1.231094in}}%
\pgfpathlineto{\pgfqpoint{4.706143in}{1.363502in}}%
\pgfpathlineto{\pgfqpoint{4.705623in}{1.355964in}}%
\pgfpathlineto{\pgfqpoint{4.703545in}{1.269978in}}%
\pgfpathlineto{\pgfqpoint{4.703025in}{1.282233in}}%
\pgfpathlineto{\pgfqpoint{4.701467in}{1.323728in}}%
\pgfpathlineto{\pgfqpoint{4.699908in}{1.289339in}}%
\pgfpathlineto{\pgfqpoint{4.696790in}{1.428257in}}%
\pgfpathlineto{\pgfqpoint{4.696271in}{1.416058in}}%
\pgfpathlineto{\pgfqpoint{4.693673in}{1.325376in}}%
\pgfpathlineto{\pgfqpoint{4.693154in}{1.312184in}}%
\pgfpathlineto{\pgfqpoint{4.692634in}{1.320924in}}%
\pgfpathlineto{\pgfqpoint{4.690036in}{1.545036in}}%
\pgfpathlineto{\pgfqpoint{4.689517in}{1.525678in}}%
\pgfpathlineto{\pgfqpoint{4.687958in}{1.422854in}}%
\pgfpathlineto{\pgfqpoint{4.687438in}{1.448187in}}%
\pgfpathlineto{\pgfqpoint{4.685360in}{1.615029in}}%
\pgfpathlineto{\pgfqpoint{4.682762in}{1.523860in}}%
\pgfpathlineto{\pgfqpoint{4.682243in}{1.527195in}}%
\pgfpathlineto{\pgfqpoint{4.680164in}{1.599974in}}%
\pgfpathlineto{\pgfqpoint{4.679645in}{1.597884in}}%
\pgfpathlineto{\pgfqpoint{4.677567in}{1.583835in}}%
\pgfpathlineto{\pgfqpoint{4.677047in}{1.585273in}}%
\pgfpathlineto{\pgfqpoint{4.675488in}{1.649188in}}%
\pgfpathlineto{\pgfqpoint{4.674449in}{1.685441in}}%
\pgfpathlineto{\pgfqpoint{4.673930in}{1.677655in}}%
\pgfpathlineto{\pgfqpoint{4.672890in}{1.630363in}}%
\pgfpathlineto{\pgfqpoint{4.672371in}{1.644353in}}%
\pgfpathlineto{\pgfqpoint{4.670293in}{1.802775in}}%
\pgfpathlineto{\pgfqpoint{4.669773in}{1.800256in}}%
\pgfpathlineto{\pgfqpoint{4.668214in}{1.739698in}}%
\pgfpathlineto{\pgfqpoint{4.667695in}{1.752269in}}%
\pgfpathlineto{\pgfqpoint{4.665616in}{1.819771in}}%
\pgfpathlineto{\pgfqpoint{4.664577in}{1.793313in}}%
\pgfpathlineto{\pgfqpoint{4.663538in}{1.768440in}}%
\pgfpathlineto{\pgfqpoint{4.662499in}{1.826971in}}%
\pgfpathlineto{\pgfqpoint{4.660421in}{1.991273in}}%
\pgfpathlineto{\pgfqpoint{4.659901in}{1.983060in}}%
\pgfpathlineto{\pgfqpoint{4.657303in}{1.811046in}}%
\pgfpathlineto{\pgfqpoint{4.656264in}{1.854446in}}%
\pgfpathlineto{\pgfqpoint{4.653666in}{1.995389in}}%
\pgfpathlineto{\pgfqpoint{4.653147in}{2.001130in}}%
\pgfpathlineto{\pgfqpoint{4.652627in}{1.995482in}}%
\pgfpathlineto{\pgfqpoint{4.651588in}{1.975604in}}%
\pgfpathlineto{\pgfqpoint{4.650549in}{2.008914in}}%
\pgfpathlineto{\pgfqpoint{4.649510in}{2.035138in}}%
\pgfpathlineto{\pgfqpoint{4.647432in}{1.972988in}}%
\pgfpathlineto{\pgfqpoint{4.646912in}{1.982024in}}%
\pgfpathlineto{\pgfqpoint{4.644834in}{2.015886in}}%
\pgfpathlineto{\pgfqpoint{4.643795in}{2.011924in}}%
\pgfpathlineto{\pgfqpoint{4.642756in}{2.050444in}}%
\pgfpathlineto{\pgfqpoint{4.638079in}{2.264914in}}%
\pgfpathlineto{\pgfqpoint{4.637560in}{2.257940in}}%
\pgfpathlineto{\pgfqpoint{4.636521in}{2.207907in}}%
\pgfpathlineto{\pgfqpoint{4.636001in}{2.216354in}}%
\pgfpathlineto{\pgfqpoint{4.634442in}{2.331470in}}%
\pgfpathlineto{\pgfqpoint{4.633923in}{2.327363in}}%
\pgfpathlineto{\pgfqpoint{4.630806in}{2.239331in}}%
\pgfpathlineto{\pgfqpoint{4.630286in}{2.241202in}}%
\pgfpathlineto{\pgfqpoint{4.626649in}{2.307858in}}%
\pgfpathlineto{\pgfqpoint{4.625610in}{2.333284in}}%
\pgfpathlineto{\pgfqpoint{4.624571in}{2.307832in}}%
\pgfpathlineto{\pgfqpoint{4.623532in}{2.361585in}}%
\pgfpathlineto{\pgfqpoint{4.621973in}{2.414775in}}%
\pgfpathlineto{\pgfqpoint{4.620934in}{2.395168in}}%
\pgfpathlineto{\pgfqpoint{4.619895in}{2.359947in}}%
\pgfpathlineto{\pgfqpoint{4.619375in}{2.362780in}}%
\pgfpathlineto{\pgfqpoint{4.616258in}{2.510436in}}%
\pgfpathlineto{\pgfqpoint{4.615738in}{2.493082in}}%
\pgfpathlineto{\pgfqpoint{4.615219in}{2.475899in}}%
\pgfpathlineto{\pgfqpoint{4.614699in}{2.480479in}}%
\pgfpathlineto{\pgfqpoint{4.612621in}{2.582266in}}%
\pgfpathlineto{\pgfqpoint{4.611062in}{2.515077in}}%
\pgfpathlineto{\pgfqpoint{4.609503in}{2.416740in}}%
\pgfpathlineto{\pgfqpoint{4.606905in}{2.570888in}}%
\pgfpathlineto{\pgfqpoint{4.606386in}{2.566309in}}%
\pgfpathlineto{\pgfqpoint{4.605866in}{2.566710in}}%
\pgfpathlineto{\pgfqpoint{4.603788in}{2.683167in}}%
\pgfpathlineto{\pgfqpoint{4.602749in}{2.653679in}}%
\pgfpathlineto{\pgfqpoint{4.600671in}{2.596395in}}%
\pgfpathlineto{\pgfqpoint{4.600151in}{2.597758in}}%
\pgfpathlineto{\pgfqpoint{4.599632in}{2.599697in}}%
\pgfpathlineto{\pgfqpoint{4.598073in}{2.568229in}}%
\pgfpathlineto{\pgfqpoint{4.597553in}{2.582649in}}%
\pgfpathlineto{\pgfqpoint{4.595995in}{2.745016in}}%
\pgfpathlineto{\pgfqpoint{4.595475in}{2.740001in}}%
\pgfpathlineto{\pgfqpoint{4.594436in}{2.685485in}}%
\pgfpathlineto{\pgfqpoint{4.593916in}{2.692483in}}%
\pgfpathlineto{\pgfqpoint{4.592877in}{2.706404in}}%
\pgfpathlineto{\pgfqpoint{4.592358in}{2.699371in}}%
\pgfpathlineto{\pgfqpoint{4.591838in}{2.702087in}}%
\pgfpathlineto{\pgfqpoint{4.590799in}{2.717555in}}%
\pgfpathlineto{\pgfqpoint{4.589240in}{2.638671in}}%
\pgfpathlineto{\pgfqpoint{4.588721in}{2.640110in}}%
\pgfpathlineto{\pgfqpoint{4.586123in}{2.836294in}}%
\pgfpathlineto{\pgfqpoint{4.585603in}{2.816348in}}%
\pgfpathlineto{\pgfqpoint{4.583005in}{2.638894in}}%
\pgfpathlineto{\pgfqpoint{4.582486in}{2.639447in}}%
\pgfpathlineto{\pgfqpoint{4.578849in}{2.811773in}}%
\pgfpathlineto{\pgfqpoint{4.578329in}{2.809250in}}%
\pgfpathlineto{\pgfqpoint{4.576771in}{2.693150in}}%
\pgfpathlineto{\pgfqpoint{4.576251in}{2.702088in}}%
\pgfpathlineto{\pgfqpoint{4.573134in}{2.835461in}}%
\pgfpathlineto{\pgfqpoint{4.572614in}{2.825391in}}%
\pgfpathlineto{\pgfqpoint{4.570536in}{2.747080in}}%
\pgfpathlineto{\pgfqpoint{4.570016in}{2.758355in}}%
\pgfpathlineto{\pgfqpoint{4.567938in}{2.823325in}}%
\pgfpathlineto{\pgfqpoint{4.566899in}{2.780584in}}%
\pgfpathlineto{\pgfqpoint{4.566379in}{2.754000in}}%
\pgfpathlineto{\pgfqpoint{4.565860in}{2.759023in}}%
\pgfpathlineto{\pgfqpoint{4.564301in}{2.857998in}}%
\pgfpathlineto{\pgfqpoint{4.563781in}{2.842368in}}%
\pgfpathlineto{\pgfqpoint{4.563262in}{2.822254in}}%
\pgfpathlineto{\pgfqpoint{4.562742in}{2.824472in}}%
\pgfpathlineto{\pgfqpoint{4.560664in}{2.894554in}}%
\pgfpathlineto{\pgfqpoint{4.560144in}{2.894196in}}%
\pgfpathlineto{\pgfqpoint{4.559625in}{2.897343in}}%
\pgfpathlineto{\pgfqpoint{4.559105in}{2.896937in}}%
\pgfpathlineto{\pgfqpoint{4.557547in}{2.869964in}}%
\pgfpathlineto{\pgfqpoint{4.556508in}{2.895852in}}%
\pgfpathlineto{\pgfqpoint{4.555988in}{2.895412in}}%
\pgfpathlineto{\pgfqpoint{4.554429in}{2.851446in}}%
\pgfpathlineto{\pgfqpoint{4.552871in}{2.899842in}}%
\pgfpathlineto{\pgfqpoint{4.551312in}{2.837191in}}%
\pgfpathlineto{\pgfqpoint{4.550792in}{2.846671in}}%
\pgfpathlineto{\pgfqpoint{4.549753in}{2.871915in}}%
\pgfpathlineto{\pgfqpoint{4.549234in}{2.868402in}}%
\pgfpathlineto{\pgfqpoint{4.547155in}{2.810621in}}%
\pgfpathlineto{\pgfqpoint{4.546116in}{2.786687in}}%
\pgfpathlineto{\pgfqpoint{4.545597in}{2.789321in}}%
\pgfpathlineto{\pgfqpoint{4.544038in}{2.860566in}}%
\pgfpathlineto{\pgfqpoint{4.543518in}{2.839904in}}%
\pgfpathlineto{\pgfqpoint{4.542479in}{2.752902in}}%
\pgfpathlineto{\pgfqpoint{4.541960in}{2.753085in}}%
\pgfpathlineto{\pgfqpoint{4.540401in}{2.879479in}}%
\pgfpathlineto{\pgfqpoint{4.539881in}{2.874733in}}%
\pgfpathlineto{\pgfqpoint{4.538842in}{2.844521in}}%
\pgfpathlineto{\pgfqpoint{4.537803in}{2.903747in}}%
\pgfpathlineto{\pgfqpoint{4.537284in}{2.900091in}}%
\pgfpathlineto{\pgfqpoint{4.535725in}{2.781977in}}%
\pgfpathlineto{\pgfqpoint{4.535205in}{2.811243in}}%
\pgfpathlineto{\pgfqpoint{4.533647in}{2.910314in}}%
\pgfpathlineto{\pgfqpoint{4.531568in}{2.710075in}}%
\pgfpathlineto{\pgfqpoint{4.530529in}{2.743859in}}%
\pgfpathlineto{\pgfqpoint{4.527412in}{2.837066in}}%
\pgfpathlineto{\pgfqpoint{4.526892in}{2.834277in}}%
\pgfpathlineto{\pgfqpoint{4.525334in}{2.775776in}}%
\pgfpathlineto{\pgfqpoint{4.524814in}{2.781380in}}%
\pgfpathlineto{\pgfqpoint{4.522736in}{2.865432in}}%
\pgfpathlineto{\pgfqpoint{4.520138in}{2.763827in}}%
\pgfpathlineto{\pgfqpoint{4.519618in}{2.759497in}}%
\pgfpathlineto{\pgfqpoint{4.518060in}{2.808992in}}%
\pgfpathlineto{\pgfqpoint{4.517540in}{2.795755in}}%
\pgfpathlineto{\pgfqpoint{4.514942in}{2.710077in}}%
\pgfpathlineto{\pgfqpoint{4.514423in}{2.705353in}}%
\pgfpathlineto{\pgfqpoint{4.513383in}{2.730483in}}%
\pgfpathlineto{\pgfqpoint{4.512864in}{2.723598in}}%
\pgfpathlineto{\pgfqpoint{4.511825in}{2.698861in}}%
\pgfpathlineto{\pgfqpoint{4.509747in}{2.774876in}}%
\pgfpathlineto{\pgfqpoint{4.509227in}{2.767089in}}%
\pgfpathlineto{\pgfqpoint{4.504551in}{2.636669in}}%
\pgfpathlineto{\pgfqpoint{4.503512in}{2.598211in}}%
\pgfpathlineto{\pgfqpoint{4.501953in}{2.667707in}}%
\pgfpathlineto{\pgfqpoint{4.501433in}{2.661489in}}%
\pgfpathlineto{\pgfqpoint{4.499355in}{2.471313in}}%
\pgfpathlineto{\pgfqpoint{4.498316in}{2.522691in}}%
\pgfpathlineto{\pgfqpoint{4.495718in}{2.628141in}}%
\pgfpathlineto{\pgfqpoint{4.495199in}{2.625511in}}%
\pgfpathlineto{\pgfqpoint{4.493120in}{2.566580in}}%
\pgfpathlineto{\pgfqpoint{4.492081in}{2.516647in}}%
\pgfpathlineto{\pgfqpoint{4.491562in}{2.520407in}}%
\pgfpathlineto{\pgfqpoint{4.488964in}{2.587854in}}%
\pgfpathlineto{\pgfqpoint{4.488444in}{2.581360in}}%
\pgfpathlineto{\pgfqpoint{4.486886in}{2.470637in}}%
\pgfpathlineto{\pgfqpoint{4.486366in}{2.444473in}}%
\pgfpathlineto{\pgfqpoint{4.485846in}{2.459336in}}%
\pgfpathlineto{\pgfqpoint{4.484288in}{2.544810in}}%
\pgfpathlineto{\pgfqpoint{4.482210in}{2.450293in}}%
\pgfpathlineto{\pgfqpoint{4.481690in}{2.454586in}}%
\pgfpathlineto{\pgfqpoint{4.481170in}{2.455141in}}%
\pgfpathlineto{\pgfqpoint{4.480131in}{2.406590in}}%
\pgfpathlineto{\pgfqpoint{4.478573in}{2.333392in}}%
\pgfpathlineto{\pgfqpoint{4.477533in}{2.389137in}}%
\pgfpathlineto{\pgfqpoint{4.475975in}{2.492232in}}%
\pgfpathlineto{\pgfqpoint{4.473377in}{2.329165in}}%
\pgfpathlineto{\pgfqpoint{4.472857in}{2.342893in}}%
\pgfpathlineto{\pgfqpoint{4.471818in}{2.374088in}}%
\pgfpathlineto{\pgfqpoint{4.470779in}{2.344943in}}%
\pgfpathlineto{\pgfqpoint{4.470259in}{2.348867in}}%
\pgfpathlineto{\pgfqpoint{4.469220in}{2.379224in}}%
\pgfpathlineto{\pgfqpoint{4.468181in}{2.324740in}}%
\pgfpathlineto{\pgfqpoint{4.465583in}{2.189870in}}%
\pgfpathlineto{\pgfqpoint{4.463505in}{2.143569in}}%
\pgfpathlineto{\pgfqpoint{4.462986in}{2.150165in}}%
\pgfpathlineto{\pgfqpoint{4.461427in}{2.226446in}}%
\pgfpathlineto{\pgfqpoint{4.457270in}{2.014789in}}%
\pgfpathlineto{\pgfqpoint{4.456231in}{1.996898in}}%
\pgfpathlineto{\pgfqpoint{4.455712in}{2.000771in}}%
\pgfpathlineto{\pgfqpoint{4.454153in}{2.057287in}}%
\pgfpathlineto{\pgfqpoint{4.452594in}{2.109980in}}%
\pgfpathlineto{\pgfqpoint{4.449996in}{2.086514in}}%
\pgfpathlineto{\pgfqpoint{4.448438in}{1.985119in}}%
\pgfpathlineto{\pgfqpoint{4.447918in}{1.959657in}}%
\pgfpathlineto{\pgfqpoint{4.447399in}{1.971213in}}%
\pgfpathlineto{\pgfqpoint{4.445840in}{2.032984in}}%
\pgfpathlineto{\pgfqpoint{4.445320in}{2.027105in}}%
\pgfpathlineto{\pgfqpoint{4.444801in}{2.028324in}}%
\pgfpathlineto{\pgfqpoint{4.444281in}{2.029377in}}%
\pgfpathlineto{\pgfqpoint{4.443242in}{1.967447in}}%
\pgfpathlineto{\pgfqpoint{4.442203in}{1.890985in}}%
\pgfpathlineto{\pgfqpoint{4.441683in}{1.891014in}}%
\pgfpathlineto{\pgfqpoint{4.440644in}{1.907913in}}%
\pgfpathlineto{\pgfqpoint{4.439605in}{1.893909in}}%
\pgfpathlineto{\pgfqpoint{4.439085in}{1.899873in}}%
\pgfpathlineto{\pgfqpoint{4.438566in}{1.905934in}}%
\pgfpathlineto{\pgfqpoint{4.437527in}{1.856272in}}%
\pgfpathlineto{\pgfqpoint{4.434929in}{1.728174in}}%
\pgfpathlineto{\pgfqpoint{4.433370in}{1.795483in}}%
\pgfpathlineto{\pgfqpoint{4.432851in}{1.779924in}}%
\pgfpathlineto{\pgfqpoint{4.430772in}{1.682584in}}%
\pgfpathlineto{\pgfqpoint{4.428694in}{1.731856in}}%
\pgfpathlineto{\pgfqpoint{4.428175in}{1.731727in}}%
\pgfpathlineto{\pgfqpoint{4.427135in}{1.676801in}}%
\pgfpathlineto{\pgfqpoint{4.426096in}{1.605946in}}%
\pgfpathlineto{\pgfqpoint{4.425577in}{1.629430in}}%
\pgfpathlineto{\pgfqpoint{4.424538in}{1.706179in}}%
\pgfpathlineto{\pgfqpoint{4.424018in}{1.679875in}}%
\pgfpathlineto{\pgfqpoint{4.422459in}{1.565480in}}%
\pgfpathlineto{\pgfqpoint{4.420901in}{1.641748in}}%
\pgfpathlineto{\pgfqpoint{4.420381in}{1.639321in}}%
\pgfpathlineto{\pgfqpoint{4.418822in}{1.611327in}}%
\pgfpathlineto{\pgfqpoint{4.418303in}{1.614946in}}%
\pgfpathlineto{\pgfqpoint{4.417783in}{1.612039in}}%
\pgfpathlineto{\pgfqpoint{4.416225in}{1.551016in}}%
\pgfpathlineto{\pgfqpoint{4.415705in}{1.555625in}}%
\pgfpathlineto{\pgfqpoint{4.414146in}{1.607292in}}%
\pgfpathlineto{\pgfqpoint{4.412068in}{1.526004in}}%
\pgfpathlineto{\pgfqpoint{4.411029in}{1.540878in}}%
\pgfpathlineto{\pgfqpoint{4.409990in}{1.516928in}}%
\pgfpathlineto{\pgfqpoint{4.405833in}{1.368679in}}%
\pgfpathlineto{\pgfqpoint{4.404275in}{1.400226in}}%
\pgfpathlineto{\pgfqpoint{4.403755in}{1.392926in}}%
\pgfpathlineto{\pgfqpoint{4.402716in}{1.381685in}}%
\pgfpathlineto{\pgfqpoint{4.401677in}{1.397642in}}%
\pgfpathlineto{\pgfqpoint{4.401157in}{1.393404in}}%
\pgfpathlineto{\pgfqpoint{4.400638in}{1.387228in}}%
\pgfpathlineto{\pgfqpoint{4.399598in}{1.415203in}}%
\pgfpathlineto{\pgfqpoint{4.399079in}{1.429286in}}%
\pgfpathlineto{\pgfqpoint{4.398559in}{1.415127in}}%
\pgfpathlineto{\pgfqpoint{4.396481in}{1.289245in}}%
\pgfpathlineto{\pgfqpoint{4.395961in}{1.296433in}}%
\pgfpathlineto{\pgfqpoint{4.394403in}{1.385871in}}%
\pgfpathlineto{\pgfqpoint{4.393883in}{1.371904in}}%
\pgfpathlineto{\pgfqpoint{4.391285in}{1.250129in}}%
\pgfpathlineto{\pgfqpoint{4.389207in}{1.293445in}}%
\pgfpathlineto{\pgfqpoint{4.388688in}{1.292114in}}%
\pgfpathlineto{\pgfqpoint{4.388168in}{1.287826in}}%
\pgfpathlineto{\pgfqpoint{4.387129in}{1.241963in}}%
\pgfpathlineto{\pgfqpoint{4.385570in}{1.182530in}}%
\pgfpathlineto{\pgfqpoint{4.384011in}{1.194353in}}%
\pgfpathlineto{\pgfqpoint{4.382972in}{1.154671in}}%
\pgfpathlineto{\pgfqpoint{4.380894in}{1.044221in}}%
\pgfpathlineto{\pgfqpoint{4.380374in}{1.043734in}}%
\pgfpathlineto{\pgfqpoint{4.379855in}{1.049716in}}%
\pgfpathlineto{\pgfqpoint{4.378816in}{1.119947in}}%
\pgfpathlineto{\pgfqpoint{4.377257in}{1.215181in}}%
\pgfpathlineto{\pgfqpoint{4.376218in}{1.193646in}}%
\pgfpathlineto{\pgfqpoint{4.375179in}{1.234008in}}%
\pgfpathlineto{\pgfqpoint{4.374659in}{1.227818in}}%
\pgfpathlineto{\pgfqpoint{4.372061in}{1.062092in}}%
\pgfpathlineto{\pgfqpoint{4.371542in}{1.070235in}}%
\pgfpathlineto{\pgfqpoint{4.368944in}{1.249582in}}%
\pgfpathlineto{\pgfqpoint{4.368424in}{1.219594in}}%
\pgfpathlineto{\pgfqpoint{4.366346in}{0.964601in}}%
\pgfpathlineto{\pgfqpoint{4.365827in}{0.968204in}}%
\pgfpathlineto{\pgfqpoint{4.364787in}{0.994133in}}%
\pgfpathlineto{\pgfqpoint{4.363748in}{0.931184in}}%
\pgfpathlineto{\pgfqpoint{4.362709in}{0.883464in}}%
\pgfpathlineto{\pgfqpoint{4.360631in}{1.098426in}}%
\pgfpathlineto{\pgfqpoint{4.360111in}{1.095669in}}%
\pgfpathlineto{\pgfqpoint{4.357514in}{1.018921in}}%
\pgfpathlineto{\pgfqpoint{4.356994in}{1.028500in}}%
\pgfpathlineto{\pgfqpoint{4.355955in}{1.043738in}}%
\pgfpathlineto{\pgfqpoint{4.353877in}{0.957804in}}%
\pgfpathlineto{\pgfqpoint{4.353357in}{0.967559in}}%
\pgfpathlineto{\pgfqpoint{4.352837in}{0.979828in}}%
\pgfpathlineto{\pgfqpoint{4.352318in}{0.979229in}}%
\pgfpathlineto{\pgfqpoint{4.351279in}{0.957052in}}%
\pgfpathlineto{\pgfqpoint{4.350759in}{0.964655in}}%
\pgfpathlineto{\pgfqpoint{4.347642in}{1.066000in}}%
\pgfpathlineto{\pgfqpoint{4.346083in}{1.047638in}}%
\pgfpathlineto{\pgfqpoint{4.345564in}{1.053021in}}%
\pgfpathlineto{\pgfqpoint{4.345044in}{1.047952in}}%
\pgfpathlineto{\pgfqpoint{4.342446in}{0.936134in}}%
\pgfpathlineto{\pgfqpoint{4.341927in}{0.941334in}}%
\pgfpathlineto{\pgfqpoint{4.341407in}{0.947794in}}%
\pgfpathlineto{\pgfqpoint{4.340887in}{0.946328in}}%
\pgfpathlineto{\pgfqpoint{4.339329in}{0.896153in}}%
\pgfpathlineto{\pgfqpoint{4.338809in}{0.909020in}}%
\pgfpathlineto{\pgfqpoint{4.337250in}{0.982703in}}%
\pgfpathlineto{\pgfqpoint{4.336731in}{0.972310in}}%
\pgfpathlineto{\pgfqpoint{4.335692in}{0.925920in}}%
\pgfpathlineto{\pgfqpoint{4.335172in}{0.934336in}}%
\pgfpathlineto{\pgfqpoint{4.334133in}{0.971156in}}%
\pgfpathlineto{\pgfqpoint{4.332574in}{0.926130in}}%
\pgfpathlineto{\pgfqpoint{4.330496in}{0.994774in}}%
\pgfpathlineto{\pgfqpoint{4.329977in}{0.989268in}}%
\pgfpathlineto{\pgfqpoint{4.327898in}{0.897502in}}%
\pgfpathlineto{\pgfqpoint{4.327379in}{0.908990in}}%
\pgfpathlineto{\pgfqpoint{4.326340in}{0.925741in}}%
\pgfpathlineto{\pgfqpoint{4.325820in}{0.920926in}}%
\pgfpathlineto{\pgfqpoint{4.325300in}{0.931331in}}%
\pgfpathlineto{\pgfqpoint{4.323742in}{1.049992in}}%
\pgfpathlineto{\pgfqpoint{4.323222in}{1.032699in}}%
\pgfpathlineto{\pgfqpoint{4.320624in}{0.906589in}}%
\pgfpathlineto{\pgfqpoint{4.315948in}{1.144789in}}%
\pgfpathlineto{\pgfqpoint{4.315429in}{1.113856in}}%
\pgfpathlineto{\pgfqpoint{4.313350in}{1.020576in}}%
\pgfpathlineto{\pgfqpoint{4.311792in}{0.956981in}}%
\pgfpathlineto{\pgfqpoint{4.311272in}{0.958767in}}%
\pgfpathlineto{\pgfqpoint{4.309194in}{1.074277in}}%
\pgfpathlineto{\pgfqpoint{4.308674in}{1.067928in}}%
\pgfpathlineto{\pgfqpoint{4.306076in}{0.981773in}}%
\pgfpathlineto{\pgfqpoint{4.305557in}{1.003271in}}%
\pgfpathlineto{\pgfqpoint{4.304518in}{1.035671in}}%
\pgfpathlineto{\pgfqpoint{4.302440in}{0.976531in}}%
\pgfpathlineto{\pgfqpoint{4.301920in}{0.979747in}}%
\pgfpathlineto{\pgfqpoint{4.300361in}{1.086565in}}%
\pgfpathlineto{\pgfqpoint{4.298283in}{1.173503in}}%
\pgfpathlineto{\pgfqpoint{4.297244in}{1.149931in}}%
\pgfpathlineto{\pgfqpoint{4.294646in}{1.013592in}}%
\pgfpathlineto{\pgfqpoint{4.294126in}{1.034229in}}%
\pgfpathlineto{\pgfqpoint{4.292048in}{1.140895in}}%
\pgfpathlineto{\pgfqpoint{4.290489in}{1.061679in}}%
\pgfpathlineto{\pgfqpoint{4.289970in}{1.079024in}}%
\pgfpathlineto{\pgfqpoint{4.288411in}{1.166743in}}%
\pgfpathlineto{\pgfqpoint{4.287892in}{1.158083in}}%
\pgfpathlineto{\pgfqpoint{4.286853in}{1.125128in}}%
\pgfpathlineto{\pgfqpoint{4.285813in}{1.184595in}}%
\pgfpathlineto{\pgfqpoint{4.284774in}{1.227443in}}%
\pgfpathlineto{\pgfqpoint{4.282696in}{1.117970in}}%
\pgfpathlineto{\pgfqpoint{4.277500in}{1.362562in}}%
\pgfpathlineto{\pgfqpoint{4.276981in}{1.364960in}}%
\pgfpathlineto{\pgfqpoint{4.274902in}{1.322909in}}%
\pgfpathlineto{\pgfqpoint{4.274383in}{1.327207in}}%
\pgfpathlineto{\pgfqpoint{4.272824in}{1.362894in}}%
\pgfpathlineto{\pgfqpoint{4.272305in}{1.352585in}}%
\pgfpathlineto{\pgfqpoint{4.270746in}{1.266525in}}%
\pgfpathlineto{\pgfqpoint{4.270226in}{1.279408in}}%
\pgfpathlineto{\pgfqpoint{4.266070in}{1.547679in}}%
\pgfpathlineto{\pgfqpoint{4.265031in}{1.502923in}}%
\pgfpathlineto{\pgfqpoint{4.263992in}{1.462425in}}%
\pgfpathlineto{\pgfqpoint{4.263472in}{1.472413in}}%
\pgfpathlineto{\pgfqpoint{4.262952in}{1.480160in}}%
\pgfpathlineto{\pgfqpoint{4.261913in}{1.433086in}}%
\pgfpathlineto{\pgfqpoint{4.260874in}{1.381887in}}%
\pgfpathlineto{\pgfqpoint{4.259835in}{1.454450in}}%
\pgfpathlineto{\pgfqpoint{4.258796in}{1.517607in}}%
\pgfpathlineto{\pgfqpoint{4.258276in}{1.511036in}}%
\pgfpathlineto{\pgfqpoint{4.257757in}{1.504398in}}%
\pgfpathlineto{\pgfqpoint{4.257237in}{1.511342in}}%
\pgfpathlineto{\pgfqpoint{4.255159in}{1.539868in}}%
\pgfpathlineto{\pgfqpoint{4.254639in}{1.539802in}}%
\pgfpathlineto{\pgfqpoint{4.254120in}{1.543756in}}%
\pgfpathlineto{\pgfqpoint{4.252561in}{1.585170in}}%
\pgfpathlineto{\pgfqpoint{4.252042in}{1.576891in}}%
\pgfpathlineto{\pgfqpoint{4.249963in}{1.539787in}}%
\pgfpathlineto{\pgfqpoint{4.249444in}{1.536232in}}%
\pgfpathlineto{\pgfqpoint{4.248405in}{1.574538in}}%
\pgfpathlineto{\pgfqpoint{4.245807in}{1.859837in}}%
\pgfpathlineto{\pgfqpoint{4.245287in}{1.829010in}}%
\pgfpathlineto{\pgfqpoint{4.243209in}{1.602156in}}%
\pgfpathlineto{\pgfqpoint{4.242689in}{1.636924in}}%
\pgfpathlineto{\pgfqpoint{4.241131in}{1.769656in}}%
\pgfpathlineto{\pgfqpoint{4.239572in}{1.697829in}}%
\pgfpathlineto{\pgfqpoint{4.239052in}{1.713940in}}%
\pgfpathlineto{\pgfqpoint{4.236974in}{1.868114in}}%
\pgfpathlineto{\pgfqpoint{4.236455in}{1.853159in}}%
\pgfpathlineto{\pgfqpoint{4.234376in}{1.820114in}}%
\pgfpathlineto{\pgfqpoint{4.233337in}{1.812509in}}%
\pgfpathlineto{\pgfqpoint{4.230739in}{1.854164in}}%
\pgfpathlineto{\pgfqpoint{4.229181in}{1.978654in}}%
\pgfpathlineto{\pgfqpoint{4.227622in}{1.898274in}}%
\pgfpathlineto{\pgfqpoint{4.224505in}{2.039728in}}%
\pgfpathlineto{\pgfqpoint{4.223985in}{2.041053in}}%
\pgfpathlineto{\pgfqpoint{4.222946in}{2.016026in}}%
\pgfpathlineto{\pgfqpoint{4.221387in}{2.077133in}}%
\pgfpathlineto{\pgfqpoint{4.220868in}{2.069684in}}%
\pgfpathlineto{\pgfqpoint{4.218789in}{2.000046in}}%
\pgfpathlineto{\pgfqpoint{4.217231in}{2.049482in}}%
\pgfpathlineto{\pgfqpoint{4.216191in}{2.071416in}}%
\pgfpathlineto{\pgfqpoint{4.215672in}{2.071277in}}%
\pgfpathlineto{\pgfqpoint{4.214633in}{2.099115in}}%
\pgfpathlineto{\pgfqpoint{4.213594in}{2.117303in}}%
\pgfpathlineto{\pgfqpoint{4.212035in}{2.073225in}}%
\pgfpathlineto{\pgfqpoint{4.208398in}{2.194193in}}%
\pgfpathlineto{\pgfqpoint{4.206320in}{2.374437in}}%
\pgfpathlineto{\pgfqpoint{4.205800in}{2.369213in}}%
\pgfpathlineto{\pgfqpoint{4.203722in}{2.330600in}}%
\pgfpathlineto{\pgfqpoint{4.202683in}{2.391894in}}%
\pgfpathlineto{\pgfqpoint{4.201644in}{2.467213in}}%
\pgfpathlineto{\pgfqpoint{4.201124in}{2.440940in}}%
\pgfpathlineto{\pgfqpoint{4.199046in}{2.330611in}}%
\pgfpathlineto{\pgfqpoint{4.198526in}{2.329825in}}%
\pgfpathlineto{\pgfqpoint{4.196968in}{2.405148in}}%
\pgfpathlineto{\pgfqpoint{4.195409in}{2.434648in}}%
\pgfpathlineto{\pgfqpoint{4.194370in}{2.429533in}}%
\pgfpathlineto{\pgfqpoint{4.190733in}{2.464248in}}%
\pgfpathlineto{\pgfqpoint{4.189174in}{2.509633in}}%
\pgfpathlineto{\pgfqpoint{4.187615in}{2.456726in}}%
\pgfpathlineto{\pgfqpoint{4.187096in}{2.471128in}}%
\pgfpathlineto{\pgfqpoint{4.184498in}{2.604940in}}%
\pgfpathlineto{\pgfqpoint{4.183978in}{2.602384in}}%
\pgfpathlineto{\pgfqpoint{4.182420in}{2.553166in}}%
\pgfpathlineto{\pgfqpoint{4.181900in}{2.558532in}}%
\pgfpathlineto{\pgfqpoint{4.179822in}{2.654264in}}%
\pgfpathlineto{\pgfqpoint{4.179302in}{2.648462in}}%
\pgfpathlineto{\pgfqpoint{4.177744in}{2.558863in}}%
\pgfpathlineto{\pgfqpoint{4.177224in}{2.581341in}}%
\pgfpathlineto{\pgfqpoint{4.175146in}{2.744087in}}%
\pgfpathlineto{\pgfqpoint{4.174107in}{2.695018in}}%
\pgfpathlineto{\pgfqpoint{4.173067in}{2.627956in}}%
\pgfpathlineto{\pgfqpoint{4.172548in}{2.628218in}}%
\pgfpathlineto{\pgfqpoint{4.170470in}{2.737453in}}%
\pgfpathlineto{\pgfqpoint{4.169950in}{2.717396in}}%
\pgfpathlineto{\pgfqpoint{4.168391in}{2.650848in}}%
\pgfpathlineto{\pgfqpoint{4.166833in}{2.762872in}}%
\pgfpathlineto{\pgfqpoint{4.166313in}{2.760558in}}%
\pgfpathlineto{\pgfqpoint{4.165274in}{2.732299in}}%
\pgfpathlineto{\pgfqpoint{4.164754in}{2.740129in}}%
\pgfpathlineto{\pgfqpoint{4.164235in}{2.750879in}}%
\pgfpathlineto{\pgfqpoint{4.163715in}{2.750593in}}%
\pgfpathlineto{\pgfqpoint{4.163196in}{2.743148in}}%
\pgfpathlineto{\pgfqpoint{4.162676in}{2.743220in}}%
\pgfpathlineto{\pgfqpoint{4.159559in}{2.838043in}}%
\pgfpathlineto{\pgfqpoint{4.158520in}{2.887771in}}%
\pgfpathlineto{\pgfqpoint{4.158000in}{2.876260in}}%
\pgfpathlineto{\pgfqpoint{4.156441in}{2.806483in}}%
\pgfpathlineto{\pgfqpoint{4.155402in}{2.848433in}}%
\pgfpathlineto{\pgfqpoint{4.153843in}{2.905586in}}%
\pgfpathlineto{\pgfqpoint{4.153324in}{2.901212in}}%
\pgfpathlineto{\pgfqpoint{4.152285in}{2.862715in}}%
\pgfpathlineto{\pgfqpoint{4.150726in}{2.777961in}}%
\pgfpathlineto{\pgfqpoint{4.150207in}{2.792322in}}%
\pgfpathlineto{\pgfqpoint{4.147609in}{2.940072in}}%
\pgfpathlineto{\pgfqpoint{4.146570in}{2.916098in}}%
\pgfpathlineto{\pgfqpoint{4.145530in}{2.895454in}}%
\pgfpathlineto{\pgfqpoint{4.140854in}{3.028774in}}%
\pgfpathlineto{\pgfqpoint{4.138776in}{2.910671in}}%
\pgfpathlineto{\pgfqpoint{4.138256in}{2.931883in}}%
\pgfpathlineto{\pgfqpoint{4.137217in}{2.956522in}}%
\pgfpathlineto{\pgfqpoint{4.135659in}{2.902636in}}%
\pgfpathlineto{\pgfqpoint{4.135139in}{2.916233in}}%
\pgfpathlineto{\pgfqpoint{4.132541in}{3.007077in}}%
\pgfpathlineto{\pgfqpoint{4.130983in}{3.061277in}}%
\pgfpathlineto{\pgfqpoint{4.130463in}{3.054068in}}%
\pgfpathlineto{\pgfqpoint{4.127346in}{2.889660in}}%
\pgfpathlineto{\pgfqpoint{4.126306in}{2.924359in}}%
\pgfpathlineto{\pgfqpoint{4.124228in}{3.056216in}}%
\pgfpathlineto{\pgfqpoint{4.123189in}{3.022190in}}%
\pgfpathlineto{\pgfqpoint{4.122150in}{2.973377in}}%
\pgfpathlineto{\pgfqpoint{4.121630in}{2.980799in}}%
\pgfpathlineto{\pgfqpoint{4.120072in}{3.029149in}}%
\pgfpathlineto{\pgfqpoint{4.117993in}{2.954603in}}%
\pgfpathlineto{\pgfqpoint{4.117474in}{2.968362in}}%
\pgfpathlineto{\pgfqpoint{4.115915in}{3.109040in}}%
\pgfpathlineto{\pgfqpoint{4.115396in}{3.085219in}}%
\pgfpathlineto{\pgfqpoint{4.114356in}{3.027218in}}%
\pgfpathlineto{\pgfqpoint{4.113837in}{3.033636in}}%
\pgfpathlineto{\pgfqpoint{4.113317in}{3.033961in}}%
\pgfpathlineto{\pgfqpoint{4.111239in}{2.942268in}}%
\pgfpathlineto{\pgfqpoint{4.110719in}{2.958941in}}%
\pgfpathlineto{\pgfqpoint{4.109161in}{3.066886in}}%
\pgfpathlineto{\pgfqpoint{4.108641in}{3.065517in}}%
\pgfpathlineto{\pgfqpoint{4.107083in}{3.027907in}}%
\pgfpathlineto{\pgfqpoint{4.105524in}{3.053510in}}%
\pgfpathlineto{\pgfqpoint{4.103965in}{3.092674in}}%
\pgfpathlineto{\pgfqpoint{4.103446in}{3.084295in}}%
\pgfpathlineto{\pgfqpoint{4.099809in}{2.956326in}}%
\pgfpathlineto{\pgfqpoint{4.099289in}{2.950638in}}%
\pgfpathlineto{\pgfqpoint{4.098769in}{2.957098in}}%
\pgfpathlineto{\pgfqpoint{4.096691in}{2.988475in}}%
\pgfpathlineto{\pgfqpoint{4.095652in}{3.017833in}}%
\pgfpathlineto{\pgfqpoint{4.093574in}{2.892711in}}%
\pgfpathlineto{\pgfqpoint{4.093054in}{2.906039in}}%
\pgfpathlineto{\pgfqpoint{4.092535in}{2.915859in}}%
\pgfpathlineto{\pgfqpoint{4.092015in}{2.914112in}}%
\pgfpathlineto{\pgfqpoint{4.091496in}{2.911335in}}%
\pgfpathlineto{\pgfqpoint{4.090456in}{2.922386in}}%
\pgfpathlineto{\pgfqpoint{4.089937in}{2.920274in}}%
\pgfpathlineto{\pgfqpoint{4.089417in}{2.917817in}}%
\pgfpathlineto{\pgfqpoint{4.087339in}{2.978642in}}%
\pgfpathlineto{\pgfqpoint{4.086819in}{2.963970in}}%
\pgfpathlineto{\pgfqpoint{4.084741in}{2.883526in}}%
\pgfpathlineto{\pgfqpoint{4.083182in}{2.956477in}}%
\pgfpathlineto{\pgfqpoint{4.080585in}{2.873978in}}%
\pgfpathlineto{\pgfqpoint{4.079545in}{2.856701in}}%
\pgfpathlineto{\pgfqpoint{4.078506in}{2.873383in}}%
\pgfpathlineto{\pgfqpoint{4.076428in}{2.767060in}}%
\pgfpathlineto{\pgfqpoint{4.075909in}{2.788837in}}%
\pgfpathlineto{\pgfqpoint{4.073830in}{2.854244in}}%
\pgfpathlineto{\pgfqpoint{4.073311in}{2.861584in}}%
\pgfpathlineto{\pgfqpoint{4.072791in}{2.856158in}}%
\pgfpathlineto{\pgfqpoint{4.069674in}{2.732875in}}%
\pgfpathlineto{\pgfqpoint{4.069154in}{2.735786in}}%
\pgfpathlineto{\pgfqpoint{4.066556in}{2.844742in}}%
\pgfpathlineto{\pgfqpoint{4.066037in}{2.840666in}}%
\pgfpathlineto{\pgfqpoint{4.064998in}{2.792006in}}%
\pgfpathlineto{\pgfqpoint{4.063439in}{2.705830in}}%
\pgfpathlineto{\pgfqpoint{4.062919in}{2.713623in}}%
\pgfpathlineto{\pgfqpoint{4.060841in}{2.786967in}}%
\pgfpathlineto{\pgfqpoint{4.060322in}{2.775421in}}%
\pgfpathlineto{\pgfqpoint{4.058243in}{2.645048in}}%
\pgfpathlineto{\pgfqpoint{4.057724in}{2.661204in}}%
\pgfpathlineto{\pgfqpoint{4.057204in}{2.674121in}}%
\pgfpathlineto{\pgfqpoint{4.056685in}{2.671529in}}%
\pgfpathlineto{\pgfqpoint{4.056165in}{2.663390in}}%
\pgfpathlineto{\pgfqpoint{4.055645in}{2.664372in}}%
\pgfpathlineto{\pgfqpoint{4.054606in}{2.675153in}}%
\pgfpathlineto{\pgfqpoint{4.053567in}{2.660835in}}%
\pgfpathlineto{\pgfqpoint{4.052008in}{2.695094in}}%
\pgfpathlineto{\pgfqpoint{4.051489in}{2.685690in}}%
\pgfpathlineto{\pgfqpoint{4.048891in}{2.544056in}}%
\pgfpathlineto{\pgfqpoint{4.047332in}{2.478935in}}%
\pgfpathlineto{\pgfqpoint{4.046813in}{2.484335in}}%
\pgfpathlineto{\pgfqpoint{4.044215in}{2.619993in}}%
\pgfpathlineto{\pgfqpoint{4.043695in}{2.607972in}}%
\pgfpathlineto{\pgfqpoint{4.042137in}{2.597670in}}%
\pgfpathlineto{\pgfqpoint{4.040058in}{2.410228in}}%
\pgfpathlineto{\pgfqpoint{4.039539in}{2.428508in}}%
\pgfpathlineto{\pgfqpoint{4.037980in}{2.521779in}}%
\pgfpathlineto{\pgfqpoint{4.037461in}{2.516919in}}%
\pgfpathlineto{\pgfqpoint{4.033824in}{2.322272in}}%
\pgfpathlineto{\pgfqpoint{4.033304in}{2.325863in}}%
\pgfpathlineto{\pgfqpoint{4.031745in}{2.373617in}}%
\pgfpathlineto{\pgfqpoint{4.030706in}{2.329688in}}%
\pgfpathlineto{\pgfqpoint{4.029148in}{2.239325in}}%
\pgfpathlineto{\pgfqpoint{4.028628in}{2.239577in}}%
\pgfpathlineto{\pgfqpoint{4.027069in}{2.297800in}}%
\pgfpathlineto{\pgfqpoint{4.026550in}{2.294954in}}%
\pgfpathlineto{\pgfqpoint{4.023952in}{2.226809in}}%
\pgfpathlineto{\pgfqpoint{4.022393in}{2.148814in}}%
\pgfpathlineto{\pgfqpoint{4.021874in}{2.155265in}}%
\pgfpathlineto{\pgfqpoint{4.020834in}{2.183097in}}%
\pgfpathlineto{\pgfqpoint{4.019276in}{2.123939in}}%
\pgfpathlineto{\pgfqpoint{4.018756in}{2.130692in}}%
\pgfpathlineto{\pgfqpoint{4.017717in}{2.149447in}}%
\pgfpathlineto{\pgfqpoint{4.016158in}{2.051719in}}%
\pgfpathlineto{\pgfqpoint{4.015639in}{2.071736in}}%
\pgfpathlineto{\pgfqpoint{4.014600in}{2.153795in}}%
\pgfpathlineto{\pgfqpoint{4.014080in}{2.150869in}}%
\pgfpathlineto{\pgfqpoint{4.011482in}{2.026930in}}%
\pgfpathlineto{\pgfqpoint{4.010963in}{2.028580in}}%
\pgfpathlineto{\pgfqpoint{4.009404in}{2.059237in}}%
\pgfpathlineto{\pgfqpoint{4.008365in}{2.049403in}}%
\pgfpathlineto{\pgfqpoint{4.006806in}{2.012251in}}%
\pgfpathlineto{\pgfqpoint{4.005247in}{2.044576in}}%
\pgfpathlineto{\pgfqpoint{4.004728in}{2.037578in}}%
\pgfpathlineto{\pgfqpoint{4.001091in}{1.988083in}}%
\pgfpathlineto{\pgfqpoint{3.999532in}{1.896611in}}%
\pgfpathlineto{\pgfqpoint{3.999013in}{1.901363in}}%
\pgfpathlineto{\pgfqpoint{3.997454in}{1.965197in}}%
\pgfpathlineto{\pgfqpoint{3.996934in}{1.949302in}}%
\pgfpathlineto{\pgfqpoint{3.994856in}{1.808525in}}%
\pgfpathlineto{\pgfqpoint{3.994337in}{1.815160in}}%
\pgfpathlineto{\pgfqpoint{3.993817in}{1.824951in}}%
\pgfpathlineto{\pgfqpoint{3.993297in}{1.823842in}}%
\pgfpathlineto{\pgfqpoint{3.991739in}{1.766306in}}%
\pgfpathlineto{\pgfqpoint{3.991219in}{1.769646in}}%
\pgfpathlineto{\pgfqpoint{3.990700in}{1.777629in}}%
\pgfpathlineto{\pgfqpoint{3.990180in}{1.775530in}}%
\pgfpathlineto{\pgfqpoint{3.984984in}{1.655299in}}%
\pgfpathlineto{\pgfqpoint{3.983426in}{1.675165in}}%
\pgfpathlineto{\pgfqpoint{3.981867in}{1.754745in}}%
\pgfpathlineto{\pgfqpoint{3.981347in}{1.743755in}}%
\pgfpathlineto{\pgfqpoint{3.978750in}{1.619407in}}%
\pgfpathlineto{\pgfqpoint{3.977710in}{1.649344in}}%
\pgfpathlineto{\pgfqpoint{3.976152in}{1.788450in}}%
\pgfpathlineto{\pgfqpoint{3.975632in}{1.777593in}}%
\pgfpathlineto{\pgfqpoint{3.973034in}{1.269829in}}%
\pgfpathlineto{\pgfqpoint{3.972515in}{1.328780in}}%
\pgfpathlineto{\pgfqpoint{3.970437in}{1.586285in}}%
\pgfpathlineto{\pgfqpoint{3.967839in}{1.439552in}}%
\pgfpathlineto{\pgfqpoint{3.965760in}{1.039463in}}%
\pgfpathlineto{\pgfqpoint{3.963682in}{1.536002in}}%
\pgfpathlineto{\pgfqpoint{3.962643in}{1.493653in}}%
\pgfpathlineto{\pgfqpoint{3.960565in}{1.366883in}}%
\pgfpathlineto{\pgfqpoint{3.959006in}{1.459701in}}%
\pgfpathlineto{\pgfqpoint{3.958486in}{1.459156in}}%
\pgfpathlineto{\pgfqpoint{3.955369in}{1.257785in}}%
\pgfpathlineto{\pgfqpoint{3.954850in}{1.262563in}}%
\pgfpathlineto{\pgfqpoint{3.953291in}{1.306345in}}%
\pgfpathlineto{\pgfqpoint{3.952771in}{1.301784in}}%
\pgfpathlineto{\pgfqpoint{3.950693in}{1.232601in}}%
\pgfpathlineto{\pgfqpoint{3.950173in}{1.241729in}}%
\pgfpathlineto{\pgfqpoint{3.949134in}{1.269142in}}%
\pgfpathlineto{\pgfqpoint{3.948615in}{1.265430in}}%
\pgfpathlineto{\pgfqpoint{3.947056in}{1.219895in}}%
\pgfpathlineto{\pgfqpoint{3.946536in}{1.226510in}}%
\pgfpathlineto{\pgfqpoint{3.945497in}{1.251430in}}%
\pgfpathlineto{\pgfqpoint{3.940821in}{1.106929in}}%
\pgfpathlineto{\pgfqpoint{3.939263in}{1.172415in}}%
\pgfpathlineto{\pgfqpoint{3.938743in}{1.163239in}}%
\pgfpathlineto{\pgfqpoint{3.937704in}{1.121170in}}%
\pgfpathlineto{\pgfqpoint{3.937184in}{1.133373in}}%
\pgfpathlineto{\pgfqpoint{3.936145in}{1.164056in}}%
\pgfpathlineto{\pgfqpoint{3.935626in}{1.155902in}}%
\pgfpathlineto{\pgfqpoint{3.933547in}{1.086591in}}%
\pgfpathlineto{\pgfqpoint{3.933028in}{1.090198in}}%
\pgfpathlineto{\pgfqpoint{3.932508in}{1.092002in}}%
\pgfpathlineto{\pgfqpoint{3.930430in}{0.994697in}}%
\pgfpathlineto{\pgfqpoint{3.929910in}{1.021290in}}%
\pgfpathlineto{\pgfqpoint{3.928871in}{1.076475in}}%
\pgfpathlineto{\pgfqpoint{3.928352in}{1.064288in}}%
\pgfpathlineto{\pgfqpoint{3.927312in}{1.020143in}}%
\pgfpathlineto{\pgfqpoint{3.926793in}{1.022145in}}%
\pgfpathlineto{\pgfqpoint{3.924195in}{1.084465in}}%
\pgfpathlineto{\pgfqpoint{3.923676in}{1.080200in}}%
\pgfpathlineto{\pgfqpoint{3.923156in}{1.071621in}}%
\pgfpathlineto{\pgfqpoint{3.922636in}{1.074931in}}%
\pgfpathlineto{\pgfqpoint{3.921078in}{1.110503in}}%
\pgfpathlineto{\pgfqpoint{3.920039in}{1.095495in}}%
\pgfpathlineto{\pgfqpoint{3.916402in}{1.005635in}}%
\pgfpathlineto{\pgfqpoint{3.915362in}{1.048282in}}%
\pgfpathlineto{\pgfqpoint{3.913284in}{1.173718in}}%
\pgfpathlineto{\pgfqpoint{3.912765in}{1.165546in}}%
\pgfpathlineto{\pgfqpoint{3.910167in}{1.083877in}}%
\pgfpathlineto{\pgfqpoint{3.908089in}{0.924109in}}%
\pgfpathlineto{\pgfqpoint{3.907049in}{0.943408in}}%
\pgfpathlineto{\pgfqpoint{3.906530in}{0.939656in}}%
\pgfpathlineto{\pgfqpoint{3.906010in}{0.930167in}}%
\pgfpathlineto{\pgfqpoint{3.905491in}{0.933544in}}%
\pgfpathlineto{\pgfqpoint{3.903412in}{1.027608in}}%
\pgfpathlineto{\pgfqpoint{3.902373in}{0.959313in}}%
\pgfpathlineto{\pgfqpoint{3.900815in}{0.841700in}}%
\pgfpathlineto{\pgfqpoint{3.895619in}{1.147345in}}%
\pgfpathlineto{\pgfqpoint{3.891982in}{1.070498in}}%
\pgfpathlineto{\pgfqpoint{3.891462in}{1.074693in}}%
\pgfpathlineto{\pgfqpoint{3.890423in}{1.105936in}}%
\pgfpathlineto{\pgfqpoint{3.889904in}{1.087911in}}%
\pgfpathlineto{\pgfqpoint{3.888345in}{0.921321in}}%
\pgfpathlineto{\pgfqpoint{3.887825in}{0.930097in}}%
\pgfpathlineto{\pgfqpoint{3.886267in}{0.994524in}}%
\pgfpathlineto{\pgfqpoint{3.884188in}{0.939182in}}%
\pgfpathlineto{\pgfqpoint{3.882110in}{1.068692in}}%
\pgfpathlineto{\pgfqpoint{3.881071in}{1.057604in}}%
\pgfpathlineto{\pgfqpoint{3.879512in}{1.046620in}}%
\pgfpathlineto{\pgfqpoint{3.878473in}{1.054058in}}%
\pgfpathlineto{\pgfqpoint{3.877954in}{1.044479in}}%
\pgfpathlineto{\pgfqpoint{3.876395in}{0.944942in}}%
\pgfpathlineto{\pgfqpoint{3.875875in}{0.961106in}}%
\pgfpathlineto{\pgfqpoint{3.873797in}{1.034120in}}%
\pgfpathlineto{\pgfqpoint{3.871719in}{1.084883in}}%
\pgfpathlineto{\pgfqpoint{3.871199in}{1.081693in}}%
\pgfpathlineto{\pgfqpoint{3.869641in}{1.037676in}}%
\pgfpathlineto{\pgfqpoint{3.869121in}{1.048944in}}%
\pgfpathlineto{\pgfqpoint{3.868082in}{1.110698in}}%
\pgfpathlineto{\pgfqpoint{3.867562in}{1.110120in}}%
\pgfpathlineto{\pgfqpoint{3.866004in}{1.008094in}}%
\pgfpathlineto{\pgfqpoint{3.865484in}{1.023152in}}%
\pgfpathlineto{\pgfqpoint{3.863925in}{1.165701in}}%
\pgfpathlineto{\pgfqpoint{3.863406in}{1.141138in}}%
\pgfpathlineto{\pgfqpoint{3.861847in}{1.006170in}}%
\pgfpathlineto{\pgfqpoint{3.860808in}{1.086163in}}%
\pgfpathlineto{\pgfqpoint{3.858210in}{1.263647in}}%
\pgfpathlineto{\pgfqpoint{3.857171in}{1.188976in}}%
\pgfpathlineto{\pgfqpoint{3.856132in}{1.125083in}}%
\pgfpathlineto{\pgfqpoint{3.854054in}{1.333488in}}%
\pgfpathlineto{\pgfqpoint{3.853014in}{1.306286in}}%
\pgfpathlineto{\pgfqpoint{3.849897in}{1.129487in}}%
\pgfpathlineto{\pgfqpoint{3.849378in}{1.135429in}}%
\pgfpathlineto{\pgfqpoint{3.848338in}{1.200138in}}%
\pgfpathlineto{\pgfqpoint{3.847299in}{1.287866in}}%
\pgfpathlineto{\pgfqpoint{3.846780in}{1.283422in}}%
\pgfpathlineto{\pgfqpoint{3.845221in}{1.204147in}}%
\pgfpathlineto{\pgfqpoint{3.844701in}{1.211557in}}%
\pgfpathlineto{\pgfqpoint{3.841064in}{1.286293in}}%
\pgfpathlineto{\pgfqpoint{3.840545in}{1.283689in}}%
\pgfpathlineto{\pgfqpoint{3.837427in}{1.353524in}}%
\pgfpathlineto{\pgfqpoint{3.836388in}{1.415443in}}%
\pgfpathlineto{\pgfqpoint{3.835869in}{1.395586in}}%
\pgfpathlineto{\pgfqpoint{3.834310in}{1.309568in}}%
\pgfpathlineto{\pgfqpoint{3.833271in}{1.381346in}}%
\pgfpathlineto{\pgfqpoint{3.831712in}{1.490391in}}%
\pgfpathlineto{\pgfqpoint{3.830154in}{1.452055in}}%
\pgfpathlineto{\pgfqpoint{3.829634in}{1.456062in}}%
\pgfpathlineto{\pgfqpoint{3.829114in}{1.457331in}}%
\pgfpathlineto{\pgfqpoint{3.828075in}{1.428663in}}%
\pgfpathlineto{\pgfqpoint{3.827556in}{1.431338in}}%
\pgfpathlineto{\pgfqpoint{3.825997in}{1.500421in}}%
\pgfpathlineto{\pgfqpoint{3.823399in}{1.446108in}}%
\pgfpathlineto{\pgfqpoint{3.822360in}{1.449976in}}%
\pgfpathlineto{\pgfqpoint{3.821321in}{1.435491in}}%
\pgfpathlineto{\pgfqpoint{3.820801in}{1.444368in}}%
\pgfpathlineto{\pgfqpoint{3.818204in}{1.551317in}}%
\pgfpathlineto{\pgfqpoint{3.817684in}{1.554404in}}%
\pgfpathlineto{\pgfqpoint{3.817164in}{1.549076in}}%
\pgfpathlineto{\pgfqpoint{3.816645in}{1.551550in}}%
\pgfpathlineto{\pgfqpoint{3.815086in}{1.646846in}}%
\pgfpathlineto{\pgfqpoint{3.814567in}{1.641114in}}%
\pgfpathlineto{\pgfqpoint{3.813527in}{1.600264in}}%
\pgfpathlineto{\pgfqpoint{3.813008in}{1.613189in}}%
\pgfpathlineto{\pgfqpoint{3.811449in}{1.716667in}}%
\pgfpathlineto{\pgfqpoint{3.810930in}{1.702710in}}%
\pgfpathlineto{\pgfqpoint{3.809890in}{1.649804in}}%
\pgfpathlineto{\pgfqpoint{3.809371in}{1.665954in}}%
\pgfpathlineto{\pgfqpoint{3.807812in}{1.744931in}}%
\pgfpathlineto{\pgfqpoint{3.806254in}{1.677550in}}%
\pgfpathlineto{\pgfqpoint{3.805734in}{1.681768in}}%
\pgfpathlineto{\pgfqpoint{3.804175in}{1.761005in}}%
\pgfpathlineto{\pgfqpoint{3.802617in}{1.808705in}}%
\pgfpathlineto{\pgfqpoint{3.801577in}{1.853889in}}%
\pgfpathlineto{\pgfqpoint{3.800538in}{1.911891in}}%
\pgfpathlineto{\pgfqpoint{3.800019in}{1.895979in}}%
\pgfpathlineto{\pgfqpoint{3.797940in}{1.779202in}}%
\pgfpathlineto{\pgfqpoint{3.795343in}{1.976584in}}%
\pgfpathlineto{\pgfqpoint{3.794823in}{1.961679in}}%
\pgfpathlineto{\pgfqpoint{3.793264in}{1.883623in}}%
\pgfpathlineto{\pgfqpoint{3.792745in}{1.886123in}}%
\pgfpathlineto{\pgfqpoint{3.790667in}{2.013297in}}%
\pgfpathlineto{\pgfqpoint{3.790147in}{1.999771in}}%
\pgfpathlineto{\pgfqpoint{3.789627in}{1.985264in}}%
\pgfpathlineto{\pgfqpoint{3.789108in}{1.993965in}}%
\pgfpathlineto{\pgfqpoint{3.785990in}{2.141815in}}%
\pgfpathlineto{\pgfqpoint{3.785471in}{2.134843in}}%
\pgfpathlineto{\pgfqpoint{3.783393in}{2.043239in}}%
\pgfpathlineto{\pgfqpoint{3.780275in}{2.170248in}}%
\pgfpathlineto{\pgfqpoint{3.779756in}{2.175356in}}%
\pgfpathlineto{\pgfqpoint{3.779236in}{2.170150in}}%
\pgfpathlineto{\pgfqpoint{3.777677in}{2.130558in}}%
\pgfpathlineto{\pgfqpoint{3.773001in}{2.327634in}}%
\pgfpathlineto{\pgfqpoint{3.772482in}{2.318668in}}%
\pgfpathlineto{\pgfqpoint{3.771962in}{2.316294in}}%
\pgfpathlineto{\pgfqpoint{3.770923in}{2.337763in}}%
\pgfpathlineto{\pgfqpoint{3.770403in}{2.336983in}}%
\pgfpathlineto{\pgfqpoint{3.769364in}{2.320407in}}%
\pgfpathlineto{\pgfqpoint{3.768845in}{2.325216in}}%
\pgfpathlineto{\pgfqpoint{3.767286in}{2.382896in}}%
\pgfpathlineto{\pgfqpoint{3.766766in}{2.382721in}}%
\pgfpathlineto{\pgfqpoint{3.765208in}{2.336153in}}%
\pgfpathlineto{\pgfqpoint{3.761051in}{2.476074in}}%
\pgfpathlineto{\pgfqpoint{3.760532in}{2.463777in}}%
\pgfpathlineto{\pgfqpoint{3.758973in}{2.423528in}}%
\pgfpathlineto{\pgfqpoint{3.757414in}{2.476268in}}%
\pgfpathlineto{\pgfqpoint{3.755336in}{2.548586in}}%
\pgfpathlineto{\pgfqpoint{3.754816in}{2.546507in}}%
\pgfpathlineto{\pgfqpoint{3.752738in}{2.513769in}}%
\pgfpathlineto{\pgfqpoint{3.752219in}{2.524051in}}%
\pgfpathlineto{\pgfqpoint{3.748582in}{2.689629in}}%
\pgfpathlineto{\pgfqpoint{3.748062in}{2.693570in}}%
\pgfpathlineto{\pgfqpoint{3.747023in}{2.662108in}}%
\pgfpathlineto{\pgfqpoint{3.746503in}{2.666812in}}%
\pgfpathlineto{\pgfqpoint{3.744425in}{2.758061in}}%
\pgfpathlineto{\pgfqpoint{3.743906in}{2.755698in}}%
\pgfpathlineto{\pgfqpoint{3.741827in}{2.664704in}}%
\pgfpathlineto{\pgfqpoint{3.741308in}{2.685332in}}%
\pgfpathlineto{\pgfqpoint{3.739749in}{2.801012in}}%
\pgfpathlineto{\pgfqpoint{3.739229in}{2.781753in}}%
\pgfpathlineto{\pgfqpoint{3.738190in}{2.747029in}}%
\pgfpathlineto{\pgfqpoint{3.737671in}{2.753951in}}%
\pgfpathlineto{\pgfqpoint{3.737151in}{2.757615in}}%
\pgfpathlineto{\pgfqpoint{3.735592in}{2.737344in}}%
\pgfpathlineto{\pgfqpoint{3.733514in}{2.764520in}}%
\pgfpathlineto{\pgfqpoint{3.731436in}{2.707205in}}%
\pgfpathlineto{\pgfqpoint{3.730916in}{2.714917in}}%
\pgfpathlineto{\pgfqpoint{3.729877in}{2.783156in}}%
\pgfpathlineto{\pgfqpoint{3.727799in}{2.975699in}}%
\pgfpathlineto{\pgfqpoint{3.727279in}{2.968329in}}%
\pgfpathlineto{\pgfqpoint{3.724162in}{2.891064in}}%
\pgfpathlineto{\pgfqpoint{3.722603in}{2.946137in}}%
\pgfpathlineto{\pgfqpoint{3.722084in}{2.939762in}}%
\pgfpathlineto{\pgfqpoint{3.720005in}{2.907502in}}%
\pgfpathlineto{\pgfqpoint{3.719486in}{2.913551in}}%
\pgfpathlineto{\pgfqpoint{3.717927in}{3.006567in}}%
\pgfpathlineto{\pgfqpoint{3.717408in}{3.000143in}}%
\pgfpathlineto{\pgfqpoint{3.716369in}{2.943364in}}%
\pgfpathlineto{\pgfqpoint{3.715849in}{2.946933in}}%
\pgfpathlineto{\pgfqpoint{3.712732in}{3.063614in}}%
\pgfpathlineto{\pgfqpoint{3.711692in}{3.009557in}}%
\pgfpathlineto{\pgfqpoint{3.710134in}{2.925256in}}%
\pgfpathlineto{\pgfqpoint{3.709614in}{2.926718in}}%
\pgfpathlineto{\pgfqpoint{3.708575in}{2.938308in}}%
\pgfpathlineto{\pgfqpoint{3.708055in}{2.930591in}}%
\pgfpathlineto{\pgfqpoint{3.707536in}{2.939938in}}%
\pgfpathlineto{\pgfqpoint{3.705977in}{3.149768in}}%
\pgfpathlineto{\pgfqpoint{3.705458in}{3.143659in}}%
\pgfpathlineto{\pgfqpoint{3.703379in}{3.024460in}}%
\pgfpathlineto{\pgfqpoint{3.702860in}{3.021484in}}%
\pgfpathlineto{\pgfqpoint{3.700782in}{2.985989in}}%
\pgfpathlineto{\pgfqpoint{3.697145in}{3.084904in}}%
\pgfpathlineto{\pgfqpoint{3.694547in}{2.967583in}}%
\pgfpathlineto{\pgfqpoint{3.694027in}{2.988618in}}%
\pgfpathlineto{\pgfqpoint{3.692468in}{3.053631in}}%
\pgfpathlineto{\pgfqpoint{3.691429in}{3.032589in}}%
\pgfpathlineto{\pgfqpoint{3.690910in}{3.038362in}}%
\pgfpathlineto{\pgfqpoint{3.688831in}{3.099510in}}%
\pgfpathlineto{\pgfqpoint{3.686234in}{3.076742in}}%
\pgfpathlineto{\pgfqpoint{3.685714in}{3.078190in}}%
\pgfpathlineto{\pgfqpoint{3.683636in}{3.102514in}}%
\pgfpathlineto{\pgfqpoint{3.682597in}{3.075443in}}%
\pgfpathlineto{\pgfqpoint{3.681558in}{3.038121in}}%
\pgfpathlineto{\pgfqpoint{3.681038in}{3.051846in}}%
\pgfpathlineto{\pgfqpoint{3.679999in}{3.092076in}}%
\pgfpathlineto{\pgfqpoint{3.679479in}{3.079423in}}%
\pgfpathlineto{\pgfqpoint{3.678440in}{3.049608in}}%
\pgfpathlineto{\pgfqpoint{3.677401in}{3.073062in}}%
\pgfpathlineto{\pgfqpoint{3.676881in}{3.071299in}}%
\pgfpathlineto{\pgfqpoint{3.675842in}{3.054963in}}%
\pgfpathlineto{\pgfqpoint{3.674284in}{3.099814in}}%
\pgfpathlineto{\pgfqpoint{3.672725in}{3.019947in}}%
\pgfpathlineto{\pgfqpoint{3.672205in}{3.039658in}}%
\pgfpathlineto{\pgfqpoint{3.671166in}{3.091965in}}%
\pgfpathlineto{\pgfqpoint{3.670647in}{3.083267in}}%
\pgfpathlineto{\pgfqpoint{3.669608in}{3.045621in}}%
\pgfpathlineto{\pgfqpoint{3.669088in}{3.058589in}}%
\pgfpathlineto{\pgfqpoint{3.667529in}{3.161329in}}%
\pgfpathlineto{\pgfqpoint{3.667010in}{3.158057in}}%
\pgfpathlineto{\pgfqpoint{3.665971in}{3.145084in}}%
\pgfpathlineto{\pgfqpoint{3.664931in}{3.166932in}}%
\pgfpathlineto{\pgfqpoint{3.664412in}{3.154295in}}%
\pgfpathlineto{\pgfqpoint{3.662853in}{3.018886in}}%
\pgfpathlineto{\pgfqpoint{3.662334in}{3.024229in}}%
\pgfpathlineto{\pgfqpoint{3.661294in}{3.104372in}}%
\pgfpathlineto{\pgfqpoint{3.660775in}{3.102337in}}%
\pgfpathlineto{\pgfqpoint{3.658697in}{2.942014in}}%
\pgfpathlineto{\pgfqpoint{3.656618in}{3.054035in}}%
\pgfpathlineto{\pgfqpoint{3.656099in}{3.041010in}}%
\pgfpathlineto{\pgfqpoint{3.653501in}{2.997047in}}%
\pgfpathlineto{\pgfqpoint{3.652981in}{2.994434in}}%
\pgfpathlineto{\pgfqpoint{3.651942in}{3.011879in}}%
\pgfpathlineto{\pgfqpoint{3.650384in}{3.069878in}}%
\pgfpathlineto{\pgfqpoint{3.649864in}{3.067162in}}%
\pgfpathlineto{\pgfqpoint{3.646747in}{2.994428in}}%
\pgfpathlineto{\pgfqpoint{3.646227in}{2.991236in}}%
\pgfpathlineto{\pgfqpoint{3.644149in}{3.079094in}}%
\pgfpathlineto{\pgfqpoint{3.643110in}{3.064694in}}%
\pgfpathlineto{\pgfqpoint{3.642590in}{3.059970in}}%
\pgfpathlineto{\pgfqpoint{3.642070in}{3.063334in}}%
\pgfpathlineto{\pgfqpoint{3.641551in}{3.063617in}}%
\pgfpathlineto{\pgfqpoint{3.640512in}{3.003577in}}%
\pgfpathlineto{\pgfqpoint{3.638953in}{2.918463in}}%
\pgfpathlineto{\pgfqpoint{3.637914in}{2.940130in}}%
\pgfpathlineto{\pgfqpoint{3.637394in}{2.935192in}}%
\pgfpathlineto{\pgfqpoint{3.636355in}{2.917925in}}%
\pgfpathlineto{\pgfqpoint{3.634797in}{2.951102in}}%
\pgfpathlineto{\pgfqpoint{3.634277in}{2.939435in}}%
\pgfpathlineto{\pgfqpoint{3.632718in}{2.872074in}}%
\pgfpathlineto{\pgfqpoint{3.630640in}{2.964751in}}%
\pgfpathlineto{\pgfqpoint{3.629601in}{2.909026in}}%
\pgfpathlineto{\pgfqpoint{3.627003in}{2.737394in}}%
\pgfpathlineto{\pgfqpoint{3.626483in}{2.730427in}}%
\pgfpathlineto{\pgfqpoint{3.625964in}{2.734707in}}%
\pgfpathlineto{\pgfqpoint{3.621807in}{2.875417in}}%
\pgfpathlineto{\pgfqpoint{3.620768in}{2.869095in}}%
\pgfpathlineto{\pgfqpoint{3.620249in}{2.870481in}}%
\pgfpathlineto{\pgfqpoint{3.619729in}{2.864638in}}%
\pgfpathlineto{\pgfqpoint{3.618170in}{2.766556in}}%
\pgfpathlineto{\pgfqpoint{3.615053in}{2.604864in}}%
\pgfpathlineto{\pgfqpoint{3.614014in}{2.632969in}}%
\pgfpathlineto{\pgfqpoint{3.612455in}{2.692300in}}%
\pgfpathlineto{\pgfqpoint{3.611416in}{2.668994in}}%
\pgfpathlineto{\pgfqpoint{3.610897in}{2.669070in}}%
\pgfpathlineto{\pgfqpoint{3.610377in}{2.674636in}}%
\pgfpathlineto{\pgfqpoint{3.609857in}{2.672089in}}%
\pgfpathlineto{\pgfqpoint{3.608818in}{2.661389in}}%
\pgfpathlineto{\pgfqpoint{3.608299in}{2.665611in}}%
\pgfpathlineto{\pgfqpoint{3.607779in}{2.662513in}}%
\pgfpathlineto{\pgfqpoint{3.606220in}{2.601174in}}%
\pgfpathlineto{\pgfqpoint{3.604142in}{2.435010in}}%
\pgfpathlineto{\pgfqpoint{3.603623in}{2.484642in}}%
\pgfpathlineto{\pgfqpoint{3.602583in}{2.598307in}}%
\pgfpathlineto{\pgfqpoint{3.602064in}{2.578624in}}%
\pgfpathlineto{\pgfqpoint{3.600505in}{2.454662in}}%
\pgfpathlineto{\pgfqpoint{3.599986in}{2.455680in}}%
\pgfpathlineto{\pgfqpoint{3.598946in}{2.430275in}}%
\pgfpathlineto{\pgfqpoint{3.598427in}{2.436441in}}%
\pgfpathlineto{\pgfqpoint{3.596868in}{2.538180in}}%
\pgfpathlineto{\pgfqpoint{3.594270in}{2.290876in}}%
\pgfpathlineto{\pgfqpoint{3.593751in}{2.310037in}}%
\pgfpathlineto{\pgfqpoint{3.590633in}{2.451263in}}%
\pgfpathlineto{\pgfqpoint{3.590114in}{2.449564in}}%
\pgfpathlineto{\pgfqpoint{3.588036in}{2.362756in}}%
\pgfpathlineto{\pgfqpoint{3.586996in}{2.379207in}}%
\pgfpathlineto{\pgfqpoint{3.586477in}{2.373687in}}%
\pgfpathlineto{\pgfqpoint{3.584918in}{2.242315in}}%
\pgfpathlineto{\pgfqpoint{3.583879in}{2.197088in}}%
\pgfpathlineto{\pgfqpoint{3.583359in}{2.207254in}}%
\pgfpathlineto{\pgfqpoint{3.580242in}{2.261570in}}%
\pgfpathlineto{\pgfqpoint{3.578683in}{2.213160in}}%
\pgfpathlineto{\pgfqpoint{3.577644in}{2.226635in}}%
\pgfpathlineto{\pgfqpoint{3.577125in}{2.227461in}}%
\pgfpathlineto{\pgfqpoint{3.574527in}{2.117894in}}%
\pgfpathlineto{\pgfqpoint{3.574007in}{2.119991in}}%
\pgfpathlineto{\pgfqpoint{3.573488in}{2.117714in}}%
\pgfpathlineto{\pgfqpoint{3.572449in}{2.093251in}}%
\pgfpathlineto{\pgfqpoint{3.569851in}{1.980886in}}%
\pgfpathlineto{\pgfqpoint{3.569331in}{1.988217in}}%
\pgfpathlineto{\pgfqpoint{3.567253in}{2.039026in}}%
\pgfpathlineto{\pgfqpoint{3.566733in}{2.035200in}}%
\pgfpathlineto{\pgfqpoint{3.564655in}{1.915390in}}%
\pgfpathlineto{\pgfqpoint{3.564136in}{1.916442in}}%
\pgfpathlineto{\pgfqpoint{3.563096in}{1.916208in}}%
\pgfpathlineto{\pgfqpoint{3.562057in}{1.950161in}}%
\pgfpathlineto{\pgfqpoint{3.561018in}{1.985193in}}%
\pgfpathlineto{\pgfqpoint{3.560499in}{1.974222in}}%
\pgfpathlineto{\pgfqpoint{3.559459in}{1.939039in}}%
\pgfpathlineto{\pgfqpoint{3.558940in}{1.941316in}}%
\pgfpathlineto{\pgfqpoint{3.558420in}{1.946566in}}%
\pgfpathlineto{\pgfqpoint{3.557381in}{1.893431in}}%
\pgfpathlineto{\pgfqpoint{3.554783in}{1.746059in}}%
\pgfpathlineto{\pgfqpoint{3.553225in}{1.808766in}}%
\pgfpathlineto{\pgfqpoint{3.552705in}{1.798389in}}%
\pgfpathlineto{\pgfqpoint{3.549588in}{1.642925in}}%
\pgfpathlineto{\pgfqpoint{3.549068in}{1.649133in}}%
\pgfpathlineto{\pgfqpoint{3.546470in}{1.812666in}}%
\pgfpathlineto{\pgfqpoint{3.545951in}{1.787798in}}%
\pgfpathlineto{\pgfqpoint{3.543872in}{1.653742in}}%
\pgfpathlineto{\pgfqpoint{3.543353in}{1.650348in}}%
\pgfpathlineto{\pgfqpoint{3.542314in}{1.622556in}}%
\pgfpathlineto{\pgfqpoint{3.541794in}{1.625215in}}%
\pgfpathlineto{\pgfqpoint{3.540755in}{1.651011in}}%
\pgfpathlineto{\pgfqpoint{3.540235in}{1.642369in}}%
\pgfpathlineto{\pgfqpoint{3.536598in}{1.486870in}}%
\pgfpathlineto{\pgfqpoint{3.536079in}{1.479108in}}%
\pgfpathlineto{\pgfqpoint{3.535559in}{1.486233in}}%
\pgfpathlineto{\pgfqpoint{3.532442in}{1.576625in}}%
\pgfpathlineto{\pgfqpoint{3.531403in}{1.547765in}}%
\pgfpathlineto{\pgfqpoint{3.527766in}{1.341518in}}%
\pgfpathlineto{\pgfqpoint{3.527246in}{1.358746in}}%
\pgfpathlineto{\pgfqpoint{3.526727in}{1.374054in}}%
\pgfpathlineto{\pgfqpoint{3.526207in}{1.372947in}}%
\pgfpathlineto{\pgfqpoint{3.525168in}{1.349298in}}%
\pgfpathlineto{\pgfqpoint{3.524648in}{1.355253in}}%
\pgfpathlineto{\pgfqpoint{3.523609in}{1.384061in}}%
\pgfpathlineto{\pgfqpoint{3.523090in}{1.373668in}}%
\pgfpathlineto{\pgfqpoint{3.522051in}{1.329150in}}%
\pgfpathlineto{\pgfqpoint{3.521531in}{1.335928in}}%
\pgfpathlineto{\pgfqpoint{3.519453in}{1.444309in}}%
\pgfpathlineto{\pgfqpoint{3.515816in}{1.304714in}}%
\pgfpathlineto{\pgfqpoint{3.515296in}{1.298745in}}%
\pgfpathlineto{\pgfqpoint{3.513738in}{1.333889in}}%
\pgfpathlineto{\pgfqpoint{3.513218in}{1.321954in}}%
\pgfpathlineto{\pgfqpoint{3.510620in}{1.183479in}}%
\pgfpathlineto{\pgfqpoint{3.510101in}{1.194011in}}%
\pgfpathlineto{\pgfqpoint{3.509061in}{1.215777in}}%
\pgfpathlineto{\pgfqpoint{3.508542in}{1.207159in}}%
\pgfpathlineto{\pgfqpoint{3.504385in}{1.048273in}}%
\pgfpathlineto{\pgfqpoint{3.503346in}{1.088497in}}%
\pgfpathlineto{\pgfqpoint{3.500748in}{1.227253in}}%
\pgfpathlineto{\pgfqpoint{3.500229in}{1.227187in}}%
\pgfpathlineto{\pgfqpoint{3.496072in}{1.085047in}}%
\pgfpathlineto{\pgfqpoint{3.495553in}{1.095294in}}%
\pgfpathlineto{\pgfqpoint{3.495033in}{1.114750in}}%
\pgfpathlineto{\pgfqpoint{3.494514in}{1.111766in}}%
\pgfpathlineto{\pgfqpoint{3.492955in}{1.042340in}}%
\pgfpathlineto{\pgfqpoint{3.492435in}{1.051513in}}%
\pgfpathlineto{\pgfqpoint{3.491396in}{1.078818in}}%
\pgfpathlineto{\pgfqpoint{3.490877in}{1.073282in}}%
\pgfpathlineto{\pgfqpoint{3.489838in}{1.053580in}}%
\pgfpathlineto{\pgfqpoint{3.488279in}{1.182693in}}%
\pgfpathlineto{\pgfqpoint{3.487759in}{1.180080in}}%
\pgfpathlineto{\pgfqpoint{3.484642in}{0.995205in}}%
\pgfpathlineto{\pgfqpoint{3.482564in}{0.951008in}}%
\pgfpathlineto{\pgfqpoint{3.481524in}{0.951659in}}%
\pgfpathlineto{\pgfqpoint{3.479966in}{0.925905in}}%
\pgfpathlineto{\pgfqpoint{3.475809in}{1.025371in}}%
\pgfpathlineto{\pgfqpoint{3.474770in}{1.024029in}}%
\pgfpathlineto{\pgfqpoint{3.474251in}{1.026664in}}%
\pgfpathlineto{\pgfqpoint{3.473731in}{1.019270in}}%
\pgfpathlineto{\pgfqpoint{3.471653in}{0.936895in}}%
\pgfpathlineto{\pgfqpoint{3.471133in}{0.947408in}}%
\pgfpathlineto{\pgfqpoint{3.468535in}{1.017307in}}%
\pgfpathlineto{\pgfqpoint{3.466457in}{0.991781in}}%
\pgfpathlineto{\pgfqpoint{3.465937in}{0.993554in}}%
\pgfpathlineto{\pgfqpoint{3.464898in}{1.023473in}}%
\pgfpathlineto{\pgfqpoint{3.464379in}{1.017417in}}%
\pgfpathlineto{\pgfqpoint{3.463340in}{0.966730in}}%
\pgfpathlineto{\pgfqpoint{3.462820in}{0.989242in}}%
\pgfpathlineto{\pgfqpoint{3.461781in}{1.066955in}}%
\pgfpathlineto{\pgfqpoint{3.461261in}{1.038709in}}%
\pgfpathlineto{\pgfqpoint{3.459703in}{0.890530in}}%
\pgfpathlineto{\pgfqpoint{3.459183in}{0.893609in}}%
\pgfpathlineto{\pgfqpoint{3.457105in}{1.030396in}}%
\pgfpathlineto{\pgfqpoint{3.456585in}{1.012495in}}%
\pgfpathlineto{\pgfqpoint{3.454507in}{0.962561in}}%
\pgfpathlineto{\pgfqpoint{3.453987in}{0.965045in}}%
\pgfpathlineto{\pgfqpoint{3.452948in}{0.979834in}}%
\pgfpathlineto{\pgfqpoint{3.451390in}{0.932770in}}%
\pgfpathlineto{\pgfqpoint{3.450870in}{0.943774in}}%
\pgfpathlineto{\pgfqpoint{3.449831in}{0.967468in}}%
\pgfpathlineto{\pgfqpoint{3.449311in}{0.959646in}}%
\pgfpathlineto{\pgfqpoint{3.447753in}{0.873111in}}%
\pgfpathlineto{\pgfqpoint{3.447233in}{0.841754in}}%
\pgfpathlineto{\pgfqpoint{3.446713in}{0.843041in}}%
\pgfpathlineto{\pgfqpoint{3.444116in}{0.986899in}}%
\pgfpathlineto{\pgfqpoint{3.443077in}{0.958384in}}%
\pgfpathlineto{\pgfqpoint{3.442557in}{0.959385in}}%
\pgfpathlineto{\pgfqpoint{3.440998in}{1.038392in}}%
\pgfpathlineto{\pgfqpoint{3.439440in}{0.939669in}}%
\pgfpathlineto{\pgfqpoint{3.438920in}{0.961786in}}%
\pgfpathlineto{\pgfqpoint{3.436842in}{1.030643in}}%
\pgfpathlineto{\pgfqpoint{3.435803in}{1.035028in}}%
\pgfpathlineto{\pgfqpoint{3.434244in}{1.000574in}}%
\pgfpathlineto{\pgfqpoint{3.433205in}{1.032926in}}%
\pgfpathlineto{\pgfqpoint{3.432685in}{1.017826in}}%
\pgfpathlineto{\pgfqpoint{3.431126in}{0.936558in}}%
\pgfpathlineto{\pgfqpoint{3.427490in}{1.093793in}}%
\pgfpathlineto{\pgfqpoint{3.425411in}{1.124121in}}%
\pgfpathlineto{\pgfqpoint{3.424372in}{1.106468in}}%
\pgfpathlineto{\pgfqpoint{3.423333in}{1.077324in}}%
\pgfpathlineto{\pgfqpoint{3.422813in}{1.088334in}}%
\pgfpathlineto{\pgfqpoint{3.421774in}{1.128973in}}%
\pgfpathlineto{\pgfqpoint{3.421255in}{1.124758in}}%
\pgfpathlineto{\pgfqpoint{3.419696in}{1.052679in}}%
\pgfpathlineto{\pgfqpoint{3.419176in}{1.058649in}}%
\pgfpathlineto{\pgfqpoint{3.417618in}{1.204836in}}%
\pgfpathlineto{\pgfqpoint{3.417098in}{1.200710in}}%
\pgfpathlineto{\pgfqpoint{3.415540in}{1.138485in}}%
\pgfpathlineto{\pgfqpoint{3.414500in}{1.169510in}}%
\pgfpathlineto{\pgfqpoint{3.412422in}{1.269748in}}%
\pgfpathlineto{\pgfqpoint{3.409824in}{1.167818in}}%
\pgfpathlineto{\pgfqpoint{3.409305in}{1.182635in}}%
\pgfpathlineto{\pgfqpoint{3.407746in}{1.232836in}}%
\pgfpathlineto{\pgfqpoint{3.406707in}{1.182000in}}%
\pgfpathlineto{\pgfqpoint{3.406187in}{1.186612in}}%
\pgfpathlineto{\pgfqpoint{3.404629in}{1.312962in}}%
\pgfpathlineto{\pgfqpoint{3.404109in}{1.308640in}}%
\pgfpathlineto{\pgfqpoint{3.403589in}{1.296525in}}%
\pgfpathlineto{\pgfqpoint{3.403070in}{1.306217in}}%
\pgfpathlineto{\pgfqpoint{3.401511in}{1.396484in}}%
\pgfpathlineto{\pgfqpoint{3.399433in}{1.251751in}}%
\pgfpathlineto{\pgfqpoint{3.398394in}{1.295110in}}%
\pgfpathlineto{\pgfqpoint{3.396835in}{1.442685in}}%
\pgfpathlineto{\pgfqpoint{3.396316in}{1.441058in}}%
\pgfpathlineto{\pgfqpoint{3.395796in}{1.431849in}}%
\pgfpathlineto{\pgfqpoint{3.395276in}{1.437368in}}%
\pgfpathlineto{\pgfqpoint{3.393198in}{1.477451in}}%
\pgfpathlineto{\pgfqpoint{3.392159in}{1.484139in}}%
\pgfpathlineto{\pgfqpoint{3.390600in}{1.448702in}}%
\pgfpathlineto{\pgfqpoint{3.389042in}{1.401376in}}%
\pgfpathlineto{\pgfqpoint{3.387483in}{1.457889in}}%
\pgfpathlineto{\pgfqpoint{3.386444in}{1.430493in}}%
\pgfpathlineto{\pgfqpoint{3.385924in}{1.441841in}}%
\pgfpathlineto{\pgfqpoint{3.383326in}{1.561565in}}%
\pgfpathlineto{\pgfqpoint{3.382807in}{1.557952in}}%
\pgfpathlineto{\pgfqpoint{3.381768in}{1.494337in}}%
\pgfpathlineto{\pgfqpoint{3.381248in}{1.495860in}}%
\pgfpathlineto{\pgfqpoint{3.378650in}{1.604089in}}%
\pgfpathlineto{\pgfqpoint{3.377092in}{1.665268in}}%
\pgfpathlineto{\pgfqpoint{3.375533in}{1.740098in}}%
\pgfpathlineto{\pgfqpoint{3.374494in}{1.681322in}}%
\pgfpathlineto{\pgfqpoint{3.372935in}{1.558339in}}%
\pgfpathlineto{\pgfqpoint{3.372415in}{1.560269in}}%
\pgfpathlineto{\pgfqpoint{3.369298in}{1.793361in}}%
\pgfpathlineto{\pgfqpoint{3.368779in}{1.781331in}}%
\pgfpathlineto{\pgfqpoint{3.367220in}{1.697603in}}%
\pgfpathlineto{\pgfqpoint{3.363583in}{1.832476in}}%
\pgfpathlineto{\pgfqpoint{3.362544in}{1.824973in}}%
\pgfpathlineto{\pgfqpoint{3.361505in}{1.811241in}}%
\pgfpathlineto{\pgfqpoint{3.356309in}{1.995831in}}%
\pgfpathlineto{\pgfqpoint{3.355270in}{2.003197in}}%
\pgfpathlineto{\pgfqpoint{3.354750in}{2.001645in}}%
\pgfpathlineto{\pgfqpoint{3.354231in}{2.001245in}}%
\pgfpathlineto{\pgfqpoint{3.353192in}{2.007940in}}%
\pgfpathlineto{\pgfqpoint{3.352152in}{1.986534in}}%
\pgfpathlineto{\pgfqpoint{3.351113in}{1.968992in}}%
\pgfpathlineto{\pgfqpoint{3.349035in}{2.067727in}}%
\pgfpathlineto{\pgfqpoint{3.348515in}{2.035501in}}%
\pgfpathlineto{\pgfqpoint{3.347476in}{1.954885in}}%
\pgfpathlineto{\pgfqpoint{3.346957in}{1.959528in}}%
\pgfpathlineto{\pgfqpoint{3.344359in}{2.117984in}}%
\pgfpathlineto{\pgfqpoint{3.343839in}{2.117755in}}%
\pgfpathlineto{\pgfqpoint{3.343320in}{2.119304in}}%
\pgfpathlineto{\pgfqpoint{3.342281in}{2.173640in}}%
\pgfpathlineto{\pgfqpoint{3.340202in}{2.258882in}}%
\pgfpathlineto{\pgfqpoint{3.339163in}{2.283985in}}%
\pgfpathlineto{\pgfqpoint{3.338644in}{2.283004in}}%
\pgfpathlineto{\pgfqpoint{3.337085in}{2.251216in}}%
\pgfpathlineto{\pgfqpoint{3.335526in}{2.200193in}}%
\pgfpathlineto{\pgfqpoint{3.332928in}{2.310745in}}%
\pgfpathlineto{\pgfqpoint{3.331889in}{2.362453in}}%
\pgfpathlineto{\pgfqpoint{3.330331in}{2.289823in}}%
\pgfpathlineto{\pgfqpoint{3.329811in}{2.304212in}}%
\pgfpathlineto{\pgfqpoint{3.325655in}{2.473931in}}%
\pgfpathlineto{\pgfqpoint{3.324096in}{2.410410in}}%
\pgfpathlineto{\pgfqpoint{3.323576in}{2.428552in}}%
\pgfpathlineto{\pgfqpoint{3.319939in}{2.604547in}}%
\pgfpathlineto{\pgfqpoint{3.318900in}{2.598603in}}%
\pgfpathlineto{\pgfqpoint{3.318381in}{2.597549in}}%
\pgfpathlineto{\pgfqpoint{3.316822in}{2.578798in}}%
\pgfpathlineto{\pgfqpoint{3.315263in}{2.613758in}}%
\pgfpathlineto{\pgfqpoint{3.314224in}{2.639942in}}%
\pgfpathlineto{\pgfqpoint{3.313704in}{2.628817in}}%
\pgfpathlineto{\pgfqpoint{3.311626in}{2.514544in}}%
\pgfpathlineto{\pgfqpoint{3.310068in}{2.584435in}}%
\pgfpathlineto{\pgfqpoint{3.309548in}{2.572818in}}%
\pgfpathlineto{\pgfqpoint{3.309028in}{2.565122in}}%
\pgfpathlineto{\pgfqpoint{3.307989in}{2.644300in}}%
\pgfpathlineto{\pgfqpoint{3.306950in}{2.721937in}}%
\pgfpathlineto{\pgfqpoint{3.306431in}{2.716828in}}%
\pgfpathlineto{\pgfqpoint{3.304872in}{2.663019in}}%
\pgfpathlineto{\pgfqpoint{3.300715in}{2.865964in}}%
\pgfpathlineto{\pgfqpoint{3.299157in}{2.816823in}}%
\pgfpathlineto{\pgfqpoint{3.298637in}{2.834837in}}%
\pgfpathlineto{\pgfqpoint{3.298117in}{2.847063in}}%
\pgfpathlineto{\pgfqpoint{3.296559in}{2.789251in}}%
\pgfpathlineto{\pgfqpoint{3.295520in}{2.802567in}}%
\pgfpathlineto{\pgfqpoint{3.294481in}{2.857067in}}%
\pgfpathlineto{\pgfqpoint{3.293441in}{2.930727in}}%
\pgfpathlineto{\pgfqpoint{3.292922in}{2.917775in}}%
\pgfpathlineto{\pgfqpoint{3.290844in}{2.794556in}}%
\pgfpathlineto{\pgfqpoint{3.290324in}{2.796953in}}%
\pgfpathlineto{\pgfqpoint{3.288765in}{2.901650in}}%
\pgfpathlineto{\pgfqpoint{3.287726in}{2.926717in}}%
\pgfpathlineto{\pgfqpoint{3.286687in}{2.902982in}}%
\pgfpathlineto{\pgfqpoint{3.286167in}{2.906857in}}%
\pgfpathlineto{\pgfqpoint{3.284609in}{2.947572in}}%
\pgfpathlineto{\pgfqpoint{3.283570in}{2.920598in}}%
\pgfpathlineto{\pgfqpoint{3.283050in}{2.921648in}}%
\pgfpathlineto{\pgfqpoint{3.282530in}{2.923871in}}%
\pgfpathlineto{\pgfqpoint{3.281491in}{2.909140in}}%
\pgfpathlineto{\pgfqpoint{3.279933in}{3.023469in}}%
\pgfpathlineto{\pgfqpoint{3.279413in}{3.006181in}}%
\pgfpathlineto{\pgfqpoint{3.278374in}{2.940971in}}%
\pgfpathlineto{\pgfqpoint{3.277854in}{2.946705in}}%
\pgfpathlineto{\pgfqpoint{3.277335in}{2.950116in}}%
\pgfpathlineto{\pgfqpoint{3.275776in}{2.884968in}}%
\pgfpathlineto{\pgfqpoint{3.275257in}{2.898476in}}%
\pgfpathlineto{\pgfqpoint{3.269541in}{3.184143in}}%
\pgfpathlineto{\pgfqpoint{3.269022in}{3.167518in}}%
\pgfpathlineto{\pgfqpoint{3.264346in}{2.954944in}}%
\pgfpathlineto{\pgfqpoint{3.263826in}{2.965520in}}%
\pgfpathlineto{\pgfqpoint{3.262267in}{3.054692in}}%
\pgfpathlineto{\pgfqpoint{3.261228in}{2.981181in}}%
\pgfpathlineto{\pgfqpoint{3.260709in}{2.989857in}}%
\pgfpathlineto{\pgfqpoint{3.259150in}{3.134829in}}%
\pgfpathlineto{\pgfqpoint{3.258630in}{3.127517in}}%
\pgfpathlineto{\pgfqpoint{3.256033in}{3.036588in}}%
\pgfpathlineto{\pgfqpoint{3.253954in}{3.079792in}}%
\pgfpathlineto{\pgfqpoint{3.253435in}{3.069207in}}%
\pgfpathlineto{\pgfqpoint{3.251356in}{2.987566in}}%
\pgfpathlineto{\pgfqpoint{3.250317in}{2.993896in}}%
\pgfpathlineto{\pgfqpoint{3.249278in}{2.981624in}}%
\pgfpathlineto{\pgfqpoint{3.248239in}{3.038333in}}%
\pgfpathlineto{\pgfqpoint{3.246680in}{3.209783in}}%
\pgfpathlineto{\pgfqpoint{3.246161in}{3.184021in}}%
\pgfpathlineto{\pgfqpoint{3.244602in}{2.947314in}}%
\pgfpathlineto{\pgfqpoint{3.244083in}{2.972729in}}%
\pgfpathlineto{\pgfqpoint{3.242524in}{3.090485in}}%
\pgfpathlineto{\pgfqpoint{3.239926in}{3.008281in}}%
\pgfpathlineto{\pgfqpoint{3.239406in}{3.013507in}}%
\pgfpathlineto{\pgfqpoint{3.238367in}{3.050716in}}%
\pgfpathlineto{\pgfqpoint{3.237848in}{3.044327in}}%
\pgfpathlineto{\pgfqpoint{3.236289in}{3.017682in}}%
\pgfpathlineto{\pgfqpoint{3.235769in}{3.014895in}}%
\pgfpathlineto{\pgfqpoint{3.234211in}{2.984016in}}%
\pgfpathlineto{\pgfqpoint{3.233172in}{3.029514in}}%
\pgfpathlineto{\pgfqpoint{3.231613in}{3.102522in}}%
\pgfpathlineto{\pgfqpoint{3.230054in}{3.039219in}}%
\pgfpathlineto{\pgfqpoint{3.227976in}{2.946314in}}%
\pgfpathlineto{\pgfqpoint{3.226937in}{2.934303in}}%
\pgfpathlineto{\pgfqpoint{3.225378in}{3.013004in}}%
\pgfpathlineto{\pgfqpoint{3.224859in}{2.991350in}}%
\pgfpathlineto{\pgfqpoint{3.223300in}{2.868040in}}%
\pgfpathlineto{\pgfqpoint{3.222780in}{2.869300in}}%
\pgfpathlineto{\pgfqpoint{3.222261in}{2.867379in}}%
\pgfpathlineto{\pgfqpoint{3.221222in}{2.849672in}}%
\pgfpathlineto{\pgfqpoint{3.220183in}{2.913566in}}%
\pgfpathlineto{\pgfqpoint{3.218624in}{3.029478in}}%
\pgfpathlineto{\pgfqpoint{3.217065in}{2.961335in}}%
\pgfpathlineto{\pgfqpoint{3.216546in}{2.970356in}}%
\pgfpathlineto{\pgfqpoint{3.216026in}{2.975332in}}%
\pgfpathlineto{\pgfqpoint{3.214987in}{2.889038in}}%
\pgfpathlineto{\pgfqpoint{3.213428in}{2.784475in}}%
\pgfpathlineto{\pgfqpoint{3.211869in}{2.828499in}}%
\pgfpathlineto{\pgfqpoint{3.209272in}{3.045700in}}%
\pgfpathlineto{\pgfqpoint{3.208752in}{2.992892in}}%
\pgfpathlineto{\pgfqpoint{3.207193in}{2.801351in}}%
\pgfpathlineto{\pgfqpoint{3.206674in}{2.816857in}}%
\pgfpathlineto{\pgfqpoint{3.204076in}{2.896630in}}%
\pgfpathlineto{\pgfqpoint{3.203037in}{2.840521in}}%
\pgfpathlineto{\pgfqpoint{3.200439in}{2.654317in}}%
\pgfpathlineto{\pgfqpoint{3.199919in}{2.675729in}}%
\pgfpathlineto{\pgfqpoint{3.198361in}{2.873062in}}%
\pgfpathlineto{\pgfqpoint{3.197841in}{2.861042in}}%
\pgfpathlineto{\pgfqpoint{3.195763in}{2.754484in}}%
\pgfpathlineto{\pgfqpoint{3.194724in}{2.706659in}}%
\pgfpathlineto{\pgfqpoint{3.192645in}{2.641937in}}%
\pgfpathlineto{\pgfqpoint{3.191606in}{2.670302in}}%
\pgfpathlineto{\pgfqpoint{3.190567in}{2.717048in}}%
\pgfpathlineto{\pgfqpoint{3.190048in}{2.704592in}}%
\pgfpathlineto{\pgfqpoint{3.189009in}{2.632196in}}%
\pgfpathlineto{\pgfqpoint{3.188489in}{2.635722in}}%
\pgfpathlineto{\pgfqpoint{3.187450in}{2.682096in}}%
\pgfpathlineto{\pgfqpoint{3.182774in}{2.518854in}}%
\pgfpathlineto{\pgfqpoint{3.181215in}{2.557477in}}%
\pgfpathlineto{\pgfqpoint{3.180695in}{2.546458in}}%
\pgfpathlineto{\pgfqpoint{3.179137in}{2.515557in}}%
\pgfpathlineto{\pgfqpoint{3.178617in}{2.510486in}}%
\pgfpathlineto{\pgfqpoint{3.175500in}{2.427195in}}%
\pgfpathlineto{\pgfqpoint{3.173941in}{2.497053in}}%
\pgfpathlineto{\pgfqpoint{3.173422in}{2.490259in}}%
\pgfpathlineto{\pgfqpoint{3.172382in}{2.454715in}}%
\pgfpathlineto{\pgfqpoint{3.171863in}{2.466158in}}%
\pgfpathlineto{\pgfqpoint{3.170304in}{2.520044in}}%
\pgfpathlineto{\pgfqpoint{3.169785in}{2.516220in}}%
\pgfpathlineto{\pgfqpoint{3.167706in}{2.458244in}}%
\pgfpathlineto{\pgfqpoint{3.167187in}{2.464474in}}%
\pgfpathlineto{\pgfqpoint{3.166667in}{2.469097in}}%
\pgfpathlineto{\pgfqpoint{3.166148in}{2.467020in}}%
\pgfpathlineto{\pgfqpoint{3.164069in}{2.422358in}}%
\pgfpathlineto{\pgfqpoint{3.163550in}{2.432415in}}%
\pgfpathlineto{\pgfqpoint{3.163030in}{2.437823in}}%
\pgfpathlineto{\pgfqpoint{3.161991in}{2.359787in}}%
\pgfpathlineto{\pgfqpoint{3.160952in}{2.247464in}}%
\pgfpathlineto{\pgfqpoint{3.160432in}{2.251475in}}%
\pgfpathlineto{\pgfqpoint{3.157835in}{2.411973in}}%
\pgfpathlineto{\pgfqpoint{3.157315in}{2.406759in}}%
\pgfpathlineto{\pgfqpoint{3.154198in}{2.258015in}}%
\pgfpathlineto{\pgfqpoint{3.153678in}{2.265693in}}%
\pgfpathlineto{\pgfqpoint{3.152119in}{2.359697in}}%
\pgfpathlineto{\pgfqpoint{3.150041in}{2.169771in}}%
\pgfpathlineto{\pgfqpoint{3.149521in}{2.179001in}}%
\pgfpathlineto{\pgfqpoint{3.147963in}{2.204512in}}%
\pgfpathlineto{\pgfqpoint{3.147443in}{2.202720in}}%
\pgfpathlineto{\pgfqpoint{3.145884in}{2.166029in}}%
\pgfpathlineto{\pgfqpoint{3.143806in}{2.099443in}}%
\pgfpathlineto{\pgfqpoint{3.142248in}{2.215636in}}%
\pgfpathlineto{\pgfqpoint{3.141728in}{2.212353in}}%
\pgfpathlineto{\pgfqpoint{3.139650in}{1.995282in}}%
\pgfpathlineto{\pgfqpoint{3.139130in}{2.003525in}}%
\pgfpathlineto{\pgfqpoint{3.138091in}{2.032321in}}%
\pgfpathlineto{\pgfqpoint{3.136013in}{1.940502in}}%
\pgfpathlineto{\pgfqpoint{3.135493in}{1.956210in}}%
\pgfpathlineto{\pgfqpoint{3.133934in}{2.028246in}}%
\pgfpathlineto{\pgfqpoint{3.131856in}{1.767907in}}%
\pgfpathlineto{\pgfqpoint{3.131337in}{1.786994in}}%
\pgfpathlineto{\pgfqpoint{3.129258in}{1.943552in}}%
\pgfpathlineto{\pgfqpoint{3.128739in}{1.942031in}}%
\pgfpathlineto{\pgfqpoint{3.127180in}{1.886969in}}%
\pgfpathlineto{\pgfqpoint{3.126661in}{1.891348in}}%
\pgfpathlineto{\pgfqpoint{3.125102in}{1.930232in}}%
\pgfpathlineto{\pgfqpoint{3.123024in}{1.724159in}}%
\pgfpathlineto{\pgfqpoint{3.122504in}{1.753021in}}%
\pgfpathlineto{\pgfqpoint{3.120426in}{1.853650in}}%
\pgfpathlineto{\pgfqpoint{3.119906in}{1.850096in}}%
\pgfpathlineto{\pgfqpoint{3.117828in}{1.771033in}}%
\pgfpathlineto{\pgfqpoint{3.117308in}{1.777915in}}%
\pgfpathlineto{\pgfqpoint{3.116789in}{1.779741in}}%
\pgfpathlineto{\pgfqpoint{3.115750in}{1.746215in}}%
\pgfpathlineto{\pgfqpoint{3.112632in}{1.602872in}}%
\pgfpathlineto{\pgfqpoint{3.108476in}{1.629733in}}%
\pgfpathlineto{\pgfqpoint{3.106917in}{1.704164in}}%
\pgfpathlineto{\pgfqpoint{3.106397in}{1.698036in}}%
\pgfpathlineto{\pgfqpoint{3.104839in}{1.583156in}}%
\pgfpathlineto{\pgfqpoint{3.104319in}{1.585806in}}%
\pgfpathlineto{\pgfqpoint{3.103800in}{1.588132in}}%
\pgfpathlineto{\pgfqpoint{3.102760in}{1.559458in}}%
\pgfpathlineto{\pgfqpoint{3.102241in}{1.564077in}}%
\pgfpathlineto{\pgfqpoint{3.101202in}{1.603852in}}%
\pgfpathlineto{\pgfqpoint{3.100682in}{1.597611in}}%
\pgfpathlineto{\pgfqpoint{3.098084in}{1.409978in}}%
\pgfpathlineto{\pgfqpoint{3.096526in}{1.295828in}}%
\pgfpathlineto{\pgfqpoint{3.095487in}{1.356850in}}%
\pgfpathlineto{\pgfqpoint{3.094447in}{1.441092in}}%
\pgfpathlineto{\pgfqpoint{3.093928in}{1.434132in}}%
\pgfpathlineto{\pgfqpoint{3.091330in}{1.304126in}}%
\pgfpathlineto{\pgfqpoint{3.090291in}{1.297183in}}%
\pgfpathlineto{\pgfqpoint{3.089771in}{1.297929in}}%
\pgfpathlineto{\pgfqpoint{3.088732in}{1.354315in}}%
\pgfpathlineto{\pgfqpoint{3.087693in}{1.420497in}}%
\pgfpathlineto{\pgfqpoint{3.087173in}{1.403645in}}%
\pgfpathlineto{\pgfqpoint{3.085095in}{1.249871in}}%
\pgfpathlineto{\pgfqpoint{3.083017in}{1.400341in}}%
\pgfpathlineto{\pgfqpoint{3.081978in}{1.354428in}}%
\pgfpathlineto{\pgfqpoint{3.078341in}{1.210386in}}%
\pgfpathlineto{\pgfqpoint{3.077821in}{1.205296in}}%
\pgfpathlineto{\pgfqpoint{3.077302in}{1.209017in}}%
\pgfpathlineto{\pgfqpoint{3.076782in}{1.209503in}}%
\pgfpathlineto{\pgfqpoint{3.074184in}{1.145339in}}%
\pgfpathlineto{\pgfqpoint{3.073665in}{1.150119in}}%
\pgfpathlineto{\pgfqpoint{3.073145in}{1.155382in}}%
\pgfpathlineto{\pgfqpoint{3.072626in}{1.152594in}}%
\pgfpathlineto{\pgfqpoint{3.070547in}{1.107730in}}%
\pgfpathlineto{\pgfqpoint{3.069508in}{1.144091in}}%
\pgfpathlineto{\pgfqpoint{3.067950in}{1.200537in}}%
\pgfpathlineto{\pgfqpoint{3.066391in}{1.140111in}}%
\pgfpathlineto{\pgfqpoint{3.065871in}{1.142441in}}%
\pgfpathlineto{\pgfqpoint{3.064313in}{1.204418in}}%
\pgfpathlineto{\pgfqpoint{3.063793in}{1.196518in}}%
\pgfpathlineto{\pgfqpoint{3.061195in}{1.117517in}}%
\pgfpathlineto{\pgfqpoint{3.057558in}{1.044458in}}%
\pgfpathlineto{\pgfqpoint{3.056519in}{0.989678in}}%
\pgfpathlineto{\pgfqpoint{3.055999in}{1.000783in}}%
\pgfpathlineto{\pgfqpoint{3.054441in}{1.068763in}}%
\pgfpathlineto{\pgfqpoint{3.053921in}{1.057417in}}%
\pgfpathlineto{\pgfqpoint{3.050284in}{0.893059in}}%
\pgfpathlineto{\pgfqpoint{3.045089in}{1.058389in}}%
\pgfpathlineto{\pgfqpoint{3.044569in}{1.047842in}}%
\pgfpathlineto{\pgfqpoint{3.041452in}{0.884880in}}%
\pgfpathlineto{\pgfqpoint{3.040932in}{0.889761in}}%
\pgfpathlineto{\pgfqpoint{3.038854in}{0.953783in}}%
\pgfpathlineto{\pgfqpoint{3.038334in}{0.951451in}}%
\pgfpathlineto{\pgfqpoint{3.037295in}{0.939742in}}%
\pgfpathlineto{\pgfqpoint{3.035217in}{1.034623in}}%
\pgfpathlineto{\pgfqpoint{3.034697in}{1.015246in}}%
\pgfpathlineto{\pgfqpoint{3.032099in}{0.906986in}}%
\pgfpathlineto{\pgfqpoint{3.028982in}{0.948120in}}%
\pgfpathlineto{\pgfqpoint{3.027943in}{0.929128in}}%
\pgfpathlineto{\pgfqpoint{3.027423in}{0.934035in}}%
\pgfpathlineto{\pgfqpoint{3.026904in}{0.937248in}}%
\pgfpathlineto{\pgfqpoint{3.025345in}{0.838140in}}%
\pgfpathlineto{\pgfqpoint{3.024306in}{0.871386in}}%
\pgfpathlineto{\pgfqpoint{3.022747in}{0.922580in}}%
\pgfpathlineto{\pgfqpoint{3.021189in}{0.901498in}}%
\pgfpathlineto{\pgfqpoint{3.020149in}{0.942950in}}%
\pgfpathlineto{\pgfqpoint{3.018591in}{1.005409in}}%
\pgfpathlineto{\pgfqpoint{3.015993in}{0.933099in}}%
\pgfpathlineto{\pgfqpoint{3.014434in}{0.903210in}}%
\pgfpathlineto{\pgfqpoint{3.013395in}{0.910469in}}%
\pgfpathlineto{\pgfqpoint{3.012875in}{0.906715in}}%
\pgfpathlineto{\pgfqpoint{3.011836in}{0.883948in}}%
\pgfpathlineto{\pgfqpoint{3.011317in}{0.886305in}}%
\pgfpathlineto{\pgfqpoint{3.007160in}{1.019604in}}%
\pgfpathlineto{\pgfqpoint{3.006121in}{0.987195in}}%
\pgfpathlineto{\pgfqpoint{3.004562in}{0.946496in}}%
\pgfpathlineto{\pgfqpoint{3.004043in}{0.950565in}}%
\pgfpathlineto{\pgfqpoint{3.003004in}{0.928606in}}%
\pgfpathlineto{\pgfqpoint{3.000925in}{0.879714in}}%
\pgfpathlineto{\pgfqpoint{2.999886in}{0.973558in}}%
\pgfpathlineto{\pgfqpoint{2.998847in}{1.096241in}}%
\pgfpathlineto{\pgfqpoint{2.998328in}{1.085200in}}%
\pgfpathlineto{\pgfqpoint{2.996769in}{0.993992in}}%
\pgfpathlineto{\pgfqpoint{2.996249in}{1.000245in}}%
\pgfpathlineto{\pgfqpoint{2.995210in}{1.019594in}}%
\pgfpathlineto{\pgfqpoint{2.994691in}{1.011738in}}%
\pgfpathlineto{\pgfqpoint{2.992612in}{0.939722in}}%
\pgfpathlineto{\pgfqpoint{2.991573in}{0.982145in}}%
\pgfpathlineto{\pgfqpoint{2.987936in}{1.208962in}}%
\pgfpathlineto{\pgfqpoint{2.987417in}{1.201425in}}%
\pgfpathlineto{\pgfqpoint{2.986378in}{1.136273in}}%
\pgfpathlineto{\pgfqpoint{2.984299in}{0.973511in}}%
\pgfpathlineto{\pgfqpoint{2.979623in}{1.249701in}}%
\pgfpathlineto{\pgfqpoint{2.979104in}{1.226332in}}%
\pgfpathlineto{\pgfqpoint{2.977025in}{1.118625in}}%
\pgfpathlineto{\pgfqpoint{2.971830in}{1.298866in}}%
\pgfpathlineto{\pgfqpoint{2.969751in}{1.202663in}}%
\pgfpathlineto{\pgfqpoint{2.969232in}{1.212572in}}%
\pgfpathlineto{\pgfqpoint{2.968193in}{1.223956in}}%
\pgfpathlineto{\pgfqpoint{2.966634in}{1.193355in}}%
\pgfpathlineto{\pgfqpoint{2.965075in}{1.298695in}}%
\pgfpathlineto{\pgfqpoint{2.961438in}{1.447013in}}%
\pgfpathlineto{\pgfqpoint{2.960399in}{1.410540in}}%
\pgfpathlineto{\pgfqpoint{2.958841in}{1.344637in}}%
\pgfpathlineto{\pgfqpoint{2.958321in}{1.350926in}}%
\pgfpathlineto{\pgfqpoint{2.957282in}{1.413368in}}%
\pgfpathlineto{\pgfqpoint{2.955723in}{1.520923in}}%
\pgfpathlineto{\pgfqpoint{2.953645in}{1.447485in}}%
\pgfpathlineto{\pgfqpoint{2.953125in}{1.448549in}}%
\pgfpathlineto{\pgfqpoint{2.949488in}{1.543850in}}%
\pgfpathlineto{\pgfqpoint{2.948969in}{1.548111in}}%
\pgfpathlineto{\pgfqpoint{2.948449in}{1.544307in}}%
\pgfpathlineto{\pgfqpoint{2.946371in}{1.515651in}}%
\pgfpathlineto{\pgfqpoint{2.945332in}{1.549011in}}%
\pgfpathlineto{\pgfqpoint{2.943254in}{1.674976in}}%
\pgfpathlineto{\pgfqpoint{2.942214in}{1.604149in}}%
\pgfpathlineto{\pgfqpoint{2.941175in}{1.531633in}}%
\pgfpathlineto{\pgfqpoint{2.940656in}{1.555114in}}%
\pgfpathlineto{\pgfqpoint{2.939097in}{1.699955in}}%
\pgfpathlineto{\pgfqpoint{2.938577in}{1.691821in}}%
\pgfpathlineto{\pgfqpoint{2.937019in}{1.610178in}}%
\pgfpathlineto{\pgfqpoint{2.936499in}{1.626773in}}%
\pgfpathlineto{\pgfqpoint{2.934421in}{1.772492in}}%
\pgfpathlineto{\pgfqpoint{2.933382in}{1.732718in}}%
\pgfpathlineto{\pgfqpoint{2.932862in}{1.737251in}}%
\pgfpathlineto{\pgfqpoint{2.927667in}{1.886846in}}%
\pgfpathlineto{\pgfqpoint{2.927147in}{1.884705in}}%
\pgfpathlineto{\pgfqpoint{2.926108in}{1.862629in}}%
\pgfpathlineto{\pgfqpoint{2.925588in}{1.866968in}}%
\pgfpathlineto{\pgfqpoint{2.923510in}{1.940656in}}%
\pgfpathlineto{\pgfqpoint{2.922990in}{1.933334in}}%
\pgfpathlineto{\pgfqpoint{2.921432in}{1.896362in}}%
\pgfpathlineto{\pgfqpoint{2.919873in}{1.796701in}}%
\pgfpathlineto{\pgfqpoint{2.918314in}{1.973527in}}%
\pgfpathlineto{\pgfqpoint{2.917275in}{2.030965in}}%
\pgfpathlineto{\pgfqpoint{2.916756in}{2.030684in}}%
\pgfpathlineto{\pgfqpoint{2.916236in}{2.025567in}}%
\pgfpathlineto{\pgfqpoint{2.915197in}{1.973572in}}%
\pgfpathlineto{\pgfqpoint{2.914158in}{1.896829in}}%
\pgfpathlineto{\pgfqpoint{2.913638in}{1.913072in}}%
\pgfpathlineto{\pgfqpoint{2.911040in}{2.231636in}}%
\pgfpathlineto{\pgfqpoint{2.910521in}{2.223420in}}%
\pgfpathlineto{\pgfqpoint{2.908443in}{2.106238in}}%
\pgfpathlineto{\pgfqpoint{2.907923in}{2.117591in}}%
\pgfpathlineto{\pgfqpoint{2.905845in}{2.187551in}}%
\pgfpathlineto{\pgfqpoint{2.905325in}{2.182476in}}%
\pgfpathlineto{\pgfqpoint{2.902727in}{2.145463in}}%
\pgfpathlineto{\pgfqpoint{2.902208in}{2.143309in}}%
\pgfpathlineto{\pgfqpoint{2.900130in}{2.289167in}}%
\pgfpathlineto{\pgfqpoint{2.899090in}{2.266044in}}%
\pgfpathlineto{\pgfqpoint{2.898571in}{2.258727in}}%
\pgfpathlineto{\pgfqpoint{2.897532in}{2.290928in}}%
\pgfpathlineto{\pgfqpoint{2.895453in}{2.335503in}}%
\pgfpathlineto{\pgfqpoint{2.894414in}{2.364934in}}%
\pgfpathlineto{\pgfqpoint{2.892336in}{2.505184in}}%
\pgfpathlineto{\pgfqpoint{2.891816in}{2.499014in}}%
\pgfpathlineto{\pgfqpoint{2.889738in}{2.445114in}}%
\pgfpathlineto{\pgfqpoint{2.888699in}{2.477399in}}%
\pgfpathlineto{\pgfqpoint{2.887660in}{2.516537in}}%
\pgfpathlineto{\pgfqpoint{2.887140in}{2.508852in}}%
\pgfpathlineto{\pgfqpoint{2.885582in}{2.462170in}}%
\pgfpathlineto{\pgfqpoint{2.885062in}{2.465397in}}%
\pgfpathlineto{\pgfqpoint{2.881425in}{2.627029in}}%
\pgfpathlineto{\pgfqpoint{2.880906in}{2.618113in}}%
\pgfpathlineto{\pgfqpoint{2.878308in}{2.561476in}}%
\pgfpathlineto{\pgfqpoint{2.877788in}{2.569377in}}%
\pgfpathlineto{\pgfqpoint{2.875190in}{2.695898in}}%
\pgfpathlineto{\pgfqpoint{2.874151in}{2.673222in}}%
\pgfpathlineto{\pgfqpoint{2.873632in}{2.673254in}}%
\pgfpathlineto{\pgfqpoint{2.871034in}{2.771581in}}%
\pgfpathlineto{\pgfqpoint{2.869475in}{2.842906in}}%
\pgfpathlineto{\pgfqpoint{2.868956in}{2.839199in}}%
\pgfpathlineto{\pgfqpoint{2.867397in}{2.859050in}}%
\pgfpathlineto{\pgfqpoint{2.865838in}{2.764019in}}%
\pgfpathlineto{\pgfqpoint{2.864799in}{2.732274in}}%
\pgfpathlineto{\pgfqpoint{2.862721in}{2.843136in}}%
\pgfpathlineto{\pgfqpoint{2.862201in}{2.840379in}}%
\pgfpathlineto{\pgfqpoint{2.861162in}{2.857569in}}%
\pgfpathlineto{\pgfqpoint{2.860642in}{2.853780in}}%
\pgfpathlineto{\pgfqpoint{2.859603in}{2.808769in}}%
\pgfpathlineto{\pgfqpoint{2.859084in}{2.809819in}}%
\pgfpathlineto{\pgfqpoint{2.854927in}{2.998122in}}%
\pgfpathlineto{\pgfqpoint{2.854408in}{3.005184in}}%
\pgfpathlineto{\pgfqpoint{2.853888in}{3.001171in}}%
\pgfpathlineto{\pgfqpoint{2.853369in}{2.998207in}}%
\pgfpathlineto{\pgfqpoint{2.852329in}{3.015733in}}%
\pgfpathlineto{\pgfqpoint{2.850251in}{3.102127in}}%
\pgfpathlineto{\pgfqpoint{2.849732in}{3.096616in}}%
\pgfpathlineto{\pgfqpoint{2.849212in}{3.093819in}}%
\pgfpathlineto{\pgfqpoint{2.848692in}{3.095210in}}%
\pgfpathlineto{\pgfqpoint{2.847653in}{3.042222in}}%
\pgfpathlineto{\pgfqpoint{2.846614in}{2.978805in}}%
\pgfpathlineto{\pgfqpoint{2.846095in}{2.990222in}}%
\pgfpathlineto{\pgfqpoint{2.844016in}{3.074736in}}%
\pgfpathlineto{\pgfqpoint{2.843497in}{3.067627in}}%
\pgfpathlineto{\pgfqpoint{2.842458in}{3.048044in}}%
\pgfpathlineto{\pgfqpoint{2.840899in}{3.115827in}}%
\pgfpathlineto{\pgfqpoint{2.840379in}{3.115657in}}%
\pgfpathlineto{\pgfqpoint{2.838301in}{3.049559in}}%
\pgfpathlineto{\pgfqpoint{2.837782in}{3.067073in}}%
\pgfpathlineto{\pgfqpoint{2.834145in}{3.281434in}}%
\pgfpathlineto{\pgfqpoint{2.833625in}{3.269336in}}%
\pgfpathlineto{\pgfqpoint{2.831547in}{3.133100in}}%
\pgfpathlineto{\pgfqpoint{2.829988in}{3.212403in}}%
\pgfpathlineto{\pgfqpoint{2.829469in}{3.210674in}}%
\pgfpathlineto{\pgfqpoint{2.828949in}{3.204393in}}%
\pgfpathlineto{\pgfqpoint{2.827910in}{3.234398in}}%
\pgfpathlineto{\pgfqpoint{2.826871in}{3.258462in}}%
\pgfpathlineto{\pgfqpoint{2.825832in}{3.244371in}}%
\pgfpathlineto{\pgfqpoint{2.824273in}{3.278257in}}%
\pgfpathlineto{\pgfqpoint{2.823753in}{3.268001in}}%
\pgfpathlineto{\pgfqpoint{2.821675in}{3.163518in}}%
\pgfpathlineto{\pgfqpoint{2.820116in}{3.252887in}}%
\pgfpathlineto{\pgfqpoint{2.819597in}{3.239174in}}%
\pgfpathlineto{\pgfqpoint{2.819077in}{3.223652in}}%
\pgfpathlineto{\pgfqpoint{2.818558in}{3.231913in}}%
\pgfpathlineto{\pgfqpoint{2.816999in}{3.297664in}}%
\pgfpathlineto{\pgfqpoint{2.815440in}{3.231589in}}%
\pgfpathlineto{\pgfqpoint{2.813882in}{3.303593in}}%
\pgfpathlineto{\pgfqpoint{2.813362in}{3.289977in}}%
\pgfpathlineto{\pgfqpoint{2.812323in}{3.266687in}}%
\pgfpathlineto{\pgfqpoint{2.810764in}{3.306145in}}%
\pgfpathlineto{\pgfqpoint{2.810245in}{3.304290in}}%
\pgfpathlineto{\pgfqpoint{2.808686in}{3.248091in}}%
\pgfpathlineto{\pgfqpoint{2.808166in}{3.257124in}}%
\pgfpathlineto{\pgfqpoint{2.807647in}{3.261771in}}%
\pgfpathlineto{\pgfqpoint{2.805568in}{3.133820in}}%
\pgfpathlineto{\pgfqpoint{2.805049in}{3.153879in}}%
\pgfpathlineto{\pgfqpoint{2.803490in}{3.213615in}}%
\pgfpathlineto{\pgfqpoint{2.802971in}{3.211650in}}%
\pgfpathlineto{\pgfqpoint{2.801412in}{3.248875in}}%
\pgfpathlineto{\pgfqpoint{2.800373in}{3.239759in}}%
\pgfpathlineto{\pgfqpoint{2.798814in}{3.230986in}}%
\pgfpathlineto{\pgfqpoint{2.797775in}{3.229746in}}%
\pgfpathlineto{\pgfqpoint{2.796736in}{3.243777in}}%
\pgfpathlineto{\pgfqpoint{2.795697in}{3.260879in}}%
\pgfpathlineto{\pgfqpoint{2.792579in}{3.204838in}}%
\pgfpathlineto{\pgfqpoint{2.792060in}{3.201131in}}%
\pgfpathlineto{\pgfqpoint{2.790501in}{3.227568in}}%
\pgfpathlineto{\pgfqpoint{2.789981in}{3.220620in}}%
\pgfpathlineto{\pgfqpoint{2.789462in}{3.214614in}}%
\pgfpathlineto{\pgfqpoint{2.788423in}{3.257615in}}%
\pgfpathlineto{\pgfqpoint{2.787384in}{3.325523in}}%
\pgfpathlineto{\pgfqpoint{2.786864in}{3.323070in}}%
\pgfpathlineto{\pgfqpoint{2.783747in}{3.196480in}}%
\pgfpathlineto{\pgfqpoint{2.782188in}{3.247101in}}%
\pgfpathlineto{\pgfqpoint{2.781668in}{3.241884in}}%
\pgfpathlineto{\pgfqpoint{2.781149in}{3.233625in}}%
\pgfpathlineto{\pgfqpoint{2.780629in}{3.235726in}}%
\pgfpathlineto{\pgfqpoint{2.779590in}{3.253892in}}%
\pgfpathlineto{\pgfqpoint{2.779071in}{3.251136in}}%
\pgfpathlineto{\pgfqpoint{2.775953in}{3.169787in}}%
\pgfpathlineto{\pgfqpoint{2.774394in}{3.068950in}}%
\pgfpathlineto{\pgfqpoint{2.773875in}{3.072044in}}%
\pgfpathlineto{\pgfqpoint{2.773355in}{3.070717in}}%
\pgfpathlineto{\pgfqpoint{2.771277in}{3.026208in}}%
\pgfpathlineto{\pgfqpoint{2.770757in}{3.031562in}}%
\pgfpathlineto{\pgfqpoint{2.768160in}{3.074015in}}%
\pgfpathlineto{\pgfqpoint{2.767640in}{3.071163in}}%
\pgfpathlineto{\pgfqpoint{2.766081in}{3.019536in}}%
\pgfpathlineto{\pgfqpoint{2.765042in}{3.052421in}}%
\pgfpathlineto{\pgfqpoint{2.764523in}{3.048949in}}%
\pgfpathlineto{\pgfqpoint{2.761925in}{2.955357in}}%
\pgfpathlineto{\pgfqpoint{2.761405in}{2.960463in}}%
\pgfpathlineto{\pgfqpoint{2.756729in}{3.059785in}}%
\pgfpathlineto{\pgfqpoint{2.756210in}{3.062590in}}%
\pgfpathlineto{\pgfqpoint{2.755170in}{2.994410in}}%
\pgfpathlineto{\pgfqpoint{2.753092in}{2.865788in}}%
\pgfpathlineto{\pgfqpoint{2.752053in}{2.877109in}}%
\pgfpathlineto{\pgfqpoint{2.749975in}{2.950832in}}%
\pgfpathlineto{\pgfqpoint{2.749455in}{2.936066in}}%
\pgfpathlineto{\pgfqpoint{2.746857in}{2.815407in}}%
\pgfpathlineto{\pgfqpoint{2.746338in}{2.813501in}}%
\pgfpathlineto{\pgfqpoint{2.745818in}{2.814661in}}%
\pgfpathlineto{\pgfqpoint{2.745299in}{2.818986in}}%
\pgfpathlineto{\pgfqpoint{2.744779in}{2.818110in}}%
\pgfpathlineto{\pgfqpoint{2.743220in}{2.780384in}}%
\pgfpathlineto{\pgfqpoint{2.742701in}{2.788361in}}%
\pgfpathlineto{\pgfqpoint{2.741142in}{2.819508in}}%
\pgfpathlineto{\pgfqpoint{2.740103in}{2.794390in}}%
\pgfpathlineto{\pgfqpoint{2.738544in}{2.736030in}}%
\pgfpathlineto{\pgfqpoint{2.738025in}{2.738167in}}%
\pgfpathlineto{\pgfqpoint{2.737505in}{2.737469in}}%
\pgfpathlineto{\pgfqpoint{2.735427in}{2.661243in}}%
\pgfpathlineto{\pgfqpoint{2.732829in}{2.591120in}}%
\pgfpathlineto{\pgfqpoint{2.731790in}{2.570052in}}%
\pgfpathlineto{\pgfqpoint{2.731270in}{2.570362in}}%
\pgfpathlineto{\pgfqpoint{2.729192in}{2.603467in}}%
\pgfpathlineto{\pgfqpoint{2.728153in}{2.604089in}}%
\pgfpathlineto{\pgfqpoint{2.727114in}{2.607742in}}%
\pgfpathlineto{\pgfqpoint{2.726075in}{2.615787in}}%
\pgfpathlineto{\pgfqpoint{2.725555in}{2.612780in}}%
\pgfpathlineto{\pgfqpoint{2.724516in}{2.576201in}}%
\pgfpathlineto{\pgfqpoint{2.721918in}{2.442735in}}%
\pgfpathlineto{\pgfqpoint{2.720879in}{2.490201in}}%
\pgfpathlineto{\pgfqpoint{2.719320in}{2.557748in}}%
\pgfpathlineto{\pgfqpoint{2.718281in}{2.541808in}}%
\pgfpathlineto{\pgfqpoint{2.717242in}{2.472670in}}%
\pgfpathlineto{\pgfqpoint{2.715164in}{2.222479in}}%
\pgfpathlineto{\pgfqpoint{2.714644in}{2.231374in}}%
\pgfpathlineto{\pgfqpoint{2.713605in}{2.274094in}}%
\pgfpathlineto{\pgfqpoint{2.713086in}{2.265416in}}%
\pgfpathlineto{\pgfqpoint{2.712046in}{2.247342in}}%
\pgfpathlineto{\pgfqpoint{2.708929in}{2.349476in}}%
\pgfpathlineto{\pgfqpoint{2.707890in}{2.301962in}}%
\pgfpathlineto{\pgfqpoint{2.706331in}{2.178928in}}%
\pgfpathlineto{\pgfqpoint{2.705812in}{2.181487in}}%
\pgfpathlineto{\pgfqpoint{2.704773in}{2.219679in}}%
\pgfpathlineto{\pgfqpoint{2.704253in}{2.218788in}}%
\pgfpathlineto{\pgfqpoint{2.703733in}{2.212370in}}%
\pgfpathlineto{\pgfqpoint{2.703214in}{2.215782in}}%
\pgfpathlineto{\pgfqpoint{2.702694in}{2.223443in}}%
\pgfpathlineto{\pgfqpoint{2.702175in}{2.217065in}}%
\pgfpathlineto{\pgfqpoint{2.699057in}{2.058590in}}%
\pgfpathlineto{\pgfqpoint{2.698538in}{2.061959in}}%
\pgfpathlineto{\pgfqpoint{2.696979in}{2.083775in}}%
\pgfpathlineto{\pgfqpoint{2.696459in}{2.082493in}}%
\pgfpathlineto{\pgfqpoint{2.694901in}{2.040177in}}%
\pgfpathlineto{\pgfqpoint{2.694381in}{2.031458in}}%
\pgfpathlineto{\pgfqpoint{2.693862in}{2.035231in}}%
\pgfpathlineto{\pgfqpoint{2.693342in}{2.043483in}}%
\pgfpathlineto{\pgfqpoint{2.692823in}{2.042122in}}%
\pgfpathlineto{\pgfqpoint{2.689186in}{1.928861in}}%
\pgfpathlineto{\pgfqpoint{2.688146in}{1.940469in}}%
\pgfpathlineto{\pgfqpoint{2.687627in}{1.938905in}}%
\pgfpathlineto{\pgfqpoint{2.686588in}{1.928286in}}%
\pgfpathlineto{\pgfqpoint{2.685549in}{1.933478in}}%
\pgfpathlineto{\pgfqpoint{2.683990in}{1.914926in}}%
\pgfpathlineto{\pgfqpoint{2.683470in}{1.915871in}}%
\pgfpathlineto{\pgfqpoint{2.682431in}{1.934203in}}%
\pgfpathlineto{\pgfqpoint{2.681912in}{1.950833in}}%
\pgfpathlineto{\pgfqpoint{2.681392in}{1.949359in}}%
\pgfpathlineto{\pgfqpoint{2.677755in}{1.772718in}}%
\pgfpathlineto{\pgfqpoint{2.673599in}{1.688137in}}%
\pgfpathlineto{\pgfqpoint{2.672559in}{1.706413in}}%
\pgfpathlineto{\pgfqpoint{2.672040in}{1.717538in}}%
\pgfpathlineto{\pgfqpoint{2.671520in}{1.715960in}}%
\pgfpathlineto{\pgfqpoint{2.669442in}{1.612040in}}%
\pgfpathlineto{\pgfqpoint{2.668403in}{1.623022in}}%
\pgfpathlineto{\pgfqpoint{2.667364in}{1.651416in}}%
\pgfpathlineto{\pgfqpoint{2.666325in}{1.692651in}}%
\pgfpathlineto{\pgfqpoint{2.665805in}{1.675270in}}%
\pgfpathlineto{\pgfqpoint{2.663727in}{1.509807in}}%
\pgfpathlineto{\pgfqpoint{2.662688in}{1.574994in}}%
\pgfpathlineto{\pgfqpoint{2.662168in}{1.571584in}}%
\pgfpathlineto{\pgfqpoint{2.659570in}{1.446028in}}%
\pgfpathlineto{\pgfqpoint{2.658531in}{1.422490in}}%
\pgfpathlineto{\pgfqpoint{2.656972in}{1.473778in}}%
\pgfpathlineto{\pgfqpoint{2.656453in}{1.468256in}}%
\pgfpathlineto{\pgfqpoint{2.654894in}{1.427630in}}%
\pgfpathlineto{\pgfqpoint{2.654375in}{1.434324in}}%
\pgfpathlineto{\pgfqpoint{2.653855in}{1.427497in}}%
\pgfpathlineto{\pgfqpoint{2.651257in}{1.348201in}}%
\pgfpathlineto{\pgfqpoint{2.650218in}{1.366156in}}%
\pgfpathlineto{\pgfqpoint{2.649698in}{1.361227in}}%
\pgfpathlineto{\pgfqpoint{2.648140in}{1.294887in}}%
\pgfpathlineto{\pgfqpoint{2.647620in}{1.302382in}}%
\pgfpathlineto{\pgfqpoint{2.645542in}{1.387565in}}%
\pgfpathlineto{\pgfqpoint{2.645022in}{1.368229in}}%
\pgfpathlineto{\pgfqpoint{2.642425in}{1.213032in}}%
\pgfpathlineto{\pgfqpoint{2.641905in}{1.219486in}}%
\pgfpathlineto{\pgfqpoint{2.641385in}{1.218962in}}%
\pgfpathlineto{\pgfqpoint{2.639827in}{1.181406in}}%
\pgfpathlineto{\pgfqpoint{2.637748in}{1.265995in}}%
\pgfpathlineto{\pgfqpoint{2.637229in}{1.252129in}}%
\pgfpathlineto{\pgfqpoint{2.635670in}{1.178790in}}%
\pgfpathlineto{\pgfqpoint{2.633592in}{1.272149in}}%
\pgfpathlineto{\pgfqpoint{2.633072in}{1.238949in}}%
\pgfpathlineto{\pgfqpoint{2.630994in}{1.117139in}}%
\pgfpathlineto{\pgfqpoint{2.628916in}{1.132243in}}%
\pgfpathlineto{\pgfqpoint{2.626838in}{1.045800in}}%
\pgfpathlineto{\pgfqpoint{2.626318in}{1.055019in}}%
\pgfpathlineto{\pgfqpoint{2.623201in}{1.137934in}}%
\pgfpathlineto{\pgfqpoint{2.622681in}{1.133671in}}%
\pgfpathlineto{\pgfqpoint{2.620603in}{1.027538in}}%
\pgfpathlineto{\pgfqpoint{2.620083in}{1.033405in}}%
\pgfpathlineto{\pgfqpoint{2.619044in}{1.061096in}}%
\pgfpathlineto{\pgfqpoint{2.618525in}{1.051095in}}%
\pgfpathlineto{\pgfqpoint{2.616966in}{0.952928in}}%
\pgfpathlineto{\pgfqpoint{2.616446in}{0.970423in}}%
\pgfpathlineto{\pgfqpoint{2.614888in}{1.027711in}}%
\pgfpathlineto{\pgfqpoint{2.613848in}{1.038772in}}%
\pgfpathlineto{\pgfqpoint{2.613329in}{1.037214in}}%
\pgfpathlineto{\pgfqpoint{2.611770in}{0.946543in}}%
\pgfpathlineto{\pgfqpoint{2.611251in}{0.951873in}}%
\pgfpathlineto{\pgfqpoint{2.610211in}{0.994191in}}%
\pgfpathlineto{\pgfqpoint{2.609692in}{0.983184in}}%
\pgfpathlineto{\pgfqpoint{2.608133in}{0.933455in}}%
\pgfpathlineto{\pgfqpoint{2.606055in}{0.995995in}}%
\pgfpathlineto{\pgfqpoint{2.605535in}{0.992863in}}%
\pgfpathlineto{\pgfqpoint{2.604496in}{0.976568in}}%
\pgfpathlineto{\pgfqpoint{2.602418in}{1.055896in}}%
\pgfpathlineto{\pgfqpoint{2.601898in}{1.036664in}}%
\pgfpathlineto{\pgfqpoint{2.599820in}{0.911714in}}%
\pgfpathlineto{\pgfqpoint{2.599301in}{0.917114in}}%
\pgfpathlineto{\pgfqpoint{2.597742in}{1.020536in}}%
\pgfpathlineto{\pgfqpoint{2.596703in}{1.073873in}}%
\pgfpathlineto{\pgfqpoint{2.596183in}{1.062263in}}%
\pgfpathlineto{\pgfqpoint{2.593585in}{0.907931in}}%
\pgfpathlineto{\pgfqpoint{2.593066in}{0.916249in}}%
\pgfpathlineto{\pgfqpoint{2.589948in}{0.980306in}}%
\pgfpathlineto{\pgfqpoint{2.588909in}{0.972139in}}%
\pgfpathlineto{\pgfqpoint{2.587870in}{0.960937in}}%
\pgfpathlineto{\pgfqpoint{2.587351in}{0.965717in}}%
\pgfpathlineto{\pgfqpoint{2.585792in}{1.021307in}}%
\pgfpathlineto{\pgfqpoint{2.584753in}{0.944384in}}%
\pgfpathlineto{\pgfqpoint{2.583194in}{0.862488in}}%
\pgfpathlineto{\pgfqpoint{2.582674in}{0.867354in}}%
\pgfpathlineto{\pgfqpoint{2.582155in}{0.866456in}}%
\pgfpathlineto{\pgfqpoint{2.581635in}{0.864146in}}%
\pgfpathlineto{\pgfqpoint{2.580596in}{0.891705in}}%
\pgfpathlineto{\pgfqpoint{2.579037in}{0.967113in}}%
\pgfpathlineto{\pgfqpoint{2.578518in}{0.965812in}}%
\pgfpathlineto{\pgfqpoint{2.576959in}{0.896883in}}%
\pgfpathlineto{\pgfqpoint{2.576440in}{0.906923in}}%
\pgfpathlineto{\pgfqpoint{2.574881in}{1.047999in}}%
\pgfpathlineto{\pgfqpoint{2.574361in}{1.031114in}}%
\pgfpathlineto{\pgfqpoint{2.572803in}{0.908900in}}%
\pgfpathlineto{\pgfqpoint{2.572283in}{0.924185in}}%
\pgfpathlineto{\pgfqpoint{2.571244in}{0.949040in}}%
\pgfpathlineto{\pgfqpoint{2.570724in}{0.944758in}}%
\pgfpathlineto{\pgfqpoint{2.570205in}{0.941456in}}%
\pgfpathlineto{\pgfqpoint{2.569166in}{0.963397in}}%
\pgfpathlineto{\pgfqpoint{2.565529in}{1.070138in}}%
\pgfpathlineto{\pgfqpoint{2.565009in}{1.067506in}}%
\pgfpathlineto{\pgfqpoint{2.564490in}{1.057958in}}%
\pgfpathlineto{\pgfqpoint{2.563970in}{1.058575in}}%
\pgfpathlineto{\pgfqpoint{2.562931in}{1.071663in}}%
\pgfpathlineto{\pgfqpoint{2.561372in}{1.006980in}}%
\pgfpathlineto{\pgfqpoint{2.560853in}{1.015987in}}%
\pgfpathlineto{\pgfqpoint{2.557735in}{1.147448in}}%
\pgfpathlineto{\pgfqpoint{2.557216in}{1.154399in}}%
\pgfpathlineto{\pgfqpoint{2.556177in}{1.136764in}}%
\pgfpathlineto{\pgfqpoint{2.555137in}{1.159584in}}%
\pgfpathlineto{\pgfqpoint{2.554618in}{1.157236in}}%
\pgfpathlineto{\pgfqpoint{2.552540in}{1.087414in}}%
\pgfpathlineto{\pgfqpoint{2.551500in}{1.130158in}}%
\pgfpathlineto{\pgfqpoint{2.550461in}{1.165314in}}%
\pgfpathlineto{\pgfqpoint{2.548903in}{1.078863in}}%
\pgfpathlineto{\pgfqpoint{2.548383in}{1.081235in}}%
\pgfpathlineto{\pgfqpoint{2.544227in}{1.264313in}}%
\pgfpathlineto{\pgfqpoint{2.543707in}{1.260397in}}%
\pgfpathlineto{\pgfqpoint{2.541629in}{1.149435in}}%
\pgfpathlineto{\pgfqpoint{2.540070in}{1.197895in}}%
\pgfpathlineto{\pgfqpoint{2.537992in}{1.268658in}}%
\pgfpathlineto{\pgfqpoint{2.535913in}{1.190569in}}%
\pgfpathlineto{\pgfqpoint{2.534355in}{1.301061in}}%
\pgfpathlineto{\pgfqpoint{2.533316in}{1.345903in}}%
\pgfpathlineto{\pgfqpoint{2.531757in}{1.250591in}}%
\pgfpathlineto{\pgfqpoint{2.531237in}{1.261192in}}%
\pgfpathlineto{\pgfqpoint{2.528640in}{1.434930in}}%
\pgfpathlineto{\pgfqpoint{2.527600in}{1.412919in}}%
\pgfpathlineto{\pgfqpoint{2.527081in}{1.415521in}}%
\pgfpathlineto{\pgfqpoint{2.526561in}{1.412707in}}%
\pgfpathlineto{\pgfqpoint{2.525003in}{1.344611in}}%
\pgfpathlineto{\pgfqpoint{2.523444in}{1.422104in}}%
\pgfpathlineto{\pgfqpoint{2.522405in}{1.420896in}}%
\pgfpathlineto{\pgfqpoint{2.520326in}{1.454262in}}%
\pgfpathlineto{\pgfqpoint{2.519807in}{1.449607in}}%
\pgfpathlineto{\pgfqpoint{2.518768in}{1.433560in}}%
\pgfpathlineto{\pgfqpoint{2.516689in}{1.531878in}}%
\pgfpathlineto{\pgfqpoint{2.516170in}{1.511602in}}%
\pgfpathlineto{\pgfqpoint{2.515131in}{1.477751in}}%
\pgfpathlineto{\pgfqpoint{2.514092in}{1.524347in}}%
\pgfpathlineto{\pgfqpoint{2.512533in}{1.608670in}}%
\pgfpathlineto{\pgfqpoint{2.512013in}{1.604381in}}%
\pgfpathlineto{\pgfqpoint{2.510974in}{1.578738in}}%
\pgfpathlineto{\pgfqpoint{2.510455in}{1.582179in}}%
\pgfpathlineto{\pgfqpoint{2.507337in}{1.674689in}}%
\pgfpathlineto{\pgfqpoint{2.506818in}{1.676142in}}%
\pgfpathlineto{\pgfqpoint{2.506298in}{1.675765in}}%
\pgfpathlineto{\pgfqpoint{2.505259in}{1.701760in}}%
\pgfpathlineto{\pgfqpoint{2.503700in}{1.750084in}}%
\pgfpathlineto{\pgfqpoint{2.502661in}{1.726664in}}%
\pgfpathlineto{\pgfqpoint{2.502142in}{1.714765in}}%
\pgfpathlineto{\pgfqpoint{2.501622in}{1.719225in}}%
\pgfpathlineto{\pgfqpoint{2.501102in}{1.729166in}}%
\pgfpathlineto{\pgfqpoint{2.500583in}{1.724252in}}%
\pgfpathlineto{\pgfqpoint{2.500063in}{1.708822in}}%
\pgfpathlineto{\pgfqpoint{2.499544in}{1.713641in}}%
\pgfpathlineto{\pgfqpoint{2.496426in}{1.881214in}}%
\pgfpathlineto{\pgfqpoint{2.495907in}{1.891555in}}%
\pgfpathlineto{\pgfqpoint{2.495387in}{1.887582in}}%
\pgfpathlineto{\pgfqpoint{2.494348in}{1.878849in}}%
\pgfpathlineto{\pgfqpoint{2.491231in}{1.930058in}}%
\pgfpathlineto{\pgfqpoint{2.488633in}{1.997511in}}%
\pgfpathlineto{\pgfqpoint{2.487074in}{2.033810in}}%
\pgfpathlineto{\pgfqpoint{2.486555in}{2.024763in}}%
\pgfpathlineto{\pgfqpoint{2.484476in}{1.942187in}}%
\pgfpathlineto{\pgfqpoint{2.483957in}{1.962801in}}%
\pgfpathlineto{\pgfqpoint{2.482398in}{2.063341in}}%
\pgfpathlineto{\pgfqpoint{2.481879in}{2.053684in}}%
\pgfpathlineto{\pgfqpoint{2.480320in}{2.026871in}}%
\pgfpathlineto{\pgfqpoint{2.479281in}{2.053361in}}%
\pgfpathlineto{\pgfqpoint{2.475644in}{2.199320in}}%
\pgfpathlineto{\pgfqpoint{2.475124in}{2.185020in}}%
\pgfpathlineto{\pgfqpoint{2.474605in}{2.167473in}}%
\pgfpathlineto{\pgfqpoint{2.474085in}{2.175171in}}%
\pgfpathlineto{\pgfqpoint{2.472007in}{2.314753in}}%
\pgfpathlineto{\pgfqpoint{2.471487in}{2.309649in}}%
\pgfpathlineto{\pgfqpoint{2.470448in}{2.290029in}}%
\pgfpathlineto{\pgfqpoint{2.466811in}{2.403940in}}%
\pgfpathlineto{\pgfqpoint{2.466292in}{2.398656in}}%
\pgfpathlineto{\pgfqpoint{2.464213in}{2.380085in}}%
\pgfpathlineto{\pgfqpoint{2.463694in}{2.378889in}}%
\pgfpathlineto{\pgfqpoint{2.459537in}{2.439633in}}%
\pgfpathlineto{\pgfqpoint{2.455900in}{2.566540in}}%
\pgfpathlineto{\pgfqpoint{2.454341in}{2.501891in}}%
\pgfpathlineto{\pgfqpoint{2.453822in}{2.509830in}}%
\pgfpathlineto{\pgfqpoint{2.449665in}{2.649457in}}%
\pgfpathlineto{\pgfqpoint{2.447587in}{2.616103in}}%
\pgfpathlineto{\pgfqpoint{2.447068in}{2.621618in}}%
\pgfpathlineto{\pgfqpoint{2.444989in}{2.707378in}}%
\pgfpathlineto{\pgfqpoint{2.444470in}{2.695370in}}%
\pgfpathlineto{\pgfqpoint{2.443950in}{2.691437in}}%
\pgfpathlineto{\pgfqpoint{2.441872in}{2.775412in}}%
\pgfpathlineto{\pgfqpoint{2.441352in}{2.758841in}}%
\pgfpathlineto{\pgfqpoint{2.439794in}{2.704335in}}%
\pgfpathlineto{\pgfqpoint{2.439274in}{2.706835in}}%
\pgfpathlineto{\pgfqpoint{2.436676in}{2.819565in}}%
\pgfpathlineto{\pgfqpoint{2.436157in}{2.806237in}}%
\pgfpathlineto{\pgfqpoint{2.435637in}{2.795943in}}%
\pgfpathlineto{\pgfqpoint{2.435118in}{2.802446in}}%
\pgfpathlineto{\pgfqpoint{2.433039in}{2.887837in}}%
\pgfpathlineto{\pgfqpoint{2.432520in}{2.887477in}}%
\pgfpathlineto{\pgfqpoint{2.432000in}{2.886043in}}%
\pgfpathlineto{\pgfqpoint{2.430961in}{2.928041in}}%
\pgfpathlineto{\pgfqpoint{2.429922in}{2.957505in}}%
\pgfpathlineto{\pgfqpoint{2.427844in}{2.912335in}}%
\pgfpathlineto{\pgfqpoint{2.426804in}{2.901290in}}%
\pgfpathlineto{\pgfqpoint{2.425765in}{2.935666in}}%
\pgfpathlineto{\pgfqpoint{2.424207in}{2.995671in}}%
\pgfpathlineto{\pgfqpoint{2.422128in}{2.976486in}}%
\pgfpathlineto{\pgfqpoint{2.421609in}{2.983268in}}%
\pgfpathlineto{\pgfqpoint{2.420050in}{3.000169in}}%
\pgfpathlineto{\pgfqpoint{2.419531in}{2.998378in}}%
\pgfpathlineto{\pgfqpoint{2.419011in}{3.002853in}}%
\pgfpathlineto{\pgfqpoint{2.414335in}{3.169247in}}%
\pgfpathlineto{\pgfqpoint{2.413296in}{3.227506in}}%
\pgfpathlineto{\pgfqpoint{2.412776in}{3.223047in}}%
\pgfpathlineto{\pgfqpoint{2.409659in}{3.072487in}}%
\pgfpathlineto{\pgfqpoint{2.407581in}{3.101889in}}%
\pgfpathlineto{\pgfqpoint{2.406541in}{3.156675in}}%
\pgfpathlineto{\pgfqpoint{2.405502in}{3.207146in}}%
\pgfpathlineto{\pgfqpoint{2.404463in}{3.167780in}}%
\pgfpathlineto{\pgfqpoint{2.403944in}{3.176823in}}%
\pgfpathlineto{\pgfqpoint{2.402904in}{3.234330in}}%
\pgfpathlineto{\pgfqpoint{2.402385in}{3.224521in}}%
\pgfpathlineto{\pgfqpoint{2.400826in}{3.158310in}}%
\pgfpathlineto{\pgfqpoint{2.397189in}{3.344818in}}%
\pgfpathlineto{\pgfqpoint{2.396670in}{3.343260in}}%
\pgfpathlineto{\pgfqpoint{2.394591in}{3.164000in}}%
\pgfpathlineto{\pgfqpoint{2.394072in}{3.184348in}}%
\pgfpathlineto{\pgfqpoint{2.392513in}{3.240540in}}%
\pgfpathlineto{\pgfqpoint{2.390954in}{3.213624in}}%
\pgfpathlineto{\pgfqpoint{2.389915in}{3.180154in}}%
\pgfpathlineto{\pgfqpoint{2.389396in}{3.191111in}}%
\pgfpathlineto{\pgfqpoint{2.385759in}{3.378949in}}%
\pgfpathlineto{\pgfqpoint{2.383680in}{3.294463in}}%
\pgfpathlineto{\pgfqpoint{2.383161in}{3.301163in}}%
\pgfpathlineto{\pgfqpoint{2.381083in}{3.358615in}}%
\pgfpathlineto{\pgfqpoint{2.380563in}{3.357658in}}%
\pgfpathlineto{\pgfqpoint{2.378485in}{3.261856in}}%
\pgfpathlineto{\pgfqpoint{2.377965in}{3.267195in}}%
\pgfpathlineto{\pgfqpoint{2.376407in}{3.308897in}}%
\pgfpathlineto{\pgfqpoint{2.374848in}{3.374881in}}%
\pgfpathlineto{\pgfqpoint{2.374328in}{3.367630in}}%
\pgfpathlineto{\pgfqpoint{2.372770in}{3.329083in}}%
\pgfpathlineto{\pgfqpoint{2.370691in}{3.403996in}}%
\pgfpathlineto{\pgfqpoint{2.370172in}{3.385830in}}%
\pgfpathlineto{\pgfqpoint{2.369133in}{3.331917in}}%
\pgfpathlineto{\pgfqpoint{2.368613in}{3.335338in}}%
\pgfpathlineto{\pgfqpoint{2.367574in}{3.384487in}}%
\pgfpathlineto{\pgfqpoint{2.367054in}{3.380952in}}%
\pgfpathlineto{\pgfqpoint{2.365496in}{3.254421in}}%
\pgfpathlineto{\pgfqpoint{2.364976in}{3.257600in}}%
\pgfpathlineto{\pgfqpoint{2.363417in}{3.338631in}}%
\pgfpathlineto{\pgfqpoint{2.362898in}{3.324029in}}%
\pgfpathlineto{\pgfqpoint{2.361339in}{3.223372in}}%
\pgfpathlineto{\pgfqpoint{2.360820in}{3.227514in}}%
\pgfpathlineto{\pgfqpoint{2.359261in}{3.291876in}}%
\pgfpathlineto{\pgfqpoint{2.357702in}{3.253177in}}%
\pgfpathlineto{\pgfqpoint{2.356663in}{3.277837in}}%
\pgfpathlineto{\pgfqpoint{2.355104in}{3.348374in}}%
\pgfpathlineto{\pgfqpoint{2.354585in}{3.332480in}}%
\pgfpathlineto{\pgfqpoint{2.351987in}{3.191942in}}%
\pgfpathlineto{\pgfqpoint{2.351467in}{3.185289in}}%
\pgfpathlineto{\pgfqpoint{2.350948in}{3.190106in}}%
\pgfpathlineto{\pgfqpoint{2.349389in}{3.253115in}}%
\pgfpathlineto{\pgfqpoint{2.348870in}{3.252766in}}%
\pgfpathlineto{\pgfqpoint{2.346791in}{3.211054in}}%
\pgfpathlineto{\pgfqpoint{2.346272in}{3.214379in}}%
\pgfpathlineto{\pgfqpoint{2.344193in}{3.256777in}}%
\pgfpathlineto{\pgfqpoint{2.343674in}{3.247622in}}%
\pgfpathlineto{\pgfqpoint{2.343154in}{3.243446in}}%
\pgfpathlineto{\pgfqpoint{2.341596in}{3.281732in}}%
\pgfpathlineto{\pgfqpoint{2.341076in}{3.270244in}}%
\pgfpathlineto{\pgfqpoint{2.338998in}{3.193870in}}%
\pgfpathlineto{\pgfqpoint{2.337439in}{3.277938in}}%
\pgfpathlineto{\pgfqpoint{2.336919in}{3.263163in}}%
\pgfpathlineto{\pgfqpoint{2.335361in}{3.194227in}}%
\pgfpathlineto{\pgfqpoint{2.334841in}{3.199619in}}%
\pgfpathlineto{\pgfqpoint{2.334322in}{3.196420in}}%
\pgfpathlineto{\pgfqpoint{2.330685in}{3.061972in}}%
\pgfpathlineto{\pgfqpoint{2.329126in}{3.062156in}}%
\pgfpathlineto{\pgfqpoint{2.328606in}{3.063448in}}%
\pgfpathlineto{\pgfqpoint{2.327048in}{3.085832in}}%
\pgfpathlineto{\pgfqpoint{2.326528in}{3.083641in}}%
\pgfpathlineto{\pgfqpoint{2.325489in}{3.073233in}}%
\pgfpathlineto{\pgfqpoint{2.323930in}{3.018425in}}%
\pgfpathlineto{\pgfqpoint{2.323411in}{3.019930in}}%
\pgfpathlineto{\pgfqpoint{2.322372in}{3.044581in}}%
\pgfpathlineto{\pgfqpoint{2.321852in}{3.040002in}}%
\pgfpathlineto{\pgfqpoint{2.318735in}{2.983423in}}%
\pgfpathlineto{\pgfqpoint{2.317176in}{3.016281in}}%
\pgfpathlineto{\pgfqpoint{2.313019in}{2.875728in}}%
\pgfpathlineto{\pgfqpoint{2.312500in}{2.886455in}}%
\pgfpathlineto{\pgfqpoint{2.310941in}{2.928579in}}%
\pgfpathlineto{\pgfqpoint{2.309902in}{2.897838in}}%
\pgfpathlineto{\pgfqpoint{2.308343in}{2.803604in}}%
\pgfpathlineto{\pgfqpoint{2.307824in}{2.809827in}}%
\pgfpathlineto{\pgfqpoint{2.306265in}{2.926201in}}%
\pgfpathlineto{\pgfqpoint{2.305745in}{2.908945in}}%
\pgfpathlineto{\pgfqpoint{2.302109in}{2.712997in}}%
\pgfpathlineto{\pgfqpoint{2.300550in}{2.671834in}}%
\pgfpathlineto{\pgfqpoint{2.298991in}{2.740238in}}%
\pgfpathlineto{\pgfqpoint{2.297952in}{2.759508in}}%
\pgfpathlineto{\pgfqpoint{2.294835in}{2.646110in}}%
\pgfpathlineto{\pgfqpoint{2.293795in}{2.608704in}}%
\pgfpathlineto{\pgfqpoint{2.293276in}{2.617074in}}%
\pgfpathlineto{\pgfqpoint{2.291717in}{2.666266in}}%
\pgfpathlineto{\pgfqpoint{2.291198in}{2.654824in}}%
\pgfpathlineto{\pgfqpoint{2.288600in}{2.550439in}}%
\pgfpathlineto{\pgfqpoint{2.288080in}{2.544700in}}%
\pgfpathlineto{\pgfqpoint{2.285482in}{2.427901in}}%
\pgfpathlineto{\pgfqpoint{2.284963in}{2.446716in}}%
\pgfpathlineto{\pgfqpoint{2.283924in}{2.485567in}}%
\pgfpathlineto{\pgfqpoint{2.283404in}{2.483936in}}%
\pgfpathlineto{\pgfqpoint{2.280806in}{2.420503in}}%
\pgfpathlineto{\pgfqpoint{2.280287in}{2.432805in}}%
\pgfpathlineto{\pgfqpoint{2.279767in}{2.447400in}}%
\pgfpathlineto{\pgfqpoint{2.279248in}{2.435995in}}%
\pgfpathlineto{\pgfqpoint{2.277689in}{2.283258in}}%
\pgfpathlineto{\pgfqpoint{2.277169in}{2.289759in}}%
\pgfpathlineto{\pgfqpoint{2.275611in}{2.353155in}}%
\pgfpathlineto{\pgfqpoint{2.271974in}{2.307519in}}%
\pgfpathlineto{\pgfqpoint{2.265739in}{2.095604in}}%
\pgfpathlineto{\pgfqpoint{2.265219in}{2.093152in}}%
\pgfpathlineto{\pgfqpoint{2.264180in}{2.045077in}}%
\pgfpathlineto{\pgfqpoint{2.263141in}{1.979305in}}%
\pgfpathlineto{\pgfqpoint{2.262621in}{2.003823in}}%
\pgfpathlineto{\pgfqpoint{2.261063in}{2.129426in}}%
\pgfpathlineto{\pgfqpoint{2.260543in}{2.123645in}}%
\pgfpathlineto{\pgfqpoint{2.257945in}{2.038480in}}%
\pgfpathlineto{\pgfqpoint{2.257426in}{2.041772in}}%
\pgfpathlineto{\pgfqpoint{2.256906in}{2.047414in}}%
\pgfpathlineto{\pgfqpoint{2.256387in}{2.034999in}}%
\pgfpathlineto{\pgfqpoint{2.254828in}{1.929259in}}%
\pgfpathlineto{\pgfqpoint{2.254308in}{1.951054in}}%
\pgfpathlineto{\pgfqpoint{2.253269in}{2.009115in}}%
\pgfpathlineto{\pgfqpoint{2.252750in}{1.991186in}}%
\pgfpathlineto{\pgfqpoint{2.249113in}{1.793454in}}%
\pgfpathlineto{\pgfqpoint{2.248593in}{1.794600in}}%
\pgfpathlineto{\pgfqpoint{2.246515in}{1.909941in}}%
\pgfpathlineto{\pgfqpoint{2.245995in}{1.895585in}}%
\pgfpathlineto{\pgfqpoint{2.243917in}{1.772575in}}%
\pgfpathlineto{\pgfqpoint{2.242878in}{1.732965in}}%
\pgfpathlineto{\pgfqpoint{2.242358in}{1.741744in}}%
\pgfpathlineto{\pgfqpoint{2.241839in}{1.747431in}}%
\pgfpathlineto{\pgfqpoint{2.240800in}{1.706709in}}%
\pgfpathlineto{\pgfqpoint{2.239761in}{1.663563in}}%
\pgfpathlineto{\pgfqpoint{2.239241in}{1.664026in}}%
\pgfpathlineto{\pgfqpoint{2.237682in}{1.709090in}}%
\pgfpathlineto{\pgfqpoint{2.237163in}{1.707496in}}%
\pgfpathlineto{\pgfqpoint{2.236124in}{1.683633in}}%
\pgfpathlineto{\pgfqpoint{2.235604in}{1.686664in}}%
\pgfpathlineto{\pgfqpoint{2.235084in}{1.691133in}}%
\pgfpathlineto{\pgfqpoint{2.234045in}{1.620504in}}%
\pgfpathlineto{\pgfqpoint{2.231967in}{1.452798in}}%
\pgfpathlineto{\pgfqpoint{2.230928in}{1.482104in}}%
\pgfpathlineto{\pgfqpoint{2.228850in}{1.552025in}}%
\pgfpathlineto{\pgfqpoint{2.227291in}{1.515128in}}%
\pgfpathlineto{\pgfqpoint{2.224693in}{1.400453in}}%
\pgfpathlineto{\pgfqpoint{2.224174in}{1.410849in}}%
\pgfpathlineto{\pgfqpoint{2.222095in}{1.503080in}}%
\pgfpathlineto{\pgfqpoint{2.221576in}{1.501544in}}%
\pgfpathlineto{\pgfqpoint{2.217419in}{1.366523in}}%
\pgfpathlineto{\pgfqpoint{2.216380in}{1.374876in}}%
\pgfpathlineto{\pgfqpoint{2.215341in}{1.340603in}}%
\pgfpathlineto{\pgfqpoint{2.213263in}{1.260271in}}%
\pgfpathlineto{\pgfqpoint{2.212743in}{1.262059in}}%
\pgfpathlineto{\pgfqpoint{2.211704in}{1.290465in}}%
\pgfpathlineto{\pgfqpoint{2.210665in}{1.334289in}}%
\pgfpathlineto{\pgfqpoint{2.210145in}{1.332863in}}%
\pgfpathlineto{\pgfqpoint{2.206508in}{1.207522in}}%
\pgfpathlineto{\pgfqpoint{2.205989in}{1.229137in}}%
\pgfpathlineto{\pgfqpoint{2.204950in}{1.269422in}}%
\pgfpathlineto{\pgfqpoint{2.204430in}{1.262407in}}%
\pgfpathlineto{\pgfqpoint{2.200793in}{1.120359in}}%
\pgfpathlineto{\pgfqpoint{2.200273in}{1.121416in}}%
\pgfpathlineto{\pgfqpoint{2.199234in}{1.127174in}}%
\pgfpathlineto{\pgfqpoint{2.197156in}{1.155009in}}%
\pgfpathlineto{\pgfqpoint{2.196117in}{1.155943in}}%
\pgfpathlineto{\pgfqpoint{2.195078in}{1.116393in}}%
\pgfpathlineto{\pgfqpoint{2.194039in}{1.066271in}}%
\pgfpathlineto{\pgfqpoint{2.193519in}{1.066632in}}%
\pgfpathlineto{\pgfqpoint{2.191960in}{1.113826in}}%
\pgfpathlineto{\pgfqpoint{2.191441in}{1.109170in}}%
\pgfpathlineto{\pgfqpoint{2.189882in}{1.026938in}}%
\pgfpathlineto{\pgfqpoint{2.189363in}{1.027938in}}%
\pgfpathlineto{\pgfqpoint{2.186245in}{1.103886in}}%
\pgfpathlineto{\pgfqpoint{2.185726in}{1.101384in}}%
\pgfpathlineto{\pgfqpoint{2.181569in}{1.005359in}}%
\pgfpathlineto{\pgfqpoint{2.180010in}{1.033435in}}%
\pgfpathlineto{\pgfqpoint{2.178452in}{0.991160in}}%
\pgfpathlineto{\pgfqpoint{2.177932in}{0.996946in}}%
\pgfpathlineto{\pgfqpoint{2.176893in}{1.024799in}}%
\pgfpathlineto{\pgfqpoint{2.176373in}{1.013153in}}%
\pgfpathlineto{\pgfqpoint{2.174295in}{0.906522in}}%
\pgfpathlineto{\pgfqpoint{2.173776in}{0.912015in}}%
\pgfpathlineto{\pgfqpoint{2.172736in}{0.924634in}}%
\pgfpathlineto{\pgfqpoint{2.172217in}{0.924323in}}%
\pgfpathlineto{\pgfqpoint{2.169619in}{0.873037in}}%
\pgfpathlineto{\pgfqpoint{2.169099in}{0.885401in}}%
\pgfpathlineto{\pgfqpoint{2.167541in}{0.961677in}}%
\pgfpathlineto{\pgfqpoint{2.167021in}{0.959937in}}%
\pgfpathlineto{\pgfqpoint{2.166502in}{0.956160in}}%
\pgfpathlineto{\pgfqpoint{2.165463in}{0.968489in}}%
\pgfpathlineto{\pgfqpoint{2.164943in}{0.966022in}}%
\pgfpathlineto{\pgfqpoint{2.163904in}{0.948988in}}%
\pgfpathlineto{\pgfqpoint{2.163384in}{0.952246in}}%
\pgfpathlineto{\pgfqpoint{2.161826in}{1.010083in}}%
\pgfpathlineto{\pgfqpoint{2.161306in}{1.027551in}}%
\pgfpathlineto{\pgfqpoint{2.160786in}{1.019781in}}%
\pgfpathlineto{\pgfqpoint{2.158708in}{0.910587in}}%
\pgfpathlineto{\pgfqpoint{2.158189in}{0.913484in}}%
\pgfpathlineto{\pgfqpoint{2.157669in}{0.910153in}}%
\pgfpathlineto{\pgfqpoint{2.156630in}{0.889697in}}%
\pgfpathlineto{\pgfqpoint{2.156110in}{0.894654in}}%
\pgfpathlineto{\pgfqpoint{2.153512in}{0.951579in}}%
\pgfpathlineto{\pgfqpoint{2.152993in}{0.943005in}}%
\pgfpathlineto{\pgfqpoint{2.150915in}{0.798712in}}%
\pgfpathlineto{\pgfqpoint{2.150395in}{0.834162in}}%
\pgfpathlineto{\pgfqpoint{2.148317in}{0.959555in}}%
\pgfpathlineto{\pgfqpoint{2.147797in}{0.961228in}}%
\pgfpathlineto{\pgfqpoint{2.143641in}{0.880060in}}%
\pgfpathlineto{\pgfqpoint{2.143121in}{0.877789in}}%
\pgfpathlineto{\pgfqpoint{2.140523in}{0.939326in}}%
\pgfpathlineto{\pgfqpoint{2.140004in}{0.936512in}}%
\pgfpathlineto{\pgfqpoint{2.138965in}{0.921527in}}%
\pgfpathlineto{\pgfqpoint{2.136886in}{0.998749in}}%
\pgfpathlineto{\pgfqpoint{2.136367in}{0.988263in}}%
\pgfpathlineto{\pgfqpoint{2.133769in}{0.952783in}}%
\pgfpathlineto{\pgfqpoint{2.133249in}{0.955471in}}%
\pgfpathlineto{\pgfqpoint{2.131691in}{0.998064in}}%
\pgfpathlineto{\pgfqpoint{2.131171in}{0.996612in}}%
\pgfpathlineto{\pgfqpoint{2.128573in}{0.929508in}}%
\pgfpathlineto{\pgfqpoint{2.127534in}{0.970389in}}%
\pgfpathlineto{\pgfqpoint{2.126495in}{1.012710in}}%
\pgfpathlineto{\pgfqpoint{2.125975in}{1.007046in}}%
\pgfpathlineto{\pgfqpoint{2.124417in}{0.976126in}}%
\pgfpathlineto{\pgfqpoint{2.123897in}{0.980313in}}%
\pgfpathlineto{\pgfqpoint{2.123378in}{0.979322in}}%
\pgfpathlineto{\pgfqpoint{2.122339in}{0.964448in}}%
\pgfpathlineto{\pgfqpoint{2.121299in}{1.019785in}}%
\pgfpathlineto{\pgfqpoint{2.120260in}{1.068381in}}%
\pgfpathlineto{\pgfqpoint{2.119741in}{1.051498in}}%
\pgfpathlineto{\pgfqpoint{2.117662in}{0.948012in}}%
\pgfpathlineto{\pgfqpoint{2.115065in}{1.058270in}}%
\pgfpathlineto{\pgfqpoint{2.112986in}{1.176771in}}%
\pgfpathlineto{\pgfqpoint{2.111947in}{1.166578in}}%
\pgfpathlineto{\pgfqpoint{2.111428in}{1.166884in}}%
\pgfpathlineto{\pgfqpoint{2.110908in}{1.171092in}}%
\pgfpathlineto{\pgfqpoint{2.110388in}{1.165010in}}%
\pgfpathlineto{\pgfqpoint{2.108830in}{1.123354in}}%
\pgfpathlineto{\pgfqpoint{2.106232in}{1.210999in}}%
\pgfpathlineto{\pgfqpoint{2.105712in}{1.194530in}}%
\pgfpathlineto{\pgfqpoint{2.104673in}{1.161829in}}%
\pgfpathlineto{\pgfqpoint{2.104154in}{1.167853in}}%
\pgfpathlineto{\pgfqpoint{2.094801in}{1.359326in}}%
\pgfpathlineto{\pgfqpoint{2.092723in}{1.325233in}}%
\pgfpathlineto{\pgfqpoint{2.091684in}{1.369005in}}%
\pgfpathlineto{\pgfqpoint{2.090125in}{1.435146in}}%
\pgfpathlineto{\pgfqpoint{2.088047in}{1.400114in}}%
\pgfpathlineto{\pgfqpoint{2.087528in}{1.401802in}}%
\pgfpathlineto{\pgfqpoint{2.087008in}{1.402950in}}%
\pgfpathlineto{\pgfqpoint{2.086488in}{1.402320in}}%
\pgfpathlineto{\pgfqpoint{2.085969in}{1.402077in}}%
\pgfpathlineto{\pgfqpoint{2.084930in}{1.411580in}}%
\pgfpathlineto{\pgfqpoint{2.083891in}{1.451271in}}%
\pgfpathlineto{\pgfqpoint{2.081812in}{1.602638in}}%
\pgfpathlineto{\pgfqpoint{2.081293in}{1.596903in}}%
\pgfpathlineto{\pgfqpoint{2.078695in}{1.508199in}}%
\pgfpathlineto{\pgfqpoint{2.078175in}{1.516864in}}%
\pgfpathlineto{\pgfqpoint{2.076097in}{1.617845in}}%
\pgfpathlineto{\pgfqpoint{2.075578in}{1.615066in}}%
\pgfpathlineto{\pgfqpoint{2.074538in}{1.603453in}}%
\pgfpathlineto{\pgfqpoint{2.073499in}{1.655923in}}%
\pgfpathlineto{\pgfqpoint{2.072460in}{1.703995in}}%
\pgfpathlineto{\pgfqpoint{2.070901in}{1.637140in}}%
\pgfpathlineto{\pgfqpoint{2.070382in}{1.640313in}}%
\pgfpathlineto{\pgfqpoint{2.069343in}{1.656260in}}%
\pgfpathlineto{\pgfqpoint{2.068304in}{1.629982in}}%
\pgfpathlineto{\pgfqpoint{2.067784in}{1.634092in}}%
\pgfpathlineto{\pgfqpoint{2.065706in}{1.809502in}}%
\pgfpathlineto{\pgfqpoint{2.065186in}{1.793033in}}%
\pgfpathlineto{\pgfqpoint{2.062588in}{1.750502in}}%
\pgfpathlineto{\pgfqpoint{2.062069in}{1.747800in}}%
\pgfpathlineto{\pgfqpoint{2.061549in}{1.750701in}}%
\pgfpathlineto{\pgfqpoint{2.060510in}{1.796672in}}%
\pgfpathlineto{\pgfqpoint{2.058432in}{1.878086in}}%
\pgfpathlineto{\pgfqpoint{2.057393in}{1.882349in}}%
\pgfpathlineto{\pgfqpoint{2.056354in}{1.915651in}}%
\pgfpathlineto{\pgfqpoint{2.054275in}{2.046658in}}%
\pgfpathlineto{\pgfqpoint{2.053756in}{2.034314in}}%
\pgfpathlineto{\pgfqpoint{2.052197in}{1.963821in}}%
\pgfpathlineto{\pgfqpoint{2.051677in}{1.970215in}}%
\pgfpathlineto{\pgfqpoint{2.048560in}{2.028815in}}%
\pgfpathlineto{\pgfqpoint{2.047001in}{2.099126in}}%
\pgfpathlineto{\pgfqpoint{2.046482in}{2.092807in}}%
\pgfpathlineto{\pgfqpoint{2.045443in}{2.072829in}}%
\pgfpathlineto{\pgfqpoint{2.043364in}{2.190283in}}%
\pgfpathlineto{\pgfqpoint{2.042325in}{2.167821in}}%
\pgfpathlineto{\pgfqpoint{2.041806in}{2.163282in}}%
\pgfpathlineto{\pgfqpoint{2.040767in}{2.175206in}}%
\pgfpathlineto{\pgfqpoint{2.039727in}{2.162465in}}%
\pgfpathlineto{\pgfqpoint{2.036090in}{2.294241in}}%
\pgfpathlineto{\pgfqpoint{2.032973in}{2.359954in}}%
\pgfpathlineto{\pgfqpoint{2.032454in}{2.358450in}}%
\pgfpathlineto{\pgfqpoint{2.030895in}{2.336315in}}%
\pgfpathlineto{\pgfqpoint{2.029856in}{2.377597in}}%
\pgfpathlineto{\pgfqpoint{2.028817in}{2.426711in}}%
\pgfpathlineto{\pgfqpoint{2.028297in}{2.416585in}}%
\pgfpathlineto{\pgfqpoint{2.027258in}{2.375249in}}%
\pgfpathlineto{\pgfqpoint{2.026219in}{2.445087in}}%
\pgfpathlineto{\pgfqpoint{2.025180in}{2.525350in}}%
\pgfpathlineto{\pgfqpoint{2.024660in}{2.510485in}}%
\pgfpathlineto{\pgfqpoint{2.023621in}{2.456287in}}%
\pgfpathlineto{\pgfqpoint{2.023101in}{2.470293in}}%
\pgfpathlineto{\pgfqpoint{2.020503in}{2.592279in}}%
\pgfpathlineto{\pgfqpoint{2.019464in}{2.611379in}}%
\pgfpathlineto{\pgfqpoint{2.018945in}{2.607430in}}%
\pgfpathlineto{\pgfqpoint{2.017906in}{2.586569in}}%
\pgfpathlineto{\pgfqpoint{2.016347in}{2.635320in}}%
\pgfpathlineto{\pgfqpoint{2.015827in}{2.621887in}}%
\pgfpathlineto{\pgfqpoint{2.014788in}{2.578829in}}%
\pgfpathlineto{\pgfqpoint{2.014269in}{2.583710in}}%
\pgfpathlineto{\pgfqpoint{2.009593in}{2.808668in}}%
\pgfpathlineto{\pgfqpoint{2.008553in}{2.777140in}}%
\pgfpathlineto{\pgfqpoint{2.006475in}{2.691349in}}%
\pgfpathlineto{\pgfqpoint{2.005956in}{2.701317in}}%
\pgfpathlineto{\pgfqpoint{2.003358in}{2.825250in}}%
\pgfpathlineto{\pgfqpoint{2.000760in}{2.806408in}}%
\pgfpathlineto{\pgfqpoint{2.000240in}{2.802245in}}%
\pgfpathlineto{\pgfqpoint{1.999721in}{2.804178in}}%
\pgfpathlineto{\pgfqpoint{1.998682in}{2.847183in}}%
\pgfpathlineto{\pgfqpoint{1.996603in}{2.989515in}}%
\pgfpathlineto{\pgfqpoint{1.994525in}{2.903941in}}%
\pgfpathlineto{\pgfqpoint{1.994006in}{2.908162in}}%
\pgfpathlineto{\pgfqpoint{1.991927in}{3.016728in}}%
\pgfpathlineto{\pgfqpoint{1.991408in}{3.006489in}}%
\pgfpathlineto{\pgfqpoint{1.990369in}{2.963631in}}%
\pgfpathlineto{\pgfqpoint{1.989849in}{2.970700in}}%
\pgfpathlineto{\pgfqpoint{1.987771in}{3.074282in}}%
\pgfpathlineto{\pgfqpoint{1.987251in}{3.071195in}}%
\pgfpathlineto{\pgfqpoint{1.985693in}{3.039693in}}%
\pgfpathlineto{\pgfqpoint{1.981536in}{3.117558in}}%
\pgfpathlineto{\pgfqpoint{1.978419in}{3.304399in}}%
\pgfpathlineto{\pgfqpoint{1.977899in}{3.304080in}}%
\pgfpathlineto{\pgfqpoint{1.975301in}{3.147510in}}%
\pgfpathlineto{\pgfqpoint{1.974262in}{3.163050in}}%
\pgfpathlineto{\pgfqpoint{1.973223in}{3.175558in}}%
\pgfpathlineto{\pgfqpoint{1.971664in}{3.129219in}}%
\pgfpathlineto{\pgfqpoint{1.971145in}{3.139439in}}%
\pgfpathlineto{\pgfqpoint{1.968547in}{3.273430in}}%
\pgfpathlineto{\pgfqpoint{1.968027in}{3.279195in}}%
\pgfpathlineto{\pgfqpoint{1.967508in}{3.275148in}}%
\pgfpathlineto{\pgfqpoint{1.964910in}{3.226723in}}%
\pgfpathlineto{\pgfqpoint{1.964390in}{3.226007in}}%
\pgfpathlineto{\pgfqpoint{1.962832in}{3.251941in}}%
\pgfpathlineto{\pgfqpoint{1.962312in}{3.244180in}}%
\pgfpathlineto{\pgfqpoint{1.961792in}{3.233780in}}%
\pgfpathlineto{\pgfqpoint{1.961273in}{3.241297in}}%
\pgfpathlineto{\pgfqpoint{1.958155in}{3.392928in}}%
\pgfpathlineto{\pgfqpoint{1.955038in}{3.267144in}}%
\pgfpathlineto{\pgfqpoint{1.954519in}{3.270818in}}%
\pgfpathlineto{\pgfqpoint{1.950882in}{3.386388in}}%
\pgfpathlineto{\pgfqpoint{1.950362in}{3.381284in}}%
\pgfpathlineto{\pgfqpoint{1.948803in}{3.331159in}}%
\pgfpathlineto{\pgfqpoint{1.946725in}{3.420286in}}%
\pgfpathlineto{\pgfqpoint{1.945686in}{3.417487in}}%
\pgfpathlineto{\pgfqpoint{1.944647in}{3.421719in}}%
\pgfpathlineto{\pgfqpoint{1.943088in}{3.394855in}}%
\pgfpathlineto{\pgfqpoint{1.942569in}{3.400926in}}%
\pgfpathlineto{\pgfqpoint{1.941529in}{3.425840in}}%
\pgfpathlineto{\pgfqpoint{1.941010in}{3.419502in}}%
\pgfpathlineto{\pgfqpoint{1.938932in}{3.375532in}}%
\pgfpathlineto{\pgfqpoint{1.937892in}{3.370220in}}%
\pgfpathlineto{\pgfqpoint{1.933736in}{3.437612in}}%
\pgfpathlineto{\pgfqpoint{1.932697in}{3.408062in}}%
\pgfpathlineto{\pgfqpoint{1.931658in}{3.373811in}}%
\pgfpathlineto{\pgfqpoint{1.931138in}{3.379538in}}%
\pgfpathlineto{\pgfqpoint{1.930099in}{3.400042in}}%
\pgfpathlineto{\pgfqpoint{1.928540in}{3.356667in}}%
\pgfpathlineto{\pgfqpoint{1.928021in}{3.358542in}}%
\pgfpathlineto{\pgfqpoint{1.925942in}{3.422934in}}%
\pgfpathlineto{\pgfqpoint{1.925423in}{3.420332in}}%
\pgfpathlineto{\pgfqpoint{1.924384in}{3.376061in}}%
\pgfpathlineto{\pgfqpoint{1.923345in}{3.311599in}}%
\pgfpathlineto{\pgfqpoint{1.922825in}{3.316261in}}%
\pgfpathlineto{\pgfqpoint{1.921786in}{3.360747in}}%
\pgfpathlineto{\pgfqpoint{1.921266in}{3.344760in}}%
\pgfpathlineto{\pgfqpoint{1.919708in}{3.275796in}}%
\pgfpathlineto{\pgfqpoint{1.916071in}{3.356123in}}%
\pgfpathlineto{\pgfqpoint{1.914512in}{3.296852in}}%
\pgfpathlineto{\pgfqpoint{1.913473in}{3.311909in}}%
\pgfpathlineto{\pgfqpoint{1.912953in}{3.313011in}}%
\pgfpathlineto{\pgfqpoint{1.909836in}{3.259203in}}%
\pgfpathlineto{\pgfqpoint{1.908277in}{3.249421in}}%
\pgfpathlineto{\pgfqpoint{1.906718in}{3.263431in}}%
\pgfpathlineto{\pgfqpoint{1.905679in}{3.290372in}}%
\pgfpathlineto{\pgfqpoint{1.904121in}{3.339730in}}%
\pgfpathlineto{\pgfqpoint{1.903601in}{3.337291in}}%
\pgfpathlineto{\pgfqpoint{1.901523in}{3.282294in}}%
\pgfpathlineto{\pgfqpoint{1.900484in}{3.289288in}}%
\pgfpathlineto{\pgfqpoint{1.898925in}{3.297179in}}%
\pgfpathlineto{\pgfqpoint{1.897886in}{3.242802in}}%
\pgfpathlineto{\pgfqpoint{1.896327in}{3.170649in}}%
\pgfpathlineto{\pgfqpoint{1.895808in}{3.173043in}}%
\pgfpathlineto{\pgfqpoint{1.894768in}{3.126605in}}%
\pgfpathlineto{\pgfqpoint{1.893210in}{3.069653in}}%
\pgfpathlineto{\pgfqpoint{1.892171in}{3.101718in}}%
\pgfpathlineto{\pgfqpoint{1.891131in}{3.156239in}}%
\pgfpathlineto{\pgfqpoint{1.890612in}{3.151521in}}%
\pgfpathlineto{\pgfqpoint{1.889053in}{3.095206in}}%
\pgfpathlineto{\pgfqpoint{1.888014in}{3.105794in}}%
\pgfpathlineto{\pgfqpoint{1.885416in}{3.039011in}}%
\pgfpathlineto{\pgfqpoint{1.883857in}{3.125963in}}%
\pgfpathlineto{\pgfqpoint{1.883338in}{3.117297in}}%
\pgfpathlineto{\pgfqpoint{1.880221in}{2.912069in}}%
\pgfpathlineto{\pgfqpoint{1.879181in}{2.892569in}}%
\pgfpathlineto{\pgfqpoint{1.878662in}{2.897695in}}%
\pgfpathlineto{\pgfqpoint{1.876064in}{2.983844in}}%
\pgfpathlineto{\pgfqpoint{1.875544in}{2.983315in}}%
\pgfpathlineto{\pgfqpoint{1.874505in}{2.961429in}}%
\pgfpathlineto{\pgfqpoint{1.872427in}{2.884849in}}%
\pgfpathlineto{\pgfqpoint{1.870868in}{2.913119in}}%
\pgfpathlineto{\pgfqpoint{1.870349in}{2.910216in}}%
\pgfpathlineto{\pgfqpoint{1.868790in}{2.900770in}}%
\pgfpathlineto{\pgfqpoint{1.866712in}{2.855188in}}%
\pgfpathlineto{\pgfqpoint{1.863594in}{2.709370in}}%
\pgfpathlineto{\pgfqpoint{1.863075in}{2.716505in}}%
\pgfpathlineto{\pgfqpoint{1.860477in}{2.763101in}}%
\pgfpathlineto{\pgfqpoint{1.859957in}{2.762492in}}%
\pgfpathlineto{\pgfqpoint{1.858918in}{2.736153in}}%
\pgfpathlineto{\pgfqpoint{1.857879in}{2.692851in}}%
\pgfpathlineto{\pgfqpoint{1.857360in}{2.693532in}}%
\pgfpathlineto{\pgfqpoint{1.856320in}{2.711361in}}%
\pgfpathlineto{\pgfqpoint{1.855801in}{2.709802in}}%
\pgfpathlineto{\pgfqpoint{1.854762in}{2.672370in}}%
\pgfpathlineto{\pgfqpoint{1.852684in}{2.549915in}}%
\pgfpathlineto{\pgfqpoint{1.851125in}{2.614014in}}%
\pgfpathlineto{\pgfqpoint{1.850605in}{2.602833in}}%
\pgfpathlineto{\pgfqpoint{1.849566in}{2.582171in}}%
\pgfpathlineto{\pgfqpoint{1.848527in}{2.593801in}}%
\pgfpathlineto{\pgfqpoint{1.841773in}{2.348371in}}%
\pgfpathlineto{\pgfqpoint{1.841253in}{2.355662in}}%
\pgfpathlineto{\pgfqpoint{1.839175in}{2.401842in}}%
\pgfpathlineto{\pgfqpoint{1.838655in}{2.401354in}}%
\pgfpathlineto{\pgfqpoint{1.837616in}{2.376855in}}%
\pgfpathlineto{\pgfqpoint{1.835538in}{2.321904in}}%
\pgfpathlineto{\pgfqpoint{1.833979in}{2.285027in}}%
\pgfpathlineto{\pgfqpoint{1.831901in}{2.195152in}}%
\pgfpathlineto{\pgfqpoint{1.831381in}{2.210296in}}%
\pgfpathlineto{\pgfqpoint{1.830342in}{2.232861in}}%
\pgfpathlineto{\pgfqpoint{1.829823in}{2.226361in}}%
\pgfpathlineto{\pgfqpoint{1.828264in}{2.165611in}}%
\pgfpathlineto{\pgfqpoint{1.827225in}{2.120857in}}%
\pgfpathlineto{\pgfqpoint{1.826705in}{2.121241in}}%
\pgfpathlineto{\pgfqpoint{1.825666in}{2.132090in}}%
\pgfpathlineto{\pgfqpoint{1.824627in}{2.099392in}}%
\pgfpathlineto{\pgfqpoint{1.822029in}{2.012365in}}%
\pgfpathlineto{\pgfqpoint{1.821510in}{2.011178in}}%
\pgfpathlineto{\pgfqpoint{1.819951in}{2.033164in}}%
\pgfpathlineto{\pgfqpoint{1.818912in}{2.002589in}}%
\pgfpathlineto{\pgfqpoint{1.816314in}{1.870528in}}%
\pgfpathlineto{\pgfqpoint{1.815794in}{1.882684in}}%
\pgfpathlineto{\pgfqpoint{1.814236in}{1.910208in}}%
\pgfpathlineto{\pgfqpoint{1.813196in}{1.863507in}}%
\pgfpathlineto{\pgfqpoint{1.812157in}{1.815130in}}%
\pgfpathlineto{\pgfqpoint{1.811638in}{1.825609in}}%
\pgfpathlineto{\pgfqpoint{1.810599in}{1.857380in}}%
\pgfpathlineto{\pgfqpoint{1.810079in}{1.848947in}}%
\pgfpathlineto{\pgfqpoint{1.808001in}{1.814216in}}%
\pgfpathlineto{\pgfqpoint{1.805923in}{1.786540in}}%
\pgfpathlineto{\pgfqpoint{1.805403in}{1.787061in}}%
\pgfpathlineto{\pgfqpoint{1.804883in}{1.780730in}}%
\pgfpathlineto{\pgfqpoint{1.799688in}{1.554701in}}%
\pgfpathlineto{\pgfqpoint{1.799168in}{1.573330in}}%
\pgfpathlineto{\pgfqpoint{1.797609in}{1.642546in}}%
\pgfpathlineto{\pgfqpoint{1.792933in}{1.544140in}}%
\pgfpathlineto{\pgfqpoint{1.790855in}{1.439667in}}%
\pgfpathlineto{\pgfqpoint{1.790336in}{1.454307in}}%
\pgfpathlineto{\pgfqpoint{1.787738in}{1.545204in}}%
\pgfpathlineto{\pgfqpoint{1.778385in}{1.267422in}}%
\pgfpathlineto{\pgfqpoint{1.777866in}{1.267474in}}%
\pgfpathlineto{\pgfqpoint{1.775788in}{1.329808in}}%
\pgfpathlineto{\pgfqpoint{1.775268in}{1.323470in}}%
\pgfpathlineto{\pgfqpoint{1.774749in}{1.319303in}}%
\pgfpathlineto{\pgfqpoint{1.774229in}{1.323280in}}%
\pgfpathlineto{\pgfqpoint{1.773190in}{1.332649in}}%
\pgfpathlineto{\pgfqpoint{1.772151in}{1.299585in}}%
\pgfpathlineto{\pgfqpoint{1.770592in}{1.235978in}}%
\pgfpathlineto{\pgfqpoint{1.769033in}{1.307039in}}%
\pgfpathlineto{\pgfqpoint{1.768514in}{1.303643in}}%
\pgfpathlineto{\pgfqpoint{1.765396in}{1.160247in}}%
\pgfpathlineto{\pgfqpoint{1.764357in}{1.114312in}}%
\pgfpathlineto{\pgfqpoint{1.762279in}{1.182212in}}%
\pgfpathlineto{\pgfqpoint{1.761759in}{1.172423in}}%
\pgfpathlineto{\pgfqpoint{1.760720in}{1.155160in}}%
\pgfpathlineto{\pgfqpoint{1.760201in}{1.156521in}}%
\pgfpathlineto{\pgfqpoint{1.759681in}{1.155323in}}%
\pgfpathlineto{\pgfqpoint{1.758642in}{1.124029in}}%
\pgfpathlineto{\pgfqpoint{1.756564in}{1.031658in}}%
\pgfpathlineto{\pgfqpoint{1.756044in}{1.042419in}}%
\pgfpathlineto{\pgfqpoint{1.755005in}{1.072290in}}%
\pgfpathlineto{\pgfqpoint{1.754485in}{1.068237in}}%
\pgfpathlineto{\pgfqpoint{1.751888in}{1.021444in}}%
\pgfpathlineto{\pgfqpoint{1.751368in}{1.033161in}}%
\pgfpathlineto{\pgfqpoint{1.750329in}{1.050152in}}%
\pgfpathlineto{\pgfqpoint{1.748251in}{0.974465in}}%
\pgfpathlineto{\pgfqpoint{1.747731in}{0.976814in}}%
\pgfpathlineto{\pgfqpoint{1.745133in}{1.053349in}}%
\pgfpathlineto{\pgfqpoint{1.744094in}{1.051242in}}%
\pgfpathlineto{\pgfqpoint{1.743055in}{1.057158in}}%
\pgfpathlineto{\pgfqpoint{1.742016in}{1.047396in}}%
\pgfpathlineto{\pgfqpoint{1.740457in}{1.016657in}}%
\pgfpathlineto{\pgfqpoint{1.738898in}{0.989789in}}%
\pgfpathlineto{\pgfqpoint{1.735781in}{1.020168in}}%
\pgfpathlineto{\pgfqpoint{1.735261in}{1.016786in}}%
\pgfpathlineto{\pgfqpoint{1.733183in}{0.946054in}}%
\pgfpathlineto{\pgfqpoint{1.732664in}{0.951617in}}%
\pgfpathlineto{\pgfqpoint{1.732144in}{0.951287in}}%
\pgfpathlineto{\pgfqpoint{1.731105in}{0.928886in}}%
\pgfpathlineto{\pgfqpoint{1.730585in}{0.940059in}}%
\pgfpathlineto{\pgfqpoint{1.729027in}{1.020056in}}%
\pgfpathlineto{\pgfqpoint{1.724870in}{0.856441in}}%
\pgfpathlineto{\pgfqpoint{1.724351in}{0.857756in}}%
\pgfpathlineto{\pgfqpoint{1.723831in}{0.862994in}}%
\pgfpathlineto{\pgfqpoint{1.723311in}{0.860720in}}%
\pgfpathlineto{\pgfqpoint{1.722792in}{0.857953in}}%
\pgfpathlineto{\pgfqpoint{1.721753in}{0.888503in}}%
\pgfpathlineto{\pgfqpoint{1.719155in}{0.970305in}}%
\pgfpathlineto{\pgfqpoint{1.717077in}{0.849962in}}%
\pgfpathlineto{\pgfqpoint{1.716557in}{0.870590in}}%
\pgfpathlineto{\pgfqpoint{1.716038in}{0.889632in}}%
\pgfpathlineto{\pgfqpoint{1.715518in}{0.883255in}}%
\pgfpathlineto{\pgfqpoint{1.713959in}{0.807897in}}%
\pgfpathlineto{\pgfqpoint{1.713440in}{0.823514in}}%
\pgfpathlineto{\pgfqpoint{1.711881in}{0.913320in}}%
\pgfpathlineto{\pgfqpoint{1.711361in}{0.911326in}}%
\pgfpathlineto{\pgfqpoint{1.709803in}{0.875949in}}%
\pgfpathlineto{\pgfqpoint{1.709283in}{0.881428in}}%
\pgfpathlineto{\pgfqpoint{1.708244in}{0.892984in}}%
\pgfpathlineto{\pgfqpoint{1.707724in}{0.889287in}}%
\pgfpathlineto{\pgfqpoint{1.707205in}{0.884739in}}%
\pgfpathlineto{\pgfqpoint{1.706685in}{0.888427in}}%
\pgfpathlineto{\pgfqpoint{1.703048in}{0.977289in}}%
\pgfpathlineto{\pgfqpoint{1.701490in}{0.933387in}}%
\pgfpathlineto{\pgfqpoint{1.700970in}{0.945083in}}%
\pgfpathlineto{\pgfqpoint{1.699931in}{0.960114in}}%
\pgfpathlineto{\pgfqpoint{1.698892in}{0.946153in}}%
\pgfpathlineto{\pgfqpoint{1.697853in}{0.978418in}}%
\pgfpathlineto{\pgfqpoint{1.696294in}{1.037265in}}%
\pgfpathlineto{\pgfqpoint{1.695255in}{0.996663in}}%
\pgfpathlineto{\pgfqpoint{1.694216in}{0.968765in}}%
\pgfpathlineto{\pgfqpoint{1.692657in}{1.014640in}}%
\pgfpathlineto{\pgfqpoint{1.691618in}{1.002529in}}%
\pgfpathlineto{\pgfqpoint{1.690059in}{1.065971in}}%
\pgfpathlineto{\pgfqpoint{1.689540in}{1.053908in}}%
\pgfpathlineto{\pgfqpoint{1.687981in}{0.980430in}}%
\pgfpathlineto{\pgfqpoint{1.686422in}{1.037173in}}%
\pgfpathlineto{\pgfqpoint{1.685903in}{1.029259in}}%
\pgfpathlineto{\pgfqpoint{1.683824in}{0.959500in}}%
\pgfpathlineto{\pgfqpoint{1.682785in}{1.029643in}}%
\pgfpathlineto{\pgfqpoint{1.681746in}{1.111529in}}%
\pgfpathlineto{\pgfqpoint{1.681227in}{1.111416in}}%
\pgfpathlineto{\pgfqpoint{1.680187in}{1.098569in}}%
\pgfpathlineto{\pgfqpoint{1.679668in}{1.100009in}}%
\pgfpathlineto{\pgfqpoint{1.679148in}{1.100511in}}%
\pgfpathlineto{\pgfqpoint{1.678629in}{1.104437in}}%
\pgfpathlineto{\pgfqpoint{1.676031in}{1.173158in}}%
\pgfpathlineto{\pgfqpoint{1.675511in}{1.172141in}}%
\pgfpathlineto{\pgfqpoint{1.674472in}{1.164147in}}%
\pgfpathlineto{\pgfqpoint{1.673433in}{1.151764in}}%
\pgfpathlineto{\pgfqpoint{1.672394in}{1.176416in}}%
\pgfpathlineto{\pgfqpoint{1.671355in}{1.213005in}}%
\pgfpathlineto{\pgfqpoint{1.670835in}{1.205996in}}%
\pgfpathlineto{\pgfqpoint{1.669796in}{1.176098in}}%
\pgfpathlineto{\pgfqpoint{1.669277in}{1.187101in}}%
\pgfpathlineto{\pgfqpoint{1.668237in}{1.216794in}}%
\pgfpathlineto{\pgfqpoint{1.667718in}{1.212288in}}%
\pgfpathlineto{\pgfqpoint{1.667198in}{1.208003in}}%
\pgfpathlineto{\pgfqpoint{1.666679in}{1.211158in}}%
\pgfpathlineto{\pgfqpoint{1.666159in}{1.211891in}}%
\pgfpathlineto{\pgfqpoint{1.665120in}{1.187981in}}%
\pgfpathlineto{\pgfqpoint{1.664600in}{1.188484in}}%
\pgfpathlineto{\pgfqpoint{1.662003in}{1.259184in}}%
\pgfpathlineto{\pgfqpoint{1.660963in}{1.238616in}}%
\pgfpathlineto{\pgfqpoint{1.660444in}{1.240755in}}%
\pgfpathlineto{\pgfqpoint{1.656287in}{1.417398in}}%
\pgfpathlineto{\pgfqpoint{1.655768in}{1.417993in}}%
\pgfpathlineto{\pgfqpoint{1.654729in}{1.391517in}}%
\pgfpathlineto{\pgfqpoint{1.654209in}{1.397720in}}%
\pgfpathlineto{\pgfqpoint{1.651092in}{1.467768in}}%
\pgfpathlineto{\pgfqpoint{1.649533in}{1.424887in}}%
\pgfpathlineto{\pgfqpoint{1.649013in}{1.433016in}}%
\pgfpathlineto{\pgfqpoint{1.646416in}{1.540448in}}%
\pgfpathlineto{\pgfqpoint{1.645376in}{1.533572in}}%
\pgfpathlineto{\pgfqpoint{1.642259in}{1.492732in}}%
\pgfpathlineto{\pgfqpoint{1.641220in}{1.537146in}}%
\pgfpathlineto{\pgfqpoint{1.638622in}{1.653780in}}%
\pgfpathlineto{\pgfqpoint{1.637063in}{1.695582in}}%
\pgfpathlineto{\pgfqpoint{1.636544in}{1.694227in}}%
\pgfpathlineto{\pgfqpoint{1.636024in}{1.691829in}}%
\pgfpathlineto{\pgfqpoint{1.635505in}{1.695701in}}%
\pgfpathlineto{\pgfqpoint{1.633946in}{1.752025in}}%
\pgfpathlineto{\pgfqpoint{1.633426in}{1.749850in}}%
\pgfpathlineto{\pgfqpoint{1.631868in}{1.659864in}}%
\pgfpathlineto{\pgfqpoint{1.631348in}{1.663340in}}%
\pgfpathlineto{\pgfqpoint{1.625633in}{1.941016in}}%
\pgfpathlineto{\pgfqpoint{1.625113in}{1.957119in}}%
\pgfpathlineto{\pgfqpoint{1.624594in}{1.952477in}}%
\pgfpathlineto{\pgfqpoint{1.621996in}{1.856059in}}%
\pgfpathlineto{\pgfqpoint{1.620437in}{1.907667in}}%
\pgfpathlineto{\pgfqpoint{1.618359in}{2.022996in}}%
\pgfpathlineto{\pgfqpoint{1.617839in}{2.014361in}}%
\pgfpathlineto{\pgfqpoint{1.616800in}{1.988291in}}%
\pgfpathlineto{\pgfqpoint{1.615242in}{2.026959in}}%
\pgfpathlineto{\pgfqpoint{1.614722in}{2.022736in}}%
\pgfpathlineto{\pgfqpoint{1.614202in}{2.015575in}}%
\pgfpathlineto{\pgfqpoint{1.613683in}{2.020135in}}%
\pgfpathlineto{\pgfqpoint{1.611085in}{2.185507in}}%
\pgfpathlineto{\pgfqpoint{1.610566in}{2.169511in}}%
\pgfpathlineto{\pgfqpoint{1.608487in}{2.069545in}}%
\pgfpathlineto{\pgfqpoint{1.605889in}{2.234518in}}%
\pgfpathlineto{\pgfqpoint{1.605370in}{2.219722in}}%
\pgfpathlineto{\pgfqpoint{1.604850in}{2.205934in}}%
\pgfpathlineto{\pgfqpoint{1.604331in}{2.216634in}}%
\pgfpathlineto{\pgfqpoint{1.601733in}{2.373033in}}%
\pgfpathlineto{\pgfqpoint{1.599655in}{2.287794in}}%
\pgfpathlineto{\pgfqpoint{1.599135in}{2.298942in}}%
\pgfpathlineto{\pgfqpoint{1.592381in}{2.560443in}}%
\pgfpathlineto{\pgfqpoint{1.591861in}{2.544408in}}%
\pgfpathlineto{\pgfqpoint{1.590302in}{2.458584in}}%
\pgfpathlineto{\pgfqpoint{1.586665in}{2.626033in}}%
\pgfpathlineto{\pgfqpoint{1.585626in}{2.609879in}}%
\pgfpathlineto{\pgfqpoint{1.583548in}{2.557346in}}%
\pgfpathlineto{\pgfqpoint{1.581470in}{2.643432in}}%
\pgfpathlineto{\pgfqpoint{1.580950in}{2.628725in}}%
\pgfpathlineto{\pgfqpoint{1.580431in}{2.626609in}}%
\pgfpathlineto{\pgfqpoint{1.575755in}{2.871994in}}%
\pgfpathlineto{\pgfqpoint{1.573676in}{2.822587in}}%
\pgfpathlineto{\pgfqpoint{1.573157in}{2.829161in}}%
\pgfpathlineto{\pgfqpoint{1.571078in}{2.858757in}}%
\pgfpathlineto{\pgfqpoint{1.570039in}{2.835301in}}%
\pgfpathlineto{\pgfqpoint{1.569520in}{2.816263in}}%
\pgfpathlineto{\pgfqpoint{1.569000in}{2.816940in}}%
\pgfpathlineto{\pgfqpoint{1.566922in}{2.933403in}}%
\pgfpathlineto{\pgfqpoint{1.566402in}{2.924983in}}%
\pgfpathlineto{\pgfqpoint{1.565883in}{2.918743in}}%
\pgfpathlineto{\pgfqpoint{1.565363in}{2.924260in}}%
\pgfpathlineto{\pgfqpoint{1.563805in}{2.961190in}}%
\pgfpathlineto{\pgfqpoint{1.562246in}{2.923330in}}%
\pgfpathlineto{\pgfqpoint{1.561726in}{2.933539in}}%
\pgfpathlineto{\pgfqpoint{1.558609in}{3.031120in}}%
\pgfpathlineto{\pgfqpoint{1.558089in}{3.031380in}}%
\pgfpathlineto{\pgfqpoint{1.556011in}{3.067139in}}%
\pgfpathlineto{\pgfqpoint{1.553413in}{3.121314in}}%
\pgfpathlineto{\pgfqpoint{1.551335in}{3.247709in}}%
\pgfpathlineto{\pgfqpoint{1.550815in}{3.234732in}}%
\pgfpathlineto{\pgfqpoint{1.547698in}{3.096120in}}%
\pgfpathlineto{\pgfqpoint{1.546139in}{3.125283in}}%
\pgfpathlineto{\pgfqpoint{1.544581in}{3.210231in}}%
\pgfpathlineto{\pgfqpoint{1.543022in}{3.286881in}}%
\pgfpathlineto{\pgfqpoint{1.541463in}{3.231748in}}%
\pgfpathlineto{\pgfqpoint{1.540944in}{3.236471in}}%
\pgfpathlineto{\pgfqpoint{1.540424in}{3.245855in}}%
\pgfpathlineto{\pgfqpoint{1.539904in}{3.244389in}}%
\pgfpathlineto{\pgfqpoint{1.538865in}{3.231469in}}%
\pgfpathlineto{\pgfqpoint{1.537307in}{3.260821in}}%
\pgfpathlineto{\pgfqpoint{1.536268in}{3.242449in}}%
\pgfpathlineto{\pgfqpoint{1.535748in}{3.248526in}}%
\pgfpathlineto{\pgfqpoint{1.533150in}{3.335909in}}%
\pgfpathlineto{\pgfqpoint{1.532631in}{3.371047in}}%
\pgfpathlineto{\pgfqpoint{1.532111in}{3.370059in}}%
\pgfpathlineto{\pgfqpoint{1.530552in}{3.284844in}}%
\pgfpathlineto{\pgfqpoint{1.530033in}{3.301462in}}%
\pgfpathlineto{\pgfqpoint{1.528474in}{3.359996in}}%
\pgfpathlineto{\pgfqpoint{1.527954in}{3.359027in}}%
\pgfpathlineto{\pgfqpoint{1.527435in}{3.351838in}}%
\pgfpathlineto{\pgfqpoint{1.526915in}{3.353961in}}%
\pgfpathlineto{\pgfqpoint{1.525357in}{3.407201in}}%
\pgfpathlineto{\pgfqpoint{1.524837in}{3.391678in}}%
\pgfpathlineto{\pgfqpoint{1.523798in}{3.358707in}}%
\pgfpathlineto{\pgfqpoint{1.523278in}{3.366379in}}%
\pgfpathlineto{\pgfqpoint{1.521200in}{3.455818in}}%
\pgfpathlineto{\pgfqpoint{1.520681in}{3.439641in}}%
\pgfpathlineto{\pgfqpoint{1.519122in}{3.345097in}}%
\pgfpathlineto{\pgfqpoint{1.518602in}{3.355824in}}%
\pgfpathlineto{\pgfqpoint{1.516004in}{3.451212in}}%
\pgfpathlineto{\pgfqpoint{1.514446in}{3.440736in}}%
\pgfpathlineto{\pgfqpoint{1.512367in}{3.493341in}}%
\pgfpathlineto{\pgfqpoint{1.511848in}{3.486162in}}%
\pgfpathlineto{\pgfqpoint{1.510809in}{3.472499in}}%
\pgfpathlineto{\pgfqpoint{1.509770in}{3.480676in}}%
\pgfpathlineto{\pgfqpoint{1.508730in}{3.454721in}}%
\pgfpathlineto{\pgfqpoint{1.507172in}{3.420672in}}%
\pgfpathlineto{\pgfqpoint{1.506133in}{3.420448in}}%
\pgfpathlineto{\pgfqpoint{1.505094in}{3.429194in}}%
\pgfpathlineto{\pgfqpoint{1.500937in}{3.509857in}}%
\pgfpathlineto{\pgfqpoint{1.499898in}{3.501227in}}%
\pgfpathlineto{\pgfqpoint{1.498339in}{3.465856in}}%
\pgfpathlineto{\pgfqpoint{1.497820in}{3.466366in}}%
\pgfpathlineto{\pgfqpoint{1.496261in}{3.531795in}}%
\pgfpathlineto{\pgfqpoint{1.495741in}{3.513572in}}%
\pgfpathlineto{\pgfqpoint{1.494183in}{3.443612in}}%
\pgfpathlineto{\pgfqpoint{1.492104in}{3.511926in}}%
\pgfpathlineto{\pgfqpoint{1.491585in}{3.505459in}}%
\pgfpathlineto{\pgfqpoint{1.490026in}{3.473923in}}%
\pgfpathlineto{\pgfqpoint{1.488987in}{3.436907in}}%
\pgfpathlineto{\pgfqpoint{1.488467in}{3.436970in}}%
\pgfpathlineto{\pgfqpoint{1.485870in}{3.502393in}}%
\pgfpathlineto{\pgfqpoint{1.485350in}{3.496206in}}%
\pgfpathlineto{\pgfqpoint{1.482752in}{3.382287in}}%
\pgfpathlineto{\pgfqpoint{1.482233in}{3.388260in}}%
\pgfpathlineto{\pgfqpoint{1.479635in}{3.449894in}}%
\pgfpathlineto{\pgfqpoint{1.479115in}{3.471114in}}%
\pgfpathlineto{\pgfqpoint{1.478596in}{3.469916in}}%
\pgfpathlineto{\pgfqpoint{1.475998in}{3.371922in}}%
\pgfpathlineto{\pgfqpoint{1.475478in}{3.378018in}}%
\pgfpathlineto{\pgfqpoint{1.474959in}{3.386146in}}%
\pgfpathlineto{\pgfqpoint{1.474439in}{3.385789in}}%
\pgfpathlineto{\pgfqpoint{1.471322in}{3.343429in}}%
\pgfpathlineto{\pgfqpoint{1.470283in}{3.371786in}}%
\pgfpathlineto{\pgfqpoint{1.469763in}{3.385769in}}%
\pgfpathlineto{\pgfqpoint{1.469243in}{3.371549in}}%
\pgfpathlineto{\pgfqpoint{1.467685in}{3.291084in}}%
\pgfpathlineto{\pgfqpoint{1.466126in}{3.364867in}}%
\pgfpathlineto{\pgfqpoint{1.465606in}{3.351806in}}%
\pgfpathlineto{\pgfqpoint{1.463009in}{3.295489in}}%
\pgfpathlineto{\pgfqpoint{1.461450in}{3.257478in}}%
\pgfpathlineto{\pgfqpoint{1.458852in}{3.175415in}}%
\pgfpathlineto{\pgfqpoint{1.457813in}{3.190520in}}%
\pgfpathlineto{\pgfqpoint{1.455215in}{3.252799in}}%
\pgfpathlineto{\pgfqpoint{1.454696in}{3.252988in}}%
\pgfpathlineto{\pgfqpoint{1.454176in}{3.251329in}}%
\pgfpathlineto{\pgfqpoint{1.453137in}{3.227489in}}%
\pgfpathlineto{\pgfqpoint{1.451578in}{3.154066in}}%
\pgfpathlineto{\pgfqpoint{1.451059in}{3.164238in}}%
\pgfpathlineto{\pgfqpoint{1.450019in}{3.185892in}}%
\pgfpathlineto{\pgfqpoint{1.448461in}{3.141183in}}%
\pgfpathlineto{\pgfqpoint{1.447941in}{3.146689in}}%
\pgfpathlineto{\pgfqpoint{1.447422in}{3.150031in}}%
\pgfpathlineto{\pgfqpoint{1.446383in}{3.116958in}}%
\pgfpathlineto{\pgfqpoint{1.443785in}{3.008086in}}%
\pgfpathlineto{\pgfqpoint{1.443265in}{3.009600in}}%
\pgfpathlineto{\pgfqpoint{1.442746in}{3.012990in}}%
\pgfpathlineto{\pgfqpoint{1.441706in}{2.975001in}}%
\pgfpathlineto{\pgfqpoint{1.441187in}{2.951929in}}%
\pgfpathlineto{\pgfqpoint{1.440667in}{2.953053in}}%
\pgfpathlineto{\pgfqpoint{1.439109in}{2.996793in}}%
\pgfpathlineto{\pgfqpoint{1.438069in}{2.973270in}}%
\pgfpathlineto{\pgfqpoint{1.437550in}{2.976827in}}%
\pgfpathlineto{\pgfqpoint{1.437030in}{2.982340in}}%
\pgfpathlineto{\pgfqpoint{1.436511in}{2.977069in}}%
\pgfpathlineto{\pgfqpoint{1.432354in}{2.860055in}}%
\pgfpathlineto{\pgfqpoint{1.431835in}{2.865495in}}%
\pgfpathlineto{\pgfqpoint{1.430276in}{2.908249in}}%
\pgfpathlineto{\pgfqpoint{1.429756in}{2.900081in}}%
\pgfpathlineto{\pgfqpoint{1.425600in}{2.751227in}}%
\pgfpathlineto{\pgfqpoint{1.425080in}{2.751306in}}%
\pgfpathlineto{\pgfqpoint{1.423002in}{2.785667in}}%
\pgfpathlineto{\pgfqpoint{1.422482in}{2.783572in}}%
\pgfpathlineto{\pgfqpoint{1.420924in}{2.741688in}}%
\pgfpathlineto{\pgfqpoint{1.411572in}{2.473866in}}%
\pgfpathlineto{\pgfqpoint{1.411052in}{2.481543in}}%
\pgfpathlineto{\pgfqpoint{1.410013in}{2.512976in}}%
\pgfpathlineto{\pgfqpoint{1.409493in}{2.502101in}}%
\pgfpathlineto{\pgfqpoint{1.408454in}{2.477683in}}%
\pgfpathlineto{\pgfqpoint{1.407935in}{2.479184in}}%
\pgfpathlineto{\pgfqpoint{1.407415in}{2.473652in}}%
\pgfpathlineto{\pgfqpoint{1.405856in}{2.427625in}}%
\pgfpathlineto{\pgfqpoint{1.405337in}{2.429678in}}%
\pgfpathlineto{\pgfqpoint{1.404817in}{2.433928in}}%
\pgfpathlineto{\pgfqpoint{1.404298in}{2.430549in}}%
\pgfpathlineto{\pgfqpoint{1.400141in}{2.332098in}}%
\pgfpathlineto{\pgfqpoint{1.397543in}{2.359185in}}%
\pgfpathlineto{\pgfqpoint{1.396504in}{2.294683in}}%
\pgfpathlineto{\pgfqpoint{1.394426in}{2.167842in}}%
\pgfpathlineto{\pgfqpoint{1.393906in}{2.166579in}}%
\pgfpathlineto{\pgfqpoint{1.392348in}{2.195173in}}%
\pgfpathlineto{\pgfqpoint{1.391828in}{2.185962in}}%
\pgfpathlineto{\pgfqpoint{1.390789in}{2.160113in}}%
\pgfpathlineto{\pgfqpoint{1.390269in}{2.162692in}}%
\pgfpathlineto{\pgfqpoint{1.389230in}{2.185021in}}%
\pgfpathlineto{\pgfqpoint{1.388711in}{2.181831in}}%
\pgfpathlineto{\pgfqpoint{1.387152in}{2.079991in}}%
\pgfpathlineto{\pgfqpoint{1.385593in}{1.992464in}}%
\pgfpathlineto{\pgfqpoint{1.385074in}{1.994930in}}%
\pgfpathlineto{\pgfqpoint{1.383515in}{2.022096in}}%
\pgfpathlineto{\pgfqpoint{1.382476in}{1.989097in}}%
\pgfpathlineto{\pgfqpoint{1.381437in}{1.940261in}}%
\pgfpathlineto{\pgfqpoint{1.380917in}{1.941300in}}%
\pgfpathlineto{\pgfqpoint{1.379878in}{1.967859in}}%
\pgfpathlineto{\pgfqpoint{1.379358in}{1.963673in}}%
\pgfpathlineto{\pgfqpoint{1.376761in}{1.884818in}}%
\pgfpathlineto{\pgfqpoint{1.375202in}{1.846252in}}%
\pgfpathlineto{\pgfqpoint{1.374682in}{1.847709in}}%
\pgfpathlineto{\pgfqpoint{1.373643in}{1.863044in}}%
\pgfpathlineto{\pgfqpoint{1.373124in}{1.857143in}}%
\pgfpathlineto{\pgfqpoint{1.366369in}{1.680270in}}%
\pgfpathlineto{\pgfqpoint{1.365850in}{1.682169in}}%
\pgfpathlineto{\pgfqpoint{1.363771in}{1.687569in}}%
\pgfpathlineto{\pgfqpoint{1.362732in}{1.679460in}}%
\pgfpathlineto{\pgfqpoint{1.362213in}{1.682101in}}%
\pgfpathlineto{\pgfqpoint{1.361174in}{1.705845in}}%
\pgfpathlineto{\pgfqpoint{1.360134in}{1.635795in}}%
\pgfpathlineto{\pgfqpoint{1.358576in}{1.559139in}}%
\pgfpathlineto{\pgfqpoint{1.355978in}{1.568168in}}%
\pgfpathlineto{\pgfqpoint{1.354939in}{1.560218in}}%
\pgfpathlineto{\pgfqpoint{1.353900in}{1.526125in}}%
\pgfpathlineto{\pgfqpoint{1.351302in}{1.393487in}}%
\pgfpathlineto{\pgfqpoint{1.350263in}{1.402790in}}%
\pgfpathlineto{\pgfqpoint{1.348704in}{1.424944in}}%
\pgfpathlineto{\pgfqpoint{1.348184in}{1.423525in}}%
\pgfpathlineto{\pgfqpoint{1.346106in}{1.409255in}}%
\pgfpathlineto{\pgfqpoint{1.345587in}{1.414270in}}%
\pgfpathlineto{\pgfqpoint{1.344547in}{1.434938in}}%
\pgfpathlineto{\pgfqpoint{1.344028in}{1.425939in}}%
\pgfpathlineto{\pgfqpoint{1.341950in}{1.350483in}}%
\pgfpathlineto{\pgfqpoint{1.341430in}{1.352588in}}%
\pgfpathlineto{\pgfqpoint{1.340911in}{1.347176in}}%
\pgfpathlineto{\pgfqpoint{1.338313in}{1.230463in}}%
\pgfpathlineto{\pgfqpoint{1.337793in}{1.246111in}}%
\pgfpathlineto{\pgfqpoint{1.336754in}{1.282302in}}%
\pgfpathlineto{\pgfqpoint{1.336234in}{1.280921in}}%
\pgfpathlineto{\pgfqpoint{1.333117in}{1.211855in}}%
\pgfpathlineto{\pgfqpoint{1.331039in}{1.196795in}}%
\pgfpathlineto{\pgfqpoint{1.329480in}{1.136742in}}%
\pgfpathlineto{\pgfqpoint{1.328441in}{1.159444in}}%
\pgfpathlineto{\pgfqpoint{1.327921in}{1.157777in}}%
\pgfpathlineto{\pgfqpoint{1.325324in}{1.064115in}}%
\pgfpathlineto{\pgfqpoint{1.324804in}{1.076575in}}%
\pgfpathlineto{\pgfqpoint{1.322726in}{1.118307in}}%
\pgfpathlineto{\pgfqpoint{1.322206in}{1.120122in}}%
\pgfpathlineto{\pgfqpoint{1.321167in}{1.085361in}}%
\pgfpathlineto{\pgfqpoint{1.319608in}{1.007604in}}%
\pgfpathlineto{\pgfqpoint{1.319089in}{1.010425in}}%
\pgfpathlineto{\pgfqpoint{1.316491in}{1.055689in}}%
\pgfpathlineto{\pgfqpoint{1.314413in}{1.090825in}}%
\pgfpathlineto{\pgfqpoint{1.313893in}{1.087590in}}%
\pgfpathlineto{\pgfqpoint{1.312334in}{0.970715in}}%
\pgfpathlineto{\pgfqpoint{1.310776in}{0.882537in}}%
\pgfpathlineto{\pgfqpoint{1.310256in}{0.888734in}}%
\pgfpathlineto{\pgfqpoint{1.307658in}{0.990446in}}%
\pgfpathlineto{\pgfqpoint{1.307139in}{0.978161in}}%
\pgfpathlineto{\pgfqpoint{1.306100in}{0.956278in}}%
\pgfpathlineto{\pgfqpoint{1.305580in}{0.958384in}}%
\pgfpathlineto{\pgfqpoint{1.305060in}{0.955206in}}%
\pgfpathlineto{\pgfqpoint{1.303502in}{0.893487in}}%
\pgfpathlineto{\pgfqpoint{1.302982in}{0.893997in}}%
\pgfpathlineto{\pgfqpoint{1.299865in}{0.958106in}}%
\pgfpathlineto{\pgfqpoint{1.298826in}{0.912750in}}%
\pgfpathlineto{\pgfqpoint{1.297267in}{0.859448in}}%
\pgfpathlineto{\pgfqpoint{1.294669in}{0.912424in}}%
\pgfpathlineto{\pgfqpoint{1.293110in}{0.951546in}}%
\pgfpathlineto{\pgfqpoint{1.292071in}{0.913702in}}%
\pgfpathlineto{\pgfqpoint{1.290513in}{0.854967in}}%
\pgfpathlineto{\pgfqpoint{1.289473in}{0.867130in}}%
\pgfpathlineto{\pgfqpoint{1.287395in}{0.931765in}}%
\pgfpathlineto{\pgfqpoint{1.286876in}{0.913681in}}%
\pgfpathlineto{\pgfqpoint{1.285836in}{0.870844in}}%
\pgfpathlineto{\pgfqpoint{1.283758in}{0.975148in}}%
\pgfpathlineto{\pgfqpoint{1.283239in}{0.962765in}}%
\pgfpathlineto{\pgfqpoint{1.281160in}{0.925751in}}%
\pgfpathlineto{\pgfqpoint{1.280641in}{0.919639in}}%
\pgfpathlineto{\pgfqpoint{1.278563in}{0.843534in}}%
\pgfpathlineto{\pgfqpoint{1.278043in}{0.858199in}}%
\pgfpathlineto{\pgfqpoint{1.277523in}{0.871352in}}%
\pgfpathlineto{\pgfqpoint{1.277004in}{0.869047in}}%
\pgfpathlineto{\pgfqpoint{1.275965in}{0.853518in}}%
\pgfpathlineto{\pgfqpoint{1.273886in}{0.933995in}}%
\pgfpathlineto{\pgfqpoint{1.273367in}{0.922549in}}%
\pgfpathlineto{\pgfqpoint{1.271289in}{0.880091in}}%
\pgfpathlineto{\pgfqpoint{1.270249in}{0.902844in}}%
\pgfpathlineto{\pgfqpoint{1.269210in}{0.934874in}}%
\pgfpathlineto{\pgfqpoint{1.268691in}{0.930473in}}%
\pgfpathlineto{\pgfqpoint{1.265573in}{0.867482in}}%
\pgfpathlineto{\pgfqpoint{1.265054in}{0.872906in}}%
\pgfpathlineto{\pgfqpoint{1.263495in}{0.952632in}}%
\pgfpathlineto{\pgfqpoint{1.262976in}{0.948438in}}%
\pgfpathlineto{\pgfqpoint{1.261936in}{0.923194in}}%
\pgfpathlineto{\pgfqpoint{1.261417in}{0.927290in}}%
\pgfpathlineto{\pgfqpoint{1.260378in}{0.957144in}}%
\pgfpathlineto{\pgfqpoint{1.259858in}{0.954694in}}%
\pgfpathlineto{\pgfqpoint{1.258299in}{0.924775in}}%
\pgfpathlineto{\pgfqpoint{1.256741in}{0.972432in}}%
\pgfpathlineto{\pgfqpoint{1.254143in}{1.052988in}}%
\pgfpathlineto{\pgfqpoint{1.253623in}{1.049146in}}%
\pgfpathlineto{\pgfqpoint{1.252584in}{1.024201in}}%
\pgfpathlineto{\pgfqpoint{1.252065in}{1.025486in}}%
\pgfpathlineto{\pgfqpoint{1.251026in}{1.038558in}}%
\pgfpathlineto{\pgfqpoint{1.250506in}{1.034985in}}%
\pgfpathlineto{\pgfqpoint{1.247908in}{1.022086in}}%
\pgfpathlineto{\pgfqpoint{1.247389in}{1.023381in}}%
\pgfpathlineto{\pgfqpoint{1.246869in}{1.022628in}}%
\pgfpathlineto{\pgfqpoint{1.245830in}{1.018116in}}%
\pgfpathlineto{\pgfqpoint{1.244791in}{1.043530in}}%
\pgfpathlineto{\pgfqpoint{1.243232in}{1.105738in}}%
\pgfpathlineto{\pgfqpoint{1.241673in}{1.031861in}}%
\pgfpathlineto{\pgfqpoint{1.241154in}{1.041767in}}%
\pgfpathlineto{\pgfqpoint{1.238036in}{1.168836in}}%
\pgfpathlineto{\pgfqpoint{1.236997in}{1.186892in}}%
\pgfpathlineto{\pgfqpoint{1.235439in}{1.117126in}}%
\pgfpathlineto{\pgfqpoint{1.234919in}{1.117684in}}%
\pgfpathlineto{\pgfqpoint{1.231282in}{1.219348in}}%
\pgfpathlineto{\pgfqpoint{1.229723in}{1.296472in}}%
\pgfpathlineto{\pgfqpoint{1.229204in}{1.291716in}}%
\pgfpathlineto{\pgfqpoint{1.227645in}{1.228329in}}%
\pgfpathlineto{\pgfqpoint{1.227125in}{1.230575in}}%
\pgfpathlineto{\pgfqpoint{1.226606in}{1.236870in}}%
\pgfpathlineto{\pgfqpoint{1.226086in}{1.235966in}}%
\pgfpathlineto{\pgfqpoint{1.225047in}{1.222173in}}%
\pgfpathlineto{\pgfqpoint{1.224528in}{1.227346in}}%
\pgfpathlineto{\pgfqpoint{1.221930in}{1.312676in}}%
\pgfpathlineto{\pgfqpoint{1.220891in}{1.371685in}}%
\pgfpathlineto{\pgfqpoint{1.220371in}{1.365715in}}%
\pgfpathlineto{\pgfqpoint{1.219332in}{1.330829in}}%
\pgfpathlineto{\pgfqpoint{1.218293in}{1.380047in}}%
\pgfpathlineto{\pgfqpoint{1.216215in}{1.462292in}}%
\pgfpathlineto{\pgfqpoint{1.215695in}{1.461834in}}%
\pgfpathlineto{\pgfqpoint{1.214136in}{1.434370in}}%
\pgfpathlineto{\pgfqpoint{1.213617in}{1.439637in}}%
\pgfpathlineto{\pgfqpoint{1.211538in}{1.499552in}}%
\pgfpathlineto{\pgfqpoint{1.209980in}{1.546909in}}%
\pgfpathlineto{\pgfqpoint{1.208421in}{1.508789in}}%
\pgfpathlineto{\pgfqpoint{1.205823in}{1.626893in}}%
\pgfpathlineto{\pgfqpoint{1.205304in}{1.624496in}}%
\pgfpathlineto{\pgfqpoint{1.204784in}{1.626022in}}%
\pgfpathlineto{\pgfqpoint{1.203745in}{1.638249in}}%
\pgfpathlineto{\pgfqpoint{1.203225in}{1.635002in}}%
\pgfpathlineto{\pgfqpoint{1.202186in}{1.619498in}}%
\pgfpathlineto{\pgfqpoint{1.201147in}{1.645277in}}%
\pgfpathlineto{\pgfqpoint{1.200108in}{1.670938in}}%
\pgfpathlineto{\pgfqpoint{1.199588in}{1.668015in}}%
\pgfpathlineto{\pgfqpoint{1.199069in}{1.660938in}}%
\pgfpathlineto{\pgfqpoint{1.198549in}{1.661394in}}%
\pgfpathlineto{\pgfqpoint{1.191275in}{1.834503in}}%
\pgfpathlineto{\pgfqpoint{1.190236in}{1.831184in}}%
\pgfpathlineto{\pgfqpoint{1.188678in}{1.859255in}}%
\pgfpathlineto{\pgfqpoint{1.187638in}{1.885643in}}%
\pgfpathlineto{\pgfqpoint{1.185560in}{1.978667in}}%
\pgfpathlineto{\pgfqpoint{1.184001in}{1.952437in}}%
\pgfpathlineto{\pgfqpoint{1.183482in}{1.952713in}}%
\pgfpathlineto{\pgfqpoint{1.182443in}{1.948623in}}%
\pgfpathlineto{\pgfqpoint{1.181923in}{1.952013in}}%
\pgfpathlineto{\pgfqpoint{1.177247in}{2.074072in}}%
\pgfpathlineto{\pgfqpoint{1.176727in}{2.078057in}}%
\pgfpathlineto{\pgfqpoint{1.175688in}{2.064074in}}%
\pgfpathlineto{\pgfqpoint{1.175169in}{2.072391in}}%
\pgfpathlineto{\pgfqpoint{1.171532in}{2.288315in}}%
\pgfpathlineto{\pgfqpoint{1.170493in}{2.280365in}}%
\pgfpathlineto{\pgfqpoint{1.168414in}{2.265385in}}%
\pgfpathlineto{\pgfqpoint{1.165297in}{2.327789in}}%
\pgfpathlineto{\pgfqpoint{1.164777in}{2.344134in}}%
\pgfpathlineto{\pgfqpoint{1.164258in}{2.342601in}}%
\pgfpathlineto{\pgfqpoint{1.163219in}{2.326073in}}%
\pgfpathlineto{\pgfqpoint{1.162180in}{2.358990in}}%
\pgfpathlineto{\pgfqpoint{1.158543in}{2.479145in}}%
\pgfpathlineto{\pgfqpoint{1.156984in}{2.414362in}}%
\pgfpathlineto{\pgfqpoint{1.156464in}{2.428252in}}%
\pgfpathlineto{\pgfqpoint{1.153347in}{2.607788in}}%
\pgfpathlineto{\pgfqpoint{1.152827in}{2.609917in}}%
\pgfpathlineto{\pgfqpoint{1.150749in}{2.535955in}}%
\pgfpathlineto{\pgfqpoint{1.150230in}{2.539190in}}%
\pgfpathlineto{\pgfqpoint{1.148671in}{2.545320in}}%
\pgfpathlineto{\pgfqpoint{1.148151in}{2.546264in}}%
\pgfpathlineto{\pgfqpoint{1.147112in}{2.575571in}}%
\pgfpathlineto{\pgfqpoint{1.143475in}{2.719923in}}%
\pgfpathlineto{\pgfqpoint{1.142956in}{2.719396in}}%
\pgfpathlineto{\pgfqpoint{1.142436in}{2.723440in}}%
\pgfpathlineto{\pgfqpoint{1.137760in}{2.825440in}}%
\pgfpathlineto{\pgfqpoint{1.136201in}{2.799107in}}%
\pgfpathlineto{\pgfqpoint{1.135682in}{2.809453in}}%
\pgfpathlineto{\pgfqpoint{1.132045in}{2.982040in}}%
\pgfpathlineto{\pgfqpoint{1.131006in}{2.936287in}}%
\pgfpathlineto{\pgfqpoint{1.129967in}{2.904042in}}%
\pgfpathlineto{\pgfqpoint{1.125810in}{3.041492in}}%
\pgfpathlineto{\pgfqpoint{1.125290in}{3.049711in}}%
\pgfpathlineto{\pgfqpoint{1.124251in}{3.008363in}}%
\pgfpathlineto{\pgfqpoint{1.123212in}{2.972476in}}%
\pgfpathlineto{\pgfqpoint{1.114899in}{3.205669in}}%
\pgfpathlineto{\pgfqpoint{1.112821in}{3.155818in}}%
\pgfpathlineto{\pgfqpoint{1.112301in}{3.169170in}}%
\pgfpathlineto{\pgfqpoint{1.110743in}{3.219858in}}%
\pgfpathlineto{\pgfqpoint{1.109184in}{3.180598in}}%
\pgfpathlineto{\pgfqpoint{1.107625in}{3.259912in}}%
\pgfpathlineto{\pgfqpoint{1.105547in}{3.331116in}}%
\pgfpathlineto{\pgfqpoint{1.105027in}{3.332846in}}%
\pgfpathlineto{\pgfqpoint{1.104508in}{3.328251in}}%
\pgfpathlineto{\pgfqpoint{1.102429in}{3.278542in}}%
\pgfpathlineto{\pgfqpoint{1.099832in}{3.314081in}}%
\pgfpathlineto{\pgfqpoint{1.097234in}{3.383937in}}%
\pgfpathlineto{\pgfqpoint{1.095675in}{3.428119in}}%
\pgfpathlineto{\pgfqpoint{1.094116in}{3.337075in}}%
\pgfpathlineto{\pgfqpoint{1.093597in}{3.352974in}}%
\pgfpathlineto{\pgfqpoint{1.092038in}{3.431469in}}%
\pgfpathlineto{\pgfqpoint{1.090999in}{3.414637in}}%
\pgfpathlineto{\pgfqpoint{1.090479in}{3.419031in}}%
\pgfpathlineto{\pgfqpoint{1.089440in}{3.428581in}}%
\pgfpathlineto{\pgfqpoint{1.088401in}{3.407344in}}%
\pgfpathlineto{\pgfqpoint{1.086842in}{3.356984in}}%
\pgfpathlineto{\pgfqpoint{1.086323in}{3.364232in}}%
\pgfpathlineto{\pgfqpoint{1.081127in}{3.516490in}}%
\pgfpathlineto{\pgfqpoint{1.080088in}{3.529984in}}%
\pgfpathlineto{\pgfqpoint{1.078529in}{3.482244in}}%
\pgfpathlineto{\pgfqpoint{1.078010in}{3.485748in}}%
\pgfpathlineto{\pgfqpoint{1.076971in}{3.513675in}}%
\pgfpathlineto{\pgfqpoint{1.076451in}{3.503844in}}%
\pgfpathlineto{\pgfqpoint{1.075412in}{3.467260in}}%
\pgfpathlineto{\pgfqpoint{1.074892in}{3.475644in}}%
\pgfpathlineto{\pgfqpoint{1.072814in}{3.594051in}}%
\pgfpathlineto{\pgfqpoint{1.072295in}{3.586552in}}%
\pgfpathlineto{\pgfqpoint{1.070216in}{3.494854in}}%
\pgfpathlineto{\pgfqpoint{1.069697in}{3.499726in}}%
\pgfpathlineto{\pgfqpoint{1.068138in}{3.536087in}}%
\pgfpathlineto{\pgfqpoint{1.067099in}{3.517262in}}%
\pgfpathlineto{\pgfqpoint{1.066579in}{3.517555in}}%
\pgfpathlineto{\pgfqpoint{1.066060in}{3.519421in}}%
\pgfpathlineto{\pgfqpoint{1.064501in}{3.492981in}}%
\pgfpathlineto{\pgfqpoint{1.063462in}{3.498842in}}%
\pgfpathlineto{\pgfqpoint{1.060345in}{3.520743in}}%
\pgfpathlineto{\pgfqpoint{1.059825in}{3.515702in}}%
\pgfpathlineto{\pgfqpoint{1.058266in}{3.493225in}}%
\pgfpathlineto{\pgfqpoint{1.057747in}{3.496469in}}%
\pgfpathlineto{\pgfqpoint{1.057227in}{3.495191in}}%
\pgfpathlineto{\pgfqpoint{1.055669in}{3.478417in}}%
\pgfpathlineto{\pgfqpoint{1.055149in}{3.479372in}}%
\pgfpathlineto{\pgfqpoint{1.054110in}{3.487485in}}%
\pgfpathlineto{\pgfqpoint{1.053590in}{3.493681in}}%
\pgfpathlineto{\pgfqpoint{1.053071in}{3.487872in}}%
\pgfpathlineto{\pgfqpoint{1.052032in}{3.457127in}}%
\pgfpathlineto{\pgfqpoint{1.051512in}{3.470807in}}%
\pgfpathlineto{\pgfqpoint{1.050473in}{3.500002in}}%
\pgfpathlineto{\pgfqpoint{1.048395in}{3.367866in}}%
\pgfpathlineto{\pgfqpoint{1.047875in}{3.391043in}}%
\pgfpathlineto{\pgfqpoint{1.045797in}{3.474861in}}%
\pgfpathlineto{\pgfqpoint{1.041640in}{3.422259in}}%
\pgfpathlineto{\pgfqpoint{1.038523in}{3.473216in}}%
\pgfpathlineto{\pgfqpoint{1.038003in}{3.475830in}}%
\pgfpathlineto{\pgfqpoint{1.034366in}{3.420444in}}%
\pgfpathlineto{\pgfqpoint{1.033327in}{3.429255in}}%
\pgfpathlineto{\pgfqpoint{1.032808in}{3.423010in}}%
\pgfpathlineto{\pgfqpoint{1.031249in}{3.344001in}}%
\pgfpathlineto{\pgfqpoint{1.030729in}{3.344652in}}%
\pgfpathlineto{\pgfqpoint{1.029690in}{3.355866in}}%
\pgfpathlineto{\pgfqpoint{1.028651in}{3.342567in}}%
\pgfpathlineto{\pgfqpoint{1.028131in}{3.348449in}}%
\pgfpathlineto{\pgfqpoint{1.027092in}{3.373408in}}%
\pgfpathlineto{\pgfqpoint{1.026573in}{3.364615in}}%
\pgfpathlineto{\pgfqpoint{1.023975in}{3.245007in}}%
\pgfpathlineto{\pgfqpoint{1.023975in}{3.245007in}}%
\pgfusepath{stroke}%
\end{pgfscope}%
\begin{pgfscope}%
\pgfpathrectangle{\pgfqpoint{0.795366in}{0.646140in}}{\pgfqpoint{5.029404in}{3.088289in}}%
\pgfusepath{clip}%
\pgfsetbuttcap%
\pgfsetroundjoin%
\definecolor{currentfill}{rgb}{1.000000,0.000000,0.000000}%
\pgfsetfillcolor{currentfill}%
\pgfsetlinewidth{1.003750pt}%
\definecolor{currentstroke}{rgb}{1.000000,0.000000,0.000000}%
\pgfsetstrokecolor{currentstroke}%
\pgfsetdash{}{0pt}%
\pgfsys@defobject{currentmarker}{\pgfqpoint{-0.027778in}{-0.027778in}}{\pgfqpoint{0.027778in}{0.027778in}}{%
\pgfpathmoveto{\pgfqpoint{0.000000in}{-0.027778in}}%
\pgfpathcurveto{\pgfqpoint{0.007367in}{-0.027778in}}{\pgfqpoint{0.014433in}{-0.024851in}}{\pgfqpoint{0.019642in}{-0.019642in}}%
\pgfpathcurveto{\pgfqpoint{0.024851in}{-0.014433in}}{\pgfqpoint{0.027778in}{-0.007367in}}{\pgfqpoint{0.027778in}{0.000000in}}%
\pgfpathcurveto{\pgfqpoint{0.027778in}{0.007367in}}{\pgfqpoint{0.024851in}{0.014433in}}{\pgfqpoint{0.019642in}{0.019642in}}%
\pgfpathcurveto{\pgfqpoint{0.014433in}{0.024851in}}{\pgfqpoint{0.007367in}{0.027778in}}{\pgfqpoint{0.000000in}{0.027778in}}%
\pgfpathcurveto{\pgfqpoint{-0.007367in}{0.027778in}}{\pgfqpoint{-0.014433in}{0.024851in}}{\pgfqpoint{-0.019642in}{0.019642in}}%
\pgfpathcurveto{\pgfqpoint{-0.024851in}{0.014433in}}{\pgfqpoint{-0.027778in}{0.007367in}}{\pgfqpoint{-0.027778in}{0.000000in}}%
\pgfpathcurveto{\pgfqpoint{-0.027778in}{-0.007367in}}{\pgfqpoint{-0.024851in}{-0.014433in}}{\pgfqpoint{-0.019642in}{-0.019642in}}%
\pgfpathcurveto{\pgfqpoint{-0.014433in}{-0.024851in}}{\pgfqpoint{-0.007367in}{-0.027778in}}{\pgfqpoint{0.000000in}{-0.027778in}}%
\pgfpathlineto{\pgfqpoint{0.000000in}{-0.027778in}}%
\pgfpathclose%
\pgfusepath{stroke,fill}%
}%
\begin{pgfscope}%
\pgfsys@transformshift{1.278563in}{0.844303in}%
\pgfsys@useobject{currentmarker}{}%
\end{pgfscope}%
\end{pgfscope}%
\begin{pgfscope}%
\pgfpathrectangle{\pgfqpoint{0.795366in}{0.646140in}}{\pgfqpoint{5.029404in}{3.088289in}}%
\pgfusepath{clip}%
\pgfsetbuttcap%
\pgfsetroundjoin%
\definecolor{currentfill}{rgb}{1.000000,0.000000,0.000000}%
\pgfsetfillcolor{currentfill}%
\pgfsetlinewidth{1.003750pt}%
\definecolor{currentstroke}{rgb}{1.000000,0.000000,0.000000}%
\pgfsetstrokecolor{currentstroke}%
\pgfsetdash{}{0pt}%
\pgfsys@defobject{currentmarker}{\pgfqpoint{-0.027778in}{-0.027778in}}{\pgfqpoint{0.027778in}{0.027778in}}{%
\pgfpathmoveto{\pgfqpoint{0.000000in}{-0.027778in}}%
\pgfpathcurveto{\pgfqpoint{0.007367in}{-0.027778in}}{\pgfqpoint{0.014433in}{-0.024851in}}{\pgfqpoint{0.019642in}{-0.019642in}}%
\pgfpathcurveto{\pgfqpoint{0.024851in}{-0.014433in}}{\pgfqpoint{0.027778in}{-0.007367in}}{\pgfqpoint{0.027778in}{0.000000in}}%
\pgfpathcurveto{\pgfqpoint{0.027778in}{0.007367in}}{\pgfqpoint{0.024851in}{0.014433in}}{\pgfqpoint{0.019642in}{0.019642in}}%
\pgfpathcurveto{\pgfqpoint{0.014433in}{0.024851in}}{\pgfqpoint{0.007367in}{0.027778in}}{\pgfqpoint{0.000000in}{0.027778in}}%
\pgfpathcurveto{\pgfqpoint{-0.007367in}{0.027778in}}{\pgfqpoint{-0.014433in}{0.024851in}}{\pgfqpoint{-0.019642in}{0.019642in}}%
\pgfpathcurveto{\pgfqpoint{-0.024851in}{0.014433in}}{\pgfqpoint{-0.027778in}{0.007367in}}{\pgfqpoint{-0.027778in}{0.000000in}}%
\pgfpathcurveto{\pgfqpoint{-0.027778in}{-0.007367in}}{\pgfqpoint{-0.024851in}{-0.014433in}}{\pgfqpoint{-0.019642in}{-0.019642in}}%
\pgfpathcurveto{\pgfqpoint{-0.014433in}{-0.024851in}}{\pgfqpoint{-0.007367in}{-0.027778in}}{\pgfqpoint{0.000000in}{-0.027778in}}%
\pgfpathlineto{\pgfqpoint{0.000000in}{-0.027778in}}%
\pgfpathclose%
\pgfusepath{stroke,fill}%
}%
\begin{pgfscope}%
\pgfsys@transformshift{1.713959in}{0.805679in}%
\pgfsys@useobject{currentmarker}{}%
\end{pgfscope}%
\end{pgfscope}%
\begin{pgfscope}%
\pgfpathrectangle{\pgfqpoint{0.795366in}{0.646140in}}{\pgfqpoint{5.029404in}{3.088289in}}%
\pgfusepath{clip}%
\pgfsetbuttcap%
\pgfsetroundjoin%
\definecolor{currentfill}{rgb}{1.000000,0.000000,0.000000}%
\pgfsetfillcolor{currentfill}%
\pgfsetlinewidth{1.003750pt}%
\definecolor{currentstroke}{rgb}{1.000000,0.000000,0.000000}%
\pgfsetstrokecolor{currentstroke}%
\pgfsetdash{}{0pt}%
\pgfsys@defobject{currentmarker}{\pgfqpoint{-0.027778in}{-0.027778in}}{\pgfqpoint{0.027778in}{0.027778in}}{%
\pgfpathmoveto{\pgfqpoint{0.000000in}{-0.027778in}}%
\pgfpathcurveto{\pgfqpoint{0.007367in}{-0.027778in}}{\pgfqpoint{0.014433in}{-0.024851in}}{\pgfqpoint{0.019642in}{-0.019642in}}%
\pgfpathcurveto{\pgfqpoint{0.024851in}{-0.014433in}}{\pgfqpoint{0.027778in}{-0.007367in}}{\pgfqpoint{0.027778in}{0.000000in}}%
\pgfpathcurveto{\pgfqpoint{0.027778in}{0.007367in}}{\pgfqpoint{0.024851in}{0.014433in}}{\pgfqpoint{0.019642in}{0.019642in}}%
\pgfpathcurveto{\pgfqpoint{0.014433in}{0.024851in}}{\pgfqpoint{0.007367in}{0.027778in}}{\pgfqpoint{0.000000in}{0.027778in}}%
\pgfpathcurveto{\pgfqpoint{-0.007367in}{0.027778in}}{\pgfqpoint{-0.014433in}{0.024851in}}{\pgfqpoint{-0.019642in}{0.019642in}}%
\pgfpathcurveto{\pgfqpoint{-0.024851in}{0.014433in}}{\pgfqpoint{-0.027778in}{0.007367in}}{\pgfqpoint{-0.027778in}{0.000000in}}%
\pgfpathcurveto{\pgfqpoint{-0.027778in}{-0.007367in}}{\pgfqpoint{-0.024851in}{-0.014433in}}{\pgfqpoint{-0.019642in}{-0.019642in}}%
\pgfpathcurveto{\pgfqpoint{-0.014433in}{-0.024851in}}{\pgfqpoint{-0.007367in}{-0.027778in}}{\pgfqpoint{0.000000in}{-0.027778in}}%
\pgfpathlineto{\pgfqpoint{0.000000in}{-0.027778in}}%
\pgfpathclose%
\pgfusepath{stroke,fill}%
}%
\begin{pgfscope}%
\pgfsys@transformshift{2.150915in}{0.883464in}%
\pgfsys@useobject{currentmarker}{}%
\end{pgfscope}%
\end{pgfscope}%
\begin{pgfscope}%
\pgfpathrectangle{\pgfqpoint{0.795366in}{0.646140in}}{\pgfqpoint{5.029404in}{3.088289in}}%
\pgfusepath{clip}%
\pgfsetbuttcap%
\pgfsetroundjoin%
\definecolor{currentfill}{rgb}{1.000000,0.000000,0.000000}%
\pgfsetfillcolor{currentfill}%
\pgfsetlinewidth{1.003750pt}%
\definecolor{currentstroke}{rgb}{1.000000,0.000000,0.000000}%
\pgfsetstrokecolor{currentstroke}%
\pgfsetdash{}{0pt}%
\pgfsys@defobject{currentmarker}{\pgfqpoint{-0.027778in}{-0.027778in}}{\pgfqpoint{0.027778in}{0.027778in}}{%
\pgfpathmoveto{\pgfqpoint{0.000000in}{-0.027778in}}%
\pgfpathcurveto{\pgfqpoint{0.007367in}{-0.027778in}}{\pgfqpoint{0.014433in}{-0.024851in}}{\pgfqpoint{0.019642in}{-0.019642in}}%
\pgfpathcurveto{\pgfqpoint{0.024851in}{-0.014433in}}{\pgfqpoint{0.027778in}{-0.007367in}}{\pgfqpoint{0.027778in}{0.000000in}}%
\pgfpathcurveto{\pgfqpoint{0.027778in}{0.007367in}}{\pgfqpoint{0.024851in}{0.014433in}}{\pgfqpoint{0.019642in}{0.019642in}}%
\pgfpathcurveto{\pgfqpoint{0.014433in}{0.024851in}}{\pgfqpoint{0.007367in}{0.027778in}}{\pgfqpoint{0.000000in}{0.027778in}}%
\pgfpathcurveto{\pgfqpoint{-0.007367in}{0.027778in}}{\pgfqpoint{-0.014433in}{0.024851in}}{\pgfqpoint{-0.019642in}{0.019642in}}%
\pgfpathcurveto{\pgfqpoint{-0.024851in}{0.014433in}}{\pgfqpoint{-0.027778in}{0.007367in}}{\pgfqpoint{-0.027778in}{0.000000in}}%
\pgfpathcurveto{\pgfqpoint{-0.027778in}{-0.007367in}}{\pgfqpoint{-0.024851in}{-0.014433in}}{\pgfqpoint{-0.019642in}{-0.019642in}}%
\pgfpathcurveto{\pgfqpoint{-0.014433in}{-0.024851in}}{\pgfqpoint{-0.007367in}{-0.027778in}}{\pgfqpoint{0.000000in}{-0.027778in}}%
\pgfpathlineto{\pgfqpoint{0.000000in}{-0.027778in}}%
\pgfpathclose%
\pgfusepath{stroke,fill}%
}%
\begin{pgfscope}%
\pgfsys@transformshift{2.583194in}{0.841700in}%
\pgfsys@useobject{currentmarker}{}%
\end{pgfscope}%
\end{pgfscope}%
\begin{pgfscope}%
\pgfpathrectangle{\pgfqpoint{0.795366in}{0.646140in}}{\pgfqpoint{5.029404in}{3.088289in}}%
\pgfusepath{clip}%
\pgfsetbuttcap%
\pgfsetroundjoin%
\definecolor{currentfill}{rgb}{1.000000,0.000000,0.000000}%
\pgfsetfillcolor{currentfill}%
\pgfsetlinewidth{1.003750pt}%
\definecolor{currentstroke}{rgb}{1.000000,0.000000,0.000000}%
\pgfsetstrokecolor{currentstroke}%
\pgfsetdash{}{0pt}%
\pgfsys@defobject{currentmarker}{\pgfqpoint{-0.027778in}{-0.027778in}}{\pgfqpoint{0.027778in}{0.027778in}}{%
\pgfpathmoveto{\pgfqpoint{0.000000in}{-0.027778in}}%
\pgfpathcurveto{\pgfqpoint{0.007367in}{-0.027778in}}{\pgfqpoint{0.014433in}{-0.024851in}}{\pgfqpoint{0.019642in}{-0.019642in}}%
\pgfpathcurveto{\pgfqpoint{0.024851in}{-0.014433in}}{\pgfqpoint{0.027778in}{-0.007367in}}{\pgfqpoint{0.027778in}{0.000000in}}%
\pgfpathcurveto{\pgfqpoint{0.027778in}{0.007367in}}{\pgfqpoint{0.024851in}{0.014433in}}{\pgfqpoint{0.019642in}{0.019642in}}%
\pgfpathcurveto{\pgfqpoint{0.014433in}{0.024851in}}{\pgfqpoint{0.007367in}{0.027778in}}{\pgfqpoint{0.000000in}{0.027778in}}%
\pgfpathcurveto{\pgfqpoint{-0.007367in}{0.027778in}}{\pgfqpoint{-0.014433in}{0.024851in}}{\pgfqpoint{-0.019642in}{0.019642in}}%
\pgfpathcurveto{\pgfqpoint{-0.024851in}{0.014433in}}{\pgfqpoint{-0.027778in}{0.007367in}}{\pgfqpoint{-0.027778in}{0.000000in}}%
\pgfpathcurveto{\pgfqpoint{-0.027778in}{-0.007367in}}{\pgfqpoint{-0.024851in}{-0.014433in}}{\pgfqpoint{-0.019642in}{-0.019642in}}%
\pgfpathcurveto{\pgfqpoint{-0.014433in}{-0.024851in}}{\pgfqpoint{-0.007367in}{-0.027778in}}{\pgfqpoint{0.000000in}{-0.027778in}}%
\pgfpathlineto{\pgfqpoint{0.000000in}{-0.027778in}}%
\pgfpathclose%
\pgfusepath{stroke,fill}%
}%
\begin{pgfscope}%
\pgfsys@transformshift{3.025345in}{0.841754in}%
\pgfsys@useobject{currentmarker}{}%
\end{pgfscope}%
\end{pgfscope}%
\begin{pgfscope}%
\pgfpathrectangle{\pgfqpoint{0.795366in}{0.646140in}}{\pgfqpoint{5.029404in}{3.088289in}}%
\pgfusepath{clip}%
\pgfsetbuttcap%
\pgfsetroundjoin%
\definecolor{currentfill}{rgb}{1.000000,0.000000,0.000000}%
\pgfsetfillcolor{currentfill}%
\pgfsetlinewidth{1.003750pt}%
\definecolor{currentstroke}{rgb}{1.000000,0.000000,0.000000}%
\pgfsetstrokecolor{currentstroke}%
\pgfsetdash{}{0pt}%
\pgfsys@defobject{currentmarker}{\pgfqpoint{-0.027778in}{-0.027778in}}{\pgfqpoint{0.027778in}{0.027778in}}{%
\pgfpathmoveto{\pgfqpoint{0.000000in}{-0.027778in}}%
\pgfpathcurveto{\pgfqpoint{0.007367in}{-0.027778in}}{\pgfqpoint{0.014433in}{-0.024851in}}{\pgfqpoint{0.019642in}{-0.019642in}}%
\pgfpathcurveto{\pgfqpoint{0.024851in}{-0.014433in}}{\pgfqpoint{0.027778in}{-0.007367in}}{\pgfqpoint{0.027778in}{0.000000in}}%
\pgfpathcurveto{\pgfqpoint{0.027778in}{0.007367in}}{\pgfqpoint{0.024851in}{0.014433in}}{\pgfqpoint{0.019642in}{0.019642in}}%
\pgfpathcurveto{\pgfqpoint{0.014433in}{0.024851in}}{\pgfqpoint{0.007367in}{0.027778in}}{\pgfqpoint{0.000000in}{0.027778in}}%
\pgfpathcurveto{\pgfqpoint{-0.007367in}{0.027778in}}{\pgfqpoint{-0.014433in}{0.024851in}}{\pgfqpoint{-0.019642in}{0.019642in}}%
\pgfpathcurveto{\pgfqpoint{-0.024851in}{0.014433in}}{\pgfqpoint{-0.027778in}{0.007367in}}{\pgfqpoint{-0.027778in}{0.000000in}}%
\pgfpathcurveto{\pgfqpoint{-0.027778in}{-0.007367in}}{\pgfqpoint{-0.024851in}{-0.014433in}}{\pgfqpoint{-0.019642in}{-0.019642in}}%
\pgfpathcurveto{\pgfqpoint{-0.014433in}{-0.024851in}}{\pgfqpoint{-0.007367in}{-0.027778in}}{\pgfqpoint{0.000000in}{-0.027778in}}%
\pgfpathlineto{\pgfqpoint{0.000000in}{-0.027778in}}%
\pgfpathclose%
\pgfusepath{stroke,fill}%
}%
\begin{pgfscope}%
\pgfsys@transformshift{3.447233in}{0.838140in}%
\pgfsys@useobject{currentmarker}{}%
\end{pgfscope}%
\end{pgfscope}%
\begin{pgfscope}%
\pgfpathrectangle{\pgfqpoint{0.795366in}{0.646140in}}{\pgfqpoint{5.029404in}{3.088289in}}%
\pgfusepath{clip}%
\pgfsetbuttcap%
\pgfsetroundjoin%
\definecolor{currentfill}{rgb}{1.000000,0.000000,0.000000}%
\pgfsetfillcolor{currentfill}%
\pgfsetlinewidth{1.003750pt}%
\definecolor{currentstroke}{rgb}{1.000000,0.000000,0.000000}%
\pgfsetstrokecolor{currentstroke}%
\pgfsetdash{}{0pt}%
\pgfsys@defobject{currentmarker}{\pgfqpoint{-0.027778in}{-0.027778in}}{\pgfqpoint{0.027778in}{0.027778in}}{%
\pgfpathmoveto{\pgfqpoint{0.000000in}{-0.027778in}}%
\pgfpathcurveto{\pgfqpoint{0.007367in}{-0.027778in}}{\pgfqpoint{0.014433in}{-0.024851in}}{\pgfqpoint{0.019642in}{-0.019642in}}%
\pgfpathcurveto{\pgfqpoint{0.024851in}{-0.014433in}}{\pgfqpoint{0.027778in}{-0.007367in}}{\pgfqpoint{0.027778in}{0.000000in}}%
\pgfpathcurveto{\pgfqpoint{0.027778in}{0.007367in}}{\pgfqpoint{0.024851in}{0.014433in}}{\pgfqpoint{0.019642in}{0.019642in}}%
\pgfpathcurveto{\pgfqpoint{0.014433in}{0.024851in}}{\pgfqpoint{0.007367in}{0.027778in}}{\pgfqpoint{0.000000in}{0.027778in}}%
\pgfpathcurveto{\pgfqpoint{-0.007367in}{0.027778in}}{\pgfqpoint{-0.014433in}{0.024851in}}{\pgfqpoint{-0.019642in}{0.019642in}}%
\pgfpathcurveto{\pgfqpoint{-0.024851in}{0.014433in}}{\pgfqpoint{-0.027778in}{0.007367in}}{\pgfqpoint{-0.027778in}{0.000000in}}%
\pgfpathcurveto{\pgfqpoint{-0.027778in}{-0.007367in}}{\pgfqpoint{-0.024851in}{-0.014433in}}{\pgfqpoint{-0.019642in}{-0.019642in}}%
\pgfpathcurveto{\pgfqpoint{-0.014433in}{-0.024851in}}{\pgfqpoint{-0.007367in}{-0.027778in}}{\pgfqpoint{0.000000in}{-0.027778in}}%
\pgfpathlineto{\pgfqpoint{0.000000in}{-0.027778in}}%
\pgfpathclose%
\pgfusepath{stroke,fill}%
}%
\begin{pgfscope}%
\pgfsys@transformshift{3.900815in}{0.862488in}%
\pgfsys@useobject{currentmarker}{}%
\end{pgfscope}%
\end{pgfscope}%
\begin{pgfscope}%
\pgfpathrectangle{\pgfqpoint{0.795366in}{0.646140in}}{\pgfqpoint{5.029404in}{3.088289in}}%
\pgfusepath{clip}%
\pgfsetbuttcap%
\pgfsetroundjoin%
\definecolor{currentfill}{rgb}{1.000000,0.000000,0.000000}%
\pgfsetfillcolor{currentfill}%
\pgfsetlinewidth{1.003750pt}%
\definecolor{currentstroke}{rgb}{1.000000,0.000000,0.000000}%
\pgfsetstrokecolor{currentstroke}%
\pgfsetdash{}{0pt}%
\pgfsys@defobject{currentmarker}{\pgfqpoint{-0.027778in}{-0.027778in}}{\pgfqpoint{0.027778in}{0.027778in}}{%
\pgfpathmoveto{\pgfqpoint{0.000000in}{-0.027778in}}%
\pgfpathcurveto{\pgfqpoint{0.007367in}{-0.027778in}}{\pgfqpoint{0.014433in}{-0.024851in}}{\pgfqpoint{0.019642in}{-0.019642in}}%
\pgfpathcurveto{\pgfqpoint{0.024851in}{-0.014433in}}{\pgfqpoint{0.027778in}{-0.007367in}}{\pgfqpoint{0.027778in}{0.000000in}}%
\pgfpathcurveto{\pgfqpoint{0.027778in}{0.007367in}}{\pgfqpoint{0.024851in}{0.014433in}}{\pgfqpoint{0.019642in}{0.019642in}}%
\pgfpathcurveto{\pgfqpoint{0.014433in}{0.024851in}}{\pgfqpoint{0.007367in}{0.027778in}}{\pgfqpoint{0.000000in}{0.027778in}}%
\pgfpathcurveto{\pgfqpoint{-0.007367in}{0.027778in}}{\pgfqpoint{-0.014433in}{0.024851in}}{\pgfqpoint{-0.019642in}{0.019642in}}%
\pgfpathcurveto{\pgfqpoint{-0.024851in}{0.014433in}}{\pgfqpoint{-0.027778in}{0.007367in}}{\pgfqpoint{-0.027778in}{0.000000in}}%
\pgfpathcurveto{\pgfqpoint{-0.027778in}{-0.007367in}}{\pgfqpoint{-0.024851in}{-0.014433in}}{\pgfqpoint{-0.019642in}{-0.019642in}}%
\pgfpathcurveto{\pgfqpoint{-0.014433in}{-0.024851in}}{\pgfqpoint{-0.007367in}{-0.027778in}}{\pgfqpoint{0.000000in}{-0.027778in}}%
\pgfpathlineto{\pgfqpoint{0.000000in}{-0.027778in}}%
\pgfpathclose%
\pgfusepath{stroke,fill}%
}%
\begin{pgfscope}%
\pgfsys@transformshift{4.362709in}{0.899838in}%
\pgfsys@useobject{currentmarker}{}%
\end{pgfscope}%
\end{pgfscope}%
\begin{pgfscope}%
\pgfpathrectangle{\pgfqpoint{0.795366in}{0.646140in}}{\pgfqpoint{5.029404in}{3.088289in}}%
\pgfusepath{clip}%
\pgfsetbuttcap%
\pgfsetroundjoin%
\definecolor{currentfill}{rgb}{1.000000,0.000000,0.000000}%
\pgfsetfillcolor{currentfill}%
\pgfsetlinewidth{1.003750pt}%
\definecolor{currentstroke}{rgb}{1.000000,0.000000,0.000000}%
\pgfsetstrokecolor{currentstroke}%
\pgfsetdash{}{0pt}%
\pgfsys@defobject{currentmarker}{\pgfqpoint{-0.027778in}{-0.027778in}}{\pgfqpoint{0.027778in}{0.027778in}}{%
\pgfpathmoveto{\pgfqpoint{0.000000in}{-0.027778in}}%
\pgfpathcurveto{\pgfqpoint{0.007367in}{-0.027778in}}{\pgfqpoint{0.014433in}{-0.024851in}}{\pgfqpoint{0.019642in}{-0.019642in}}%
\pgfpathcurveto{\pgfqpoint{0.024851in}{-0.014433in}}{\pgfqpoint{0.027778in}{-0.007367in}}{\pgfqpoint{0.027778in}{0.000000in}}%
\pgfpathcurveto{\pgfqpoint{0.027778in}{0.007367in}}{\pgfqpoint{0.024851in}{0.014433in}}{\pgfqpoint{0.019642in}{0.019642in}}%
\pgfpathcurveto{\pgfqpoint{0.014433in}{0.024851in}}{\pgfqpoint{0.007367in}{0.027778in}}{\pgfqpoint{0.000000in}{0.027778in}}%
\pgfpathcurveto{\pgfqpoint{-0.007367in}{0.027778in}}{\pgfqpoint{-0.014433in}{0.024851in}}{\pgfqpoint{-0.019642in}{0.019642in}}%
\pgfpathcurveto{\pgfqpoint{-0.024851in}{0.014433in}}{\pgfqpoint{-0.027778in}{0.007367in}}{\pgfqpoint{-0.027778in}{0.000000in}}%
\pgfpathcurveto{\pgfqpoint{-0.027778in}{-0.007367in}}{\pgfqpoint{-0.024851in}{-0.014433in}}{\pgfqpoint{-0.019642in}{-0.019642in}}%
\pgfpathcurveto{\pgfqpoint{-0.014433in}{-0.024851in}}{\pgfqpoint{-0.007367in}{-0.027778in}}{\pgfqpoint{0.000000in}{-0.027778in}}%
\pgfpathlineto{\pgfqpoint{0.000000in}{-0.027778in}}%
\pgfpathclose%
\pgfusepath{stroke,fill}%
}%
\begin{pgfscope}%
\pgfsys@transformshift{4.781480in}{0.807897in}%
\pgfsys@useobject{currentmarker}{}%
\end{pgfscope}%
\end{pgfscope}%
\begin{pgfscope}%
\pgfpathrectangle{\pgfqpoint{0.795366in}{0.646140in}}{\pgfqpoint{5.029404in}{3.088289in}}%
\pgfusepath{clip}%
\pgfsetbuttcap%
\pgfsetroundjoin%
\definecolor{currentfill}{rgb}{1.000000,0.000000,0.000000}%
\pgfsetfillcolor{currentfill}%
\pgfsetlinewidth{1.003750pt}%
\definecolor{currentstroke}{rgb}{1.000000,0.000000,0.000000}%
\pgfsetstrokecolor{currentstroke}%
\pgfsetdash{}{0pt}%
\pgfsys@defobject{currentmarker}{\pgfqpoint{-0.027778in}{-0.027778in}}{\pgfqpoint{0.027778in}{0.027778in}}{%
\pgfpathmoveto{\pgfqpoint{0.000000in}{-0.027778in}}%
\pgfpathcurveto{\pgfqpoint{0.007367in}{-0.027778in}}{\pgfqpoint{0.014433in}{-0.024851in}}{\pgfqpoint{0.019642in}{-0.019642in}}%
\pgfpathcurveto{\pgfqpoint{0.024851in}{-0.014433in}}{\pgfqpoint{0.027778in}{-0.007367in}}{\pgfqpoint{0.027778in}{0.000000in}}%
\pgfpathcurveto{\pgfqpoint{0.027778in}{0.007367in}}{\pgfqpoint{0.024851in}{0.014433in}}{\pgfqpoint{0.019642in}{0.019642in}}%
\pgfpathcurveto{\pgfqpoint{0.014433in}{0.024851in}}{\pgfqpoint{0.007367in}{0.027778in}}{\pgfqpoint{0.000000in}{0.027778in}}%
\pgfpathcurveto{\pgfqpoint{-0.007367in}{0.027778in}}{\pgfqpoint{-0.014433in}{0.024851in}}{\pgfqpoint{-0.019642in}{0.019642in}}%
\pgfpathcurveto{\pgfqpoint{-0.024851in}{0.014433in}}{\pgfqpoint{-0.027778in}{0.007367in}}{\pgfqpoint{-0.027778in}{0.000000in}}%
\pgfpathcurveto{\pgfqpoint{-0.027778in}{-0.007367in}}{\pgfqpoint{-0.024851in}{-0.014433in}}{\pgfqpoint{-0.019642in}{-0.019642in}}%
\pgfpathcurveto{\pgfqpoint{-0.014433in}{-0.024851in}}{\pgfqpoint{-0.007367in}{-0.027778in}}{\pgfqpoint{0.000000in}{-0.027778in}}%
\pgfpathlineto{\pgfqpoint{0.000000in}{-0.027778in}}%
\pgfpathclose%
\pgfusepath{stroke,fill}%
}%
\begin{pgfscope}%
\pgfsys@transformshift{5.194535in}{0.843534in}%
\pgfsys@useobject{currentmarker}{}%
\end{pgfscope}%
\end{pgfscope}%
\begin{pgfscope}%
\pgfsetrectcap%
\pgfsetmiterjoin%
\pgfsetlinewidth{0.803000pt}%
\definecolor{currentstroke}{rgb}{0.000000,0.000000,0.000000}%
\pgfsetstrokecolor{currentstroke}%
\pgfsetdash{}{0pt}%
\pgfpathmoveto{\pgfqpoint{0.795366in}{0.646140in}}%
\pgfpathlineto{\pgfqpoint{0.795366in}{3.734428in}}%
\pgfusepath{stroke}%
\end{pgfscope}%
\begin{pgfscope}%
\pgfsetrectcap%
\pgfsetmiterjoin%
\pgfsetlinewidth{0.803000pt}%
\definecolor{currentstroke}{rgb}{0.000000,0.000000,0.000000}%
\pgfsetstrokecolor{currentstroke}%
\pgfsetdash{}{0pt}%
\pgfpathmoveto{\pgfqpoint{5.824769in}{0.646140in}}%
\pgfpathlineto{\pgfqpoint{5.824769in}{3.734428in}}%
\pgfusepath{stroke}%
\end{pgfscope}%
\begin{pgfscope}%
\pgfsetrectcap%
\pgfsetmiterjoin%
\pgfsetlinewidth{0.803000pt}%
\definecolor{currentstroke}{rgb}{0.000000,0.000000,0.000000}%
\pgfsetstrokecolor{currentstroke}%
\pgfsetdash{}{0pt}%
\pgfpathmoveto{\pgfqpoint{0.795366in}{0.646140in}}%
\pgfpathlineto{\pgfqpoint{5.824769in}{0.646140in}}%
\pgfusepath{stroke}%
\end{pgfscope}%
\begin{pgfscope}%
\pgfsetrectcap%
\pgfsetmiterjoin%
\pgfsetlinewidth{0.803000pt}%
\definecolor{currentstroke}{rgb}{0.000000,0.000000,0.000000}%
\pgfsetstrokecolor{currentstroke}%
\pgfsetdash{}{0pt}%
\pgfpathmoveto{\pgfqpoint{0.795366in}{3.734428in}}%
\pgfpathlineto{\pgfqpoint{5.824769in}{3.734428in}}%
\pgfusepath{stroke}%
\end{pgfscope}%
\begin{pgfscope}%
\definecolor{textcolor}{rgb}{0.000000,0.000000,0.000000}%
\pgfsetstrokecolor{textcolor}%
\pgfsetfillcolor{textcolor}%
\pgftext[x=3.310068in,y=3.817761in,,base]{\color{textcolor}\rmfamily\fontsize{16.800000}{20.160000}\selectfont Cuvette Data with dips}%
\end{pgfscope}%
\begin{pgfscope}%
\pgfsetbuttcap%
\pgfsetmiterjoin%
\definecolor{currentfill}{rgb}{1.000000,1.000000,1.000000}%
\pgfsetfillcolor{currentfill}%
\pgfsetfillopacity{0.800000}%
\pgfsetlinewidth{1.003750pt}%
\definecolor{currentstroke}{rgb}{0.800000,0.800000,0.800000}%
\pgfsetstrokecolor{currentstroke}%
\pgfsetstrokeopacity{0.800000}%
\pgfsetdash{}{0pt}%
\pgfpathmoveto{\pgfqpoint{4.410575in}{3.036762in}}%
\pgfpathlineto{\pgfqpoint{5.688658in}{3.036762in}}%
\pgfpathquadraticcurveto{\pgfqpoint{5.727547in}{3.036762in}}{\pgfqpoint{5.727547in}{3.075651in}}%
\pgfpathlineto{\pgfqpoint{5.727547in}{3.598317in}}%
\pgfpathquadraticcurveto{\pgfqpoint{5.727547in}{3.637206in}}{\pgfqpoint{5.688658in}{3.637206in}}%
\pgfpathlineto{\pgfqpoint{4.410575in}{3.637206in}}%
\pgfpathquadraticcurveto{\pgfqpoint{4.371686in}{3.637206in}}{\pgfqpoint{4.371686in}{3.598317in}}%
\pgfpathlineto{\pgfqpoint{4.371686in}{3.075651in}}%
\pgfpathquadraticcurveto{\pgfqpoint{4.371686in}{3.036762in}}{\pgfqpoint{4.410575in}{3.036762in}}%
\pgfpathlineto{\pgfqpoint{4.410575in}{3.036762in}}%
\pgfpathclose%
\pgfusepath{stroke,fill}%
\end{pgfscope}%
\begin{pgfscope}%
\pgfsetrectcap%
\pgfsetroundjoin%
\pgfsetlinewidth{0.501875pt}%
\definecolor{currentstroke}{rgb}{0.000000,0.000000,0.000000}%
\pgfsetstrokecolor{currentstroke}%
\pgfsetdash{}{0pt}%
\pgfpathmoveto{\pgfqpoint{4.449464in}{3.491373in}}%
\pgfpathlineto{\pgfqpoint{4.643908in}{3.491373in}}%
\pgfpathlineto{\pgfqpoint{4.838353in}{3.491373in}}%
\pgfusepath{stroke}%
\end{pgfscope}%
\begin{pgfscope}%
\definecolor{textcolor}{rgb}{0.000000,0.000000,0.000000}%
\pgfsetstrokecolor{textcolor}%
\pgfsetfillcolor{textcolor}%
\pgftext[x=4.993908in,y=3.423317in,left,base]{\color{textcolor}\rmfamily\fontsize{14.000000}{16.800000}\selectfont Cuvette}%
\end{pgfscope}%
\begin{pgfscope}%
\pgfsetbuttcap%
\pgfsetroundjoin%
\definecolor{currentfill}{rgb}{1.000000,0.000000,0.000000}%
\pgfsetfillcolor{currentfill}%
\pgfsetlinewidth{1.003750pt}%
\definecolor{currentstroke}{rgb}{1.000000,0.000000,0.000000}%
\pgfsetstrokecolor{currentstroke}%
\pgfsetdash{}{0pt}%
\pgfsys@defobject{currentmarker}{\pgfqpoint{-0.027778in}{-0.027778in}}{\pgfqpoint{0.027778in}{0.027778in}}{%
\pgfpathmoveto{\pgfqpoint{0.000000in}{-0.027778in}}%
\pgfpathcurveto{\pgfqpoint{0.007367in}{-0.027778in}}{\pgfqpoint{0.014433in}{-0.024851in}}{\pgfqpoint{0.019642in}{-0.019642in}}%
\pgfpathcurveto{\pgfqpoint{0.024851in}{-0.014433in}}{\pgfqpoint{0.027778in}{-0.007367in}}{\pgfqpoint{0.027778in}{0.000000in}}%
\pgfpathcurveto{\pgfqpoint{0.027778in}{0.007367in}}{\pgfqpoint{0.024851in}{0.014433in}}{\pgfqpoint{0.019642in}{0.019642in}}%
\pgfpathcurveto{\pgfqpoint{0.014433in}{0.024851in}}{\pgfqpoint{0.007367in}{0.027778in}}{\pgfqpoint{0.000000in}{0.027778in}}%
\pgfpathcurveto{\pgfqpoint{-0.007367in}{0.027778in}}{\pgfqpoint{-0.014433in}{0.024851in}}{\pgfqpoint{-0.019642in}{0.019642in}}%
\pgfpathcurveto{\pgfqpoint{-0.024851in}{0.014433in}}{\pgfqpoint{-0.027778in}{0.007367in}}{\pgfqpoint{-0.027778in}{0.000000in}}%
\pgfpathcurveto{\pgfqpoint{-0.027778in}{-0.007367in}}{\pgfqpoint{-0.024851in}{-0.014433in}}{\pgfqpoint{-0.019642in}{-0.019642in}}%
\pgfpathcurveto{\pgfqpoint{-0.014433in}{-0.024851in}}{\pgfqpoint{-0.007367in}{-0.027778in}}{\pgfqpoint{0.000000in}{-0.027778in}}%
\pgfpathlineto{\pgfqpoint{0.000000in}{-0.027778in}}%
\pgfpathclose%
\pgfusepath{stroke,fill}%
}%
\begin{pgfscope}%
\pgfsys@transformshift{4.643908in}{3.220317in}%
\pgfsys@useobject{currentmarker}{}%
\end{pgfscope}%
\end{pgfscope}%
\begin{pgfscope}%
\definecolor{textcolor}{rgb}{0.000000,0.000000,0.000000}%
\pgfsetstrokecolor{textcolor}%
\pgfsetfillcolor{textcolor}%
\pgftext[x=4.993908in,y=3.152262in,left,base]{\color{textcolor}\rmfamily\fontsize{14.000000}{16.800000}\selectfont Dips}%
\end{pgfscope}%
\end{pgfpicture}%
\makeatother%
\endgroup%
}
		\caption{Cuvette dips}
		\label{fig:Cuvette dips}
	\end{subfigure}
	\hspace{0.5cm}
	\begin{subfigure}{0.45\textwidth}
		\centering
		\scalebox{0.50}{%% Creator: Matplotlib, PGF backend
%%
%% To include the figure in your LaTeX document, write
%%   \input{<filename>.pgf}
%%
%% Make sure the required packages are loaded in your preamble
%%   \usepackage{pgf}
%%
%% Also ensure that all the required font packages are loaded; for instance,
%% the lmodern package is sometimes necessary when using math font.
%%   \usepackage{lmodern}
%%
%% Figures using additional raster images can only be included by \input if
%% they are in the same directory as the main LaTeX file. For loading figures
%% from other directories you can use the `import` package
%%   \usepackage{import}
%%
%% and then include the figures with
%%   \import{<path to file>}{<filename>.pgf}
%%
%% Matplotlib used the following preamble
%%   
%%   \usepackage{fontspec}
%%   \makeatletter\@ifpackageloaded{underscore}{}{\usepackage[strings]{underscore}}\makeatother
%%
\begingroup%
\makeatletter%
\begin{pgfpicture}%
\pgfpathrectangle{\pgfpointorigin}{\pgfqpoint{5.654486in}{4.080862in}}%
\pgfusepath{use as bounding box, clip}%
\begin{pgfscope}%
\pgfsetbuttcap%
\pgfsetmiterjoin%
\definecolor{currentfill}{rgb}{1.000000,1.000000,1.000000}%
\pgfsetfillcolor{currentfill}%
\pgfsetlinewidth{0.000000pt}%
\definecolor{currentstroke}{rgb}{1.000000,1.000000,1.000000}%
\pgfsetstrokecolor{currentstroke}%
\pgfsetdash{}{0pt}%
\pgfpathmoveto{\pgfqpoint{0.000000in}{-0.000000in}}%
\pgfpathlineto{\pgfqpoint{5.654486in}{-0.000000in}}%
\pgfpathlineto{\pgfqpoint{5.654486in}{4.080862in}}%
\pgfpathlineto{\pgfqpoint{0.000000in}{4.080862in}}%
\pgfpathlineto{\pgfqpoint{0.000000in}{-0.000000in}}%
\pgfpathclose%
\pgfusepath{fill}%
\end{pgfscope}%
\begin{pgfscope}%
\pgfsetbuttcap%
\pgfsetmiterjoin%
\definecolor{currentfill}{rgb}{1.000000,1.000000,1.000000}%
\pgfsetfillcolor{currentfill}%
\pgfsetlinewidth{0.000000pt}%
\definecolor{currentstroke}{rgb}{0.000000,0.000000,0.000000}%
\pgfsetstrokecolor{currentstroke}%
\pgfsetstrokeopacity{0.000000}%
\pgfsetdash{}{0pt}%
\pgfpathmoveto{\pgfqpoint{0.525082in}{0.646140in}}%
\pgfpathlineto{\pgfqpoint{5.554486in}{0.646140in}}%
\pgfpathlineto{\pgfqpoint{5.554486in}{3.734428in}}%
\pgfpathlineto{\pgfqpoint{0.525082in}{3.734428in}}%
\pgfpathlineto{\pgfqpoint{0.525082in}{0.646140in}}%
\pgfpathclose%
\pgfusepath{fill}%
\end{pgfscope}%
\begin{pgfscope}%
\pgfsetbuttcap%
\pgfsetroundjoin%
\definecolor{currentfill}{rgb}{0.000000,0.000000,0.000000}%
\pgfsetfillcolor{currentfill}%
\pgfsetlinewidth{0.803000pt}%
\definecolor{currentstroke}{rgb}{0.000000,0.000000,0.000000}%
\pgfsetstrokecolor{currentstroke}%
\pgfsetdash{}{0pt}%
\pgfsys@defobject{currentmarker}{\pgfqpoint{0.000000in}{-0.048611in}}{\pgfqpoint{0.000000in}{0.000000in}}{%
\pgfpathmoveto{\pgfqpoint{0.000000in}{0.000000in}}%
\pgfpathlineto{\pgfqpoint{0.000000in}{-0.048611in}}%
\pgfusepath{stroke,fill}%
}%
\begin{pgfscope}%
\pgfsys@transformshift{1.427053in}{0.646140in}%
\pgfsys@useobject{currentmarker}{}%
\end{pgfscope}%
\end{pgfscope}%
\begin{pgfscope}%
\definecolor{textcolor}{rgb}{0.000000,0.000000,0.000000}%
\pgfsetstrokecolor{textcolor}%
\pgfsetfillcolor{textcolor}%
\pgftext[x=1.427053in,y=0.548917in,,top]{\color{textcolor}\rmfamily\fontsize{14.000000}{16.800000}\selectfont \(\displaystyle {2600}\)}%
\end{pgfscope}%
\begin{pgfscope}%
\pgfsetbuttcap%
\pgfsetroundjoin%
\definecolor{currentfill}{rgb}{0.000000,0.000000,0.000000}%
\pgfsetfillcolor{currentfill}%
\pgfsetlinewidth{0.803000pt}%
\definecolor{currentstroke}{rgb}{0.000000,0.000000,0.000000}%
\pgfsetstrokecolor{currentstroke}%
\pgfsetdash{}{0pt}%
\pgfsys@defobject{currentmarker}{\pgfqpoint{0.000000in}{-0.048611in}}{\pgfqpoint{0.000000in}{0.000000in}}{%
\pgfpathmoveto{\pgfqpoint{0.000000in}{0.000000in}}%
\pgfpathlineto{\pgfqpoint{0.000000in}{-0.048611in}}%
\pgfusepath{stroke,fill}%
}%
\begin{pgfscope}%
\pgfsys@transformshift{2.397664in}{0.646140in}%
\pgfsys@useobject{currentmarker}{}%
\end{pgfscope}%
\end{pgfscope}%
\begin{pgfscope}%
\definecolor{textcolor}{rgb}{0.000000,0.000000,0.000000}%
\pgfsetstrokecolor{textcolor}%
\pgfsetfillcolor{textcolor}%
\pgftext[x=2.397664in,y=0.548917in,,top]{\color{textcolor}\rmfamily\fontsize{14.000000}{16.800000}\selectfont \(\displaystyle {2800}\)}%
\end{pgfscope}%
\begin{pgfscope}%
\pgfsetbuttcap%
\pgfsetroundjoin%
\definecolor{currentfill}{rgb}{0.000000,0.000000,0.000000}%
\pgfsetfillcolor{currentfill}%
\pgfsetlinewidth{0.803000pt}%
\definecolor{currentstroke}{rgb}{0.000000,0.000000,0.000000}%
\pgfsetstrokecolor{currentstroke}%
\pgfsetdash{}{0pt}%
\pgfsys@defobject{currentmarker}{\pgfqpoint{0.000000in}{-0.048611in}}{\pgfqpoint{0.000000in}{0.000000in}}{%
\pgfpathmoveto{\pgfqpoint{0.000000in}{0.000000in}}%
\pgfpathlineto{\pgfqpoint{0.000000in}{-0.048611in}}%
\pgfusepath{stroke,fill}%
}%
\begin{pgfscope}%
\pgfsys@transformshift{3.368275in}{0.646140in}%
\pgfsys@useobject{currentmarker}{}%
\end{pgfscope}%
\end{pgfscope}%
\begin{pgfscope}%
\definecolor{textcolor}{rgb}{0.000000,0.000000,0.000000}%
\pgfsetstrokecolor{textcolor}%
\pgfsetfillcolor{textcolor}%
\pgftext[x=3.368275in,y=0.548917in,,top]{\color{textcolor}\rmfamily\fontsize{14.000000}{16.800000}\selectfont \(\displaystyle {3000}\)}%
\end{pgfscope}%
\begin{pgfscope}%
\pgfsetbuttcap%
\pgfsetroundjoin%
\definecolor{currentfill}{rgb}{0.000000,0.000000,0.000000}%
\pgfsetfillcolor{currentfill}%
\pgfsetlinewidth{0.803000pt}%
\definecolor{currentstroke}{rgb}{0.000000,0.000000,0.000000}%
\pgfsetstrokecolor{currentstroke}%
\pgfsetdash{}{0pt}%
\pgfsys@defobject{currentmarker}{\pgfqpoint{0.000000in}{-0.048611in}}{\pgfqpoint{0.000000in}{0.000000in}}{%
\pgfpathmoveto{\pgfqpoint{0.000000in}{0.000000in}}%
\pgfpathlineto{\pgfqpoint{0.000000in}{-0.048611in}}%
\pgfusepath{stroke,fill}%
}%
\begin{pgfscope}%
\pgfsys@transformshift{4.338886in}{0.646140in}%
\pgfsys@useobject{currentmarker}{}%
\end{pgfscope}%
\end{pgfscope}%
\begin{pgfscope}%
\definecolor{textcolor}{rgb}{0.000000,0.000000,0.000000}%
\pgfsetstrokecolor{textcolor}%
\pgfsetfillcolor{textcolor}%
\pgftext[x=4.338886in,y=0.548917in,,top]{\color{textcolor}\rmfamily\fontsize{14.000000}{16.800000}\selectfont \(\displaystyle {3200}\)}%
\end{pgfscope}%
\begin{pgfscope}%
\pgfsetbuttcap%
\pgfsetroundjoin%
\definecolor{currentfill}{rgb}{0.000000,0.000000,0.000000}%
\pgfsetfillcolor{currentfill}%
\pgfsetlinewidth{0.803000pt}%
\definecolor{currentstroke}{rgb}{0.000000,0.000000,0.000000}%
\pgfsetstrokecolor{currentstroke}%
\pgfsetdash{}{0pt}%
\pgfsys@defobject{currentmarker}{\pgfqpoint{0.000000in}{-0.048611in}}{\pgfqpoint{0.000000in}{0.000000in}}{%
\pgfpathmoveto{\pgfqpoint{0.000000in}{0.000000in}}%
\pgfpathlineto{\pgfqpoint{0.000000in}{-0.048611in}}%
\pgfusepath{stroke,fill}%
}%
\begin{pgfscope}%
\pgfsys@transformshift{5.309497in}{0.646140in}%
\pgfsys@useobject{currentmarker}{}%
\end{pgfscope}%
\end{pgfscope}%
\begin{pgfscope}%
\definecolor{textcolor}{rgb}{0.000000,0.000000,0.000000}%
\pgfsetstrokecolor{textcolor}%
\pgfsetfillcolor{textcolor}%
\pgftext[x=5.309497in,y=0.548917in,,top]{\color{textcolor}\rmfamily\fontsize{14.000000}{16.800000}\selectfont \(\displaystyle {3400}\)}%
\end{pgfscope}%
\begin{pgfscope}%
\definecolor{textcolor}{rgb}{0.000000,0.000000,0.000000}%
\pgfsetstrokecolor{textcolor}%
\pgfsetfillcolor{textcolor}%
\pgftext[x=3.039784in,y=0.320695in,,top]{\color{textcolor}\rmfamily\fontsize{14.000000}{16.800000}\selectfont Wavenumber \(\displaystyle \bar{\nu}_N\) [cm\(\displaystyle ^{-1}\)]}%
\end{pgfscope}%
\begin{pgfscope}%
\pgfsetbuttcap%
\pgfsetroundjoin%
\definecolor{currentfill}{rgb}{0.000000,0.000000,0.000000}%
\pgfsetfillcolor{currentfill}%
\pgfsetlinewidth{0.803000pt}%
\definecolor{currentstroke}{rgb}{0.000000,0.000000,0.000000}%
\pgfsetstrokecolor{currentstroke}%
\pgfsetdash{}{0pt}%
\pgfsys@defobject{currentmarker}{\pgfqpoint{-0.048611in}{0.000000in}}{\pgfqpoint{-0.000000in}{0.000000in}}{%
\pgfpathmoveto{\pgfqpoint{-0.000000in}{0.000000in}}%
\pgfpathlineto{\pgfqpoint{-0.048611in}{0.000000in}}%
\pgfusepath{stroke,fill}%
}%
\begin{pgfscope}%
\pgfsys@transformshift{0.525082in}{0.786516in}%
\pgfsys@useobject{currentmarker}{}%
\end{pgfscope}%
\end{pgfscope}%
\begin{pgfscope}%
\definecolor{textcolor}{rgb}{0.000000,0.000000,0.000000}%
\pgfsetstrokecolor{textcolor}%
\pgfsetfillcolor{textcolor}%
\pgftext[x=0.329944in, y=0.719044in, left, base]{\color{textcolor}\rmfamily\fontsize{14.000000}{16.800000}\selectfont \(\displaystyle {0}\)}%
\end{pgfscope}%
\begin{pgfscope}%
\pgfsetbuttcap%
\pgfsetroundjoin%
\definecolor{currentfill}{rgb}{0.000000,0.000000,0.000000}%
\pgfsetfillcolor{currentfill}%
\pgfsetlinewidth{0.803000pt}%
\definecolor{currentstroke}{rgb}{0.000000,0.000000,0.000000}%
\pgfsetstrokecolor{currentstroke}%
\pgfsetdash{}{0pt}%
\pgfsys@defobject{currentmarker}{\pgfqpoint{-0.048611in}{0.000000in}}{\pgfqpoint{-0.000000in}{0.000000in}}{%
\pgfpathmoveto{\pgfqpoint{-0.000000in}{0.000000in}}%
\pgfpathlineto{\pgfqpoint{-0.048611in}{0.000000in}}%
\pgfusepath{stroke,fill}%
}%
\begin{pgfscope}%
\pgfsys@transformshift{0.525082in}{1.410413in}%
\pgfsys@useobject{currentmarker}{}%
\end{pgfscope}%
\end{pgfscope}%
\begin{pgfscope}%
\definecolor{textcolor}{rgb}{0.000000,0.000000,0.000000}%
\pgfsetstrokecolor{textcolor}%
\pgfsetfillcolor{textcolor}%
\pgftext[x=0.329944in, y=1.342941in, left, base]{\color{textcolor}\rmfamily\fontsize{14.000000}{16.800000}\selectfont \(\displaystyle {2}\)}%
\end{pgfscope}%
\begin{pgfscope}%
\pgfsetbuttcap%
\pgfsetroundjoin%
\definecolor{currentfill}{rgb}{0.000000,0.000000,0.000000}%
\pgfsetfillcolor{currentfill}%
\pgfsetlinewidth{0.803000pt}%
\definecolor{currentstroke}{rgb}{0.000000,0.000000,0.000000}%
\pgfsetstrokecolor{currentstroke}%
\pgfsetdash{}{0pt}%
\pgfsys@defobject{currentmarker}{\pgfqpoint{-0.048611in}{0.000000in}}{\pgfqpoint{-0.000000in}{0.000000in}}{%
\pgfpathmoveto{\pgfqpoint{-0.000000in}{0.000000in}}%
\pgfpathlineto{\pgfqpoint{-0.048611in}{0.000000in}}%
\pgfusepath{stroke,fill}%
}%
\begin{pgfscope}%
\pgfsys@transformshift{0.525082in}{2.034310in}%
\pgfsys@useobject{currentmarker}{}%
\end{pgfscope}%
\end{pgfscope}%
\begin{pgfscope}%
\definecolor{textcolor}{rgb}{0.000000,0.000000,0.000000}%
\pgfsetstrokecolor{textcolor}%
\pgfsetfillcolor{textcolor}%
\pgftext[x=0.329944in, y=1.966837in, left, base]{\color{textcolor}\rmfamily\fontsize{14.000000}{16.800000}\selectfont \(\displaystyle {4}\)}%
\end{pgfscope}%
\begin{pgfscope}%
\pgfsetbuttcap%
\pgfsetroundjoin%
\definecolor{currentfill}{rgb}{0.000000,0.000000,0.000000}%
\pgfsetfillcolor{currentfill}%
\pgfsetlinewidth{0.803000pt}%
\definecolor{currentstroke}{rgb}{0.000000,0.000000,0.000000}%
\pgfsetstrokecolor{currentstroke}%
\pgfsetdash{}{0pt}%
\pgfsys@defobject{currentmarker}{\pgfqpoint{-0.048611in}{0.000000in}}{\pgfqpoint{-0.000000in}{0.000000in}}{%
\pgfpathmoveto{\pgfqpoint{-0.000000in}{0.000000in}}%
\pgfpathlineto{\pgfqpoint{-0.048611in}{0.000000in}}%
\pgfusepath{stroke,fill}%
}%
\begin{pgfscope}%
\pgfsys@transformshift{0.525082in}{2.658206in}%
\pgfsys@useobject{currentmarker}{}%
\end{pgfscope}%
\end{pgfscope}%
\begin{pgfscope}%
\definecolor{textcolor}{rgb}{0.000000,0.000000,0.000000}%
\pgfsetstrokecolor{textcolor}%
\pgfsetfillcolor{textcolor}%
\pgftext[x=0.329944in, y=2.590734in, left, base]{\color{textcolor}\rmfamily\fontsize{14.000000}{16.800000}\selectfont \(\displaystyle {6}\)}%
\end{pgfscope}%
\begin{pgfscope}%
\pgfsetbuttcap%
\pgfsetroundjoin%
\definecolor{currentfill}{rgb}{0.000000,0.000000,0.000000}%
\pgfsetfillcolor{currentfill}%
\pgfsetlinewidth{0.803000pt}%
\definecolor{currentstroke}{rgb}{0.000000,0.000000,0.000000}%
\pgfsetstrokecolor{currentstroke}%
\pgfsetdash{}{0pt}%
\pgfsys@defobject{currentmarker}{\pgfqpoint{-0.048611in}{0.000000in}}{\pgfqpoint{-0.000000in}{0.000000in}}{%
\pgfpathmoveto{\pgfqpoint{-0.000000in}{0.000000in}}%
\pgfpathlineto{\pgfqpoint{-0.048611in}{0.000000in}}%
\pgfusepath{stroke,fill}%
}%
\begin{pgfscope}%
\pgfsys@transformshift{0.525082in}{3.282103in}%
\pgfsys@useobject{currentmarker}{}%
\end{pgfscope}%
\end{pgfscope}%
\begin{pgfscope}%
\definecolor{textcolor}{rgb}{0.000000,0.000000,0.000000}%
\pgfsetstrokecolor{textcolor}%
\pgfsetfillcolor{textcolor}%
\pgftext[x=0.329944in, y=3.214631in, left, base]{\color{textcolor}\rmfamily\fontsize{14.000000}{16.800000}\selectfont \(\displaystyle {8}\)}%
\end{pgfscope}%
\begin{pgfscope}%
\definecolor{textcolor}{rgb}{0.000000,0.000000,0.000000}%
\pgfsetstrokecolor{textcolor}%
\pgfsetfillcolor{textcolor}%
\pgftext[x=0.274389in,y=2.190284in,,bottom,rotate=90.000000]{\color{textcolor}\rmfamily\fontsize{14.000000}{16.800000}\selectfont Dip Number \(\displaystyle \Delta N\)}%
\end{pgfscope}%
\begin{pgfscope}%
\pgfpathrectangle{\pgfqpoint{0.525082in}{0.646140in}}{\pgfqpoint{5.029404in}{3.088289in}}%
\pgfusepath{clip}%
\pgfsetbuttcap%
\pgfsetroundjoin%
\definecolor{currentfill}{rgb}{1.000000,0.000000,0.000000}%
\pgfsetfillcolor{currentfill}%
\pgfsetlinewidth{1.003750pt}%
\definecolor{currentstroke}{rgb}{1.000000,0.000000,0.000000}%
\pgfsetstrokecolor{currentstroke}%
\pgfsetdash{}{0pt}%
\pgfsys@defobject{currentmarker}{\pgfqpoint{-0.041667in}{-0.041667in}}{\pgfqpoint{0.041667in}{0.041667in}}{%
\pgfpathmoveto{\pgfqpoint{0.000000in}{-0.041667in}}%
\pgfpathcurveto{\pgfqpoint{0.011050in}{-0.041667in}}{\pgfqpoint{0.021649in}{-0.037276in}}{\pgfqpoint{0.029463in}{-0.029463in}}%
\pgfpathcurveto{\pgfqpoint{0.037276in}{-0.021649in}}{\pgfqpoint{0.041667in}{-0.011050in}}{\pgfqpoint{0.041667in}{0.000000in}}%
\pgfpathcurveto{\pgfqpoint{0.041667in}{0.011050in}}{\pgfqpoint{0.037276in}{0.021649in}}{\pgfqpoint{0.029463in}{0.029463in}}%
\pgfpathcurveto{\pgfqpoint{0.021649in}{0.037276in}}{\pgfqpoint{0.011050in}{0.041667in}}{\pgfqpoint{0.000000in}{0.041667in}}%
\pgfpathcurveto{\pgfqpoint{-0.011050in}{0.041667in}}{\pgfqpoint{-0.021649in}{0.037276in}}{\pgfqpoint{-0.029463in}{0.029463in}}%
\pgfpathcurveto{\pgfqpoint{-0.037276in}{0.021649in}}{\pgfqpoint{-0.041667in}{0.011050in}}{\pgfqpoint{-0.041667in}{0.000000in}}%
\pgfpathcurveto{\pgfqpoint{-0.041667in}{-0.011050in}}{\pgfqpoint{-0.037276in}{-0.021649in}}{\pgfqpoint{-0.029463in}{-0.029463in}}%
\pgfpathcurveto{\pgfqpoint{-0.021649in}{-0.037276in}}{\pgfqpoint{-0.011050in}{-0.041667in}}{\pgfqpoint{0.000000in}{-0.041667in}}%
\pgfpathlineto{\pgfqpoint{0.000000in}{-0.041667in}}%
\pgfpathclose%
\pgfusepath{stroke,fill}%
}%
\begin{pgfscope}%
\pgfsys@transformshift{0.753691in}{0.786516in}%
\pgfsys@useobject{currentmarker}{}%
\end{pgfscope}%
\begin{pgfscope}%
\pgfsys@transformshift{1.262049in}{1.098465in}%
\pgfsys@useobject{currentmarker}{}%
\end{pgfscope}%
\begin{pgfscope}%
\pgfsys@transformshift{1.772226in}{1.410413in}%
\pgfsys@useobject{currentmarker}{}%
\end{pgfscope}%
\begin{pgfscope}%
\pgfsys@transformshift{2.276944in}{1.722361in}%
\pgfsys@useobject{currentmarker}{}%
\end{pgfscope}%
\begin{pgfscope}%
\pgfsys@transformshift{2.793188in}{2.034310in}%
\pgfsys@useobject{currentmarker}{}%
\end{pgfscope}%
\begin{pgfscope}%
\pgfsys@transformshift{3.285773in}{2.346258in}%
\pgfsys@useobject{currentmarker}{}%
\end{pgfscope}%
\begin{pgfscope}%
\pgfsys@transformshift{3.815363in}{2.658206in}%
\pgfsys@useobject{currentmarker}{}%
\end{pgfscope}%
\begin{pgfscope}%
\pgfsys@transformshift{4.354659in}{2.970155in}%
\pgfsys@useobject{currentmarker}{}%
\end{pgfscope}%
\begin{pgfscope}%
\pgfsys@transformshift{4.843604in}{3.282103in}%
\pgfsys@useobject{currentmarker}{}%
\end{pgfscope}%
\begin{pgfscope}%
\pgfsys@transformshift{5.325876in}{3.594051in}%
\pgfsys@useobject{currentmarker}{}%
\end{pgfscope}%
\end{pgfscope}%
\begin{pgfscope}%
\pgfpathrectangle{\pgfqpoint{0.525082in}{0.646140in}}{\pgfqpoint{5.029404in}{3.088289in}}%
\pgfusepath{clip}%
\pgfsetrectcap%
\pgfsetroundjoin%
\pgfsetlinewidth{1.505625pt}%
\definecolor{currentstroke}{rgb}{0.000000,0.000000,0.000000}%
\pgfsetstrokecolor{currentstroke}%
\pgfsetdash{}{0pt}%
\pgfpathmoveto{\pgfqpoint{0.753691in}{0.788354in}}%
\pgfpathlineto{\pgfqpoint{1.262049in}{1.098938in}}%
\pgfpathlineto{\pgfqpoint{1.772226in}{1.410635in}}%
\pgfpathlineto{\pgfqpoint{2.276944in}{1.718996in}}%
\pgfpathlineto{\pgfqpoint{2.793188in}{2.034399in}}%
\pgfpathlineto{\pgfqpoint{3.285773in}{2.335347in}}%
\pgfpathlineto{\pgfqpoint{3.815363in}{2.658903in}}%
\pgfpathlineto{\pgfqpoint{4.354659in}{2.988390in}}%
\pgfpathlineto{\pgfqpoint{4.843604in}{3.287114in}}%
\pgfpathlineto{\pgfqpoint{5.325876in}{3.581762in}}%
\pgfusepath{stroke}%
\end{pgfscope}%
\begin{pgfscope}%
\pgfsetrectcap%
\pgfsetmiterjoin%
\pgfsetlinewidth{0.803000pt}%
\definecolor{currentstroke}{rgb}{0.000000,0.000000,0.000000}%
\pgfsetstrokecolor{currentstroke}%
\pgfsetdash{}{0pt}%
\pgfpathmoveto{\pgfqpoint{0.525082in}{0.646140in}}%
\pgfpathlineto{\pgfqpoint{0.525082in}{3.734428in}}%
\pgfusepath{stroke}%
\end{pgfscope}%
\begin{pgfscope}%
\pgfsetrectcap%
\pgfsetmiterjoin%
\pgfsetlinewidth{0.803000pt}%
\definecolor{currentstroke}{rgb}{0.000000,0.000000,0.000000}%
\pgfsetstrokecolor{currentstroke}%
\pgfsetdash{}{0pt}%
\pgfpathmoveto{\pgfqpoint{5.554486in}{0.646140in}}%
\pgfpathlineto{\pgfqpoint{5.554486in}{3.734428in}}%
\pgfusepath{stroke}%
\end{pgfscope}%
\begin{pgfscope}%
\pgfsetrectcap%
\pgfsetmiterjoin%
\pgfsetlinewidth{0.803000pt}%
\definecolor{currentstroke}{rgb}{0.000000,0.000000,0.000000}%
\pgfsetstrokecolor{currentstroke}%
\pgfsetdash{}{0pt}%
\pgfpathmoveto{\pgfqpoint{0.525082in}{0.646140in}}%
\pgfpathlineto{\pgfqpoint{5.554486in}{0.646140in}}%
\pgfusepath{stroke}%
\end{pgfscope}%
\begin{pgfscope}%
\pgfsetrectcap%
\pgfsetmiterjoin%
\pgfsetlinewidth{0.803000pt}%
\definecolor{currentstroke}{rgb}{0.000000,0.000000,0.000000}%
\pgfsetstrokecolor{currentstroke}%
\pgfsetdash{}{0pt}%
\pgfpathmoveto{\pgfqpoint{0.525082in}{3.734428in}}%
\pgfpathlineto{\pgfqpoint{5.554486in}{3.734428in}}%
\pgfusepath{stroke}%
\end{pgfscope}%
\begin{pgfscope}%
\definecolor{textcolor}{rgb}{0.000000,0.000000,0.000000}%
\pgfsetstrokecolor{textcolor}%
\pgfsetfillcolor{textcolor}%
\pgftext[x=3.039784in,y=3.817761in,,base]{\color{textcolor}\rmfamily\fontsize{16.800000}{20.160000}\selectfont Cuvette dips vs Wavenumbers}%
\end{pgfscope}%
\begin{pgfscope}%
\pgfsetbuttcap%
\pgfsetmiterjoin%
\definecolor{currentfill}{rgb}{1.000000,1.000000,1.000000}%
\pgfsetfillcolor{currentfill}%
\pgfsetfillopacity{0.800000}%
\pgfsetlinewidth{1.003750pt}%
\definecolor{currentstroke}{rgb}{0.800000,0.800000,0.800000}%
\pgfsetstrokecolor{currentstroke}%
\pgfsetstrokeopacity{0.800000}%
\pgfsetdash{}{0pt}%
\pgfpathmoveto{\pgfqpoint{0.661193in}{3.032818in}}%
\pgfpathlineto{\pgfqpoint{2.434533in}{3.032818in}}%
\pgfpathquadraticcurveto{\pgfqpoint{2.473422in}{3.032818in}}{\pgfqpoint{2.473422in}{3.071706in}}%
\pgfpathlineto{\pgfqpoint{2.473422in}{3.598317in}}%
\pgfpathquadraticcurveto{\pgfqpoint{2.473422in}{3.637206in}}{\pgfqpoint{2.434533in}{3.637206in}}%
\pgfpathlineto{\pgfqpoint{0.661193in}{3.637206in}}%
\pgfpathquadraticcurveto{\pgfqpoint{0.622304in}{3.637206in}}{\pgfqpoint{0.622304in}{3.598317in}}%
\pgfpathlineto{\pgfqpoint{0.622304in}{3.071706in}}%
\pgfpathquadraticcurveto{\pgfqpoint{0.622304in}{3.032818in}}{\pgfqpoint{0.661193in}{3.032818in}}%
\pgfpathlineto{\pgfqpoint{0.661193in}{3.032818in}}%
\pgfpathclose%
\pgfusepath{stroke,fill}%
\end{pgfscope}%
\begin{pgfscope}%
\pgfsetbuttcap%
\pgfsetroundjoin%
\definecolor{currentfill}{rgb}{1.000000,0.000000,0.000000}%
\pgfsetfillcolor{currentfill}%
\pgfsetlinewidth{1.003750pt}%
\definecolor{currentstroke}{rgb}{1.000000,0.000000,0.000000}%
\pgfsetstrokecolor{currentstroke}%
\pgfsetdash{}{0pt}%
\pgfsys@defobject{currentmarker}{\pgfqpoint{-0.041667in}{-0.041667in}}{\pgfqpoint{0.041667in}{0.041667in}}{%
\pgfpathmoveto{\pgfqpoint{0.000000in}{-0.041667in}}%
\pgfpathcurveto{\pgfqpoint{0.011050in}{-0.041667in}}{\pgfqpoint{0.021649in}{-0.037276in}}{\pgfqpoint{0.029463in}{-0.029463in}}%
\pgfpathcurveto{\pgfqpoint{0.037276in}{-0.021649in}}{\pgfqpoint{0.041667in}{-0.011050in}}{\pgfqpoint{0.041667in}{0.000000in}}%
\pgfpathcurveto{\pgfqpoint{0.041667in}{0.011050in}}{\pgfqpoint{0.037276in}{0.021649in}}{\pgfqpoint{0.029463in}{0.029463in}}%
\pgfpathcurveto{\pgfqpoint{0.021649in}{0.037276in}}{\pgfqpoint{0.011050in}{0.041667in}}{\pgfqpoint{0.000000in}{0.041667in}}%
\pgfpathcurveto{\pgfqpoint{-0.011050in}{0.041667in}}{\pgfqpoint{-0.021649in}{0.037276in}}{\pgfqpoint{-0.029463in}{0.029463in}}%
\pgfpathcurveto{\pgfqpoint{-0.037276in}{0.021649in}}{\pgfqpoint{-0.041667in}{0.011050in}}{\pgfqpoint{-0.041667in}{0.000000in}}%
\pgfpathcurveto{\pgfqpoint{-0.041667in}{-0.011050in}}{\pgfqpoint{-0.037276in}{-0.021649in}}{\pgfqpoint{-0.029463in}{-0.029463in}}%
\pgfpathcurveto{\pgfqpoint{-0.021649in}{-0.037276in}}{\pgfqpoint{-0.011050in}{-0.041667in}}{\pgfqpoint{0.000000in}{-0.041667in}}%
\pgfpathlineto{\pgfqpoint{0.000000in}{-0.041667in}}%
\pgfpathclose%
\pgfusepath{stroke,fill}%
}%
\begin{pgfscope}%
\pgfsys@transformshift{0.894526in}{3.491373in}%
\pgfsys@useobject{currentmarker}{}%
\end{pgfscope}%
\end{pgfscope}%
\begin{pgfscope}%
\definecolor{textcolor}{rgb}{0.000000,0.000000,0.000000}%
\pgfsetstrokecolor{textcolor}%
\pgfsetfillcolor{textcolor}%
\pgftext[x=1.244526in,y=3.423317in,left,base]{\color{textcolor}\rmfamily\fontsize{14.000000}{16.800000}\selectfont Data}%
\end{pgfscope}%
\begin{pgfscope}%
\pgfsetrectcap%
\pgfsetroundjoin%
\pgfsetlinewidth{1.505625pt}%
\definecolor{currentstroke}{rgb}{0.000000,0.000000,0.000000}%
\pgfsetstrokecolor{currentstroke}%
\pgfsetdash{}{0pt}%
\pgfpathmoveto{\pgfqpoint{0.700082in}{3.217540in}}%
\pgfpathlineto{\pgfqpoint{0.894526in}{3.217540in}}%
\pgfpathlineto{\pgfqpoint{1.088971in}{3.217540in}}%
\pgfusepath{stroke}%
\end{pgfscope}%
\begin{pgfscope}%
\definecolor{textcolor}{rgb}{0.000000,0.000000,0.000000}%
\pgfsetstrokecolor{textcolor}%
\pgfsetfillcolor{textcolor}%
\pgftext[x=1.244526in,y=3.149484in,left,base]{\color{textcolor}\rmfamily\fontsize{14.000000}{16.800000}\selectfont \(\displaystyle d = 47.52 \ \mu\)m}%
\end{pgfscope}%
\end{pgfpicture}%
\makeatother%
\endgroup%
}
		\caption{Cuvette fit}
		\label{fig:Cuvette_fit}
	\end{subfigure}
\end{figure}

\subsection{Task 5}

This task analyzes the rotation-vibration spectrum of HCl. The data in this task was provided by our supervisor. The relevant part of the spectrum (after considering the background effects) is shown in Fig. \ref{fig:HCl}. The dips corresponding to absorption of the two Chlorine isotopes $^{35}$Cl and $^{37}$Cl were identified. The left dip corresponds to $^{37}$Cl and the right dip corresponds to $^{35}$Cl.

\begin{figure}[h!]
	\centering
	\scalebox{0.50}{%% Creator: Matplotlib, PGF backend
%%
%% To include the figure in your LaTeX document, write
%%   \input{<filename>.pgf}
%%
%% Make sure the required packages are loaded in your preamble
%%   \usepackage{pgf}
%%
%% Also ensure that all the required font packages are loaded; for instance,
%% the lmodern package is sometimes necessary when using math font.
%%   \usepackage{lmodern}
%%
%% Figures using additional raster images can only be included by \input if
%% they are in the same directory as the main LaTeX file. For loading figures
%% from other directories you can use the `import` package
%%   \usepackage{import}
%%
%% and then include the figures with
%%   \import{<path to file>}{<filename>.pgf}
%%
%% Matplotlib used the following preamble
%%   
%%   \usepackage{fontspec}
%%   \makeatletter\@ifpackageloaded{underscore}{}{\usepackage[strings]{underscore}}\makeatother
%%
\begingroup%
\makeatletter%
\begin{pgfpicture}%
\pgfpathrectangle{\pgfpointorigin}{\pgfqpoint{6.120600in}{3.901900in}}%
\pgfusepath{use as bounding box, clip}%
\begin{pgfscope}%
\pgfsetbuttcap%
\pgfsetmiterjoin%
\definecolor{currentfill}{rgb}{1.000000,1.000000,1.000000}%
\pgfsetfillcolor{currentfill}%
\pgfsetlinewidth{0.000000pt}%
\definecolor{currentstroke}{rgb}{1.000000,1.000000,1.000000}%
\pgfsetstrokecolor{currentstroke}%
\pgfsetdash{}{0pt}%
\pgfpathmoveto{\pgfqpoint{0.000000in}{-0.000000in}}%
\pgfpathlineto{\pgfqpoint{6.120600in}{-0.000000in}}%
\pgfpathlineto{\pgfqpoint{6.120600in}{3.901900in}}%
\pgfpathlineto{\pgfqpoint{0.000000in}{3.901900in}}%
\pgfpathlineto{\pgfqpoint{0.000000in}{-0.000000in}}%
\pgfpathclose%
\pgfusepath{fill}%
\end{pgfscope}%
\begin{pgfscope}%
\pgfsetbuttcap%
\pgfsetmiterjoin%
\definecolor{currentfill}{rgb}{1.000000,1.000000,1.000000}%
\pgfsetfillcolor{currentfill}%
\pgfsetlinewidth{0.000000pt}%
\definecolor{currentstroke}{rgb}{0.000000,0.000000,0.000000}%
\pgfsetstrokecolor{currentstroke}%
\pgfsetstrokeopacity{0.000000}%
\pgfsetdash{}{0pt}%
\pgfpathmoveto{\pgfqpoint{0.795366in}{0.646140in}}%
\pgfpathlineto{\pgfqpoint{5.824769in}{0.646140in}}%
\pgfpathlineto{\pgfqpoint{5.824769in}{3.734428in}}%
\pgfpathlineto{\pgfqpoint{0.795366in}{3.734428in}}%
\pgfpathlineto{\pgfqpoint{0.795366in}{0.646140in}}%
\pgfpathclose%
\pgfusepath{fill}%
\end{pgfscope}%
\begin{pgfscope}%
\pgfsetbuttcap%
\pgfsetroundjoin%
\definecolor{currentfill}{rgb}{0.000000,0.000000,0.000000}%
\pgfsetfillcolor{currentfill}%
\pgfsetlinewidth{0.803000pt}%
\definecolor{currentstroke}{rgb}{0.000000,0.000000,0.000000}%
\pgfsetstrokecolor{currentstroke}%
\pgfsetdash{}{0pt}%
\pgfsys@defobject{currentmarker}{\pgfqpoint{0.000000in}{-0.048611in}}{\pgfqpoint{0.000000in}{0.000000in}}{%
\pgfpathmoveto{\pgfqpoint{0.000000in}{0.000000in}}%
\pgfpathlineto{\pgfqpoint{0.000000in}{-0.048611in}}%
\pgfusepath{stroke,fill}%
}%
\begin{pgfscope}%
\pgfsys@transformshift{1.354188in}{0.646140in}%
\pgfsys@useobject{currentmarker}{}%
\end{pgfscope}%
\end{pgfscope}%
\begin{pgfscope}%
\definecolor{textcolor}{rgb}{0.000000,0.000000,0.000000}%
\pgfsetstrokecolor{textcolor}%
\pgfsetfillcolor{textcolor}%
\pgftext[x=1.354188in,y=0.548917in,,top]{\color{textcolor}\rmfamily\fontsize{14.000000}{16.800000}\selectfont \(\displaystyle {2700}\)}%
\end{pgfscope}%
\begin{pgfscope}%
\pgfsetbuttcap%
\pgfsetroundjoin%
\definecolor{currentfill}{rgb}{0.000000,0.000000,0.000000}%
\pgfsetfillcolor{currentfill}%
\pgfsetlinewidth{0.803000pt}%
\definecolor{currentstroke}{rgb}{0.000000,0.000000,0.000000}%
\pgfsetstrokecolor{currentstroke}%
\pgfsetdash{}{0pt}%
\pgfsys@defobject{currentmarker}{\pgfqpoint{0.000000in}{-0.048611in}}{\pgfqpoint{0.000000in}{0.000000in}}{%
\pgfpathmoveto{\pgfqpoint{0.000000in}{0.000000in}}%
\pgfpathlineto{\pgfqpoint{0.000000in}{-0.048611in}}%
\pgfusepath{stroke,fill}%
}%
\begin{pgfscope}%
\pgfsys@transformshift{2.471834in}{0.646140in}%
\pgfsys@useobject{currentmarker}{}%
\end{pgfscope}%
\end{pgfscope}%
\begin{pgfscope}%
\definecolor{textcolor}{rgb}{0.000000,0.000000,0.000000}%
\pgfsetstrokecolor{textcolor}%
\pgfsetfillcolor{textcolor}%
\pgftext[x=2.471834in,y=0.548917in,,top]{\color{textcolor}\rmfamily\fontsize{14.000000}{16.800000}\selectfont \(\displaystyle {2800}\)}%
\end{pgfscope}%
\begin{pgfscope}%
\pgfsetbuttcap%
\pgfsetroundjoin%
\definecolor{currentfill}{rgb}{0.000000,0.000000,0.000000}%
\pgfsetfillcolor{currentfill}%
\pgfsetlinewidth{0.803000pt}%
\definecolor{currentstroke}{rgb}{0.000000,0.000000,0.000000}%
\pgfsetstrokecolor{currentstroke}%
\pgfsetdash{}{0pt}%
\pgfsys@defobject{currentmarker}{\pgfqpoint{0.000000in}{-0.048611in}}{\pgfqpoint{0.000000in}{0.000000in}}{%
\pgfpathmoveto{\pgfqpoint{0.000000in}{0.000000in}}%
\pgfpathlineto{\pgfqpoint{0.000000in}{-0.048611in}}%
\pgfusepath{stroke,fill}%
}%
\begin{pgfscope}%
\pgfsys@transformshift{3.589479in}{0.646140in}%
\pgfsys@useobject{currentmarker}{}%
\end{pgfscope}%
\end{pgfscope}%
\begin{pgfscope}%
\definecolor{textcolor}{rgb}{0.000000,0.000000,0.000000}%
\pgfsetstrokecolor{textcolor}%
\pgfsetfillcolor{textcolor}%
\pgftext[x=3.589479in,y=0.548917in,,top]{\color{textcolor}\rmfamily\fontsize{14.000000}{16.800000}\selectfont \(\displaystyle {2900}\)}%
\end{pgfscope}%
\begin{pgfscope}%
\pgfsetbuttcap%
\pgfsetroundjoin%
\definecolor{currentfill}{rgb}{0.000000,0.000000,0.000000}%
\pgfsetfillcolor{currentfill}%
\pgfsetlinewidth{0.803000pt}%
\definecolor{currentstroke}{rgb}{0.000000,0.000000,0.000000}%
\pgfsetstrokecolor{currentstroke}%
\pgfsetdash{}{0pt}%
\pgfsys@defobject{currentmarker}{\pgfqpoint{0.000000in}{-0.048611in}}{\pgfqpoint{0.000000in}{0.000000in}}{%
\pgfpathmoveto{\pgfqpoint{0.000000in}{0.000000in}}%
\pgfpathlineto{\pgfqpoint{0.000000in}{-0.048611in}}%
\pgfusepath{stroke,fill}%
}%
\begin{pgfscope}%
\pgfsys@transformshift{4.707124in}{0.646140in}%
\pgfsys@useobject{currentmarker}{}%
\end{pgfscope}%
\end{pgfscope}%
\begin{pgfscope}%
\definecolor{textcolor}{rgb}{0.000000,0.000000,0.000000}%
\pgfsetstrokecolor{textcolor}%
\pgfsetfillcolor{textcolor}%
\pgftext[x=4.707124in,y=0.548917in,,top]{\color{textcolor}\rmfamily\fontsize{14.000000}{16.800000}\selectfont \(\displaystyle {3000}\)}%
\end{pgfscope}%
\begin{pgfscope}%
\pgfsetbuttcap%
\pgfsetroundjoin%
\definecolor{currentfill}{rgb}{0.000000,0.000000,0.000000}%
\pgfsetfillcolor{currentfill}%
\pgfsetlinewidth{0.803000pt}%
\definecolor{currentstroke}{rgb}{0.000000,0.000000,0.000000}%
\pgfsetstrokecolor{currentstroke}%
\pgfsetdash{}{0pt}%
\pgfsys@defobject{currentmarker}{\pgfqpoint{0.000000in}{-0.048611in}}{\pgfqpoint{0.000000in}{0.000000in}}{%
\pgfpathmoveto{\pgfqpoint{0.000000in}{0.000000in}}%
\pgfpathlineto{\pgfqpoint{0.000000in}{-0.048611in}}%
\pgfusepath{stroke,fill}%
}%
\begin{pgfscope}%
\pgfsys@transformshift{5.824769in}{0.646140in}%
\pgfsys@useobject{currentmarker}{}%
\end{pgfscope}%
\end{pgfscope}%
\begin{pgfscope}%
\definecolor{textcolor}{rgb}{0.000000,0.000000,0.000000}%
\pgfsetstrokecolor{textcolor}%
\pgfsetfillcolor{textcolor}%
\pgftext[x=5.824769in,y=0.548917in,,top]{\color{textcolor}\rmfamily\fontsize{14.000000}{16.800000}\selectfont \(\displaystyle {3100}\)}%
\end{pgfscope}%
\begin{pgfscope}%
\definecolor{textcolor}{rgb}{0.000000,0.000000,0.000000}%
\pgfsetstrokecolor{textcolor}%
\pgfsetfillcolor{textcolor}%
\pgftext[x=3.310068in,y=0.320695in,,top]{\color{textcolor}\rmfamily\fontsize{14.000000}{16.800000}\selectfont Wavenumber [cm\(\displaystyle ^{-1}\)]}%
\end{pgfscope}%
\begin{pgfscope}%
\pgfsetbuttcap%
\pgfsetroundjoin%
\definecolor{currentfill}{rgb}{0.000000,0.000000,0.000000}%
\pgfsetfillcolor{currentfill}%
\pgfsetlinewidth{0.803000pt}%
\definecolor{currentstroke}{rgb}{0.000000,0.000000,0.000000}%
\pgfsetstrokecolor{currentstroke}%
\pgfsetdash{}{0pt}%
\pgfsys@defobject{currentmarker}{\pgfqpoint{-0.048611in}{0.000000in}}{\pgfqpoint{-0.000000in}{0.000000in}}{%
\pgfpathmoveto{\pgfqpoint{-0.000000in}{0.000000in}}%
\pgfpathlineto{\pgfqpoint{-0.048611in}{0.000000in}}%
\pgfusepath{stroke,fill}%
}%
\begin{pgfscope}%
\pgfsys@transformshift{0.795366in}{0.646140in}%
\pgfsys@useobject{currentmarker}{}%
\end{pgfscope}%
\end{pgfscope}%
\begin{pgfscope}%
\definecolor{textcolor}{rgb}{0.000000,0.000000,0.000000}%
\pgfsetstrokecolor{textcolor}%
\pgfsetfillcolor{textcolor}%
\pgftext[x=0.350000in, y=0.578667in, left, base]{\color{textcolor}\rmfamily\fontsize{14.000000}{16.800000}\selectfont \(\displaystyle {0.40}\)}%
\end{pgfscope}%
\begin{pgfscope}%
\pgfsetbuttcap%
\pgfsetroundjoin%
\definecolor{currentfill}{rgb}{0.000000,0.000000,0.000000}%
\pgfsetfillcolor{currentfill}%
\pgfsetlinewidth{0.803000pt}%
\definecolor{currentstroke}{rgb}{0.000000,0.000000,0.000000}%
\pgfsetstrokecolor{currentstroke}%
\pgfsetdash{}{0pt}%
\pgfsys@defobject{currentmarker}{\pgfqpoint{-0.048611in}{0.000000in}}{\pgfqpoint{-0.000000in}{0.000000in}}{%
\pgfpathmoveto{\pgfqpoint{-0.000000in}{0.000000in}}%
\pgfpathlineto{\pgfqpoint{-0.048611in}{0.000000in}}%
\pgfusepath{stroke,fill}%
}%
\begin{pgfscope}%
\pgfsys@transformshift{0.795366in}{1.160854in}%
\pgfsys@useobject{currentmarker}{}%
\end{pgfscope}%
\end{pgfscope}%
\begin{pgfscope}%
\definecolor{textcolor}{rgb}{0.000000,0.000000,0.000000}%
\pgfsetstrokecolor{textcolor}%
\pgfsetfillcolor{textcolor}%
\pgftext[x=0.350000in, y=1.093382in, left, base]{\color{textcolor}\rmfamily\fontsize{14.000000}{16.800000}\selectfont \(\displaystyle {0.45}\)}%
\end{pgfscope}%
\begin{pgfscope}%
\pgfsetbuttcap%
\pgfsetroundjoin%
\definecolor{currentfill}{rgb}{0.000000,0.000000,0.000000}%
\pgfsetfillcolor{currentfill}%
\pgfsetlinewidth{0.803000pt}%
\definecolor{currentstroke}{rgb}{0.000000,0.000000,0.000000}%
\pgfsetstrokecolor{currentstroke}%
\pgfsetdash{}{0pt}%
\pgfsys@defobject{currentmarker}{\pgfqpoint{-0.048611in}{0.000000in}}{\pgfqpoint{-0.000000in}{0.000000in}}{%
\pgfpathmoveto{\pgfqpoint{-0.000000in}{0.000000in}}%
\pgfpathlineto{\pgfqpoint{-0.048611in}{0.000000in}}%
\pgfusepath{stroke,fill}%
}%
\begin{pgfscope}%
\pgfsys@transformshift{0.795366in}{1.675569in}%
\pgfsys@useobject{currentmarker}{}%
\end{pgfscope}%
\end{pgfscope}%
\begin{pgfscope}%
\definecolor{textcolor}{rgb}{0.000000,0.000000,0.000000}%
\pgfsetstrokecolor{textcolor}%
\pgfsetfillcolor{textcolor}%
\pgftext[x=0.350000in, y=1.608097in, left, base]{\color{textcolor}\rmfamily\fontsize{14.000000}{16.800000}\selectfont \(\displaystyle {0.50}\)}%
\end{pgfscope}%
\begin{pgfscope}%
\pgfsetbuttcap%
\pgfsetroundjoin%
\definecolor{currentfill}{rgb}{0.000000,0.000000,0.000000}%
\pgfsetfillcolor{currentfill}%
\pgfsetlinewidth{0.803000pt}%
\definecolor{currentstroke}{rgb}{0.000000,0.000000,0.000000}%
\pgfsetstrokecolor{currentstroke}%
\pgfsetdash{}{0pt}%
\pgfsys@defobject{currentmarker}{\pgfqpoint{-0.048611in}{0.000000in}}{\pgfqpoint{-0.000000in}{0.000000in}}{%
\pgfpathmoveto{\pgfqpoint{-0.000000in}{0.000000in}}%
\pgfpathlineto{\pgfqpoint{-0.048611in}{0.000000in}}%
\pgfusepath{stroke,fill}%
}%
\begin{pgfscope}%
\pgfsys@transformshift{0.795366in}{2.190284in}%
\pgfsys@useobject{currentmarker}{}%
\end{pgfscope}%
\end{pgfscope}%
\begin{pgfscope}%
\definecolor{textcolor}{rgb}{0.000000,0.000000,0.000000}%
\pgfsetstrokecolor{textcolor}%
\pgfsetfillcolor{textcolor}%
\pgftext[x=0.350000in, y=2.122812in, left, base]{\color{textcolor}\rmfamily\fontsize{14.000000}{16.800000}\selectfont \(\displaystyle {0.55}\)}%
\end{pgfscope}%
\begin{pgfscope}%
\pgfsetbuttcap%
\pgfsetroundjoin%
\definecolor{currentfill}{rgb}{0.000000,0.000000,0.000000}%
\pgfsetfillcolor{currentfill}%
\pgfsetlinewidth{0.803000pt}%
\definecolor{currentstroke}{rgb}{0.000000,0.000000,0.000000}%
\pgfsetstrokecolor{currentstroke}%
\pgfsetdash{}{0pt}%
\pgfsys@defobject{currentmarker}{\pgfqpoint{-0.048611in}{0.000000in}}{\pgfqpoint{-0.000000in}{0.000000in}}{%
\pgfpathmoveto{\pgfqpoint{-0.000000in}{0.000000in}}%
\pgfpathlineto{\pgfqpoint{-0.048611in}{0.000000in}}%
\pgfusepath{stroke,fill}%
}%
\begin{pgfscope}%
\pgfsys@transformshift{0.795366in}{2.704999in}%
\pgfsys@useobject{currentmarker}{}%
\end{pgfscope}%
\end{pgfscope}%
\begin{pgfscope}%
\definecolor{textcolor}{rgb}{0.000000,0.000000,0.000000}%
\pgfsetstrokecolor{textcolor}%
\pgfsetfillcolor{textcolor}%
\pgftext[x=0.350000in, y=2.637526in, left, base]{\color{textcolor}\rmfamily\fontsize{14.000000}{16.800000}\selectfont \(\displaystyle {0.60}\)}%
\end{pgfscope}%
\begin{pgfscope}%
\pgfsetbuttcap%
\pgfsetroundjoin%
\definecolor{currentfill}{rgb}{0.000000,0.000000,0.000000}%
\pgfsetfillcolor{currentfill}%
\pgfsetlinewidth{0.803000pt}%
\definecolor{currentstroke}{rgb}{0.000000,0.000000,0.000000}%
\pgfsetstrokecolor{currentstroke}%
\pgfsetdash{}{0pt}%
\pgfsys@defobject{currentmarker}{\pgfqpoint{-0.048611in}{0.000000in}}{\pgfqpoint{-0.000000in}{0.000000in}}{%
\pgfpathmoveto{\pgfqpoint{-0.000000in}{0.000000in}}%
\pgfpathlineto{\pgfqpoint{-0.048611in}{0.000000in}}%
\pgfusepath{stroke,fill}%
}%
\begin{pgfscope}%
\pgfsys@transformshift{0.795366in}{3.219713in}%
\pgfsys@useobject{currentmarker}{}%
\end{pgfscope}%
\end{pgfscope}%
\begin{pgfscope}%
\definecolor{textcolor}{rgb}{0.000000,0.000000,0.000000}%
\pgfsetstrokecolor{textcolor}%
\pgfsetfillcolor{textcolor}%
\pgftext[x=0.350000in, y=3.152241in, left, base]{\color{textcolor}\rmfamily\fontsize{14.000000}{16.800000}\selectfont \(\displaystyle {0.65}\)}%
\end{pgfscope}%
\begin{pgfscope}%
\pgfsetbuttcap%
\pgfsetroundjoin%
\definecolor{currentfill}{rgb}{0.000000,0.000000,0.000000}%
\pgfsetfillcolor{currentfill}%
\pgfsetlinewidth{0.803000pt}%
\definecolor{currentstroke}{rgb}{0.000000,0.000000,0.000000}%
\pgfsetstrokecolor{currentstroke}%
\pgfsetdash{}{0pt}%
\pgfsys@defobject{currentmarker}{\pgfqpoint{-0.048611in}{0.000000in}}{\pgfqpoint{-0.000000in}{0.000000in}}{%
\pgfpathmoveto{\pgfqpoint{-0.000000in}{0.000000in}}%
\pgfpathlineto{\pgfqpoint{-0.048611in}{0.000000in}}%
\pgfusepath{stroke,fill}%
}%
\begin{pgfscope}%
\pgfsys@transformshift{0.795366in}{3.734428in}%
\pgfsys@useobject{currentmarker}{}%
\end{pgfscope}%
\end{pgfscope}%
\begin{pgfscope}%
\definecolor{textcolor}{rgb}{0.000000,0.000000,0.000000}%
\pgfsetstrokecolor{textcolor}%
\pgfsetfillcolor{textcolor}%
\pgftext[x=0.350000in, y=3.666956in, left, base]{\color{textcolor}\rmfamily\fontsize{14.000000}{16.800000}\selectfont \(\displaystyle {0.70}\)}%
\end{pgfscope}%
\begin{pgfscope}%
\definecolor{textcolor}{rgb}{0.000000,0.000000,0.000000}%
\pgfsetstrokecolor{textcolor}%
\pgfsetfillcolor{textcolor}%
\pgftext[x=0.294444in,y=2.190284in,,bottom,rotate=90.000000]{\color{textcolor}\rmfamily\fontsize{14.000000}{16.800000}\selectfont Transmittance [\(\displaystyle \%\)]}%
\end{pgfscope}%
\begin{pgfscope}%
\pgfpathrectangle{\pgfqpoint{0.795366in}{0.646140in}}{\pgfqpoint{5.029404in}{3.088289in}}%
\pgfusepath{clip}%
\pgfsetbuttcap%
\pgfsetroundjoin%
\definecolor{currentfill}{rgb}{1.000000,0.000000,0.000000}%
\pgfsetfillcolor{currentfill}%
\pgfsetlinewidth{1.003750pt}%
\definecolor{currentstroke}{rgb}{1.000000,0.000000,0.000000}%
\pgfsetstrokecolor{currentstroke}%
\pgfsetdash{}{0pt}%
\pgfsys@defobject{currentmarker}{\pgfqpoint{-0.020833in}{-0.020833in}}{\pgfqpoint{0.020833in}{0.020833in}}{%
\pgfpathmoveto{\pgfqpoint{0.000000in}{-0.020833in}}%
\pgfpathcurveto{\pgfqpoint{0.005525in}{-0.020833in}}{\pgfqpoint{0.010825in}{-0.018638in}}{\pgfqpoint{0.014731in}{-0.014731in}}%
\pgfpathcurveto{\pgfqpoint{0.018638in}{-0.010825in}}{\pgfqpoint{0.020833in}{-0.005525in}}{\pgfqpoint{0.020833in}{0.000000in}}%
\pgfpathcurveto{\pgfqpoint{0.020833in}{0.005525in}}{\pgfqpoint{0.018638in}{0.010825in}}{\pgfqpoint{0.014731in}{0.014731in}}%
\pgfpathcurveto{\pgfqpoint{0.010825in}{0.018638in}}{\pgfqpoint{0.005525in}{0.020833in}}{\pgfqpoint{0.000000in}{0.020833in}}%
\pgfpathcurveto{\pgfqpoint{-0.005525in}{0.020833in}}{\pgfqpoint{-0.010825in}{0.018638in}}{\pgfqpoint{-0.014731in}{0.014731in}}%
\pgfpathcurveto{\pgfqpoint{-0.018638in}{0.010825in}}{\pgfqpoint{-0.020833in}{0.005525in}}{\pgfqpoint{-0.020833in}{0.000000in}}%
\pgfpathcurveto{\pgfqpoint{-0.020833in}{-0.005525in}}{\pgfqpoint{-0.018638in}{-0.010825in}}{\pgfqpoint{-0.014731in}{-0.014731in}}%
\pgfpathcurveto{\pgfqpoint{-0.010825in}{-0.018638in}}{\pgfqpoint{-0.005525in}{-0.020833in}}{\pgfqpoint{0.000000in}{-0.020833in}}%
\pgfpathlineto{\pgfqpoint{0.000000in}{-0.020833in}}%
\pgfpathclose%
\pgfusepath{stroke,fill}%
}%
\begin{pgfscope}%
\pgfsys@transformshift{5.077344in}{2.837211in}%
\pgfsys@useobject{currentmarker}{}%
\end{pgfscope}%
\begin{pgfscope}%
\pgfsys@transformshift{5.050800in}{3.074022in}%
\pgfsys@useobject{currentmarker}{}%
\end{pgfscope}%
\begin{pgfscope}%
\pgfsys@transformshift{4.901315in}{2.587804in}%
\pgfsys@useobject{currentmarker}{}%
\end{pgfscope}%
\begin{pgfscope}%
\pgfsys@transformshift{4.876168in}{2.975439in}%
\pgfsys@useobject{currentmarker}{}%
\end{pgfscope}%
\begin{pgfscope}%
\pgfsys@transformshift{4.718301in}{2.297821in}%
\pgfsys@useobject{currentmarker}{}%
\end{pgfscope}%
\begin{pgfscope}%
\pgfsys@transformshift{4.693154in}{2.793635in}%
\pgfsys@useobject{currentmarker}{}%
\end{pgfscope}%
\begin{pgfscope}%
\pgfsys@transformshift{4.528301in}{1.899341in}%
\pgfsys@useobject{currentmarker}{}%
\end{pgfscope}%
\begin{pgfscope}%
\pgfsys@transformshift{4.503154in}{2.477932in}%
\pgfsys@useobject{currentmarker}{}%
\end{pgfscope}%
\begin{pgfscope}%
\pgfsys@transformshift{4.329919in}{0.927521in}%
\pgfsys@useobject{currentmarker}{}%
\end{pgfscope}%
\begin{pgfscope}%
\pgfsys@transformshift{4.304772in}{1.555836in}%
\pgfsys@useobject{currentmarker}{}%
\end{pgfscope}%
\begin{pgfscope}%
\pgfsys@transformshift{4.124551in}{1.553484in}%
\pgfsys@useobject{currentmarker}{}%
\end{pgfscope}%
\begin{pgfscope}%
\pgfsys@transformshift{4.099404in}{2.239707in}%
\pgfsys@useobject{currentmarker}{}%
\end{pgfscope}%
\begin{pgfscope}%
\pgfsys@transformshift{3.910802in}{1.577541in}%
\pgfsys@useobject{currentmarker}{}%
\end{pgfscope}%
\begin{pgfscope}%
\pgfsys@transformshift{3.887052in}{2.237151in}%
\pgfsys@useobject{currentmarker}{}%
\end{pgfscope}%
\begin{pgfscope}%
\pgfsys@transformshift{3.691464in}{1.919599in}%
\pgfsys@useobject{currentmarker}{}%
\end{pgfscope}%
\begin{pgfscope}%
\pgfsys@transformshift{3.667714in}{2.453588in}%
\pgfsys@useobject{currentmarker}{}%
\end{pgfscope}%
\begin{pgfscope}%
\pgfsys@transformshift{3.230435in}{1.940877in}%
\pgfsys@useobject{currentmarker}{}%
\end{pgfscope}%
\begin{pgfscope}%
\pgfsys@transformshift{3.208082in}{2.422398in}%
\pgfsys@useobject{currentmarker}{}%
\end{pgfscope}%
\begin{pgfscope}%
\pgfsys@transformshift{2.990142in}{1.449451in}%
\pgfsys@useobject{currentmarker}{}%
\end{pgfscope}%
\begin{pgfscope}%
\pgfsys@transformshift{2.967789in}{2.144803in}%
\pgfsys@useobject{currentmarker}{}%
\end{pgfscope}%
\begin{pgfscope}%
\pgfsys@transformshift{2.744260in}{1.282857in}%
\pgfsys@useobject{currentmarker}{}%
\end{pgfscope}%
\begin{pgfscope}%
\pgfsys@transformshift{2.721907in}{1.921625in}%
\pgfsys@useobject{currentmarker}{}%
\end{pgfscope}%
\begin{pgfscope}%
\pgfsys@transformshift{2.491392in}{1.325423in}%
\pgfsys@useobject{currentmarker}{}%
\end{pgfscope}%
\begin{pgfscope}%
\pgfsys@transformshift{2.469039in}{1.866335in}%
\pgfsys@useobject{currentmarker}{}%
\end{pgfscope}%
\begin{pgfscope}%
\pgfsys@transformshift{2.231540in}{1.351365in}%
\pgfsys@useobject{currentmarker}{}%
\end{pgfscope}%
\begin{pgfscope}%
\pgfsys@transformshift{2.209187in}{1.851615in}%
\pgfsys@useobject{currentmarker}{}%
\end{pgfscope}%
\begin{pgfscope}%
\pgfsys@transformshift{1.966099in}{1.452389in}%
\pgfsys@useobject{currentmarker}{}%
\end{pgfscope}%
\begin{pgfscope}%
\pgfsys@transformshift{1.945143in}{1.792400in}%
\pgfsys@useobject{currentmarker}{}%
\end{pgfscope}%
\begin{pgfscope}%
\pgfsys@transformshift{1.695070in}{1.403198in}%
\pgfsys@useobject{currentmarker}{}%
\end{pgfscope}%
\begin{pgfscope}%
\pgfsys@transformshift{1.674114in}{1.618300in}%
\pgfsys@useobject{currentmarker}{}%
\end{pgfscope}%
\begin{pgfscope}%
\pgfsys@transformshift{1.418453in}{1.260974in}%
\pgfsys@useobject{currentmarker}{}%
\end{pgfscope}%
\begin{pgfscope}%
\pgfsys@transformshift{1.397497in}{1.351459in}%
\pgfsys@useobject{currentmarker}{}%
\end{pgfscope}%
\end{pgfscope}%
\begin{pgfscope}%
\pgfpathrectangle{\pgfqpoint{0.795366in}{0.646140in}}{\pgfqpoint{5.029404in}{3.088289in}}%
\pgfusepath{clip}%
\pgfsetrectcap%
\pgfsetroundjoin%
\pgfsetlinewidth{0.501875pt}%
\definecolor{currentstroke}{rgb}{0.000000,0.000000,0.000000}%
\pgfsetstrokecolor{currentstroke}%
\pgfsetdash{}{0pt}%
\pgfpathmoveto{\pgfqpoint{5.824769in}{3.182285in}}%
\pgfpathlineto{\pgfqpoint{5.819181in}{3.209055in}}%
\pgfpathlineto{\pgfqpoint{5.817784in}{3.210373in}}%
\pgfpathlineto{\pgfqpoint{5.816387in}{3.205926in}}%
\pgfpathlineto{\pgfqpoint{5.812196in}{3.186084in}}%
\pgfpathlineto{\pgfqpoint{5.810799in}{3.183683in}}%
\pgfpathlineto{\pgfqpoint{5.808005in}{3.160294in}}%
\pgfpathlineto{\pgfqpoint{5.805211in}{3.142995in}}%
\pgfpathlineto{\pgfqpoint{5.802416in}{3.162568in}}%
\pgfpathlineto{\pgfqpoint{5.799622in}{3.179307in}}%
\pgfpathlineto{\pgfqpoint{5.798225in}{3.180545in}}%
\pgfpathlineto{\pgfqpoint{5.796828in}{3.179347in}}%
\pgfpathlineto{\pgfqpoint{5.795431in}{3.180281in}}%
\pgfpathlineto{\pgfqpoint{5.794034in}{3.188164in}}%
\pgfpathlineto{\pgfqpoint{5.788446in}{3.247302in}}%
\pgfpathlineto{\pgfqpoint{5.787049in}{3.239373in}}%
\pgfpathlineto{\pgfqpoint{5.782858in}{3.185845in}}%
\pgfpathlineto{\pgfqpoint{5.774475in}{3.156848in}}%
\pgfpathlineto{\pgfqpoint{5.771681in}{3.159091in}}%
\pgfpathlineto{\pgfqpoint{5.768887in}{3.168935in}}%
\pgfpathlineto{\pgfqpoint{5.766093in}{3.188255in}}%
\pgfpathlineto{\pgfqpoint{5.761902in}{3.227104in}}%
\pgfpathlineto{\pgfqpoint{5.759108in}{3.232758in}}%
\pgfpathlineto{\pgfqpoint{5.756314in}{3.235729in}}%
\pgfpathlineto{\pgfqpoint{5.754916in}{3.231386in}}%
\pgfpathlineto{\pgfqpoint{5.750725in}{3.201314in}}%
\pgfpathlineto{\pgfqpoint{5.749328in}{3.203577in}}%
\pgfpathlineto{\pgfqpoint{5.747931in}{3.207984in}}%
\pgfpathlineto{\pgfqpoint{5.746534in}{3.205740in}}%
\pgfpathlineto{\pgfqpoint{5.742343in}{3.166361in}}%
\pgfpathlineto{\pgfqpoint{5.740946in}{3.167848in}}%
\pgfpathlineto{\pgfqpoint{5.736755in}{3.191247in}}%
\pgfpathlineto{\pgfqpoint{5.735358in}{3.188893in}}%
\pgfpathlineto{\pgfqpoint{5.729769in}{3.171292in}}%
\pgfpathlineto{\pgfqpoint{5.728372in}{3.173627in}}%
\pgfpathlineto{\pgfqpoint{5.722784in}{3.192960in}}%
\pgfpathlineto{\pgfqpoint{5.719990in}{3.194703in}}%
\pgfpathlineto{\pgfqpoint{5.718593in}{3.194238in}}%
\pgfpathlineto{\pgfqpoint{5.717196in}{3.191974in}}%
\pgfpathlineto{\pgfqpoint{5.711608in}{3.171171in}}%
\pgfpathlineto{\pgfqpoint{5.710211in}{3.175506in}}%
\pgfpathlineto{\pgfqpoint{5.704622in}{3.213391in}}%
\pgfpathlineto{\pgfqpoint{5.703225in}{3.215969in}}%
\pgfpathlineto{\pgfqpoint{5.701828in}{3.214308in}}%
\pgfpathlineto{\pgfqpoint{5.697637in}{3.187243in}}%
\pgfpathlineto{\pgfqpoint{5.696240in}{3.191677in}}%
\pgfpathlineto{\pgfqpoint{5.693446in}{3.215665in}}%
\pgfpathlineto{\pgfqpoint{5.692049in}{3.216483in}}%
\pgfpathlineto{\pgfqpoint{5.687858in}{3.201873in}}%
\pgfpathlineto{\pgfqpoint{5.686461in}{3.201027in}}%
\pgfpathlineto{\pgfqpoint{5.685064in}{3.196406in}}%
\pgfpathlineto{\pgfqpoint{5.680872in}{3.171414in}}%
\pgfpathlineto{\pgfqpoint{5.679475in}{3.171120in}}%
\pgfpathlineto{\pgfqpoint{5.673887in}{3.183253in}}%
\pgfpathlineto{\pgfqpoint{5.671093in}{3.209633in}}%
\pgfpathlineto{\pgfqpoint{5.668299in}{3.241398in}}%
\pgfpathlineto{\pgfqpoint{5.666902in}{3.245247in}}%
\pgfpathlineto{\pgfqpoint{5.664108in}{3.231725in}}%
\pgfpathlineto{\pgfqpoint{5.661314in}{3.219578in}}%
\pgfpathlineto{\pgfqpoint{5.659917in}{3.219105in}}%
\pgfpathlineto{\pgfqpoint{5.657123in}{3.223688in}}%
\pgfpathlineto{\pgfqpoint{5.655725in}{3.222788in}}%
\pgfpathlineto{\pgfqpoint{5.651534in}{3.207167in}}%
\pgfpathlineto{\pgfqpoint{5.650137in}{3.208815in}}%
\pgfpathlineto{\pgfqpoint{5.647343in}{3.215708in}}%
\pgfpathlineto{\pgfqpoint{5.644549in}{3.216333in}}%
\pgfpathlineto{\pgfqpoint{5.641755in}{3.212776in}}%
\pgfpathlineto{\pgfqpoint{5.638961in}{3.209855in}}%
\pgfpathlineto{\pgfqpoint{5.633373in}{3.218678in}}%
\pgfpathlineto{\pgfqpoint{5.631976in}{3.219505in}}%
\pgfpathlineto{\pgfqpoint{5.629181in}{3.228947in}}%
\pgfpathlineto{\pgfqpoint{5.627784in}{3.227482in}}%
\pgfpathlineto{\pgfqpoint{5.622196in}{3.181759in}}%
\pgfpathlineto{\pgfqpoint{5.620799in}{3.185625in}}%
\pgfpathlineto{\pgfqpoint{5.619402in}{3.182259in}}%
\pgfpathlineto{\pgfqpoint{5.616608in}{3.164868in}}%
\pgfpathlineto{\pgfqpoint{5.615211in}{3.172072in}}%
\pgfpathlineto{\pgfqpoint{5.611020in}{3.227430in}}%
\pgfpathlineto{\pgfqpoint{5.609623in}{3.230464in}}%
\pgfpathlineto{\pgfqpoint{5.605431in}{3.214874in}}%
\pgfpathlineto{\pgfqpoint{5.602637in}{3.231894in}}%
\pgfpathlineto{\pgfqpoint{5.601240in}{3.231362in}}%
\pgfpathlineto{\pgfqpoint{5.597049in}{3.202915in}}%
\pgfpathlineto{\pgfqpoint{5.594255in}{3.214524in}}%
\pgfpathlineto{\pgfqpoint{5.592858in}{3.219056in}}%
\pgfpathlineto{\pgfqpoint{5.591461in}{3.217273in}}%
\pgfpathlineto{\pgfqpoint{5.588667in}{3.197518in}}%
\pgfpathlineto{\pgfqpoint{5.585873in}{3.180113in}}%
\pgfpathlineto{\pgfqpoint{5.584476in}{3.181816in}}%
\pgfpathlineto{\pgfqpoint{5.581681in}{3.208464in}}%
\pgfpathlineto{\pgfqpoint{5.578887in}{3.227459in}}%
\pgfpathlineto{\pgfqpoint{5.577490in}{3.220740in}}%
\pgfpathlineto{\pgfqpoint{5.574696in}{3.197228in}}%
\pgfpathlineto{\pgfqpoint{5.573299in}{3.194162in}}%
\pgfpathlineto{\pgfqpoint{5.569108in}{3.212286in}}%
\pgfpathlineto{\pgfqpoint{5.567711in}{3.205513in}}%
\pgfpathlineto{\pgfqpoint{5.563520in}{3.166172in}}%
\pgfpathlineto{\pgfqpoint{5.560726in}{3.171763in}}%
\pgfpathlineto{\pgfqpoint{5.556534in}{3.147960in}}%
\pgfpathlineto{\pgfqpoint{5.555137in}{3.159656in}}%
\pgfpathlineto{\pgfqpoint{5.550946in}{3.215875in}}%
\pgfpathlineto{\pgfqpoint{5.549549in}{3.215959in}}%
\pgfpathlineto{\pgfqpoint{5.546755in}{3.211484in}}%
\pgfpathlineto{\pgfqpoint{5.543961in}{3.214559in}}%
\pgfpathlineto{\pgfqpoint{5.541167in}{3.208859in}}%
\pgfpathlineto{\pgfqpoint{5.539770in}{3.212377in}}%
\pgfpathlineto{\pgfqpoint{5.536976in}{3.231903in}}%
\pgfpathlineto{\pgfqpoint{5.535579in}{3.232924in}}%
\pgfpathlineto{\pgfqpoint{5.534182in}{3.221628in}}%
\pgfpathlineto{\pgfqpoint{5.529990in}{3.165908in}}%
\pgfpathlineto{\pgfqpoint{5.528593in}{3.166364in}}%
\pgfpathlineto{\pgfqpoint{5.523005in}{3.202662in}}%
\pgfpathlineto{\pgfqpoint{5.520211in}{3.207829in}}%
\pgfpathlineto{\pgfqpoint{5.517417in}{3.209959in}}%
\pgfpathlineto{\pgfqpoint{5.514623in}{3.223848in}}%
\pgfpathlineto{\pgfqpoint{5.511829in}{3.238856in}}%
\pgfpathlineto{\pgfqpoint{5.510432in}{3.239388in}}%
\pgfpathlineto{\pgfqpoint{5.509035in}{3.234689in}}%
\pgfpathlineto{\pgfqpoint{5.506240in}{3.213366in}}%
\pgfpathlineto{\pgfqpoint{5.502049in}{3.177813in}}%
\pgfpathlineto{\pgfqpoint{5.500652in}{3.174677in}}%
\pgfpathlineto{\pgfqpoint{5.496461in}{3.181088in}}%
\pgfpathlineto{\pgfqpoint{5.495064in}{3.180261in}}%
\pgfpathlineto{\pgfqpoint{5.493667in}{3.182632in}}%
\pgfpathlineto{\pgfqpoint{5.490873in}{3.206077in}}%
\pgfpathlineto{\pgfqpoint{5.485285in}{3.265560in}}%
\pgfpathlineto{\pgfqpoint{5.483888in}{3.269413in}}%
\pgfpathlineto{\pgfqpoint{5.482490in}{3.268174in}}%
\pgfpathlineto{\pgfqpoint{5.479696in}{3.255346in}}%
\pgfpathlineto{\pgfqpoint{5.476902in}{3.241901in}}%
\pgfpathlineto{\pgfqpoint{5.474108in}{3.233370in}}%
\pgfpathlineto{\pgfqpoint{5.467123in}{3.185744in}}%
\pgfpathlineto{\pgfqpoint{5.462932in}{3.177184in}}%
\pgfpathlineto{\pgfqpoint{5.461535in}{3.186809in}}%
\pgfpathlineto{\pgfqpoint{5.458741in}{3.223961in}}%
\pgfpathlineto{\pgfqpoint{5.457343in}{3.226586in}}%
\pgfpathlineto{\pgfqpoint{5.454549in}{3.206900in}}%
\pgfpathlineto{\pgfqpoint{5.453152in}{3.211281in}}%
\pgfpathlineto{\pgfqpoint{5.450358in}{3.236985in}}%
\pgfpathlineto{\pgfqpoint{5.448961in}{3.235448in}}%
\pgfpathlineto{\pgfqpoint{5.446167in}{3.222884in}}%
\pgfpathlineto{\pgfqpoint{5.441976in}{3.239823in}}%
\pgfpathlineto{\pgfqpoint{5.439182in}{3.221277in}}%
\pgfpathlineto{\pgfqpoint{5.436388in}{3.210682in}}%
\pgfpathlineto{\pgfqpoint{5.433593in}{3.218728in}}%
\pgfpathlineto{\pgfqpoint{5.432196in}{3.217406in}}%
\pgfpathlineto{\pgfqpoint{5.428005in}{3.189346in}}%
\pgfpathlineto{\pgfqpoint{5.426608in}{3.192371in}}%
\pgfpathlineto{\pgfqpoint{5.423814in}{3.207683in}}%
\pgfpathlineto{\pgfqpoint{5.421020in}{3.213274in}}%
\pgfpathlineto{\pgfqpoint{5.415432in}{3.234790in}}%
\pgfpathlineto{\pgfqpoint{5.414035in}{3.234799in}}%
\pgfpathlineto{\pgfqpoint{5.412638in}{3.232097in}}%
\pgfpathlineto{\pgfqpoint{5.411241in}{3.222708in}}%
\pgfpathlineto{\pgfqpoint{5.408446in}{3.172854in}}%
\pgfpathlineto{\pgfqpoint{5.405652in}{3.106175in}}%
\pgfpathlineto{\pgfqpoint{5.404255in}{3.094157in}}%
\pgfpathlineto{\pgfqpoint{5.402858in}{3.107933in}}%
\pgfpathlineto{\pgfqpoint{5.397270in}{3.215984in}}%
\pgfpathlineto{\pgfqpoint{5.394476in}{3.224971in}}%
\pgfpathlineto{\pgfqpoint{5.393079in}{3.224126in}}%
\pgfpathlineto{\pgfqpoint{5.388888in}{3.210835in}}%
\pgfpathlineto{\pgfqpoint{5.387491in}{3.214744in}}%
\pgfpathlineto{\pgfqpoint{5.384697in}{3.237919in}}%
\pgfpathlineto{\pgfqpoint{5.383299in}{3.238400in}}%
\pgfpathlineto{\pgfqpoint{5.380505in}{3.193536in}}%
\pgfpathlineto{\pgfqpoint{5.377711in}{3.163833in}}%
\pgfpathlineto{\pgfqpoint{5.374917in}{3.190661in}}%
\pgfpathlineto{\pgfqpoint{5.366535in}{3.274829in}}%
\pgfpathlineto{\pgfqpoint{5.365138in}{3.263511in}}%
\pgfpathlineto{\pgfqpoint{5.360947in}{3.196385in}}%
\pgfpathlineto{\pgfqpoint{5.359549in}{3.193293in}}%
\pgfpathlineto{\pgfqpoint{5.353961in}{3.229473in}}%
\pgfpathlineto{\pgfqpoint{5.349770in}{3.215519in}}%
\pgfpathlineto{\pgfqpoint{5.346976in}{3.219514in}}%
\pgfpathlineto{\pgfqpoint{5.345579in}{3.213529in}}%
\pgfpathlineto{\pgfqpoint{5.342785in}{3.191895in}}%
\pgfpathlineto{\pgfqpoint{5.341388in}{3.189560in}}%
\pgfpathlineto{\pgfqpoint{5.338594in}{3.199697in}}%
\pgfpathlineto{\pgfqpoint{5.335800in}{3.218591in}}%
\pgfpathlineto{\pgfqpoint{5.330211in}{3.272014in}}%
\pgfpathlineto{\pgfqpoint{5.328814in}{3.272836in}}%
\pgfpathlineto{\pgfqpoint{5.327417in}{3.262086in}}%
\pgfpathlineto{\pgfqpoint{5.321829in}{3.191888in}}%
\pgfpathlineto{\pgfqpoint{5.317638in}{3.197936in}}%
\pgfpathlineto{\pgfqpoint{5.316241in}{3.196552in}}%
\pgfpathlineto{\pgfqpoint{5.314844in}{3.198827in}}%
\pgfpathlineto{\pgfqpoint{5.312050in}{3.223060in}}%
\pgfpathlineto{\pgfqpoint{5.309255in}{3.250693in}}%
\pgfpathlineto{\pgfqpoint{5.307858in}{3.255871in}}%
\pgfpathlineto{\pgfqpoint{5.306461in}{3.252869in}}%
\pgfpathlineto{\pgfqpoint{5.303667in}{3.238090in}}%
\pgfpathlineto{\pgfqpoint{5.302270in}{3.240038in}}%
\pgfpathlineto{\pgfqpoint{5.299476in}{3.255994in}}%
\pgfpathlineto{\pgfqpoint{5.298079in}{3.257142in}}%
\pgfpathlineto{\pgfqpoint{5.295285in}{3.251833in}}%
\pgfpathlineto{\pgfqpoint{5.293888in}{3.252036in}}%
\pgfpathlineto{\pgfqpoint{5.291094in}{3.253714in}}%
\pgfpathlineto{\pgfqpoint{5.289697in}{3.248742in}}%
\pgfpathlineto{\pgfqpoint{5.284108in}{3.205341in}}%
\pgfpathlineto{\pgfqpoint{5.282711in}{3.208525in}}%
\pgfpathlineto{\pgfqpoint{5.277123in}{3.237856in}}%
\pgfpathlineto{\pgfqpoint{5.272932in}{3.231177in}}%
\pgfpathlineto{\pgfqpoint{5.270138in}{3.235359in}}%
\pgfpathlineto{\pgfqpoint{5.265947in}{3.245974in}}%
\pgfpathlineto{\pgfqpoint{5.264550in}{3.238579in}}%
\pgfpathlineto{\pgfqpoint{5.258961in}{3.181613in}}%
\pgfpathlineto{\pgfqpoint{5.257564in}{3.186712in}}%
\pgfpathlineto{\pgfqpoint{5.254770in}{3.210082in}}%
\pgfpathlineto{\pgfqpoint{5.253373in}{3.213244in}}%
\pgfpathlineto{\pgfqpoint{5.250579in}{3.212483in}}%
\pgfpathlineto{\pgfqpoint{5.249182in}{3.198748in}}%
\pgfpathlineto{\pgfqpoint{5.244991in}{3.052324in}}%
\pgfpathlineto{\pgfqpoint{5.243594in}{3.055999in}}%
\pgfpathlineto{\pgfqpoint{5.238006in}{3.227914in}}%
\pgfpathlineto{\pgfqpoint{5.236609in}{3.229934in}}%
\pgfpathlineto{\pgfqpoint{5.233814in}{3.217130in}}%
\pgfpathlineto{\pgfqpoint{5.232417in}{3.216717in}}%
\pgfpathlineto{\pgfqpoint{5.229623in}{3.223718in}}%
\pgfpathlineto{\pgfqpoint{5.228226in}{3.222330in}}%
\pgfpathlineto{\pgfqpoint{5.225432in}{3.211099in}}%
\pgfpathlineto{\pgfqpoint{5.224035in}{3.199302in}}%
\pgfpathlineto{\pgfqpoint{5.219844in}{3.125217in}}%
\pgfpathlineto{\pgfqpoint{5.218447in}{3.127812in}}%
\pgfpathlineto{\pgfqpoint{5.212859in}{3.282243in}}%
\pgfpathlineto{\pgfqpoint{5.208667in}{3.274298in}}%
\pgfpathlineto{\pgfqpoint{5.207270in}{3.275297in}}%
\pgfpathlineto{\pgfqpoint{5.205873in}{3.270837in}}%
\pgfpathlineto{\pgfqpoint{5.203079in}{3.243561in}}%
\pgfpathlineto{\pgfqpoint{5.198888in}{3.191163in}}%
\pgfpathlineto{\pgfqpoint{5.197491in}{3.187668in}}%
\pgfpathlineto{\pgfqpoint{5.196094in}{3.188323in}}%
\pgfpathlineto{\pgfqpoint{5.194697in}{3.190959in}}%
\pgfpathlineto{\pgfqpoint{5.191903in}{3.210967in}}%
\pgfpathlineto{\pgfqpoint{5.187712in}{3.241534in}}%
\pgfpathlineto{\pgfqpoint{5.184917in}{3.248777in}}%
\pgfpathlineto{\pgfqpoint{5.180726in}{3.242413in}}%
\pgfpathlineto{\pgfqpoint{5.179329in}{3.242578in}}%
\pgfpathlineto{\pgfqpoint{5.177932in}{3.240446in}}%
\pgfpathlineto{\pgfqpoint{5.173741in}{3.222710in}}%
\pgfpathlineto{\pgfqpoint{5.169550in}{3.203213in}}%
\pgfpathlineto{\pgfqpoint{5.168153in}{3.204022in}}%
\pgfpathlineto{\pgfqpoint{5.165359in}{3.212248in}}%
\pgfpathlineto{\pgfqpoint{5.155579in}{3.266651in}}%
\pgfpathlineto{\pgfqpoint{5.151388in}{3.250859in}}%
\pgfpathlineto{\pgfqpoint{5.149991in}{3.256074in}}%
\pgfpathlineto{\pgfqpoint{5.147197in}{3.274109in}}%
\pgfpathlineto{\pgfqpoint{5.145800in}{3.273742in}}%
\pgfpathlineto{\pgfqpoint{5.143006in}{3.252397in}}%
\pgfpathlineto{\pgfqpoint{5.137417in}{3.205414in}}%
\pgfpathlineto{\pgfqpoint{5.136020in}{3.206802in}}%
\pgfpathlineto{\pgfqpoint{5.130432in}{3.237915in}}%
\pgfpathlineto{\pgfqpoint{5.129035in}{3.238092in}}%
\pgfpathlineto{\pgfqpoint{5.124844in}{3.235443in}}%
\pgfpathlineto{\pgfqpoint{5.123447in}{3.239546in}}%
\pgfpathlineto{\pgfqpoint{5.119256in}{3.260287in}}%
\pgfpathlineto{\pgfqpoint{5.117859in}{3.260225in}}%
\pgfpathlineto{\pgfqpoint{5.116462in}{3.258313in}}%
\pgfpathlineto{\pgfqpoint{5.115065in}{3.258344in}}%
\pgfpathlineto{\pgfqpoint{5.110873in}{3.270007in}}%
\pgfpathlineto{\pgfqpoint{5.109476in}{3.263044in}}%
\pgfpathlineto{\pgfqpoint{5.103888in}{3.199215in}}%
\pgfpathlineto{\pgfqpoint{5.102491in}{3.202484in}}%
\pgfpathlineto{\pgfqpoint{5.099697in}{3.219256in}}%
\pgfpathlineto{\pgfqpoint{5.098300in}{3.219543in}}%
\pgfpathlineto{\pgfqpoint{5.095506in}{3.213832in}}%
\pgfpathlineto{\pgfqpoint{5.094109in}{3.221207in}}%
\pgfpathlineto{\pgfqpoint{5.089918in}{3.263445in}}%
\pgfpathlineto{\pgfqpoint{5.088521in}{3.264575in}}%
\pgfpathlineto{\pgfqpoint{5.084329in}{3.230268in}}%
\pgfpathlineto{\pgfqpoint{5.082932in}{3.211103in}}%
\pgfpathlineto{\pgfqpoint{5.081535in}{3.155044in}}%
\pgfpathlineto{\pgfqpoint{5.077344in}{2.837211in}}%
\pgfpathlineto{\pgfqpoint{5.075947in}{2.857414in}}%
\pgfpathlineto{\pgfqpoint{5.070359in}{3.224061in}}%
\pgfpathlineto{\pgfqpoint{5.066168in}{3.248924in}}%
\pgfpathlineto{\pgfqpoint{5.064771in}{3.255287in}}%
\pgfpathlineto{\pgfqpoint{5.063373in}{3.252990in}}%
\pgfpathlineto{\pgfqpoint{5.059182in}{3.217247in}}%
\pgfpathlineto{\pgfqpoint{5.056388in}{3.220060in}}%
\pgfpathlineto{\pgfqpoint{5.054991in}{3.197765in}}%
\pgfpathlineto{\pgfqpoint{5.050800in}{3.074022in}}%
\pgfpathlineto{\pgfqpoint{5.049403in}{3.096083in}}%
\pgfpathlineto{\pgfqpoint{5.046609in}{3.183529in}}%
\pgfpathlineto{\pgfqpoint{5.045212in}{3.197468in}}%
\pgfpathlineto{\pgfqpoint{5.041021in}{3.177340in}}%
\pgfpathlineto{\pgfqpoint{5.038226in}{3.189115in}}%
\pgfpathlineto{\pgfqpoint{5.035432in}{3.210016in}}%
\pgfpathlineto{\pgfqpoint{5.031241in}{3.259188in}}%
\pgfpathlineto{\pgfqpoint{5.029844in}{3.258670in}}%
\pgfpathlineto{\pgfqpoint{5.025653in}{3.237540in}}%
\pgfpathlineto{\pgfqpoint{5.021462in}{3.233079in}}%
\pgfpathlineto{\pgfqpoint{5.018668in}{3.228203in}}%
\pgfpathlineto{\pgfqpoint{5.015874in}{3.233288in}}%
\pgfpathlineto{\pgfqpoint{5.014477in}{3.231998in}}%
\pgfpathlineto{\pgfqpoint{5.011682in}{3.225590in}}%
\pgfpathlineto{\pgfqpoint{5.008888in}{3.225951in}}%
\pgfpathlineto{\pgfqpoint{5.004697in}{3.209751in}}%
\pgfpathlineto{\pgfqpoint{5.003300in}{3.220030in}}%
\pgfpathlineto{\pgfqpoint{4.999109in}{3.279034in}}%
\pgfpathlineto{\pgfqpoint{4.997712in}{3.275762in}}%
\pgfpathlineto{\pgfqpoint{4.994918in}{3.256163in}}%
\pgfpathlineto{\pgfqpoint{4.993521in}{3.257179in}}%
\pgfpathlineto{\pgfqpoint{4.992124in}{3.261109in}}%
\pgfpathlineto{\pgfqpoint{4.990727in}{3.260505in}}%
\pgfpathlineto{\pgfqpoint{4.987932in}{3.245781in}}%
\pgfpathlineto{\pgfqpoint{4.980947in}{3.200931in}}%
\pgfpathlineto{\pgfqpoint{4.979550in}{3.200372in}}%
\pgfpathlineto{\pgfqpoint{4.978153in}{3.201281in}}%
\pgfpathlineto{\pgfqpoint{4.975359in}{3.210547in}}%
\pgfpathlineto{\pgfqpoint{4.972565in}{3.231618in}}%
\pgfpathlineto{\pgfqpoint{4.968374in}{3.280115in}}%
\pgfpathlineto{\pgfqpoint{4.966977in}{3.278256in}}%
\pgfpathlineto{\pgfqpoint{4.964182in}{3.265818in}}%
\pgfpathlineto{\pgfqpoint{4.961388in}{3.264917in}}%
\pgfpathlineto{\pgfqpoint{4.958594in}{3.246921in}}%
\pgfpathlineto{\pgfqpoint{4.954403in}{3.216781in}}%
\pgfpathlineto{\pgfqpoint{4.951609in}{3.208090in}}%
\pgfpathlineto{\pgfqpoint{4.950212in}{3.207497in}}%
\pgfpathlineto{\pgfqpoint{4.946021in}{3.209764in}}%
\pgfpathlineto{\pgfqpoint{4.944624in}{3.214378in}}%
\pgfpathlineto{\pgfqpoint{4.937638in}{3.273068in}}%
\pgfpathlineto{\pgfqpoint{4.936241in}{3.271684in}}%
\pgfpathlineto{\pgfqpoint{4.934844in}{3.266553in}}%
\pgfpathlineto{\pgfqpoint{4.927859in}{3.215429in}}%
\pgfpathlineto{\pgfqpoint{4.926462in}{3.211939in}}%
\pgfpathlineto{\pgfqpoint{4.925065in}{3.213807in}}%
\pgfpathlineto{\pgfqpoint{4.920874in}{3.240172in}}%
\pgfpathlineto{\pgfqpoint{4.919477in}{3.240622in}}%
\pgfpathlineto{\pgfqpoint{4.915285in}{3.224750in}}%
\pgfpathlineto{\pgfqpoint{4.913888in}{3.230008in}}%
\pgfpathlineto{\pgfqpoint{4.909697in}{3.258430in}}%
\pgfpathlineto{\pgfqpoint{4.908300in}{3.250928in}}%
\pgfpathlineto{\pgfqpoint{4.906903in}{3.198109in}}%
\pgfpathlineto{\pgfqpoint{4.904109in}{2.855383in}}%
\pgfpathlineto{\pgfqpoint{4.901315in}{2.587804in}}%
\pgfpathlineto{\pgfqpoint{4.899918in}{2.677170in}}%
\pgfpathlineto{\pgfqpoint{4.895727in}{3.168644in}}%
\pgfpathlineto{\pgfqpoint{4.892933in}{3.230942in}}%
\pgfpathlineto{\pgfqpoint{4.888741in}{3.217707in}}%
\pgfpathlineto{\pgfqpoint{4.883153in}{3.249994in}}%
\pgfpathlineto{\pgfqpoint{4.881756in}{3.242065in}}%
\pgfpathlineto{\pgfqpoint{4.880359in}{3.199232in}}%
\pgfpathlineto{\pgfqpoint{4.876168in}{2.975439in}}%
\pgfpathlineto{\pgfqpoint{4.874771in}{3.002480in}}%
\pgfpathlineto{\pgfqpoint{4.870580in}{3.226850in}}%
\pgfpathlineto{\pgfqpoint{4.869183in}{3.242602in}}%
\pgfpathlineto{\pgfqpoint{4.864991in}{3.226601in}}%
\pgfpathlineto{\pgfqpoint{4.860800in}{3.231858in}}%
\pgfpathlineto{\pgfqpoint{4.859403in}{3.226905in}}%
\pgfpathlineto{\pgfqpoint{4.855212in}{3.199361in}}%
\pgfpathlineto{\pgfqpoint{4.852418in}{3.226259in}}%
\pgfpathlineto{\pgfqpoint{4.849624in}{3.251177in}}%
\pgfpathlineto{\pgfqpoint{4.846830in}{3.258635in}}%
\pgfpathlineto{\pgfqpoint{4.844036in}{3.263837in}}%
\pgfpathlineto{\pgfqpoint{4.842639in}{3.263722in}}%
\pgfpathlineto{\pgfqpoint{4.841241in}{3.256697in}}%
\pgfpathlineto{\pgfqpoint{4.837050in}{3.217898in}}%
\pgfpathlineto{\pgfqpoint{4.832859in}{3.208888in}}%
\pgfpathlineto{\pgfqpoint{4.830065in}{3.198988in}}%
\pgfpathlineto{\pgfqpoint{4.828668in}{3.200329in}}%
\pgfpathlineto{\pgfqpoint{4.824477in}{3.212832in}}%
\pgfpathlineto{\pgfqpoint{4.821683in}{3.208682in}}%
\pgfpathlineto{\pgfqpoint{4.820286in}{3.211843in}}%
\pgfpathlineto{\pgfqpoint{4.817492in}{3.223110in}}%
\pgfpathlineto{\pgfqpoint{4.816094in}{3.222457in}}%
\pgfpathlineto{\pgfqpoint{4.814697in}{3.223222in}}%
\pgfpathlineto{\pgfqpoint{4.811903in}{3.248270in}}%
\pgfpathlineto{\pgfqpoint{4.810506in}{3.260038in}}%
\pgfpathlineto{\pgfqpoint{4.809109in}{3.258238in}}%
\pgfpathlineto{\pgfqpoint{4.806315in}{3.214902in}}%
\pgfpathlineto{\pgfqpoint{4.803521in}{3.177025in}}%
\pgfpathlineto{\pgfqpoint{4.802124in}{3.177409in}}%
\pgfpathlineto{\pgfqpoint{4.796536in}{3.204372in}}%
\pgfpathlineto{\pgfqpoint{4.795139in}{3.203007in}}%
\pgfpathlineto{\pgfqpoint{4.790947in}{3.181080in}}%
\pgfpathlineto{\pgfqpoint{4.788153in}{3.208274in}}%
\pgfpathlineto{\pgfqpoint{4.785359in}{3.234647in}}%
\pgfpathlineto{\pgfqpoint{4.782565in}{3.244547in}}%
\pgfpathlineto{\pgfqpoint{4.781168in}{3.242457in}}%
\pgfpathlineto{\pgfqpoint{4.774183in}{3.209186in}}%
\pgfpathlineto{\pgfqpoint{4.772786in}{3.204756in}}%
\pgfpathlineto{\pgfqpoint{4.768595in}{3.184001in}}%
\pgfpathlineto{\pgfqpoint{4.767198in}{3.190979in}}%
\pgfpathlineto{\pgfqpoint{4.763006in}{3.222698in}}%
\pgfpathlineto{\pgfqpoint{4.760212in}{3.225372in}}%
\pgfpathlineto{\pgfqpoint{4.756021in}{3.239441in}}%
\pgfpathlineto{\pgfqpoint{4.754624in}{3.235157in}}%
\pgfpathlineto{\pgfqpoint{4.749036in}{3.198878in}}%
\pgfpathlineto{\pgfqpoint{4.746242in}{3.202185in}}%
\pgfpathlineto{\pgfqpoint{4.744845in}{3.198918in}}%
\pgfpathlineto{\pgfqpoint{4.742050in}{3.187409in}}%
\pgfpathlineto{\pgfqpoint{4.740653in}{3.187378in}}%
\pgfpathlineto{\pgfqpoint{4.729477in}{3.227123in}}%
\pgfpathlineto{\pgfqpoint{4.726683in}{3.204482in}}%
\pgfpathlineto{\pgfqpoint{4.725286in}{3.177464in}}%
\pgfpathlineto{\pgfqpoint{4.723889in}{3.096523in}}%
\pgfpathlineto{\pgfqpoint{4.721095in}{2.644247in}}%
\pgfpathlineto{\pgfqpoint{4.718301in}{2.297821in}}%
\pgfpathlineto{\pgfqpoint{4.716903in}{2.411543in}}%
\pgfpathlineto{\pgfqpoint{4.712712in}{3.064045in}}%
\pgfpathlineto{\pgfqpoint{4.709918in}{3.178422in}}%
\pgfpathlineto{\pgfqpoint{4.707124in}{3.197490in}}%
\pgfpathlineto{\pgfqpoint{4.705727in}{3.195350in}}%
\pgfpathlineto{\pgfqpoint{4.704330in}{3.195857in}}%
\pgfpathlineto{\pgfqpoint{4.700139in}{3.214293in}}%
\pgfpathlineto{\pgfqpoint{4.698742in}{3.190405in}}%
\pgfpathlineto{\pgfqpoint{4.695948in}{2.986825in}}%
\pgfpathlineto{\pgfqpoint{4.693154in}{2.793635in}}%
\pgfpathlineto{\pgfqpoint{4.691756in}{2.835037in}}%
\pgfpathlineto{\pgfqpoint{4.687565in}{3.143354in}}%
\pgfpathlineto{\pgfqpoint{4.684771in}{3.184353in}}%
\pgfpathlineto{\pgfqpoint{4.683374in}{3.183712in}}%
\pgfpathlineto{\pgfqpoint{4.679183in}{3.161794in}}%
\pgfpathlineto{\pgfqpoint{4.677786in}{3.166110in}}%
\pgfpathlineto{\pgfqpoint{4.673595in}{3.191791in}}%
\pgfpathlineto{\pgfqpoint{4.670801in}{3.204321in}}%
\pgfpathlineto{\pgfqpoint{4.668006in}{3.219210in}}%
\pgfpathlineto{\pgfqpoint{4.666609in}{3.218650in}}%
\pgfpathlineto{\pgfqpoint{4.663815in}{3.210235in}}%
\pgfpathlineto{\pgfqpoint{4.662418in}{3.211328in}}%
\pgfpathlineto{\pgfqpoint{4.661021in}{3.210192in}}%
\pgfpathlineto{\pgfqpoint{4.658227in}{3.181472in}}%
\pgfpathlineto{\pgfqpoint{4.656830in}{3.168305in}}%
\pgfpathlineto{\pgfqpoint{4.655433in}{3.165774in}}%
\pgfpathlineto{\pgfqpoint{4.654036in}{3.167501in}}%
\pgfpathlineto{\pgfqpoint{4.652639in}{3.163946in}}%
\pgfpathlineto{\pgfqpoint{4.649845in}{3.145914in}}%
\pgfpathlineto{\pgfqpoint{4.648448in}{3.145399in}}%
\pgfpathlineto{\pgfqpoint{4.645654in}{3.166726in}}%
\pgfpathlineto{\pgfqpoint{4.642859in}{3.187463in}}%
\pgfpathlineto{\pgfqpoint{4.641462in}{3.188183in}}%
\pgfpathlineto{\pgfqpoint{4.640065in}{3.185967in}}%
\pgfpathlineto{\pgfqpoint{4.638668in}{3.186576in}}%
\pgfpathlineto{\pgfqpoint{4.633080in}{3.208231in}}%
\pgfpathlineto{\pgfqpoint{4.631683in}{3.204031in}}%
\pgfpathlineto{\pgfqpoint{4.620507in}{3.115212in}}%
\pgfpathlineto{\pgfqpoint{4.619110in}{3.115457in}}%
\pgfpathlineto{\pgfqpoint{4.614918in}{3.136119in}}%
\pgfpathlineto{\pgfqpoint{4.609330in}{3.178296in}}%
\pgfpathlineto{\pgfqpoint{4.607933in}{3.178160in}}%
\pgfpathlineto{\pgfqpoint{4.603742in}{3.158918in}}%
\pgfpathlineto{\pgfqpoint{4.600948in}{3.150696in}}%
\pgfpathlineto{\pgfqpoint{4.599551in}{3.151851in}}%
\pgfpathlineto{\pgfqpoint{4.596757in}{3.157119in}}%
\pgfpathlineto{\pgfqpoint{4.595360in}{3.157296in}}%
\pgfpathlineto{\pgfqpoint{4.593962in}{3.154222in}}%
\pgfpathlineto{\pgfqpoint{4.591168in}{3.134297in}}%
\pgfpathlineto{\pgfqpoint{4.588374in}{3.116807in}}%
\pgfpathlineto{\pgfqpoint{4.586977in}{3.117762in}}%
\pgfpathlineto{\pgfqpoint{4.582786in}{3.132116in}}%
\pgfpathlineto{\pgfqpoint{4.581389in}{3.132465in}}%
\pgfpathlineto{\pgfqpoint{4.577198in}{3.120497in}}%
\pgfpathlineto{\pgfqpoint{4.574404in}{3.143066in}}%
\pgfpathlineto{\pgfqpoint{4.571610in}{3.157323in}}%
\pgfpathlineto{\pgfqpoint{4.568815in}{3.143418in}}%
\pgfpathlineto{\pgfqpoint{4.561830in}{3.100751in}}%
\pgfpathlineto{\pgfqpoint{4.556242in}{3.085113in}}%
\pgfpathlineto{\pgfqpoint{4.547860in}{3.100817in}}%
\pgfpathlineto{\pgfqpoint{4.545066in}{3.107898in}}%
\pgfpathlineto{\pgfqpoint{4.543668in}{3.103703in}}%
\pgfpathlineto{\pgfqpoint{4.535286in}{3.041110in}}%
\pgfpathlineto{\pgfqpoint{4.533889in}{2.983887in}}%
\pgfpathlineto{\pgfqpoint{4.532492in}{2.819530in}}%
\pgfpathlineto{\pgfqpoint{4.528301in}{1.899341in}}%
\pgfpathlineto{\pgfqpoint{4.526904in}{1.909796in}}%
\pgfpathlineto{\pgfqpoint{4.519918in}{3.009977in}}%
\pgfpathlineto{\pgfqpoint{4.517124in}{3.042085in}}%
\pgfpathlineto{\pgfqpoint{4.514330in}{3.044741in}}%
\pgfpathlineto{\pgfqpoint{4.511536in}{3.053912in}}%
\pgfpathlineto{\pgfqpoint{4.510139in}{3.052289in}}%
\pgfpathlineto{\pgfqpoint{4.508742in}{3.025217in}}%
\pgfpathlineto{\pgfqpoint{4.507345in}{2.940291in}}%
\pgfpathlineto{\pgfqpoint{4.503154in}{2.477932in}}%
\pgfpathlineto{\pgfqpoint{4.501757in}{2.482702in}}%
\pgfpathlineto{\pgfqpoint{4.496169in}{2.917302in}}%
\pgfpathlineto{\pgfqpoint{4.494771in}{2.931721in}}%
\pgfpathlineto{\pgfqpoint{4.491977in}{2.931492in}}%
\pgfpathlineto{\pgfqpoint{4.487786in}{2.940134in}}%
\pgfpathlineto{\pgfqpoint{4.484992in}{2.938544in}}%
\pgfpathlineto{\pgfqpoint{4.483595in}{2.940764in}}%
\pgfpathlineto{\pgfqpoint{4.480801in}{2.951597in}}%
\pgfpathlineto{\pgfqpoint{4.479404in}{2.948837in}}%
\pgfpathlineto{\pgfqpoint{4.475213in}{2.906110in}}%
\pgfpathlineto{\pgfqpoint{4.466830in}{2.826519in}}%
\pgfpathlineto{\pgfqpoint{4.465433in}{2.829532in}}%
\pgfpathlineto{\pgfqpoint{4.458448in}{2.855282in}}%
\pgfpathlineto{\pgfqpoint{4.455654in}{2.861923in}}%
\pgfpathlineto{\pgfqpoint{4.454257in}{2.865964in}}%
\pgfpathlineto{\pgfqpoint{4.452860in}{2.865018in}}%
\pgfpathlineto{\pgfqpoint{4.451463in}{2.853475in}}%
\pgfpathlineto{\pgfqpoint{4.445874in}{2.774037in}}%
\pgfpathlineto{\pgfqpoint{4.436095in}{2.695201in}}%
\pgfpathlineto{\pgfqpoint{4.433301in}{2.694625in}}%
\pgfpathlineto{\pgfqpoint{4.431904in}{2.691856in}}%
\pgfpathlineto{\pgfqpoint{4.426316in}{2.664285in}}%
\pgfpathlineto{\pgfqpoint{4.423522in}{2.675886in}}%
\pgfpathlineto{\pgfqpoint{4.422125in}{2.674817in}}%
\pgfpathlineto{\pgfqpoint{4.419330in}{2.640638in}}%
\pgfpathlineto{\pgfqpoint{4.413742in}{2.553697in}}%
\pgfpathlineto{\pgfqpoint{4.406757in}{2.503099in}}%
\pgfpathlineto{\pgfqpoint{4.405360in}{2.498052in}}%
\pgfpathlineto{\pgfqpoint{4.401169in}{2.465614in}}%
\pgfpathlineto{\pgfqpoint{4.399772in}{2.464648in}}%
\pgfpathlineto{\pgfqpoint{4.398375in}{2.465620in}}%
\pgfpathlineto{\pgfqpoint{4.396978in}{2.460723in}}%
\pgfpathlineto{\pgfqpoint{4.389992in}{2.395085in}}%
\pgfpathlineto{\pgfqpoint{4.387198in}{2.379500in}}%
\pgfpathlineto{\pgfqpoint{4.373228in}{2.245503in}}%
\pgfpathlineto{\pgfqpoint{4.370433in}{2.234441in}}%
\pgfpathlineto{\pgfqpoint{4.367639in}{2.243872in}}%
\pgfpathlineto{\pgfqpoint{4.366242in}{2.238184in}}%
\pgfpathlineto{\pgfqpoint{4.363448in}{2.214251in}}%
\pgfpathlineto{\pgfqpoint{4.362051in}{2.214179in}}%
\pgfpathlineto{\pgfqpoint{4.360654in}{2.218130in}}%
\pgfpathlineto{\pgfqpoint{4.359257in}{2.214711in}}%
\pgfpathlineto{\pgfqpoint{4.356463in}{2.171606in}}%
\pgfpathlineto{\pgfqpoint{4.352272in}{2.094460in}}%
\pgfpathlineto{\pgfqpoint{4.350875in}{2.088034in}}%
\pgfpathlineto{\pgfqpoint{4.349478in}{2.091437in}}%
\pgfpathlineto{\pgfqpoint{4.346683in}{2.104979in}}%
\pgfpathlineto{\pgfqpoint{4.345286in}{2.101078in}}%
\pgfpathlineto{\pgfqpoint{4.342492in}{2.076200in}}%
\pgfpathlineto{\pgfqpoint{4.341095in}{2.073547in}}%
\pgfpathlineto{\pgfqpoint{4.338301in}{2.080063in}}%
\pgfpathlineto{\pgfqpoint{4.336904in}{2.060202in}}%
\pgfpathlineto{\pgfqpoint{4.335507in}{1.981062in}}%
\pgfpathlineto{\pgfqpoint{4.332713in}{1.476320in}}%
\pgfpathlineto{\pgfqpoint{4.329919in}{0.927521in}}%
\pgfpathlineto{\pgfqpoint{4.328522in}{0.981037in}}%
\pgfpathlineto{\pgfqpoint{4.322934in}{1.949336in}}%
\pgfpathlineto{\pgfqpoint{4.320139in}{2.019868in}}%
\pgfpathlineto{\pgfqpoint{4.314551in}{2.056636in}}%
\pgfpathlineto{\pgfqpoint{4.313154in}{2.054535in}}%
\pgfpathlineto{\pgfqpoint{4.311757in}{2.041834in}}%
\pgfpathlineto{\pgfqpoint{4.310360in}{2.002596in}}%
\pgfpathlineto{\pgfqpoint{4.307566in}{1.761047in}}%
\pgfpathlineto{\pgfqpoint{4.304772in}{1.555836in}}%
\pgfpathlineto{\pgfqpoint{4.303375in}{1.626875in}}%
\pgfpathlineto{\pgfqpoint{4.299184in}{2.070138in}}%
\pgfpathlineto{\pgfqpoint{4.296389in}{2.133755in}}%
\pgfpathlineto{\pgfqpoint{4.294992in}{2.133621in}}%
\pgfpathlineto{\pgfqpoint{4.293595in}{2.136164in}}%
\pgfpathlineto{\pgfqpoint{4.288007in}{2.175882in}}%
\pgfpathlineto{\pgfqpoint{4.286610in}{2.175594in}}%
\pgfpathlineto{\pgfqpoint{4.285213in}{2.174610in}}%
\pgfpathlineto{\pgfqpoint{4.283816in}{2.179193in}}%
\pgfpathlineto{\pgfqpoint{4.281022in}{2.209291in}}%
\pgfpathlineto{\pgfqpoint{4.271242in}{2.331755in}}%
\pgfpathlineto{\pgfqpoint{4.268448in}{2.351248in}}%
\pgfpathlineto{\pgfqpoint{4.265654in}{2.351688in}}%
\pgfpathlineto{\pgfqpoint{4.257272in}{2.380448in}}%
\pgfpathlineto{\pgfqpoint{4.251684in}{2.406302in}}%
\pgfpathlineto{\pgfqpoint{4.244698in}{2.487620in}}%
\pgfpathlineto{\pgfqpoint{4.241904in}{2.523636in}}%
\pgfpathlineto{\pgfqpoint{4.240507in}{2.528712in}}%
\pgfpathlineto{\pgfqpoint{4.239110in}{2.527240in}}%
\pgfpathlineto{\pgfqpoint{4.237713in}{2.529505in}}%
\pgfpathlineto{\pgfqpoint{4.232125in}{2.574090in}}%
\pgfpathlineto{\pgfqpoint{4.229331in}{2.563201in}}%
\pgfpathlineto{\pgfqpoint{4.227934in}{2.566651in}}%
\pgfpathlineto{\pgfqpoint{4.222345in}{2.607272in}}%
\pgfpathlineto{\pgfqpoint{4.220948in}{2.608209in}}%
\pgfpathlineto{\pgfqpoint{4.219551in}{2.612511in}}%
\pgfpathlineto{\pgfqpoint{4.216757in}{2.638934in}}%
\pgfpathlineto{\pgfqpoint{4.212566in}{2.688186in}}%
\pgfpathlineto{\pgfqpoint{4.209772in}{2.697245in}}%
\pgfpathlineto{\pgfqpoint{4.206978in}{2.712515in}}%
\pgfpathlineto{\pgfqpoint{4.205581in}{2.714068in}}%
\pgfpathlineto{\pgfqpoint{4.202787in}{2.711529in}}%
\pgfpathlineto{\pgfqpoint{4.201390in}{2.711238in}}%
\pgfpathlineto{\pgfqpoint{4.197198in}{2.700252in}}%
\pgfpathlineto{\pgfqpoint{4.194404in}{2.715053in}}%
\pgfpathlineto{\pgfqpoint{4.183228in}{2.797699in}}%
\pgfpathlineto{\pgfqpoint{4.180434in}{2.808167in}}%
\pgfpathlineto{\pgfqpoint{4.177640in}{2.823682in}}%
\pgfpathlineto{\pgfqpoint{4.176243in}{2.824899in}}%
\pgfpathlineto{\pgfqpoint{4.174846in}{2.820398in}}%
\pgfpathlineto{\pgfqpoint{4.172051in}{2.797757in}}%
\pgfpathlineto{\pgfqpoint{4.169257in}{2.772096in}}%
\pgfpathlineto{\pgfqpoint{4.167860in}{2.770624in}}%
\pgfpathlineto{\pgfqpoint{4.166463in}{2.778315in}}%
\pgfpathlineto{\pgfqpoint{4.162272in}{2.817331in}}%
\pgfpathlineto{\pgfqpoint{4.160875in}{2.816283in}}%
\pgfpathlineto{\pgfqpoint{4.159478in}{2.812568in}}%
\pgfpathlineto{\pgfqpoint{4.158081in}{2.819850in}}%
\pgfpathlineto{\pgfqpoint{4.153890in}{2.881756in}}%
\pgfpathlineto{\pgfqpoint{4.152493in}{2.886964in}}%
\pgfpathlineto{\pgfqpoint{4.151096in}{2.887409in}}%
\pgfpathlineto{\pgfqpoint{4.149699in}{2.885488in}}%
\pgfpathlineto{\pgfqpoint{4.145507in}{2.873742in}}%
\pgfpathlineto{\pgfqpoint{4.144110in}{2.875197in}}%
\pgfpathlineto{\pgfqpoint{4.141316in}{2.881402in}}%
\pgfpathlineto{\pgfqpoint{4.139919in}{2.879658in}}%
\pgfpathlineto{\pgfqpoint{4.137125in}{2.853446in}}%
\pgfpathlineto{\pgfqpoint{4.132934in}{2.808410in}}%
\pgfpathlineto{\pgfqpoint{4.131537in}{2.799782in}}%
\pgfpathlineto{\pgfqpoint{4.130140in}{2.747555in}}%
\pgfpathlineto{\pgfqpoint{4.128743in}{2.571941in}}%
\pgfpathlineto{\pgfqpoint{4.124551in}{1.553484in}}%
\pgfpathlineto{\pgfqpoint{4.123154in}{1.586880in}}%
\pgfpathlineto{\pgfqpoint{4.117566in}{2.796418in}}%
\pgfpathlineto{\pgfqpoint{4.114772in}{2.880544in}}%
\pgfpathlineto{\pgfqpoint{4.111978in}{2.888579in}}%
\pgfpathlineto{\pgfqpoint{4.110581in}{2.896297in}}%
\pgfpathlineto{\pgfqpoint{4.109184in}{2.895736in}}%
\pgfpathlineto{\pgfqpoint{4.107787in}{2.877955in}}%
\pgfpathlineto{\pgfqpoint{4.106390in}{2.842437in}}%
\pgfpathlineto{\pgfqpoint{4.104993in}{2.777876in}}%
\pgfpathlineto{\pgfqpoint{4.102199in}{2.478480in}}%
\pgfpathlineto{\pgfqpoint{4.099404in}{2.239707in}}%
\pgfpathlineto{\pgfqpoint{4.098007in}{2.325429in}}%
\pgfpathlineto{\pgfqpoint{4.093816in}{2.848510in}}%
\pgfpathlineto{\pgfqpoint{4.092419in}{2.908400in}}%
\pgfpathlineto{\pgfqpoint{4.091022in}{2.919276in}}%
\pgfpathlineto{\pgfqpoint{4.088228in}{2.904569in}}%
\pgfpathlineto{\pgfqpoint{4.086831in}{2.907318in}}%
\pgfpathlineto{\pgfqpoint{4.082640in}{2.933140in}}%
\pgfpathlineto{\pgfqpoint{4.081243in}{2.927775in}}%
\pgfpathlineto{\pgfqpoint{4.077052in}{2.892684in}}%
\pgfpathlineto{\pgfqpoint{4.072860in}{2.848094in}}%
\pgfpathlineto{\pgfqpoint{4.071463in}{2.844688in}}%
\pgfpathlineto{\pgfqpoint{4.070066in}{2.850148in}}%
\pgfpathlineto{\pgfqpoint{4.060287in}{2.924596in}}%
\pgfpathlineto{\pgfqpoint{4.056096in}{2.951691in}}%
\pgfpathlineto{\pgfqpoint{4.054699in}{2.952714in}}%
\pgfpathlineto{\pgfqpoint{4.051905in}{2.926162in}}%
\pgfpathlineto{\pgfqpoint{4.046316in}{2.866889in}}%
\pgfpathlineto{\pgfqpoint{4.044919in}{2.856394in}}%
\pgfpathlineto{\pgfqpoint{4.043522in}{2.855611in}}%
\pgfpathlineto{\pgfqpoint{4.040728in}{2.866364in}}%
\pgfpathlineto{\pgfqpoint{4.039331in}{2.865096in}}%
\pgfpathlineto{\pgfqpoint{4.037934in}{2.861116in}}%
\pgfpathlineto{\pgfqpoint{4.036537in}{2.861173in}}%
\pgfpathlineto{\pgfqpoint{4.035140in}{2.869709in}}%
\pgfpathlineto{\pgfqpoint{4.030949in}{2.907471in}}%
\pgfpathlineto{\pgfqpoint{4.029552in}{2.906394in}}%
\pgfpathlineto{\pgfqpoint{4.025360in}{2.896565in}}%
\pgfpathlineto{\pgfqpoint{4.019772in}{2.857053in}}%
\pgfpathlineto{\pgfqpoint{4.016978in}{2.854861in}}%
\pgfpathlineto{\pgfqpoint{4.015581in}{2.857302in}}%
\pgfpathlineto{\pgfqpoint{4.012787in}{2.874493in}}%
\pgfpathlineto{\pgfqpoint{4.009993in}{2.890598in}}%
\pgfpathlineto{\pgfqpoint{4.008596in}{2.889531in}}%
\pgfpathlineto{\pgfqpoint{4.004405in}{2.870788in}}%
\pgfpathlineto{\pgfqpoint{4.003008in}{2.875772in}}%
\pgfpathlineto{\pgfqpoint{3.998816in}{2.920786in}}%
\pgfpathlineto{\pgfqpoint{3.997419in}{2.920204in}}%
\pgfpathlineto{\pgfqpoint{3.991831in}{2.883660in}}%
\pgfpathlineto{\pgfqpoint{3.986243in}{2.796279in}}%
\pgfpathlineto{\pgfqpoint{3.984846in}{2.797492in}}%
\pgfpathlineto{\pgfqpoint{3.979258in}{2.832778in}}%
\pgfpathlineto{\pgfqpoint{3.977861in}{2.828427in}}%
\pgfpathlineto{\pgfqpoint{3.976463in}{2.830307in}}%
\pgfpathlineto{\pgfqpoint{3.969478in}{2.880479in}}%
\pgfpathlineto{\pgfqpoint{3.968081in}{2.876324in}}%
\pgfpathlineto{\pgfqpoint{3.963890in}{2.845291in}}%
\pgfpathlineto{\pgfqpoint{3.961096in}{2.842620in}}%
\pgfpathlineto{\pgfqpoint{3.958302in}{2.826827in}}%
\pgfpathlineto{\pgfqpoint{3.956905in}{2.818761in}}%
\pgfpathlineto{\pgfqpoint{3.955508in}{2.817648in}}%
\pgfpathlineto{\pgfqpoint{3.952714in}{2.832318in}}%
\pgfpathlineto{\pgfqpoint{3.948522in}{2.851622in}}%
\pgfpathlineto{\pgfqpoint{3.947125in}{2.848046in}}%
\pgfpathlineto{\pgfqpoint{3.944331in}{2.829157in}}%
\pgfpathlineto{\pgfqpoint{3.942934in}{2.831091in}}%
\pgfpathlineto{\pgfqpoint{3.938743in}{2.867672in}}%
\pgfpathlineto{\pgfqpoint{3.937346in}{2.870711in}}%
\pgfpathlineto{\pgfqpoint{3.935949in}{2.867700in}}%
\pgfpathlineto{\pgfqpoint{3.934552in}{2.856969in}}%
\pgfpathlineto{\pgfqpoint{3.928964in}{2.781229in}}%
\pgfpathlineto{\pgfqpoint{3.924772in}{2.768189in}}%
\pgfpathlineto{\pgfqpoint{3.919184in}{2.715739in}}%
\pgfpathlineto{\pgfqpoint{3.917787in}{2.679591in}}%
\pgfpathlineto{\pgfqpoint{3.916390in}{2.555761in}}%
\pgfpathlineto{\pgfqpoint{3.910802in}{1.577541in}}%
\pgfpathlineto{\pgfqpoint{3.908008in}{2.166925in}}%
\pgfpathlineto{\pgfqpoint{3.905214in}{2.651000in}}%
\pgfpathlineto{\pgfqpoint{3.902419in}{2.742653in}}%
\pgfpathlineto{\pgfqpoint{3.901022in}{2.747126in}}%
\pgfpathlineto{\pgfqpoint{3.898228in}{2.743998in}}%
\pgfpathlineto{\pgfqpoint{3.896831in}{2.745548in}}%
\pgfpathlineto{\pgfqpoint{3.895434in}{2.742671in}}%
\pgfpathlineto{\pgfqpoint{3.894037in}{2.726428in}}%
\pgfpathlineto{\pgfqpoint{3.892640in}{2.682239in}}%
\pgfpathlineto{\pgfqpoint{3.889846in}{2.444080in}}%
\pgfpathlineto{\pgfqpoint{3.887052in}{2.237151in}}%
\pgfpathlineto{\pgfqpoint{3.885655in}{2.294856in}}%
\pgfpathlineto{\pgfqpoint{3.881464in}{2.682934in}}%
\pgfpathlineto{\pgfqpoint{3.878670in}{2.753127in}}%
\pgfpathlineto{\pgfqpoint{3.875875in}{2.762069in}}%
\pgfpathlineto{\pgfqpoint{3.874478in}{2.760759in}}%
\pgfpathlineto{\pgfqpoint{3.870287in}{2.732107in}}%
\pgfpathlineto{\pgfqpoint{3.868890in}{2.733932in}}%
\pgfpathlineto{\pgfqpoint{3.864699in}{2.749326in}}%
\pgfpathlineto{\pgfqpoint{3.860508in}{2.760081in}}%
\pgfpathlineto{\pgfqpoint{3.859111in}{2.761372in}}%
\pgfpathlineto{\pgfqpoint{3.857714in}{2.760507in}}%
\pgfpathlineto{\pgfqpoint{3.853523in}{2.735916in}}%
\pgfpathlineto{\pgfqpoint{3.852125in}{2.737928in}}%
\pgfpathlineto{\pgfqpoint{3.843743in}{2.778320in}}%
\pgfpathlineto{\pgfqpoint{3.836758in}{2.752581in}}%
\pgfpathlineto{\pgfqpoint{3.831170in}{2.712489in}}%
\pgfpathlineto{\pgfqpoint{3.829773in}{2.711701in}}%
\pgfpathlineto{\pgfqpoint{3.828375in}{2.718149in}}%
\pgfpathlineto{\pgfqpoint{3.824184in}{2.751408in}}%
\pgfpathlineto{\pgfqpoint{3.821390in}{2.756839in}}%
\pgfpathlineto{\pgfqpoint{3.818596in}{2.759254in}}%
\pgfpathlineto{\pgfqpoint{3.815802in}{2.745622in}}%
\pgfpathlineto{\pgfqpoint{3.814405in}{2.739748in}}%
\pgfpathlineto{\pgfqpoint{3.813008in}{2.740843in}}%
\pgfpathlineto{\pgfqpoint{3.810214in}{2.750208in}}%
\pgfpathlineto{\pgfqpoint{3.808817in}{2.751392in}}%
\pgfpathlineto{\pgfqpoint{3.806023in}{2.751203in}}%
\pgfpathlineto{\pgfqpoint{3.804626in}{2.753406in}}%
\pgfpathlineto{\pgfqpoint{3.799037in}{2.772723in}}%
\pgfpathlineto{\pgfqpoint{3.796243in}{2.757141in}}%
\pgfpathlineto{\pgfqpoint{3.794846in}{2.750008in}}%
\pgfpathlineto{\pgfqpoint{3.793449in}{2.753247in}}%
\pgfpathlineto{\pgfqpoint{3.790655in}{2.782548in}}%
\pgfpathlineto{\pgfqpoint{3.789258in}{2.784753in}}%
\pgfpathlineto{\pgfqpoint{3.785067in}{2.759648in}}%
\pgfpathlineto{\pgfqpoint{3.782273in}{2.768263in}}%
\pgfpathlineto{\pgfqpoint{3.779479in}{2.767734in}}%
\pgfpathlineto{\pgfqpoint{3.778081in}{2.767023in}}%
\pgfpathlineto{\pgfqpoint{3.775287in}{2.759843in}}%
\pgfpathlineto{\pgfqpoint{3.771096in}{2.748219in}}%
\pgfpathlineto{\pgfqpoint{3.769699in}{2.748485in}}%
\pgfpathlineto{\pgfqpoint{3.768302in}{2.753479in}}%
\pgfpathlineto{\pgfqpoint{3.759920in}{2.804704in}}%
\pgfpathlineto{\pgfqpoint{3.755729in}{2.794623in}}%
\pgfpathlineto{\pgfqpoint{3.751537in}{2.806370in}}%
\pgfpathlineto{\pgfqpoint{3.747346in}{2.801029in}}%
\pgfpathlineto{\pgfqpoint{3.745949in}{2.800903in}}%
\pgfpathlineto{\pgfqpoint{3.740361in}{2.787822in}}%
\pgfpathlineto{\pgfqpoint{3.738964in}{2.786667in}}%
\pgfpathlineto{\pgfqpoint{3.734773in}{2.765470in}}%
\pgfpathlineto{\pgfqpoint{3.733376in}{2.767634in}}%
\pgfpathlineto{\pgfqpoint{3.727787in}{2.811674in}}%
\pgfpathlineto{\pgfqpoint{3.726390in}{2.810724in}}%
\pgfpathlineto{\pgfqpoint{3.723596in}{2.805918in}}%
\pgfpathlineto{\pgfqpoint{3.722199in}{2.806153in}}%
\pgfpathlineto{\pgfqpoint{3.720802in}{2.803436in}}%
\pgfpathlineto{\pgfqpoint{3.718008in}{2.775822in}}%
\pgfpathlineto{\pgfqpoint{3.715214in}{2.748952in}}%
\pgfpathlineto{\pgfqpoint{3.712420in}{2.743659in}}%
\pgfpathlineto{\pgfqpoint{3.709626in}{2.729199in}}%
\pgfpathlineto{\pgfqpoint{3.708229in}{2.731939in}}%
\pgfpathlineto{\pgfqpoint{3.699846in}{2.789774in}}%
\pgfpathlineto{\pgfqpoint{3.698449in}{2.775113in}}%
\pgfpathlineto{\pgfqpoint{3.697052in}{2.709988in}}%
\pgfpathlineto{\pgfqpoint{3.694258in}{2.299812in}}%
\pgfpathlineto{\pgfqpoint{3.691464in}{1.919599in}}%
\pgfpathlineto{\pgfqpoint{3.690067in}{2.004207in}}%
\pgfpathlineto{\pgfqpoint{3.685876in}{2.641983in}}%
\pgfpathlineto{\pgfqpoint{3.683082in}{2.737592in}}%
\pgfpathlineto{\pgfqpoint{3.681685in}{2.740249in}}%
\pgfpathlineto{\pgfqpoint{3.680287in}{2.740486in}}%
\pgfpathlineto{\pgfqpoint{3.676096in}{2.750173in}}%
\pgfpathlineto{\pgfqpoint{3.674699in}{2.744764in}}%
\pgfpathlineto{\pgfqpoint{3.673302in}{2.725830in}}%
\pgfpathlineto{\pgfqpoint{3.670508in}{2.586358in}}%
\pgfpathlineto{\pgfqpoint{3.667714in}{2.453588in}}%
\pgfpathlineto{\pgfqpoint{3.666317in}{2.499887in}}%
\pgfpathlineto{\pgfqpoint{3.662126in}{2.757055in}}%
\pgfpathlineto{\pgfqpoint{3.660729in}{2.778502in}}%
\pgfpathlineto{\pgfqpoint{3.659332in}{2.782252in}}%
\pgfpathlineto{\pgfqpoint{3.656538in}{2.781975in}}%
\pgfpathlineto{\pgfqpoint{3.655140in}{2.783062in}}%
\pgfpathlineto{\pgfqpoint{3.653743in}{2.781879in}}%
\pgfpathlineto{\pgfqpoint{3.649552in}{2.764621in}}%
\pgfpathlineto{\pgfqpoint{3.648155in}{2.768131in}}%
\pgfpathlineto{\pgfqpoint{3.643964in}{2.788564in}}%
\pgfpathlineto{\pgfqpoint{3.638376in}{2.796158in}}%
\pgfpathlineto{\pgfqpoint{3.632788in}{2.814729in}}%
\pgfpathlineto{\pgfqpoint{3.629993in}{2.801006in}}%
\pgfpathlineto{\pgfqpoint{3.628596in}{2.800551in}}%
\pgfpathlineto{\pgfqpoint{3.627199in}{2.801685in}}%
\pgfpathlineto{\pgfqpoint{3.625802in}{2.794867in}}%
\pgfpathlineto{\pgfqpoint{3.623008in}{2.766068in}}%
\pgfpathlineto{\pgfqpoint{3.621611in}{2.762028in}}%
\pgfpathlineto{\pgfqpoint{3.618817in}{2.772070in}}%
\pgfpathlineto{\pgfqpoint{3.614626in}{2.801778in}}%
\pgfpathlineto{\pgfqpoint{3.609038in}{2.835815in}}%
\pgfpathlineto{\pgfqpoint{3.604846in}{2.857761in}}%
\pgfpathlineto{\pgfqpoint{3.603449in}{2.855364in}}%
\pgfpathlineto{\pgfqpoint{3.599258in}{2.823914in}}%
\pgfpathlineto{\pgfqpoint{3.595067in}{2.814592in}}%
\pgfpathlineto{\pgfqpoint{3.590876in}{2.826303in}}%
\pgfpathlineto{\pgfqpoint{3.588082in}{2.812138in}}%
\pgfpathlineto{\pgfqpoint{3.586685in}{2.811269in}}%
\pgfpathlineto{\pgfqpoint{3.583891in}{2.822264in}}%
\pgfpathlineto{\pgfqpoint{3.579699in}{2.840590in}}%
\pgfpathlineto{\pgfqpoint{3.575508in}{2.846702in}}%
\pgfpathlineto{\pgfqpoint{3.572714in}{2.851628in}}%
\pgfpathlineto{\pgfqpoint{3.569920in}{2.841578in}}%
\pgfpathlineto{\pgfqpoint{3.567126in}{2.833266in}}%
\pgfpathlineto{\pgfqpoint{3.565729in}{2.830499in}}%
\pgfpathlineto{\pgfqpoint{3.561538in}{2.814709in}}%
\pgfpathlineto{\pgfqpoint{3.560141in}{2.818252in}}%
\pgfpathlineto{\pgfqpoint{3.557347in}{2.847462in}}%
\pgfpathlineto{\pgfqpoint{3.554552in}{2.879196in}}%
\pgfpathlineto{\pgfqpoint{3.553155in}{2.882278in}}%
\pgfpathlineto{\pgfqpoint{3.550361in}{2.872477in}}%
\pgfpathlineto{\pgfqpoint{3.548964in}{2.874747in}}%
\pgfpathlineto{\pgfqpoint{3.544773in}{2.897477in}}%
\pgfpathlineto{\pgfqpoint{3.543376in}{2.894600in}}%
\pgfpathlineto{\pgfqpoint{3.540582in}{2.864502in}}%
\pgfpathlineto{\pgfqpoint{3.537788in}{2.829053in}}%
\pgfpathlineto{\pgfqpoint{3.534994in}{2.821549in}}%
\pgfpathlineto{\pgfqpoint{3.533597in}{2.819160in}}%
\pgfpathlineto{\pgfqpoint{3.529405in}{2.800268in}}%
\pgfpathlineto{\pgfqpoint{3.528008in}{2.802120in}}%
\pgfpathlineto{\pgfqpoint{3.525214in}{2.820434in}}%
\pgfpathlineto{\pgfqpoint{3.516832in}{2.902156in}}%
\pgfpathlineto{\pgfqpoint{3.515435in}{2.900909in}}%
\pgfpathlineto{\pgfqpoint{3.509847in}{2.865344in}}%
\pgfpathlineto{\pgfqpoint{3.507052in}{2.854926in}}%
\pgfpathlineto{\pgfqpoint{3.500067in}{2.825422in}}%
\pgfpathlineto{\pgfqpoint{3.498670in}{2.826392in}}%
\pgfpathlineto{\pgfqpoint{3.493082in}{2.864245in}}%
\pgfpathlineto{\pgfqpoint{3.490288in}{2.869932in}}%
\pgfpathlineto{\pgfqpoint{3.488891in}{2.872836in}}%
\pgfpathlineto{\pgfqpoint{3.486097in}{2.870132in}}%
\pgfpathlineto{\pgfqpoint{3.484700in}{2.872481in}}%
\pgfpathlineto{\pgfqpoint{3.481905in}{2.878986in}}%
\pgfpathlineto{\pgfqpoint{3.470729in}{2.831991in}}%
\pgfpathlineto{\pgfqpoint{3.467935in}{2.831703in}}%
\pgfpathlineto{\pgfqpoint{3.466538in}{2.834877in}}%
\pgfpathlineto{\pgfqpoint{3.460950in}{2.871512in}}%
\pgfpathlineto{\pgfqpoint{3.458156in}{2.873100in}}%
\pgfpathlineto{\pgfqpoint{3.453964in}{2.866036in}}%
\pgfpathlineto{\pgfqpoint{3.451170in}{2.868946in}}%
\pgfpathlineto{\pgfqpoint{3.449773in}{2.864244in}}%
\pgfpathlineto{\pgfqpoint{3.445582in}{2.832158in}}%
\pgfpathlineto{\pgfqpoint{3.441391in}{2.825324in}}%
\pgfpathlineto{\pgfqpoint{3.439994in}{2.825062in}}%
\pgfpathlineto{\pgfqpoint{3.437200in}{2.840211in}}%
\pgfpathlineto{\pgfqpoint{3.434406in}{2.850044in}}%
\pgfpathlineto{\pgfqpoint{3.431611in}{2.841245in}}%
\pgfpathlineto{\pgfqpoint{3.430214in}{2.841530in}}%
\pgfpathlineto{\pgfqpoint{3.427420in}{2.860093in}}%
\pgfpathlineto{\pgfqpoint{3.424626in}{2.873771in}}%
\pgfpathlineto{\pgfqpoint{3.423229in}{2.872001in}}%
\pgfpathlineto{\pgfqpoint{3.420435in}{2.856054in}}%
\pgfpathlineto{\pgfqpoint{3.416244in}{2.828093in}}%
\pgfpathlineto{\pgfqpoint{3.413450in}{2.816928in}}%
\pgfpathlineto{\pgfqpoint{3.410656in}{2.797873in}}%
\pgfpathlineto{\pgfqpoint{3.409259in}{2.794811in}}%
\pgfpathlineto{\pgfqpoint{3.407861in}{2.796582in}}%
\pgfpathlineto{\pgfqpoint{3.406464in}{2.803042in}}%
\pgfpathlineto{\pgfqpoint{3.400876in}{2.861574in}}%
\pgfpathlineto{\pgfqpoint{3.399479in}{2.861304in}}%
\pgfpathlineto{\pgfqpoint{3.398082in}{2.860676in}}%
\pgfpathlineto{\pgfqpoint{3.396685in}{2.864938in}}%
\pgfpathlineto{\pgfqpoint{3.392494in}{2.885858in}}%
\pgfpathlineto{\pgfqpoint{3.391097in}{2.885326in}}%
\pgfpathlineto{\pgfqpoint{3.389700in}{2.876609in}}%
\pgfpathlineto{\pgfqpoint{3.382714in}{2.806055in}}%
\pgfpathlineto{\pgfqpoint{3.379920in}{2.787990in}}%
\pgfpathlineto{\pgfqpoint{3.378523in}{2.785831in}}%
\pgfpathlineto{\pgfqpoint{3.375729in}{2.796865in}}%
\pgfpathlineto{\pgfqpoint{3.371538in}{2.817935in}}%
\pgfpathlineto{\pgfqpoint{3.365950in}{2.867992in}}%
\pgfpathlineto{\pgfqpoint{3.364553in}{2.865554in}}%
\pgfpathlineto{\pgfqpoint{3.361759in}{2.846723in}}%
\pgfpathlineto{\pgfqpoint{3.356170in}{2.801407in}}%
\pgfpathlineto{\pgfqpoint{3.354773in}{2.804697in}}%
\pgfpathlineto{\pgfqpoint{3.350582in}{2.824008in}}%
\pgfpathlineto{\pgfqpoint{3.349185in}{2.825298in}}%
\pgfpathlineto{\pgfqpoint{3.347788in}{2.824323in}}%
\pgfpathlineto{\pgfqpoint{3.344994in}{2.813035in}}%
\pgfpathlineto{\pgfqpoint{3.343597in}{2.807239in}}%
\pgfpathlineto{\pgfqpoint{3.342200in}{2.806765in}}%
\pgfpathlineto{\pgfqpoint{3.336612in}{2.826979in}}%
\pgfpathlineto{\pgfqpoint{3.335215in}{2.827909in}}%
\pgfpathlineto{\pgfqpoint{3.333817in}{2.832860in}}%
\pgfpathlineto{\pgfqpoint{3.328229in}{2.884644in}}%
\pgfpathlineto{\pgfqpoint{3.326832in}{2.883883in}}%
\pgfpathlineto{\pgfqpoint{3.324038in}{2.837123in}}%
\pgfpathlineto{\pgfqpoint{3.321244in}{2.799195in}}%
\pgfpathlineto{\pgfqpoint{3.318450in}{2.805470in}}%
\pgfpathlineto{\pgfqpoint{3.317053in}{2.803909in}}%
\pgfpathlineto{\pgfqpoint{3.314259in}{2.797498in}}%
\pgfpathlineto{\pgfqpoint{3.312862in}{2.799979in}}%
\pgfpathlineto{\pgfqpoint{3.307273in}{2.820055in}}%
\pgfpathlineto{\pgfqpoint{3.303082in}{2.806186in}}%
\pgfpathlineto{\pgfqpoint{3.298891in}{2.817968in}}%
\pgfpathlineto{\pgfqpoint{3.294700in}{2.803637in}}%
\pgfpathlineto{\pgfqpoint{3.293303in}{2.798454in}}%
\pgfpathlineto{\pgfqpoint{3.286318in}{2.756594in}}%
\pgfpathlineto{\pgfqpoint{3.284920in}{2.755683in}}%
\pgfpathlineto{\pgfqpoint{3.283523in}{2.764240in}}%
\pgfpathlineto{\pgfqpoint{3.279332in}{2.828032in}}%
\pgfpathlineto{\pgfqpoint{3.277935in}{2.826627in}}%
\pgfpathlineto{\pgfqpoint{3.275141in}{2.814664in}}%
\pgfpathlineto{\pgfqpoint{3.272347in}{2.819201in}}%
\pgfpathlineto{\pgfqpoint{3.270950in}{2.816186in}}%
\pgfpathlineto{\pgfqpoint{3.268156in}{2.789798in}}%
\pgfpathlineto{\pgfqpoint{3.265362in}{2.763710in}}%
\pgfpathlineto{\pgfqpoint{3.263965in}{2.761828in}}%
\pgfpathlineto{\pgfqpoint{3.258376in}{2.772658in}}%
\pgfpathlineto{\pgfqpoint{3.255582in}{2.769340in}}%
\pgfpathlineto{\pgfqpoint{3.245803in}{2.797458in}}%
\pgfpathlineto{\pgfqpoint{3.244406in}{2.796245in}}%
\pgfpathlineto{\pgfqpoint{3.241612in}{2.773692in}}%
\pgfpathlineto{\pgfqpoint{3.238818in}{2.732583in}}%
\pgfpathlineto{\pgfqpoint{3.237421in}{2.690148in}}%
\pgfpathlineto{\pgfqpoint{3.236024in}{2.587088in}}%
\pgfpathlineto{\pgfqpoint{3.230435in}{1.940877in}}%
\pgfpathlineto{\pgfqpoint{3.223450in}{2.698812in}}%
\pgfpathlineto{\pgfqpoint{3.217862in}{2.779395in}}%
\pgfpathlineto{\pgfqpoint{3.216465in}{2.790079in}}%
\pgfpathlineto{\pgfqpoint{3.215068in}{2.788879in}}%
\pgfpathlineto{\pgfqpoint{3.213671in}{2.762463in}}%
\pgfpathlineto{\pgfqpoint{3.210876in}{2.593260in}}%
\pgfpathlineto{\pgfqpoint{3.208082in}{2.422398in}}%
\pgfpathlineto{\pgfqpoint{3.206685in}{2.439051in}}%
\pgfpathlineto{\pgfqpoint{3.201097in}{2.706195in}}%
\pgfpathlineto{\pgfqpoint{3.198303in}{2.719771in}}%
\pgfpathlineto{\pgfqpoint{3.195509in}{2.720184in}}%
\pgfpathlineto{\pgfqpoint{3.194112in}{2.719269in}}%
\pgfpathlineto{\pgfqpoint{3.189921in}{2.706567in}}%
\pgfpathlineto{\pgfqpoint{3.187127in}{2.714614in}}%
\pgfpathlineto{\pgfqpoint{3.185729in}{2.713488in}}%
\pgfpathlineto{\pgfqpoint{3.184332in}{2.709851in}}%
\pgfpathlineto{\pgfqpoint{3.182935in}{2.714323in}}%
\pgfpathlineto{\pgfqpoint{3.180141in}{2.743369in}}%
\pgfpathlineto{\pgfqpoint{3.178744in}{2.742104in}}%
\pgfpathlineto{\pgfqpoint{3.174553in}{2.694296in}}%
\pgfpathlineto{\pgfqpoint{3.171759in}{2.684240in}}%
\pgfpathlineto{\pgfqpoint{3.168965in}{2.653589in}}%
\pgfpathlineto{\pgfqpoint{3.166171in}{2.627436in}}%
\pgfpathlineto{\pgfqpoint{3.164774in}{2.629041in}}%
\pgfpathlineto{\pgfqpoint{3.159185in}{2.664581in}}%
\pgfpathlineto{\pgfqpoint{3.156391in}{2.681054in}}%
\pgfpathlineto{\pgfqpoint{3.153597in}{2.683721in}}%
\pgfpathlineto{\pgfqpoint{3.152200in}{2.681692in}}%
\pgfpathlineto{\pgfqpoint{3.149406in}{2.662658in}}%
\pgfpathlineto{\pgfqpoint{3.146612in}{2.638322in}}%
\pgfpathlineto{\pgfqpoint{3.141024in}{2.613819in}}%
\pgfpathlineto{\pgfqpoint{3.138230in}{2.592513in}}%
\pgfpathlineto{\pgfqpoint{3.136832in}{2.591932in}}%
\pgfpathlineto{\pgfqpoint{3.135435in}{2.600587in}}%
\pgfpathlineto{\pgfqpoint{3.132641in}{2.640834in}}%
\pgfpathlineto{\pgfqpoint{3.129847in}{2.685451in}}%
\pgfpathlineto{\pgfqpoint{3.128450in}{2.692655in}}%
\pgfpathlineto{\pgfqpoint{3.127053in}{2.688028in}}%
\pgfpathlineto{\pgfqpoint{3.124259in}{2.655263in}}%
\pgfpathlineto{\pgfqpoint{3.121465in}{2.612074in}}%
\pgfpathlineto{\pgfqpoint{3.120068in}{2.605483in}}%
\pgfpathlineto{\pgfqpoint{3.115877in}{2.623686in}}%
\pgfpathlineto{\pgfqpoint{3.114480in}{2.619699in}}%
\pgfpathlineto{\pgfqpoint{3.111685in}{2.596761in}}%
\pgfpathlineto{\pgfqpoint{3.108891in}{2.572051in}}%
\pgfpathlineto{\pgfqpoint{3.106097in}{2.566014in}}%
\pgfpathlineto{\pgfqpoint{3.104700in}{2.566107in}}%
\pgfpathlineto{\pgfqpoint{3.103303in}{2.570870in}}%
\pgfpathlineto{\pgfqpoint{3.099112in}{2.597117in}}%
\pgfpathlineto{\pgfqpoint{3.096318in}{2.572044in}}%
\pgfpathlineto{\pgfqpoint{3.094921in}{2.560653in}}%
\pgfpathlineto{\pgfqpoint{3.093524in}{2.559992in}}%
\pgfpathlineto{\pgfqpoint{3.087936in}{2.586538in}}%
\pgfpathlineto{\pgfqpoint{3.086538in}{2.589129in}}%
\pgfpathlineto{\pgfqpoint{3.085141in}{2.588527in}}%
\pgfpathlineto{\pgfqpoint{3.080950in}{2.571723in}}%
\pgfpathlineto{\pgfqpoint{3.078156in}{2.582116in}}%
\pgfpathlineto{\pgfqpoint{3.076759in}{2.579400in}}%
\pgfpathlineto{\pgfqpoint{3.071171in}{2.548141in}}%
\pgfpathlineto{\pgfqpoint{3.069774in}{2.556934in}}%
\pgfpathlineto{\pgfqpoint{3.065583in}{2.600631in}}%
\pgfpathlineto{\pgfqpoint{3.064186in}{2.599287in}}%
\pgfpathlineto{\pgfqpoint{3.055803in}{2.547863in}}%
\pgfpathlineto{\pgfqpoint{3.051612in}{2.507986in}}%
\pgfpathlineto{\pgfqpoint{3.050215in}{2.505038in}}%
\pgfpathlineto{\pgfqpoint{3.048818in}{2.505342in}}%
\pgfpathlineto{\pgfqpoint{3.047421in}{2.507724in}}%
\pgfpathlineto{\pgfqpoint{3.041833in}{2.525075in}}%
\pgfpathlineto{\pgfqpoint{3.040436in}{2.526407in}}%
\pgfpathlineto{\pgfqpoint{3.039039in}{2.530880in}}%
\pgfpathlineto{\pgfqpoint{3.036244in}{2.553543in}}%
\pgfpathlineto{\pgfqpoint{3.033450in}{2.576013in}}%
\pgfpathlineto{\pgfqpoint{3.029259in}{2.584875in}}%
\pgfpathlineto{\pgfqpoint{3.027862in}{2.586338in}}%
\pgfpathlineto{\pgfqpoint{3.026465in}{2.583958in}}%
\pgfpathlineto{\pgfqpoint{3.020877in}{2.563919in}}%
\pgfpathlineto{\pgfqpoint{3.019480in}{2.563497in}}%
\pgfpathlineto{\pgfqpoint{3.018083in}{2.561713in}}%
\pgfpathlineto{\pgfqpoint{3.013892in}{2.549426in}}%
\pgfpathlineto{\pgfqpoint{3.012494in}{2.555035in}}%
\pgfpathlineto{\pgfqpoint{3.009700in}{2.584056in}}%
\pgfpathlineto{\pgfqpoint{3.006906in}{2.624798in}}%
\pgfpathlineto{\pgfqpoint{3.005509in}{2.629176in}}%
\pgfpathlineto{\pgfqpoint{3.002715in}{2.605960in}}%
\pgfpathlineto{\pgfqpoint{2.998524in}{2.543366in}}%
\pgfpathlineto{\pgfqpoint{2.997127in}{2.486330in}}%
\pgfpathlineto{\pgfqpoint{2.995730in}{2.339987in}}%
\pgfpathlineto{\pgfqpoint{2.991539in}{1.455417in}}%
\pgfpathlineto{\pgfqpoint{2.990142in}{1.449451in}}%
\pgfpathlineto{\pgfqpoint{2.983156in}{2.540143in}}%
\pgfpathlineto{\pgfqpoint{2.980362in}{2.569758in}}%
\pgfpathlineto{\pgfqpoint{2.978965in}{2.574148in}}%
\pgfpathlineto{\pgfqpoint{2.974774in}{2.603263in}}%
\pgfpathlineto{\pgfqpoint{2.973377in}{2.576094in}}%
\pgfpathlineto{\pgfqpoint{2.970583in}{2.346501in}}%
\pgfpathlineto{\pgfqpoint{2.967789in}{2.144803in}}%
\pgfpathlineto{\pgfqpoint{2.966392in}{2.197803in}}%
\pgfpathlineto{\pgfqpoint{2.962200in}{2.516614in}}%
\pgfpathlineto{\pgfqpoint{2.959406in}{2.554925in}}%
\pgfpathlineto{\pgfqpoint{2.955215in}{2.557724in}}%
\pgfpathlineto{\pgfqpoint{2.952421in}{2.564349in}}%
\pgfpathlineto{\pgfqpoint{2.951024in}{2.564116in}}%
\pgfpathlineto{\pgfqpoint{2.949627in}{2.565735in}}%
\pgfpathlineto{\pgfqpoint{2.946833in}{2.585711in}}%
\pgfpathlineto{\pgfqpoint{2.942642in}{2.623526in}}%
\pgfpathlineto{\pgfqpoint{2.941245in}{2.624560in}}%
\pgfpathlineto{\pgfqpoint{2.939848in}{2.618715in}}%
\pgfpathlineto{\pgfqpoint{2.937053in}{2.592851in}}%
\pgfpathlineto{\pgfqpoint{2.934259in}{2.565185in}}%
\pgfpathlineto{\pgfqpoint{2.932862in}{2.561604in}}%
\pgfpathlineto{\pgfqpoint{2.930068in}{2.562627in}}%
\pgfpathlineto{\pgfqpoint{2.927274in}{2.559626in}}%
\pgfpathlineto{\pgfqpoint{2.925877in}{2.562409in}}%
\pgfpathlineto{\pgfqpoint{2.923083in}{2.578357in}}%
\pgfpathlineto{\pgfqpoint{2.918892in}{2.615426in}}%
\pgfpathlineto{\pgfqpoint{2.917495in}{2.617555in}}%
\pgfpathlineto{\pgfqpoint{2.914700in}{2.603067in}}%
\pgfpathlineto{\pgfqpoint{2.911906in}{2.592349in}}%
\pgfpathlineto{\pgfqpoint{2.907715in}{2.601633in}}%
\pgfpathlineto{\pgfqpoint{2.906318in}{2.599707in}}%
\pgfpathlineto{\pgfqpoint{2.903524in}{2.578955in}}%
\pgfpathlineto{\pgfqpoint{2.899333in}{2.538435in}}%
\pgfpathlineto{\pgfqpoint{2.897936in}{2.537731in}}%
\pgfpathlineto{\pgfqpoint{2.895142in}{2.550973in}}%
\pgfpathlineto{\pgfqpoint{2.892348in}{2.559960in}}%
\pgfpathlineto{\pgfqpoint{2.890951in}{2.559421in}}%
\pgfpathlineto{\pgfqpoint{2.889553in}{2.562002in}}%
\pgfpathlineto{\pgfqpoint{2.882568in}{2.602307in}}%
\pgfpathlineto{\pgfqpoint{2.881171in}{2.601138in}}%
\pgfpathlineto{\pgfqpoint{2.878377in}{2.591093in}}%
\pgfpathlineto{\pgfqpoint{2.871392in}{2.558276in}}%
\pgfpathlineto{\pgfqpoint{2.868598in}{2.537064in}}%
\pgfpathlineto{\pgfqpoint{2.865804in}{2.532013in}}%
\pgfpathlineto{\pgfqpoint{2.864406in}{2.532457in}}%
\pgfpathlineto{\pgfqpoint{2.863009in}{2.534347in}}%
\pgfpathlineto{\pgfqpoint{2.857421in}{2.554520in}}%
\pgfpathlineto{\pgfqpoint{2.851833in}{2.598387in}}%
\pgfpathlineto{\pgfqpoint{2.849039in}{2.571446in}}%
\pgfpathlineto{\pgfqpoint{2.844848in}{2.528660in}}%
\pgfpathlineto{\pgfqpoint{2.837862in}{2.500574in}}%
\pgfpathlineto{\pgfqpoint{2.836465in}{2.503797in}}%
\pgfpathlineto{\pgfqpoint{2.826686in}{2.568680in}}%
\pgfpathlineto{\pgfqpoint{2.825289in}{2.582201in}}%
\pgfpathlineto{\pgfqpoint{2.823892in}{2.583660in}}%
\pgfpathlineto{\pgfqpoint{2.821098in}{2.555890in}}%
\pgfpathlineto{\pgfqpoint{2.816907in}{2.520102in}}%
\pgfpathlineto{\pgfqpoint{2.811318in}{2.498117in}}%
\pgfpathlineto{\pgfqpoint{2.801539in}{2.519535in}}%
\pgfpathlineto{\pgfqpoint{2.800142in}{2.520372in}}%
\pgfpathlineto{\pgfqpoint{2.795951in}{2.539713in}}%
\pgfpathlineto{\pgfqpoint{2.793157in}{2.530022in}}%
\pgfpathlineto{\pgfqpoint{2.791760in}{2.532394in}}%
\pgfpathlineto{\pgfqpoint{2.788965in}{2.542260in}}%
\pgfpathlineto{\pgfqpoint{2.787568in}{2.535464in}}%
\pgfpathlineto{\pgfqpoint{2.780583in}{2.457470in}}%
\pgfpathlineto{\pgfqpoint{2.779186in}{2.462580in}}%
\pgfpathlineto{\pgfqpoint{2.773598in}{2.514383in}}%
\pgfpathlineto{\pgfqpoint{2.772201in}{2.515488in}}%
\pgfpathlineto{\pgfqpoint{2.770804in}{2.513994in}}%
\pgfpathlineto{\pgfqpoint{2.768010in}{2.508992in}}%
\pgfpathlineto{\pgfqpoint{2.766613in}{2.509634in}}%
\pgfpathlineto{\pgfqpoint{2.765215in}{2.508872in}}%
\pgfpathlineto{\pgfqpoint{2.762421in}{2.502078in}}%
\pgfpathlineto{\pgfqpoint{2.758230in}{2.524158in}}%
\pgfpathlineto{\pgfqpoint{2.756833in}{2.517545in}}%
\pgfpathlineto{\pgfqpoint{2.752642in}{2.467109in}}%
\pgfpathlineto{\pgfqpoint{2.751245in}{2.431702in}}%
\pgfpathlineto{\pgfqpoint{2.749848in}{2.337249in}}%
\pgfpathlineto{\pgfqpoint{2.747054in}{1.791429in}}%
\pgfpathlineto{\pgfqpoint{2.744260in}{1.282857in}}%
\pgfpathlineto{\pgfqpoint{2.742863in}{1.394614in}}%
\pgfpathlineto{\pgfqpoint{2.737274in}{2.428287in}}%
\pgfpathlineto{\pgfqpoint{2.734480in}{2.489182in}}%
\pgfpathlineto{\pgfqpoint{2.733083in}{2.490085in}}%
\pgfpathlineto{\pgfqpoint{2.728892in}{2.480342in}}%
\pgfpathlineto{\pgfqpoint{2.727495in}{2.451892in}}%
\pgfpathlineto{\pgfqpoint{2.726098in}{2.366626in}}%
\pgfpathlineto{\pgfqpoint{2.721907in}{1.921625in}}%
\pgfpathlineto{\pgfqpoint{2.720510in}{1.940629in}}%
\pgfpathlineto{\pgfqpoint{2.714921in}{2.409446in}}%
\pgfpathlineto{\pgfqpoint{2.712127in}{2.458050in}}%
\pgfpathlineto{\pgfqpoint{2.707936in}{2.476919in}}%
\pgfpathlineto{\pgfqpoint{2.706539in}{2.480518in}}%
\pgfpathlineto{\pgfqpoint{2.705142in}{2.477938in}}%
\pgfpathlineto{\pgfqpoint{2.698157in}{2.445417in}}%
\pgfpathlineto{\pgfqpoint{2.695363in}{2.449696in}}%
\pgfpathlineto{\pgfqpoint{2.693966in}{2.444988in}}%
\pgfpathlineto{\pgfqpoint{2.691171in}{2.427093in}}%
\pgfpathlineto{\pgfqpoint{2.689774in}{2.425759in}}%
\pgfpathlineto{\pgfqpoint{2.686980in}{2.434067in}}%
\pgfpathlineto{\pgfqpoint{2.679995in}{2.465825in}}%
\pgfpathlineto{\pgfqpoint{2.675804in}{2.502568in}}%
\pgfpathlineto{\pgfqpoint{2.674407in}{2.507475in}}%
\pgfpathlineto{\pgfqpoint{2.673010in}{2.506752in}}%
\pgfpathlineto{\pgfqpoint{2.670216in}{2.486604in}}%
\pgfpathlineto{\pgfqpoint{2.664627in}{2.419005in}}%
\pgfpathlineto{\pgfqpoint{2.663230in}{2.419182in}}%
\pgfpathlineto{\pgfqpoint{2.661833in}{2.423175in}}%
\pgfpathlineto{\pgfqpoint{2.660436in}{2.421795in}}%
\pgfpathlineto{\pgfqpoint{2.656245in}{2.403713in}}%
\pgfpathlineto{\pgfqpoint{2.654848in}{2.404377in}}%
\pgfpathlineto{\pgfqpoint{2.653451in}{2.408735in}}%
\pgfpathlineto{\pgfqpoint{2.646466in}{2.453253in}}%
\pgfpathlineto{\pgfqpoint{2.643672in}{2.452773in}}%
\pgfpathlineto{\pgfqpoint{2.639480in}{2.471391in}}%
\pgfpathlineto{\pgfqpoint{2.638083in}{2.465311in}}%
\pgfpathlineto{\pgfqpoint{2.631098in}{2.397189in}}%
\pgfpathlineto{\pgfqpoint{2.629701in}{2.393102in}}%
\pgfpathlineto{\pgfqpoint{2.628304in}{2.393442in}}%
\pgfpathlineto{\pgfqpoint{2.625510in}{2.405990in}}%
\pgfpathlineto{\pgfqpoint{2.622716in}{2.423027in}}%
\pgfpathlineto{\pgfqpoint{2.621319in}{2.425636in}}%
\pgfpathlineto{\pgfqpoint{2.619922in}{2.422547in}}%
\pgfpathlineto{\pgfqpoint{2.615730in}{2.403991in}}%
\pgfpathlineto{\pgfqpoint{2.614333in}{2.412160in}}%
\pgfpathlineto{\pgfqpoint{2.610142in}{2.457396in}}%
\pgfpathlineto{\pgfqpoint{2.608745in}{2.452650in}}%
\pgfpathlineto{\pgfqpoint{2.603157in}{2.384308in}}%
\pgfpathlineto{\pgfqpoint{2.601760in}{2.383665in}}%
\pgfpathlineto{\pgfqpoint{2.600363in}{2.384596in}}%
\pgfpathlineto{\pgfqpoint{2.597569in}{2.378134in}}%
\pgfpathlineto{\pgfqpoint{2.596172in}{2.380831in}}%
\pgfpathlineto{\pgfqpoint{2.590583in}{2.424494in}}%
\pgfpathlineto{\pgfqpoint{2.587789in}{2.414917in}}%
\pgfpathlineto{\pgfqpoint{2.586392in}{2.415744in}}%
\pgfpathlineto{\pgfqpoint{2.584995in}{2.418624in}}%
\pgfpathlineto{\pgfqpoint{2.583598in}{2.418287in}}%
\pgfpathlineto{\pgfqpoint{2.582201in}{2.416726in}}%
\pgfpathlineto{\pgfqpoint{2.580804in}{2.420665in}}%
\pgfpathlineto{\pgfqpoint{2.578010in}{2.434950in}}%
\pgfpathlineto{\pgfqpoint{2.576613in}{2.426535in}}%
\pgfpathlineto{\pgfqpoint{2.571025in}{2.359980in}}%
\pgfpathlineto{\pgfqpoint{2.565436in}{2.347872in}}%
\pgfpathlineto{\pgfqpoint{2.564039in}{2.348559in}}%
\pgfpathlineto{\pgfqpoint{2.561245in}{2.375552in}}%
\pgfpathlineto{\pgfqpoint{2.557054in}{2.434150in}}%
\pgfpathlineto{\pgfqpoint{2.555657in}{2.437911in}}%
\pgfpathlineto{\pgfqpoint{2.551466in}{2.403956in}}%
\pgfpathlineto{\pgfqpoint{2.550069in}{2.404703in}}%
\pgfpathlineto{\pgfqpoint{2.548672in}{2.408140in}}%
\pgfpathlineto{\pgfqpoint{2.547275in}{2.405764in}}%
\pgfpathlineto{\pgfqpoint{2.538892in}{2.354433in}}%
\pgfpathlineto{\pgfqpoint{2.537495in}{2.351197in}}%
\pgfpathlineto{\pgfqpoint{2.533304in}{2.363636in}}%
\pgfpathlineto{\pgfqpoint{2.529113in}{2.347699in}}%
\pgfpathlineto{\pgfqpoint{2.527716in}{2.352879in}}%
\pgfpathlineto{\pgfqpoint{2.523525in}{2.386930in}}%
\pgfpathlineto{\pgfqpoint{2.519333in}{2.393176in}}%
\pgfpathlineto{\pgfqpoint{2.517936in}{2.389801in}}%
\pgfpathlineto{\pgfqpoint{2.509554in}{2.301546in}}%
\pgfpathlineto{\pgfqpoint{2.508157in}{2.301908in}}%
\pgfpathlineto{\pgfqpoint{2.502569in}{2.352099in}}%
\pgfpathlineto{\pgfqpoint{2.501172in}{2.348716in}}%
\pgfpathlineto{\pgfqpoint{2.499775in}{2.340036in}}%
\pgfpathlineto{\pgfqpoint{2.498378in}{2.323959in}}%
\pgfpathlineto{\pgfqpoint{2.496981in}{2.272683in}}%
\pgfpathlineto{\pgfqpoint{2.495584in}{2.128143in}}%
\pgfpathlineto{\pgfqpoint{2.491392in}{1.325423in}}%
\pgfpathlineto{\pgfqpoint{2.489995in}{1.355346in}}%
\pgfpathlineto{\pgfqpoint{2.484407in}{2.269445in}}%
\pgfpathlineto{\pgfqpoint{2.483010in}{2.304460in}}%
\pgfpathlineto{\pgfqpoint{2.480216in}{2.311997in}}%
\pgfpathlineto{\pgfqpoint{2.477422in}{2.313095in}}%
\pgfpathlineto{\pgfqpoint{2.476025in}{2.307491in}}%
\pgfpathlineto{\pgfqpoint{2.474628in}{2.279992in}}%
\pgfpathlineto{\pgfqpoint{2.471834in}{2.075394in}}%
\pgfpathlineto{\pgfqpoint{2.469039in}{1.866335in}}%
\pgfpathlineto{\pgfqpoint{2.467642in}{1.917118in}}%
\pgfpathlineto{\pgfqpoint{2.463451in}{2.307255in}}%
\pgfpathlineto{\pgfqpoint{2.462054in}{2.344394in}}%
\pgfpathlineto{\pgfqpoint{2.460657in}{2.350080in}}%
\pgfpathlineto{\pgfqpoint{2.459260in}{2.345413in}}%
\pgfpathlineto{\pgfqpoint{2.453672in}{2.311641in}}%
\pgfpathlineto{\pgfqpoint{2.452275in}{2.308396in}}%
\pgfpathlineto{\pgfqpoint{2.450878in}{2.309759in}}%
\pgfpathlineto{\pgfqpoint{2.448084in}{2.318079in}}%
\pgfpathlineto{\pgfqpoint{2.445289in}{2.318078in}}%
\pgfpathlineto{\pgfqpoint{2.442495in}{2.324473in}}%
\pgfpathlineto{\pgfqpoint{2.441098in}{2.323220in}}%
\pgfpathlineto{\pgfqpoint{2.438304in}{2.317935in}}%
\pgfpathlineto{\pgfqpoint{2.436907in}{2.320980in}}%
\pgfpathlineto{\pgfqpoint{2.434113in}{2.339253in}}%
\pgfpathlineto{\pgfqpoint{2.431319in}{2.357811in}}%
\pgfpathlineto{\pgfqpoint{2.429922in}{2.357990in}}%
\pgfpathlineto{\pgfqpoint{2.427128in}{2.339840in}}%
\pgfpathlineto{\pgfqpoint{2.424334in}{2.314392in}}%
\pgfpathlineto{\pgfqpoint{2.422937in}{2.311137in}}%
\pgfpathlineto{\pgfqpoint{2.420142in}{2.314606in}}%
\pgfpathlineto{\pgfqpoint{2.414554in}{2.299820in}}%
\pgfpathlineto{\pgfqpoint{2.411760in}{2.289318in}}%
\pgfpathlineto{\pgfqpoint{2.410363in}{2.293064in}}%
\pgfpathlineto{\pgfqpoint{2.406172in}{2.312303in}}%
\pgfpathlineto{\pgfqpoint{2.401981in}{2.321488in}}%
\pgfpathlineto{\pgfqpoint{2.400584in}{2.322679in}}%
\pgfpathlineto{\pgfqpoint{2.399187in}{2.321747in}}%
\pgfpathlineto{\pgfqpoint{2.397790in}{2.314343in}}%
\pgfpathlineto{\pgfqpoint{2.393598in}{2.277190in}}%
\pgfpathlineto{\pgfqpoint{2.392201in}{2.279294in}}%
\pgfpathlineto{\pgfqpoint{2.389407in}{2.286432in}}%
\pgfpathlineto{\pgfqpoint{2.388010in}{2.287018in}}%
\pgfpathlineto{\pgfqpoint{2.386613in}{2.290835in}}%
\pgfpathlineto{\pgfqpoint{2.382422in}{2.320666in}}%
\pgfpathlineto{\pgfqpoint{2.381025in}{2.319364in}}%
\pgfpathlineto{\pgfqpoint{2.378231in}{2.312021in}}%
\pgfpathlineto{\pgfqpoint{2.372643in}{2.330890in}}%
\pgfpathlineto{\pgfqpoint{2.369848in}{2.326402in}}%
\pgfpathlineto{\pgfqpoint{2.367054in}{2.316168in}}%
\pgfpathlineto{\pgfqpoint{2.358672in}{2.268111in}}%
\pgfpathlineto{\pgfqpoint{2.355878in}{2.263737in}}%
\pgfpathlineto{\pgfqpoint{2.354481in}{2.264752in}}%
\pgfpathlineto{\pgfqpoint{2.353084in}{2.267601in}}%
\pgfpathlineto{\pgfqpoint{2.351687in}{2.273628in}}%
\pgfpathlineto{\pgfqpoint{2.348893in}{2.304004in}}%
\pgfpathlineto{\pgfqpoint{2.346098in}{2.328557in}}%
\pgfpathlineto{\pgfqpoint{2.344701in}{2.324876in}}%
\pgfpathlineto{\pgfqpoint{2.339113in}{2.277856in}}%
\pgfpathlineto{\pgfqpoint{2.336319in}{2.283389in}}%
\pgfpathlineto{\pgfqpoint{2.329334in}{2.243125in}}%
\pgfpathlineto{\pgfqpoint{2.325143in}{2.237081in}}%
\pgfpathlineto{\pgfqpoint{2.323746in}{2.240877in}}%
\pgfpathlineto{\pgfqpoint{2.319554in}{2.269273in}}%
\pgfpathlineto{\pgfqpoint{2.313966in}{2.303652in}}%
\pgfpathlineto{\pgfqpoint{2.312569in}{2.301883in}}%
\pgfpathlineto{\pgfqpoint{2.308378in}{2.285851in}}%
\pgfpathlineto{\pgfqpoint{2.306981in}{2.284787in}}%
\pgfpathlineto{\pgfqpoint{2.305584in}{2.285282in}}%
\pgfpathlineto{\pgfqpoint{2.304187in}{2.282929in}}%
\pgfpathlineto{\pgfqpoint{2.301393in}{2.258192in}}%
\pgfpathlineto{\pgfqpoint{2.298599in}{2.235760in}}%
\pgfpathlineto{\pgfqpoint{2.297201in}{2.235206in}}%
\pgfpathlineto{\pgfqpoint{2.294407in}{2.249683in}}%
\pgfpathlineto{\pgfqpoint{2.291613in}{2.264052in}}%
\pgfpathlineto{\pgfqpoint{2.288819in}{2.270993in}}%
\pgfpathlineto{\pgfqpoint{2.286025in}{2.284528in}}%
\pgfpathlineto{\pgfqpoint{2.284628in}{2.285197in}}%
\pgfpathlineto{\pgfqpoint{2.281834in}{2.276032in}}%
\pgfpathlineto{\pgfqpoint{2.276246in}{2.251983in}}%
\pgfpathlineto{\pgfqpoint{2.263672in}{2.223033in}}%
\pgfpathlineto{\pgfqpoint{2.262275in}{2.228004in}}%
\pgfpathlineto{\pgfqpoint{2.258084in}{2.264733in}}%
\pgfpathlineto{\pgfqpoint{2.255290in}{2.267556in}}%
\pgfpathlineto{\pgfqpoint{2.253893in}{2.265993in}}%
\pgfpathlineto{\pgfqpoint{2.249702in}{2.235568in}}%
\pgfpathlineto{\pgfqpoint{2.248305in}{2.238097in}}%
\pgfpathlineto{\pgfqpoint{2.245510in}{2.246219in}}%
\pgfpathlineto{\pgfqpoint{2.244113in}{2.239724in}}%
\pgfpathlineto{\pgfqpoint{2.241319in}{2.210142in}}%
\pgfpathlineto{\pgfqpoint{2.238525in}{2.165688in}}%
\pgfpathlineto{\pgfqpoint{2.237128in}{2.097549in}}%
\pgfpathlineto{\pgfqpoint{2.234334in}{1.686037in}}%
\pgfpathlineto{\pgfqpoint{2.231540in}{1.351365in}}%
\pgfpathlineto{\pgfqpoint{2.230143in}{1.467139in}}%
\pgfpathlineto{\pgfqpoint{2.225952in}{2.131147in}}%
\pgfpathlineto{\pgfqpoint{2.223158in}{2.240319in}}%
\pgfpathlineto{\pgfqpoint{2.221760in}{2.251851in}}%
\pgfpathlineto{\pgfqpoint{2.220363in}{2.252992in}}%
\pgfpathlineto{\pgfqpoint{2.217569in}{2.240779in}}%
\pgfpathlineto{\pgfqpoint{2.216172in}{2.221460in}}%
\pgfpathlineto{\pgfqpoint{2.214775in}{2.169500in}}%
\pgfpathlineto{\pgfqpoint{2.209187in}{1.851615in}}%
\pgfpathlineto{\pgfqpoint{2.202202in}{2.202930in}}%
\pgfpathlineto{\pgfqpoint{2.199408in}{2.216513in}}%
\pgfpathlineto{\pgfqpoint{2.198010in}{2.217074in}}%
\pgfpathlineto{\pgfqpoint{2.196613in}{2.222463in}}%
\pgfpathlineto{\pgfqpoint{2.192422in}{2.258554in}}%
\pgfpathlineto{\pgfqpoint{2.191025in}{2.257723in}}%
\pgfpathlineto{\pgfqpoint{2.188231in}{2.227889in}}%
\pgfpathlineto{\pgfqpoint{2.184040in}{2.182328in}}%
\pgfpathlineto{\pgfqpoint{2.182643in}{2.179155in}}%
\pgfpathlineto{\pgfqpoint{2.181246in}{2.179928in}}%
\pgfpathlineto{\pgfqpoint{2.172863in}{2.196220in}}%
\pgfpathlineto{\pgfqpoint{2.168672in}{2.201966in}}%
\pgfpathlineto{\pgfqpoint{2.163084in}{2.219017in}}%
\pgfpathlineto{\pgfqpoint{2.160290in}{2.198739in}}%
\pgfpathlineto{\pgfqpoint{2.157496in}{2.179508in}}%
\pgfpathlineto{\pgfqpoint{2.156099in}{2.180159in}}%
\pgfpathlineto{\pgfqpoint{2.154702in}{2.183162in}}%
\pgfpathlineto{\pgfqpoint{2.153305in}{2.183231in}}%
\pgfpathlineto{\pgfqpoint{2.150511in}{2.170612in}}%
\pgfpathlineto{\pgfqpoint{2.147716in}{2.160578in}}%
\pgfpathlineto{\pgfqpoint{2.146319in}{2.162930in}}%
\pgfpathlineto{\pgfqpoint{2.139334in}{2.191271in}}%
\pgfpathlineto{\pgfqpoint{2.135143in}{2.191983in}}%
\pgfpathlineto{\pgfqpoint{2.132349in}{2.206126in}}%
\pgfpathlineto{\pgfqpoint{2.129555in}{2.219795in}}%
\pgfpathlineto{\pgfqpoint{2.128158in}{2.216865in}}%
\pgfpathlineto{\pgfqpoint{2.121172in}{2.173353in}}%
\pgfpathlineto{\pgfqpoint{2.116981in}{2.143635in}}%
\pgfpathlineto{\pgfqpoint{2.115584in}{2.140506in}}%
\pgfpathlineto{\pgfqpoint{2.114187in}{2.146043in}}%
\pgfpathlineto{\pgfqpoint{2.109996in}{2.189488in}}%
\pgfpathlineto{\pgfqpoint{2.108599in}{2.188806in}}%
\pgfpathlineto{\pgfqpoint{2.105805in}{2.184382in}}%
\pgfpathlineto{\pgfqpoint{2.104408in}{2.183424in}}%
\pgfpathlineto{\pgfqpoint{2.100217in}{2.167916in}}%
\pgfpathlineto{\pgfqpoint{2.098819in}{2.168914in}}%
\pgfpathlineto{\pgfqpoint{2.096025in}{2.174579in}}%
\pgfpathlineto{\pgfqpoint{2.094628in}{2.171645in}}%
\pgfpathlineto{\pgfqpoint{2.091834in}{2.146586in}}%
\pgfpathlineto{\pgfqpoint{2.089040in}{2.122328in}}%
\pgfpathlineto{\pgfqpoint{2.087643in}{2.120226in}}%
\pgfpathlineto{\pgfqpoint{2.084849in}{2.125270in}}%
\pgfpathlineto{\pgfqpoint{2.083452in}{2.124906in}}%
\pgfpathlineto{\pgfqpoint{2.080658in}{2.119192in}}%
\pgfpathlineto{\pgfqpoint{2.079261in}{2.121306in}}%
\pgfpathlineto{\pgfqpoint{2.076467in}{2.140103in}}%
\pgfpathlineto{\pgfqpoint{2.072275in}{2.167831in}}%
\pgfpathlineto{\pgfqpoint{2.070878in}{2.168072in}}%
\pgfpathlineto{\pgfqpoint{2.065290in}{2.145906in}}%
\pgfpathlineto{\pgfqpoint{2.059702in}{2.104349in}}%
\pgfpathlineto{\pgfqpoint{2.056908in}{2.107132in}}%
\pgfpathlineto{\pgfqpoint{2.052717in}{2.098713in}}%
\pgfpathlineto{\pgfqpoint{2.042937in}{2.141150in}}%
\pgfpathlineto{\pgfqpoint{2.038746in}{2.143930in}}%
\pgfpathlineto{\pgfqpoint{2.037349in}{2.147069in}}%
\pgfpathlineto{\pgfqpoint{2.035952in}{2.145049in}}%
\pgfpathlineto{\pgfqpoint{2.030364in}{2.110796in}}%
\pgfpathlineto{\pgfqpoint{2.028967in}{2.113889in}}%
\pgfpathlineto{\pgfqpoint{2.027570in}{2.112175in}}%
\pgfpathlineto{\pgfqpoint{2.024775in}{2.097754in}}%
\pgfpathlineto{\pgfqpoint{2.023378in}{2.098635in}}%
\pgfpathlineto{\pgfqpoint{2.020584in}{2.109199in}}%
\pgfpathlineto{\pgfqpoint{2.017790in}{2.110709in}}%
\pgfpathlineto{\pgfqpoint{2.016393in}{2.118118in}}%
\pgfpathlineto{\pgfqpoint{2.010805in}{2.170272in}}%
\pgfpathlineto{\pgfqpoint{2.009408in}{2.167250in}}%
\pgfpathlineto{\pgfqpoint{2.006614in}{2.141732in}}%
\pgfpathlineto{\pgfqpoint{1.999628in}{2.065349in}}%
\pgfpathlineto{\pgfqpoint{1.996834in}{2.049403in}}%
\pgfpathlineto{\pgfqpoint{1.995437in}{2.054326in}}%
\pgfpathlineto{\pgfqpoint{1.991246in}{2.086434in}}%
\pgfpathlineto{\pgfqpoint{1.987055in}{2.067597in}}%
\pgfpathlineto{\pgfqpoint{1.985658in}{2.072432in}}%
\pgfpathlineto{\pgfqpoint{1.982864in}{2.086443in}}%
\pgfpathlineto{\pgfqpoint{1.980070in}{2.080274in}}%
\pgfpathlineto{\pgfqpoint{1.978673in}{2.082748in}}%
\pgfpathlineto{\pgfqpoint{1.975878in}{2.091215in}}%
\pgfpathlineto{\pgfqpoint{1.974481in}{2.081878in}}%
\pgfpathlineto{\pgfqpoint{1.973084in}{2.057097in}}%
\pgfpathlineto{\pgfqpoint{1.971687in}{1.996567in}}%
\pgfpathlineto{\pgfqpoint{1.968893in}{1.685248in}}%
\pgfpathlineto{\pgfqpoint{1.966099in}{1.452389in}}%
\pgfpathlineto{\pgfqpoint{1.964702in}{1.542913in}}%
\pgfpathlineto{\pgfqpoint{1.960511in}{2.005281in}}%
\pgfpathlineto{\pgfqpoint{1.957717in}{2.063884in}}%
\pgfpathlineto{\pgfqpoint{1.954923in}{2.077690in}}%
\pgfpathlineto{\pgfqpoint{1.953526in}{2.079224in}}%
\pgfpathlineto{\pgfqpoint{1.952129in}{2.076762in}}%
\pgfpathlineto{\pgfqpoint{1.950731in}{2.060233in}}%
\pgfpathlineto{\pgfqpoint{1.947937in}{1.933275in}}%
\pgfpathlineto{\pgfqpoint{1.945143in}{1.792400in}}%
\pgfpathlineto{\pgfqpoint{1.943746in}{1.801135in}}%
\pgfpathlineto{\pgfqpoint{1.938158in}{2.006335in}}%
\pgfpathlineto{\pgfqpoint{1.932570in}{2.035546in}}%
\pgfpathlineto{\pgfqpoint{1.926982in}{2.054148in}}%
\pgfpathlineto{\pgfqpoint{1.925584in}{2.054242in}}%
\pgfpathlineto{\pgfqpoint{1.924187in}{2.059378in}}%
\pgfpathlineto{\pgfqpoint{1.921393in}{2.074101in}}%
\pgfpathlineto{\pgfqpoint{1.919996in}{2.068866in}}%
\pgfpathlineto{\pgfqpoint{1.914408in}{2.025415in}}%
\pgfpathlineto{\pgfqpoint{1.913011in}{2.023692in}}%
\pgfpathlineto{\pgfqpoint{1.908820in}{2.028214in}}%
\pgfpathlineto{\pgfqpoint{1.906026in}{2.005626in}}%
\pgfpathlineto{\pgfqpoint{1.903232in}{1.983538in}}%
\pgfpathlineto{\pgfqpoint{1.901834in}{1.981905in}}%
\pgfpathlineto{\pgfqpoint{1.900437in}{1.986246in}}%
\pgfpathlineto{\pgfqpoint{1.894849in}{2.021884in}}%
\pgfpathlineto{\pgfqpoint{1.893452in}{2.023137in}}%
\pgfpathlineto{\pgfqpoint{1.892055in}{2.020264in}}%
\pgfpathlineto{\pgfqpoint{1.887864in}{2.004199in}}%
\pgfpathlineto{\pgfqpoint{1.886467in}{2.003862in}}%
\pgfpathlineto{\pgfqpoint{1.882276in}{2.014233in}}%
\pgfpathlineto{\pgfqpoint{1.880879in}{2.007503in}}%
\pgfpathlineto{\pgfqpoint{1.875290in}{1.953378in}}%
\pgfpathlineto{\pgfqpoint{1.873893in}{1.955851in}}%
\pgfpathlineto{\pgfqpoint{1.864114in}{1.992504in}}%
\pgfpathlineto{\pgfqpoint{1.859923in}{1.976531in}}%
\pgfpathlineto{\pgfqpoint{1.858526in}{1.978585in}}%
\pgfpathlineto{\pgfqpoint{1.855732in}{1.986405in}}%
\pgfpathlineto{\pgfqpoint{1.854335in}{1.980951in}}%
\pgfpathlineto{\pgfqpoint{1.850143in}{1.950937in}}%
\pgfpathlineto{\pgfqpoint{1.847349in}{1.942400in}}%
\pgfpathlineto{\pgfqpoint{1.844555in}{1.927758in}}%
\pgfpathlineto{\pgfqpoint{1.843158in}{1.929573in}}%
\pgfpathlineto{\pgfqpoint{1.834776in}{1.983373in}}%
\pgfpathlineto{\pgfqpoint{1.833379in}{1.987280in}}%
\pgfpathlineto{\pgfqpoint{1.831982in}{1.984396in}}%
\pgfpathlineto{\pgfqpoint{1.827790in}{1.964472in}}%
\pgfpathlineto{\pgfqpoint{1.824996in}{1.959425in}}%
\pgfpathlineto{\pgfqpoint{1.822202in}{1.939002in}}%
\pgfpathlineto{\pgfqpoint{1.819408in}{1.917579in}}%
\pgfpathlineto{\pgfqpoint{1.812423in}{1.902501in}}%
\pgfpathlineto{\pgfqpoint{1.811026in}{1.902563in}}%
\pgfpathlineto{\pgfqpoint{1.809629in}{1.904415in}}%
\pgfpathlineto{\pgfqpoint{1.804041in}{1.918438in}}%
\pgfpathlineto{\pgfqpoint{1.799849in}{1.923852in}}%
\pgfpathlineto{\pgfqpoint{1.797055in}{1.918802in}}%
\pgfpathlineto{\pgfqpoint{1.794261in}{1.910764in}}%
\pgfpathlineto{\pgfqpoint{1.791467in}{1.901128in}}%
\pgfpathlineto{\pgfqpoint{1.788673in}{1.903211in}}%
\pgfpathlineto{\pgfqpoint{1.787276in}{1.901304in}}%
\pgfpathlineto{\pgfqpoint{1.783085in}{1.887565in}}%
\pgfpathlineto{\pgfqpoint{1.780291in}{1.887063in}}%
\pgfpathlineto{\pgfqpoint{1.778894in}{1.888685in}}%
\pgfpathlineto{\pgfqpoint{1.776099in}{1.898602in}}%
\pgfpathlineto{\pgfqpoint{1.773305in}{1.908710in}}%
\pgfpathlineto{\pgfqpoint{1.771908in}{1.906259in}}%
\pgfpathlineto{\pgfqpoint{1.760732in}{1.853887in}}%
\pgfpathlineto{\pgfqpoint{1.757938in}{1.851721in}}%
\pgfpathlineto{\pgfqpoint{1.753746in}{1.854801in}}%
\pgfpathlineto{\pgfqpoint{1.752349in}{1.854739in}}%
\pgfpathlineto{\pgfqpoint{1.750952in}{1.857874in}}%
\pgfpathlineto{\pgfqpoint{1.746761in}{1.871610in}}%
\pgfpathlineto{\pgfqpoint{1.742570in}{1.873085in}}%
\pgfpathlineto{\pgfqpoint{1.741173in}{1.868543in}}%
\pgfpathlineto{\pgfqpoint{1.736982in}{1.836407in}}%
\pgfpathlineto{\pgfqpoint{1.732791in}{1.794765in}}%
\pgfpathlineto{\pgfqpoint{1.731394in}{1.791756in}}%
\pgfpathlineto{\pgfqpoint{1.727202in}{1.822784in}}%
\pgfpathlineto{\pgfqpoint{1.725805in}{1.822364in}}%
\pgfpathlineto{\pgfqpoint{1.723011in}{1.809690in}}%
\pgfpathlineto{\pgfqpoint{1.721614in}{1.808671in}}%
\pgfpathlineto{\pgfqpoint{1.717423in}{1.820797in}}%
\pgfpathlineto{\pgfqpoint{1.714629in}{1.818186in}}%
\pgfpathlineto{\pgfqpoint{1.713232in}{1.821618in}}%
\pgfpathlineto{\pgfqpoint{1.710438in}{1.838687in}}%
\pgfpathlineto{\pgfqpoint{1.709041in}{1.839545in}}%
\pgfpathlineto{\pgfqpoint{1.704850in}{1.819801in}}%
\pgfpathlineto{\pgfqpoint{1.702055in}{1.802584in}}%
\pgfpathlineto{\pgfqpoint{1.700658in}{1.777063in}}%
\pgfpathlineto{\pgfqpoint{1.697864in}{1.600900in}}%
\pgfpathlineto{\pgfqpoint{1.695070in}{1.403198in}}%
\pgfpathlineto{\pgfqpoint{1.693673in}{1.424782in}}%
\pgfpathlineto{\pgfqpoint{1.688085in}{1.774686in}}%
\pgfpathlineto{\pgfqpoint{1.686688in}{1.791317in}}%
\pgfpathlineto{\pgfqpoint{1.685291in}{1.793604in}}%
\pgfpathlineto{\pgfqpoint{1.683894in}{1.792455in}}%
\pgfpathlineto{\pgfqpoint{1.682497in}{1.795619in}}%
\pgfpathlineto{\pgfqpoint{1.681100in}{1.802203in}}%
\pgfpathlineto{\pgfqpoint{1.679702in}{1.798204in}}%
\pgfpathlineto{\pgfqpoint{1.678305in}{1.767041in}}%
\pgfpathlineto{\pgfqpoint{1.674114in}{1.618300in}}%
\pgfpathlineto{\pgfqpoint{1.672717in}{1.629008in}}%
\pgfpathlineto{\pgfqpoint{1.668526in}{1.721649in}}%
\pgfpathlineto{\pgfqpoint{1.664335in}{1.734256in}}%
\pgfpathlineto{\pgfqpoint{1.658747in}{1.777859in}}%
\pgfpathlineto{\pgfqpoint{1.657350in}{1.775126in}}%
\pgfpathlineto{\pgfqpoint{1.654555in}{1.763979in}}%
\pgfpathlineto{\pgfqpoint{1.653158in}{1.763068in}}%
\pgfpathlineto{\pgfqpoint{1.650364in}{1.765043in}}%
\pgfpathlineto{\pgfqpoint{1.648967in}{1.764128in}}%
\pgfpathlineto{\pgfqpoint{1.647570in}{1.759054in}}%
\pgfpathlineto{\pgfqpoint{1.640585in}{1.709681in}}%
\pgfpathlineto{\pgfqpoint{1.637791in}{1.704826in}}%
\pgfpathlineto{\pgfqpoint{1.636394in}{1.706709in}}%
\pgfpathlineto{\pgfqpoint{1.634997in}{1.706127in}}%
\pgfpathlineto{\pgfqpoint{1.632203in}{1.691125in}}%
\pgfpathlineto{\pgfqpoint{1.630806in}{1.692938in}}%
\pgfpathlineto{\pgfqpoint{1.626614in}{1.735575in}}%
\pgfpathlineto{\pgfqpoint{1.625217in}{1.737328in}}%
\pgfpathlineto{\pgfqpoint{1.623820in}{1.737107in}}%
\pgfpathlineto{\pgfqpoint{1.621026in}{1.739441in}}%
\pgfpathlineto{\pgfqpoint{1.619629in}{1.735742in}}%
\pgfpathlineto{\pgfqpoint{1.614041in}{1.702678in}}%
\pgfpathlineto{\pgfqpoint{1.609850in}{1.657668in}}%
\pgfpathlineto{\pgfqpoint{1.608453in}{1.660867in}}%
\pgfpathlineto{\pgfqpoint{1.604261in}{1.699600in}}%
\pgfpathlineto{\pgfqpoint{1.602864in}{1.700799in}}%
\pgfpathlineto{\pgfqpoint{1.600070in}{1.681082in}}%
\pgfpathlineto{\pgfqpoint{1.598673in}{1.670669in}}%
\pgfpathlineto{\pgfqpoint{1.597276in}{1.670685in}}%
\pgfpathlineto{\pgfqpoint{1.593085in}{1.705151in}}%
\pgfpathlineto{\pgfqpoint{1.591688in}{1.703856in}}%
\pgfpathlineto{\pgfqpoint{1.587497in}{1.685459in}}%
\pgfpathlineto{\pgfqpoint{1.584703in}{1.664648in}}%
\pgfpathlineto{\pgfqpoint{1.583306in}{1.661468in}}%
\pgfpathlineto{\pgfqpoint{1.581909in}{1.661623in}}%
\pgfpathlineto{\pgfqpoint{1.580511in}{1.658194in}}%
\pgfpathlineto{\pgfqpoint{1.574923in}{1.620365in}}%
\pgfpathlineto{\pgfqpoint{1.570732in}{1.652448in}}%
\pgfpathlineto{\pgfqpoint{1.569335in}{1.650399in}}%
\pgfpathlineto{\pgfqpoint{1.566541in}{1.639705in}}%
\pgfpathlineto{\pgfqpoint{1.563747in}{1.661316in}}%
\pgfpathlineto{\pgfqpoint{1.560953in}{1.676285in}}%
\pgfpathlineto{\pgfqpoint{1.559556in}{1.672423in}}%
\pgfpathlineto{\pgfqpoint{1.551173in}{1.623876in}}%
\pgfpathlineto{\pgfqpoint{1.548379in}{1.611206in}}%
\pgfpathlineto{\pgfqpoint{1.545585in}{1.591593in}}%
\pgfpathlineto{\pgfqpoint{1.544188in}{1.590241in}}%
\pgfpathlineto{\pgfqpoint{1.535806in}{1.616191in}}%
\pgfpathlineto{\pgfqpoint{1.533012in}{1.610086in}}%
\pgfpathlineto{\pgfqpoint{1.531615in}{1.610650in}}%
\pgfpathlineto{\pgfqpoint{1.530217in}{1.609223in}}%
\pgfpathlineto{\pgfqpoint{1.526026in}{1.577899in}}%
\pgfpathlineto{\pgfqpoint{1.524629in}{1.579757in}}%
\pgfpathlineto{\pgfqpoint{1.521835in}{1.586295in}}%
\pgfpathlineto{\pgfqpoint{1.520438in}{1.585510in}}%
\pgfpathlineto{\pgfqpoint{1.519041in}{1.586646in}}%
\pgfpathlineto{\pgfqpoint{1.516247in}{1.591843in}}%
\pgfpathlineto{\pgfqpoint{1.514850in}{1.588563in}}%
\pgfpathlineto{\pgfqpoint{1.510659in}{1.565319in}}%
\pgfpathlineto{\pgfqpoint{1.509262in}{1.565484in}}%
\pgfpathlineto{\pgfqpoint{1.505070in}{1.585500in}}%
\pgfpathlineto{\pgfqpoint{1.502276in}{1.596951in}}%
\pgfpathlineto{\pgfqpoint{1.500879in}{1.599319in}}%
\pgfpathlineto{\pgfqpoint{1.499482in}{1.598106in}}%
\pgfpathlineto{\pgfqpoint{1.496688in}{1.586924in}}%
\pgfpathlineto{\pgfqpoint{1.492497in}{1.555873in}}%
\pgfpathlineto{\pgfqpoint{1.488306in}{1.523499in}}%
\pgfpathlineto{\pgfqpoint{1.484115in}{1.515692in}}%
\pgfpathlineto{\pgfqpoint{1.482718in}{1.514757in}}%
\pgfpathlineto{\pgfqpoint{1.481320in}{1.519064in}}%
\pgfpathlineto{\pgfqpoint{1.477129in}{1.542760in}}%
\pgfpathlineto{\pgfqpoint{1.475732in}{1.543790in}}%
\pgfpathlineto{\pgfqpoint{1.471541in}{1.539614in}}%
\pgfpathlineto{\pgfqpoint{1.468747in}{1.541261in}}%
\pgfpathlineto{\pgfqpoint{1.465953in}{1.532289in}}%
\pgfpathlineto{\pgfqpoint{1.463159in}{1.515732in}}%
\pgfpathlineto{\pgfqpoint{1.458968in}{1.464843in}}%
\pgfpathlineto{\pgfqpoint{1.457571in}{1.466264in}}%
\pgfpathlineto{\pgfqpoint{1.453379in}{1.495721in}}%
\pgfpathlineto{\pgfqpoint{1.450585in}{1.503135in}}%
\pgfpathlineto{\pgfqpoint{1.443600in}{1.538053in}}%
\pgfpathlineto{\pgfqpoint{1.442203in}{1.535804in}}%
\pgfpathlineto{\pgfqpoint{1.440806in}{1.529302in}}%
\pgfpathlineto{\pgfqpoint{1.438012in}{1.498248in}}%
\pgfpathlineto{\pgfqpoint{1.435218in}{1.466420in}}%
\pgfpathlineto{\pgfqpoint{1.432423in}{1.459971in}}%
\pgfpathlineto{\pgfqpoint{1.431026in}{1.457834in}}%
\pgfpathlineto{\pgfqpoint{1.428232in}{1.441455in}}%
\pgfpathlineto{\pgfqpoint{1.422644in}{1.395935in}}%
\pgfpathlineto{\pgfqpoint{1.418453in}{1.260974in}}%
\pgfpathlineto{\pgfqpoint{1.417056in}{1.264451in}}%
\pgfpathlineto{\pgfqpoint{1.410071in}{1.468831in}}%
\pgfpathlineto{\pgfqpoint{1.405879in}{1.442735in}}%
\pgfpathlineto{\pgfqpoint{1.404482in}{1.441546in}}%
\pgfpathlineto{\pgfqpoint{1.403085in}{1.434011in}}%
\pgfpathlineto{\pgfqpoint{1.397497in}{1.351459in}}%
\pgfpathlineto{\pgfqpoint{1.396100in}{1.358010in}}%
\pgfpathlineto{\pgfqpoint{1.390512in}{1.410160in}}%
\pgfpathlineto{\pgfqpoint{1.384924in}{1.431867in}}%
\pgfpathlineto{\pgfqpoint{1.380732in}{1.438102in}}%
\pgfpathlineto{\pgfqpoint{1.379335in}{1.435758in}}%
\pgfpathlineto{\pgfqpoint{1.370953in}{1.396681in}}%
\pgfpathlineto{\pgfqpoint{1.365365in}{1.406654in}}%
\pgfpathlineto{\pgfqpoint{1.363968in}{1.405312in}}%
\pgfpathlineto{\pgfqpoint{1.359777in}{1.394628in}}%
\pgfpathlineto{\pgfqpoint{1.358379in}{1.400164in}}%
\pgfpathlineto{\pgfqpoint{1.354188in}{1.436086in}}%
\pgfpathlineto{\pgfqpoint{1.352791in}{1.433562in}}%
\pgfpathlineto{\pgfqpoint{1.348600in}{1.404306in}}%
\pgfpathlineto{\pgfqpoint{1.345806in}{1.399728in}}%
\pgfpathlineto{\pgfqpoint{1.343012in}{1.381371in}}%
\pgfpathlineto{\pgfqpoint{1.337424in}{1.336929in}}%
\pgfpathlineto{\pgfqpoint{1.336027in}{1.333448in}}%
\pgfpathlineto{\pgfqpoint{1.334630in}{1.334040in}}%
\pgfpathlineto{\pgfqpoint{1.329041in}{1.346875in}}%
\pgfpathlineto{\pgfqpoint{1.327644in}{1.345699in}}%
\pgfpathlineto{\pgfqpoint{1.323453in}{1.337049in}}%
\pgfpathlineto{\pgfqpoint{1.322056in}{1.339438in}}%
\pgfpathlineto{\pgfqpoint{1.319262in}{1.348095in}}%
\pgfpathlineto{\pgfqpoint{1.317865in}{1.347473in}}%
\pgfpathlineto{\pgfqpoint{1.310880in}{1.319241in}}%
\pgfpathlineto{\pgfqpoint{1.309483in}{1.318024in}}%
\pgfpathlineto{\pgfqpoint{1.308085in}{1.320819in}}%
\pgfpathlineto{\pgfqpoint{1.305291in}{1.332856in}}%
\pgfpathlineto{\pgfqpoint{1.303894in}{1.333977in}}%
\pgfpathlineto{\pgfqpoint{1.299703in}{1.326336in}}%
\pgfpathlineto{\pgfqpoint{1.298306in}{1.330296in}}%
\pgfpathlineto{\pgfqpoint{1.295512in}{1.344356in}}%
\pgfpathlineto{\pgfqpoint{1.294115in}{1.343425in}}%
\pgfpathlineto{\pgfqpoint{1.291321in}{1.334497in}}%
\pgfpathlineto{\pgfqpoint{1.288527in}{1.338962in}}%
\pgfpathlineto{\pgfqpoint{1.287130in}{1.332405in}}%
\pgfpathlineto{\pgfqpoint{1.284335in}{1.310510in}}%
\pgfpathlineto{\pgfqpoint{1.278747in}{1.295677in}}%
\pgfpathlineto{\pgfqpoint{1.275953in}{1.288243in}}%
\pgfpathlineto{\pgfqpoint{1.273159in}{1.276250in}}%
\pgfpathlineto{\pgfqpoint{1.271762in}{1.280690in}}%
\pgfpathlineto{\pgfqpoint{1.267571in}{1.306249in}}%
\pgfpathlineto{\pgfqpoint{1.263380in}{1.304521in}}%
\pgfpathlineto{\pgfqpoint{1.261983in}{1.304987in}}%
\pgfpathlineto{\pgfqpoint{1.260586in}{1.302128in}}%
\pgfpathlineto{\pgfqpoint{1.256394in}{1.283153in}}%
\pgfpathlineto{\pgfqpoint{1.253600in}{1.285081in}}%
\pgfpathlineto{\pgfqpoint{1.252203in}{1.281228in}}%
\pgfpathlineto{\pgfqpoint{1.248012in}{1.249156in}}%
\pgfpathlineto{\pgfqpoint{1.246615in}{1.252040in}}%
\pgfpathlineto{\pgfqpoint{1.242424in}{1.274406in}}%
\pgfpathlineto{\pgfqpoint{1.241027in}{1.276916in}}%
\pgfpathlineto{\pgfqpoint{1.236836in}{1.294971in}}%
\pgfpathlineto{\pgfqpoint{1.235439in}{1.291147in}}%
\pgfpathlineto{\pgfqpoint{1.231247in}{1.271317in}}%
\pgfpathlineto{\pgfqpoint{1.227056in}{1.258677in}}%
\pgfpathlineto{\pgfqpoint{1.224262in}{1.250960in}}%
\pgfpathlineto{\pgfqpoint{1.220071in}{1.243537in}}%
\pgfpathlineto{\pgfqpoint{1.217277in}{1.238734in}}%
\pgfpathlineto{\pgfqpoint{1.213086in}{1.247417in}}%
\pgfpathlineto{\pgfqpoint{1.210291in}{1.241660in}}%
\pgfpathlineto{\pgfqpoint{1.208894in}{1.243186in}}%
\pgfpathlineto{\pgfqpoint{1.204703in}{1.255650in}}%
\pgfpathlineto{\pgfqpoint{1.200512in}{1.259896in}}%
\pgfpathlineto{\pgfqpoint{1.199115in}{1.258075in}}%
\pgfpathlineto{\pgfqpoint{1.196321in}{1.240530in}}%
\pgfpathlineto{\pgfqpoint{1.192130in}{1.204586in}}%
\pgfpathlineto{\pgfqpoint{1.190733in}{1.200077in}}%
\pgfpathlineto{\pgfqpoint{1.189336in}{1.200921in}}%
\pgfpathlineto{\pgfqpoint{1.186542in}{1.206444in}}%
\pgfpathlineto{\pgfqpoint{1.183747in}{1.204864in}}%
\pgfpathlineto{\pgfqpoint{1.182350in}{1.207317in}}%
\pgfpathlineto{\pgfqpoint{1.176762in}{1.241214in}}%
\pgfpathlineto{\pgfqpoint{1.175365in}{1.238714in}}%
\pgfpathlineto{\pgfqpoint{1.172571in}{1.224744in}}%
\pgfpathlineto{\pgfqpoint{1.166983in}{1.187427in}}%
\pgfpathlineto{\pgfqpoint{1.165586in}{1.185010in}}%
\pgfpathlineto{\pgfqpoint{1.164189in}{1.186208in}}%
\pgfpathlineto{\pgfqpoint{1.161395in}{1.200624in}}%
\pgfpathlineto{\pgfqpoint{1.158600in}{1.217071in}}%
\pgfpathlineto{\pgfqpoint{1.155806in}{1.222173in}}%
\pgfpathlineto{\pgfqpoint{1.154409in}{1.220929in}}%
\pgfpathlineto{\pgfqpoint{1.151615in}{1.213987in}}%
\pgfpathlineto{\pgfqpoint{1.150218in}{1.213505in}}%
\pgfpathlineto{\pgfqpoint{1.147424in}{1.217999in}}%
\pgfpathlineto{\pgfqpoint{1.146027in}{1.216451in}}%
\pgfpathlineto{\pgfqpoint{1.143233in}{1.209507in}}%
\pgfpathlineto{\pgfqpoint{1.141836in}{1.209314in}}%
\pgfpathlineto{\pgfqpoint{1.140439in}{1.204950in}}%
\pgfpathlineto{\pgfqpoint{1.139042in}{1.185069in}}%
\pgfpathlineto{\pgfqpoint{1.134850in}{1.076008in}}%
\pgfpathlineto{\pgfqpoint{1.133453in}{1.080299in}}%
\pgfpathlineto{\pgfqpoint{1.129262in}{1.137587in}}%
\pgfpathlineto{\pgfqpoint{1.127865in}{1.137532in}}%
\pgfpathlineto{\pgfqpoint{1.126468in}{1.139198in}}%
\pgfpathlineto{\pgfqpoint{1.123674in}{1.162689in}}%
\pgfpathlineto{\pgfqpoint{1.120880in}{1.181483in}}%
\pgfpathlineto{\pgfqpoint{1.119483in}{1.176185in}}%
\pgfpathlineto{\pgfqpoint{1.115292in}{1.137125in}}%
\pgfpathlineto{\pgfqpoint{1.113895in}{1.138586in}}%
\pgfpathlineto{\pgfqpoint{1.108306in}{1.176248in}}%
\pgfpathlineto{\pgfqpoint{1.106909in}{1.174078in}}%
\pgfpathlineto{\pgfqpoint{1.099924in}{1.153037in}}%
\pgfpathlineto{\pgfqpoint{1.095733in}{1.165044in}}%
\pgfpathlineto{\pgfqpoint{1.092939in}{1.160869in}}%
\pgfpathlineto{\pgfqpoint{1.091542in}{1.161722in}}%
\pgfpathlineto{\pgfqpoint{1.083159in}{1.181224in}}%
\pgfpathlineto{\pgfqpoint{1.081762in}{1.181542in}}%
\pgfpathlineto{\pgfqpoint{1.080365in}{1.180590in}}%
\pgfpathlineto{\pgfqpoint{1.078968in}{1.176117in}}%
\pgfpathlineto{\pgfqpoint{1.076174in}{1.146764in}}%
\pgfpathlineto{\pgfqpoint{1.073380in}{1.116719in}}%
\pgfpathlineto{\pgfqpoint{1.071983in}{1.117180in}}%
\pgfpathlineto{\pgfqpoint{1.067792in}{1.140290in}}%
\pgfpathlineto{\pgfqpoint{1.066395in}{1.142028in}}%
\pgfpathlineto{\pgfqpoint{1.064998in}{1.139938in}}%
\pgfpathlineto{\pgfqpoint{1.060806in}{1.126741in}}%
\pgfpathlineto{\pgfqpoint{1.059409in}{1.129546in}}%
\pgfpathlineto{\pgfqpoint{1.055218in}{1.147719in}}%
\pgfpathlineto{\pgfqpoint{1.053821in}{1.147465in}}%
\pgfpathlineto{\pgfqpoint{1.048233in}{1.127687in}}%
\pgfpathlineto{\pgfqpoint{1.042645in}{1.138002in}}%
\pgfpathlineto{\pgfqpoint{1.041248in}{1.135781in}}%
\pgfpathlineto{\pgfqpoint{1.038454in}{1.115842in}}%
\pgfpathlineto{\pgfqpoint{1.034262in}{1.087402in}}%
\pgfpathlineto{\pgfqpoint{1.032865in}{1.087967in}}%
\pgfpathlineto{\pgfqpoint{1.030071in}{1.103029in}}%
\pgfpathlineto{\pgfqpoint{1.023086in}{1.147625in}}%
\pgfpathlineto{\pgfqpoint{1.021689in}{1.148505in}}%
\pgfpathlineto{\pgfqpoint{1.020292in}{1.144228in}}%
\pgfpathlineto{\pgfqpoint{1.013307in}{1.097716in}}%
\pgfpathlineto{\pgfqpoint{1.011909in}{1.099049in}}%
\pgfpathlineto{\pgfqpoint{1.007718in}{1.130436in}}%
\pgfpathlineto{\pgfqpoint{1.006321in}{1.129957in}}%
\pgfpathlineto{\pgfqpoint{1.002130in}{1.107495in}}%
\pgfpathlineto{\pgfqpoint{1.000733in}{1.107135in}}%
\pgfpathlineto{\pgfqpoint{0.997939in}{1.113643in}}%
\pgfpathlineto{\pgfqpoint{0.992351in}{1.130747in}}%
\pgfpathlineto{\pgfqpoint{0.990954in}{1.129150in}}%
\pgfpathlineto{\pgfqpoint{0.983968in}{1.100192in}}%
\pgfpathlineto{\pgfqpoint{0.981174in}{1.085533in}}%
\pgfpathlineto{\pgfqpoint{0.979777in}{1.086101in}}%
\pgfpathlineto{\pgfqpoint{0.978380in}{1.087925in}}%
\pgfpathlineto{\pgfqpoint{0.975586in}{1.088082in}}%
\pgfpathlineto{\pgfqpoint{0.974189in}{1.093959in}}%
\pgfpathlineto{\pgfqpoint{0.969998in}{1.120560in}}%
\pgfpathlineto{\pgfqpoint{0.968601in}{1.120183in}}%
\pgfpathlineto{\pgfqpoint{0.963012in}{1.100030in}}%
\pgfpathlineto{\pgfqpoint{0.960218in}{1.105838in}}%
\pgfpathlineto{\pgfqpoint{0.958821in}{1.102947in}}%
\pgfpathlineto{\pgfqpoint{0.954630in}{1.080267in}}%
\pgfpathlineto{\pgfqpoint{0.953233in}{1.082741in}}%
\pgfpathlineto{\pgfqpoint{0.949042in}{1.100177in}}%
\pgfpathlineto{\pgfqpoint{0.944851in}{1.092999in}}%
\pgfpathlineto{\pgfqpoint{0.940660in}{1.110234in}}%
\pgfpathlineto{\pgfqpoint{0.939263in}{1.104938in}}%
\pgfpathlineto{\pgfqpoint{0.935071in}{1.077665in}}%
\pgfpathlineto{\pgfqpoint{0.933674in}{1.080601in}}%
\pgfpathlineto{\pgfqpoint{0.928086in}{1.106458in}}%
\pgfpathlineto{\pgfqpoint{0.926689in}{1.106710in}}%
\pgfpathlineto{\pgfqpoint{0.925292in}{1.101200in}}%
\pgfpathlineto{\pgfqpoint{0.921101in}{1.069436in}}%
\pgfpathlineto{\pgfqpoint{0.919704in}{1.067647in}}%
\pgfpathlineto{\pgfqpoint{0.918307in}{1.073205in}}%
\pgfpathlineto{\pgfqpoint{0.914116in}{1.105916in}}%
\pgfpathlineto{\pgfqpoint{0.912718in}{1.108148in}}%
\pgfpathlineto{\pgfqpoint{0.908527in}{1.100534in}}%
\pgfpathlineto{\pgfqpoint{0.904336in}{1.109596in}}%
\pgfpathlineto{\pgfqpoint{0.902939in}{1.107713in}}%
\pgfpathlineto{\pgfqpoint{0.895954in}{1.074787in}}%
\pgfpathlineto{\pgfqpoint{0.891763in}{1.063250in}}%
\pgfpathlineto{\pgfqpoint{0.888968in}{1.069032in}}%
\pgfpathlineto{\pgfqpoint{0.886174in}{1.067272in}}%
\pgfpathlineto{\pgfqpoint{0.884777in}{1.069774in}}%
\pgfpathlineto{\pgfqpoint{0.879189in}{1.100165in}}%
\pgfpathlineto{\pgfqpoint{0.874998in}{1.094757in}}%
\pgfpathlineto{\pgfqpoint{0.872204in}{1.103211in}}%
\pgfpathlineto{\pgfqpoint{0.870807in}{1.099930in}}%
\pgfpathlineto{\pgfqpoint{0.865219in}{1.061638in}}%
\pgfpathlineto{\pgfqpoint{0.862424in}{1.057273in}}%
\pgfpathlineto{\pgfqpoint{0.859630in}{1.060186in}}%
\pgfpathlineto{\pgfqpoint{0.855439in}{1.066860in}}%
\pgfpathlineto{\pgfqpoint{0.852645in}{1.072979in}}%
\pgfpathlineto{\pgfqpoint{0.851248in}{1.066108in}}%
\pgfpathlineto{\pgfqpoint{0.847057in}{1.033481in}}%
\pgfpathlineto{\pgfqpoint{0.845660in}{1.041412in}}%
\pgfpathlineto{\pgfqpoint{0.840072in}{1.094057in}}%
\pgfpathlineto{\pgfqpoint{0.838674in}{1.097772in}}%
\pgfpathlineto{\pgfqpoint{0.837277in}{1.096335in}}%
\pgfpathlineto{\pgfqpoint{0.834483in}{1.084387in}}%
\pgfpathlineto{\pgfqpoint{0.831689in}{1.059373in}}%
\pgfpathlineto{\pgfqpoint{0.828895in}{1.024980in}}%
\pgfpathlineto{\pgfqpoint{0.827498in}{1.021641in}}%
\pgfpathlineto{\pgfqpoint{0.824704in}{1.050626in}}%
\pgfpathlineto{\pgfqpoint{0.821910in}{1.075581in}}%
\pgfpathlineto{\pgfqpoint{0.816322in}{1.096375in}}%
\pgfpathlineto{\pgfqpoint{0.812130in}{1.102186in}}%
\pgfpathlineto{\pgfqpoint{0.810733in}{1.100640in}}%
\pgfpathlineto{\pgfqpoint{0.807939in}{1.078914in}}%
\pgfpathlineto{\pgfqpoint{0.805145in}{1.053979in}}%
\pgfpathlineto{\pgfqpoint{0.803748in}{1.051332in}}%
\pgfpathlineto{\pgfqpoint{0.800954in}{1.055457in}}%
\pgfpathlineto{\pgfqpoint{0.796763in}{1.051298in}}%
\pgfpathlineto{\pgfqpoint{0.793969in}{1.059628in}}%
\pgfpathlineto{\pgfqpoint{0.791175in}{1.073584in}}%
\pgfpathlineto{\pgfqpoint{0.788380in}{1.089919in}}%
\pgfpathlineto{\pgfqpoint{0.785366in}{1.093716in}}%
\pgfpathlineto{\pgfqpoint{0.785366in}{1.093716in}}%
\pgfusepath{stroke}%
\end{pgfscope}%
\begin{pgfscope}%
\pgfsetrectcap%
\pgfsetmiterjoin%
\pgfsetlinewidth{0.803000pt}%
\definecolor{currentstroke}{rgb}{0.000000,0.000000,0.000000}%
\pgfsetstrokecolor{currentstroke}%
\pgfsetdash{}{0pt}%
\pgfpathmoveto{\pgfqpoint{0.795366in}{0.646140in}}%
\pgfpathlineto{\pgfqpoint{0.795366in}{3.734428in}}%
\pgfusepath{stroke}%
\end{pgfscope}%
\begin{pgfscope}%
\pgfsetrectcap%
\pgfsetmiterjoin%
\pgfsetlinewidth{0.803000pt}%
\definecolor{currentstroke}{rgb}{0.000000,0.000000,0.000000}%
\pgfsetstrokecolor{currentstroke}%
\pgfsetdash{}{0pt}%
\pgfpathmoveto{\pgfqpoint{5.824769in}{0.646140in}}%
\pgfpathlineto{\pgfqpoint{5.824769in}{3.734428in}}%
\pgfusepath{stroke}%
\end{pgfscope}%
\begin{pgfscope}%
\pgfsetrectcap%
\pgfsetmiterjoin%
\pgfsetlinewidth{0.803000pt}%
\definecolor{currentstroke}{rgb}{0.000000,0.000000,0.000000}%
\pgfsetstrokecolor{currentstroke}%
\pgfsetdash{}{0pt}%
\pgfpathmoveto{\pgfqpoint{0.795366in}{0.646140in}}%
\pgfpathlineto{\pgfqpoint{5.824769in}{0.646140in}}%
\pgfusepath{stroke}%
\end{pgfscope}%
\begin{pgfscope}%
\pgfsetrectcap%
\pgfsetmiterjoin%
\pgfsetlinewidth{0.803000pt}%
\definecolor{currentstroke}{rgb}{0.000000,0.000000,0.000000}%
\pgfsetstrokecolor{currentstroke}%
\pgfsetdash{}{0pt}%
\pgfpathmoveto{\pgfqpoint{0.795366in}{3.734428in}}%
\pgfpathlineto{\pgfqpoint{5.824769in}{3.734428in}}%
\pgfusepath{stroke}%
\end{pgfscope}%
\begin{pgfscope}%
\definecolor{textcolor}{rgb}{0.000000,0.000000,0.000000}%
\pgfsetstrokecolor{textcolor}%
\pgfsetfillcolor{textcolor}%
\pgftext[x=5.064072in,y=2.631325in,,base]{\color{textcolor}\rmfamily\fontsize{7.000000}{8.400000}\selectfont R(7)}%
\end{pgfscope}%
\begin{pgfscope}%
\definecolor{textcolor}{rgb}{0.000000,0.000000,0.000000}%
\pgfsetstrokecolor{textcolor}%
\pgfsetfillcolor{textcolor}%
\pgftext[x=4.888741in,y=2.381918in,,base]{\color{textcolor}\rmfamily\fontsize{7.000000}{8.400000}\selectfont R(6)}%
\end{pgfscope}%
\begin{pgfscope}%
\definecolor{textcolor}{rgb}{0.000000,0.000000,0.000000}%
\pgfsetstrokecolor{textcolor}%
\pgfsetfillcolor{textcolor}%
\pgftext[x=4.705727in,y=2.091935in,,base]{\color{textcolor}\rmfamily\fontsize{7.000000}{8.400000}\selectfont R(5)}%
\end{pgfscope}%
\begin{pgfscope}%
\definecolor{textcolor}{rgb}{0.000000,0.000000,0.000000}%
\pgfsetstrokecolor{textcolor}%
\pgfsetfillcolor{textcolor}%
\pgftext[x=4.515727in,y=1.693455in,,base]{\color{textcolor}\rmfamily\fontsize{7.000000}{8.400000}\selectfont R(4)}%
\end{pgfscope}%
\begin{pgfscope}%
\definecolor{textcolor}{rgb}{0.000000,0.000000,0.000000}%
\pgfsetstrokecolor{textcolor}%
\pgfsetfillcolor{textcolor}%
\pgftext[x=4.317345in,y=0.721635in,,base]{\color{textcolor}\rmfamily\fontsize{7.000000}{8.400000}\selectfont R(3)}%
\end{pgfscope}%
\begin{pgfscope}%
\definecolor{textcolor}{rgb}{0.000000,0.000000,0.000000}%
\pgfsetstrokecolor{textcolor}%
\pgfsetfillcolor{textcolor}%
\pgftext[x=4.167860in,y=1.347599in,,base]{\color{textcolor}\rmfamily\fontsize{7.000000}{8.400000}\selectfont R(2)}%
\end{pgfscope}%
\begin{pgfscope}%
\definecolor{textcolor}{rgb}{0.000000,0.000000,0.000000}%
\pgfsetstrokecolor{textcolor}%
\pgfsetfillcolor{textcolor}%
\pgftext[x=3.898927in,y=1.371655in,,base]{\color{textcolor}\rmfamily\fontsize{7.000000}{8.400000}\selectfont R(1)}%
\end{pgfscope}%
\begin{pgfscope}%
\definecolor{textcolor}{rgb}{0.000000,0.000000,0.000000}%
\pgfsetstrokecolor{textcolor}%
\pgfsetfillcolor{textcolor}%
\pgftext[x=3.679589in,y=1.713714in,,base]{\color{textcolor}\rmfamily\fontsize{7.000000}{8.400000}\selectfont R(0)}%
\end{pgfscope}%
\begin{pgfscope}%
\definecolor{textcolor}{rgb}{0.000000,0.000000,0.000000}%
\pgfsetstrokecolor{textcolor}%
\pgfsetfillcolor{textcolor}%
\pgftext[x=3.219259in,y=1.734991in,,base]{\color{textcolor}\rmfamily\fontsize{8.000000}{9.600000}\selectfont P(1)}%
\end{pgfscope}%
\begin{pgfscope}%
\definecolor{textcolor}{rgb}{0.000000,0.000000,0.000000}%
\pgfsetstrokecolor{textcolor}%
\pgfsetfillcolor{textcolor}%
\pgftext[x=2.978965in,y=1.243565in,,base]{\color{textcolor}\rmfamily\fontsize{8.000000}{9.600000}\selectfont P(2)}%
\end{pgfscope}%
\begin{pgfscope}%
\definecolor{textcolor}{rgb}{0.000000,0.000000,0.000000}%
\pgfsetstrokecolor{textcolor}%
\pgfsetfillcolor{textcolor}%
\pgftext[x=2.733083in,y=1.076971in,,base]{\color{textcolor}\rmfamily\fontsize{8.000000}{9.600000}\selectfont P(3)}%
\end{pgfscope}%
\begin{pgfscope}%
\definecolor{textcolor}{rgb}{0.000000,0.000000,0.000000}%
\pgfsetstrokecolor{textcolor}%
\pgfsetfillcolor{textcolor}%
\pgftext[x=2.480216in,y=1.119538in,,base]{\color{textcolor}\rmfamily\fontsize{8.000000}{9.600000}\selectfont P(4)}%
\end{pgfscope}%
\begin{pgfscope}%
\definecolor{textcolor}{rgb}{0.000000,0.000000,0.000000}%
\pgfsetstrokecolor{textcolor}%
\pgfsetfillcolor{textcolor}%
\pgftext[x=2.220363in,y=1.196951in,,base]{\color{textcolor}\rmfamily\fontsize{8.000000}{9.600000}\selectfont P(5)}%
\end{pgfscope}%
\begin{pgfscope}%
\definecolor{textcolor}{rgb}{0.000000,0.000000,0.000000}%
\pgfsetstrokecolor{textcolor}%
\pgfsetfillcolor{textcolor}%
\pgftext[x=1.955621in,y=1.246503in,,base]{\color{textcolor}\rmfamily\fontsize{8.000000}{9.600000}\selectfont P(6)}%
\end{pgfscope}%
\begin{pgfscope}%
\definecolor{textcolor}{rgb}{0.000000,0.000000,0.000000}%
\pgfsetstrokecolor{textcolor}%
\pgfsetfillcolor{textcolor}%
\pgftext[x=1.684592in,y=1.197312in,,base]{\color{textcolor}\rmfamily\fontsize{8.000000}{9.600000}\selectfont P(7)}%
\end{pgfscope}%
\begin{pgfscope}%
\definecolor{textcolor}{rgb}{0.000000,0.000000,0.000000}%
\pgfsetstrokecolor{textcolor}%
\pgfsetfillcolor{textcolor}%
\pgftext[x=1.407975in,y=1.055088in,,base]{\color{textcolor}\rmfamily\fontsize{8.000000}{9.600000}\selectfont P(8)}%
\end{pgfscope}%
\begin{pgfscope}%
\pgfsetbuttcap%
\pgfsetmiterjoin%
\definecolor{currentfill}{rgb}{1.000000,1.000000,1.000000}%
\pgfsetfillcolor{currentfill}%
\pgfsetfillopacity{0.800000}%
\pgfsetlinewidth{1.003750pt}%
\definecolor{currentstroke}{rgb}{0.800000,0.800000,0.800000}%
\pgfsetstrokecolor{currentstroke}%
\pgfsetstrokeopacity{0.800000}%
\pgfsetdash{}{0pt}%
\pgfpathmoveto{\pgfqpoint{0.931477in}{3.036762in}}%
\pgfpathlineto{\pgfqpoint{1.963782in}{3.036762in}}%
\pgfpathquadraticcurveto{\pgfqpoint{2.002671in}{3.036762in}}{\pgfqpoint{2.002671in}{3.075651in}}%
\pgfpathlineto{\pgfqpoint{2.002671in}{3.598317in}}%
\pgfpathquadraticcurveto{\pgfqpoint{2.002671in}{3.637206in}}{\pgfqpoint{1.963782in}{3.637206in}}%
\pgfpathlineto{\pgfqpoint{0.931477in}{3.637206in}}%
\pgfpathquadraticcurveto{\pgfqpoint{0.892588in}{3.637206in}}{\pgfqpoint{0.892588in}{3.598317in}}%
\pgfpathlineto{\pgfqpoint{0.892588in}{3.075651in}}%
\pgfpathquadraticcurveto{\pgfqpoint{0.892588in}{3.036762in}}{\pgfqpoint{0.931477in}{3.036762in}}%
\pgfpathlineto{\pgfqpoint{0.931477in}{3.036762in}}%
\pgfpathclose%
\pgfusepath{stroke,fill}%
\end{pgfscope}%
\begin{pgfscope}%
\pgfsetbuttcap%
\pgfsetroundjoin%
\definecolor{currentfill}{rgb}{1.000000,0.000000,0.000000}%
\pgfsetfillcolor{currentfill}%
\pgfsetlinewidth{1.003750pt}%
\definecolor{currentstroke}{rgb}{1.000000,0.000000,0.000000}%
\pgfsetstrokecolor{currentstroke}%
\pgfsetdash{}{0pt}%
\pgfsys@defobject{currentmarker}{\pgfqpoint{-0.020833in}{-0.020833in}}{\pgfqpoint{0.020833in}{0.020833in}}{%
\pgfpathmoveto{\pgfqpoint{0.000000in}{-0.020833in}}%
\pgfpathcurveto{\pgfqpoint{0.005525in}{-0.020833in}}{\pgfqpoint{0.010825in}{-0.018638in}}{\pgfqpoint{0.014731in}{-0.014731in}}%
\pgfpathcurveto{\pgfqpoint{0.018638in}{-0.010825in}}{\pgfqpoint{0.020833in}{-0.005525in}}{\pgfqpoint{0.020833in}{0.000000in}}%
\pgfpathcurveto{\pgfqpoint{0.020833in}{0.005525in}}{\pgfqpoint{0.018638in}{0.010825in}}{\pgfqpoint{0.014731in}{0.014731in}}%
\pgfpathcurveto{\pgfqpoint{0.010825in}{0.018638in}}{\pgfqpoint{0.005525in}{0.020833in}}{\pgfqpoint{0.000000in}{0.020833in}}%
\pgfpathcurveto{\pgfqpoint{-0.005525in}{0.020833in}}{\pgfqpoint{-0.010825in}{0.018638in}}{\pgfqpoint{-0.014731in}{0.014731in}}%
\pgfpathcurveto{\pgfqpoint{-0.018638in}{0.010825in}}{\pgfqpoint{-0.020833in}{0.005525in}}{\pgfqpoint{-0.020833in}{0.000000in}}%
\pgfpathcurveto{\pgfqpoint{-0.020833in}{-0.005525in}}{\pgfqpoint{-0.018638in}{-0.010825in}}{\pgfqpoint{-0.014731in}{-0.014731in}}%
\pgfpathcurveto{\pgfqpoint{-0.010825in}{-0.018638in}}{\pgfqpoint{-0.005525in}{-0.020833in}}{\pgfqpoint{0.000000in}{-0.020833in}}%
\pgfpathlineto{\pgfqpoint{0.000000in}{-0.020833in}}%
\pgfpathclose%
\pgfusepath{stroke,fill}%
}%
\begin{pgfscope}%
\pgfsys@transformshift{1.164810in}{3.491373in}%
\pgfsys@useobject{currentmarker}{}%
\end{pgfscope}%
\end{pgfscope}%
\begin{pgfscope}%
\definecolor{textcolor}{rgb}{0.000000,0.000000,0.000000}%
\pgfsetstrokecolor{textcolor}%
\pgfsetfillcolor{textcolor}%
\pgftext[x=1.514810in,y=3.423317in,left,base]{\color{textcolor}\rmfamily\fontsize{14.000000}{16.800000}\selectfont Dips}%
\end{pgfscope}%
\begin{pgfscope}%
\pgfsetrectcap%
\pgfsetroundjoin%
\pgfsetlinewidth{0.501875pt}%
\definecolor{currentstroke}{rgb}{0.000000,0.000000,0.000000}%
\pgfsetstrokecolor{currentstroke}%
\pgfsetdash{}{0pt}%
\pgfpathmoveto{\pgfqpoint{0.970366in}{3.220317in}}%
\pgfpathlineto{\pgfqpoint{1.164810in}{3.220317in}}%
\pgfpathlineto{\pgfqpoint{1.359255in}{3.220317in}}%
\pgfusepath{stroke}%
\end{pgfscope}%
\begin{pgfscope}%
\definecolor{textcolor}{rgb}{0.000000,0.000000,0.000000}%
\pgfsetstrokecolor{textcolor}%
\pgfsetfillcolor{textcolor}%
\pgftext[x=1.514810in,y=3.152262in,left,base]{\color{textcolor}\rmfamily\fontsize{14.000000}{16.800000}\selectfont Data}%
\end{pgfscope}%
\end{pgfpicture}%
\makeatother%
\endgroup%
}
	\caption{HCl spectrum}
	\label{fig:HCl}
\end{figure}

We can clearly see the R- and P-branches in the spectrum. The R-branch is the one with the higher wavenumbers, while the P-branch is the one with the lower wavenumbers. The wavenumbers of the R- and P-branches were determined. The results are shown in Tables \ref{table:wavenumbersP} \& \ref{table:wavenumbersR}. The gap between the two branches is due to $\Delta J = -1$ in the P-branch. The symbol $P(J)$ means an initial state with $J$ with $\Delta J = -1$ as the transition happens. Hence, we see that the P-branch must start at $J = 1$. 

\begin{table}[H]
    \centering
    \[
    \begin{array}{|c|c|c|c|}
    \hline \text{Minimum} & \text{Index } i & \bar{\nu}\left({ }^{35} \mathrm{Cl}\right)\left[\mathrm{cm}^{-1}\right] & \bar{\nu}\left({ }^{37} \mathrm{Cl}\right)\left[\mathrm{cm}^{-1}\right] \\
    \hline \mathrm{P}(8) & -8 & 2705.75 & 2703.875 \\
    \hline \mathrm{P}(7) & -7 & 2730.5 & 2728.625 \\
    \hline \mathrm{P}(6) & -6 & 2754.75 & 2752.875 \\
    \hline \mathrm{P}(5) & -5 & 2778.5 & 2776.5 \\
    \hline \mathrm{P}(4) & -4 & 2801.75 & 2799.75 \\
    \hline \mathrm{P}(3) & -3 & 2824.375 & 2822.375 \\
    \hline \mathrm{P}(2) & -2 & 2846.375 & 2844.375 \\
    \hline \mathrm{P}(1) & -1 & 2867.875 & 2865.875 \\
    \hline
    \end{array}
    \]
    \caption{P-branch wavenumbers}
    \label{table:wavenumbersP}
\end{table}

\begin{table}[H]
    \centering
    \[
    \begin{array}{|c|c|c|c|}
    \hline \text{Minimum} & \text{Index } i & \bar{\nu}\left({ }^{35} \mathrm{Cl}\right)\left[\mathrm{cm}^{-1}\right] & \bar{\nu}\left({ }^{37} \mathrm{Cl}\right)\left[\mathrm{cm}^{-1}\right] \\
    \hline \text{R(0)} & 1 & 2909.125 & 2907 \\
    \hline \text{R(1)} & 2 & 2928.75 & 2926.625 \\
    \hline \text{R(2)} & 3 & 2947.875 & 2945.625 \\
    \hline \text{R(3)} & 4 & 2966.25 & 2964 \\
    \hline \text{R(4)} & 5 & 2984 & 2981.75 \\
    \hline \text{R(5)} & 6 & 3001 & 2998.75 \\
    \hline \text{R(6)} & 7 & 3017.375 & 3015.125 \\
    \hline \text{R(7)} & 8 & 3033.125 & 3030.75 \\
    \hline
    \end{array}
    \]
    \caption{R-branch wavenumbers}
    \label{table:wavenumbersR}
\end{table}

The data can be used to find the wavenumber of the purely vibrational transition $\nu_s$, the force constant $k$, the rotation constant $B$, the distance of the two atoms in the molecule $r_0$ and the moment of inertia $I$. The results are shown in Table \ref{table:parameters}. The following equations were used:

\begin{equation}
	\nu_s = \frac{\nu_R(0) + \nu_P(1)}{2}
\end{equation}

\begin{equation}
	B = \frac{\nu_R(1) - \nu_P(1)}{4}
\end{equation}

\begin{table}[H]
    \centering
    \[
    \begin{array}{|c|c|c|}
    \hline \text{Parameter} & \text{Isotope }{ }^{35} \mathrm{Cl} & \text{Isotope }{ }^{37} \mathrm{Cl} \\
    \hline \bar{\nu}_s\left[\mathrm{cm}^{-1}\right] & 2888.50 \pm 1.37 & 2886.44 \pm 1.37 \\
    \hline B\left[\mathrm{cm}^{-1}\right] & 10.31 \pm 0.69 & 10.28 \pm 0.69 \\
    \hline \mu[\mathrm{u}] & 0.9796 & 0.9811 \\
    \hline k\left[\mathrm{Nm}^{-1}\right] & 481.55 \pm 0.46 & 481.60 \pm 0.46 \\
    \hline I\left[\mathrm{kg} \mathrm{m}^2\right] & 2.71 \cdot 10^{-47} \pm 1.82 \cdot 10^{-48} & 2.72 \cdot 10^{-47} \pm 1.83 \cdot 10^{-48} \\
    \hline r_0[\mathrm{m}] & 1.29 \cdot 10^{-10} \pm 6.12 \cdot 10^{-12} & 1.29 \cdot 10^{-10} \pm 6.15 \cdot 10^{-12} \\
    \hline
    \end{array}
    \]
    \caption{Parameters for isotopes of Chlorine}
    \label{table:parameters}
\end{table}

To determine the parameters $B_0$ , $B_1$ and $D$, the first order difference of the wavenumbers is plotted for the two isotopes (we used $D \approx D_0 \approx D_1$) against the peak indicies. A quadratic curve fit was performed using Eq. \ref{eq:quadratic} for each isotope.

\begin{figure}[H]
	\centering
	\begin{subfigure}{0.45\textwidth}
		\centering
		\scalebox{0.50}{%% Creator: Matplotlib, PGF backend
%%
%% To include the figure in your LaTeX document, write
%%   \input{<filename>.pgf}
%%
%% Make sure the required packages are loaded in your preamble
%%   \usepackage{pgf}
%%
%% Also ensure that all the required font packages are loaded; for instance,
%% the lmodern package is sometimes necessary when using math font.
%%   \usepackage{lmodern}
%%
%% Figures using additional raster images can only be included by \input if
%% they are in the same directory as the main LaTeX file. For loading figures
%% from other directories you can use the `import` package
%%   \usepackage{import}
%%
%% and then include the figures with
%%   \import{<path to file>}{<filename>.pgf}
%%
%% Matplotlib used the following preamble
%%   
%%   \usepackage{fontspec}
%%   \makeatletter\@ifpackageloaded{underscore}{}{\usepackage[strings]{underscore}}\makeatother
%%
\begingroup%
\makeatletter%
\begin{pgfpicture}%
\pgfpathrectangle{\pgfpointorigin}{\pgfqpoint{5.798708in}{3.786399in}}%
\pgfusepath{use as bounding box, clip}%
\begin{pgfscope}%
\pgfsetbuttcap%
\pgfsetmiterjoin%
\definecolor{currentfill}{rgb}{1.000000,1.000000,1.000000}%
\pgfsetfillcolor{currentfill}%
\pgfsetlinewidth{0.000000pt}%
\definecolor{currentstroke}{rgb}{1.000000,1.000000,1.000000}%
\pgfsetstrokecolor{currentstroke}%
\pgfsetdash{}{0pt}%
\pgfpathmoveto{\pgfqpoint{0.000000in}{0.000000in}}%
\pgfpathlineto{\pgfqpoint{5.798708in}{0.000000in}}%
\pgfpathlineto{\pgfqpoint{5.798708in}{3.786399in}}%
\pgfpathlineto{\pgfqpoint{0.000000in}{3.786399in}}%
\pgfpathlineto{\pgfqpoint{0.000000in}{0.000000in}}%
\pgfpathclose%
\pgfusepath{fill}%
\end{pgfscope}%
\begin{pgfscope}%
\pgfsetbuttcap%
\pgfsetmiterjoin%
\definecolor{currentfill}{rgb}{1.000000,1.000000,1.000000}%
\pgfsetfillcolor{currentfill}%
\pgfsetlinewidth{0.000000pt}%
\definecolor{currentstroke}{rgb}{0.000000,0.000000,0.000000}%
\pgfsetstrokecolor{currentstroke}%
\pgfsetstrokeopacity{0.000000}%
\pgfsetdash{}{0pt}%
\pgfpathmoveto{\pgfqpoint{0.669304in}{0.598111in}}%
\pgfpathlineto{\pgfqpoint{5.698708in}{0.598111in}}%
\pgfpathlineto{\pgfqpoint{5.698708in}{3.686399in}}%
\pgfpathlineto{\pgfqpoint{0.669304in}{3.686399in}}%
\pgfpathlineto{\pgfqpoint{0.669304in}{0.598111in}}%
\pgfpathclose%
\pgfusepath{fill}%
\end{pgfscope}%
\begin{pgfscope}%
\pgfsetbuttcap%
\pgfsetroundjoin%
\definecolor{currentfill}{rgb}{0.000000,0.000000,0.000000}%
\pgfsetfillcolor{currentfill}%
\pgfsetlinewidth{0.803000pt}%
\definecolor{currentstroke}{rgb}{0.000000,0.000000,0.000000}%
\pgfsetstrokecolor{currentstroke}%
\pgfsetdash{}{0pt}%
\pgfsys@defobject{currentmarker}{\pgfqpoint{0.000000in}{-0.048611in}}{\pgfqpoint{0.000000in}{0.000000in}}{%
\pgfpathmoveto{\pgfqpoint{0.000000in}{0.000000in}}%
\pgfpathlineto{\pgfqpoint{0.000000in}{-0.048611in}}%
\pgfusepath{stroke,fill}%
}%
\begin{pgfscope}%
\pgfsys@transformshift{0.669304in}{0.598111in}%
\pgfsys@useobject{currentmarker}{}%
\end{pgfscope}%
\end{pgfscope}%
\begin{pgfscope}%
\definecolor{textcolor}{rgb}{0.000000,0.000000,0.000000}%
\pgfsetstrokecolor{textcolor}%
\pgfsetfillcolor{textcolor}%
\pgftext[x=0.669304in,y=0.500889in,,top]{\color{textcolor}\rmfamily\fontsize{14.000000}{16.800000}\selectfont \(\displaystyle {\ensuremath{-}8}\)}%
\end{pgfscope}%
\begin{pgfscope}%
\pgfsetbuttcap%
\pgfsetroundjoin%
\definecolor{currentfill}{rgb}{0.000000,0.000000,0.000000}%
\pgfsetfillcolor{currentfill}%
\pgfsetlinewidth{0.803000pt}%
\definecolor{currentstroke}{rgb}{0.000000,0.000000,0.000000}%
\pgfsetstrokecolor{currentstroke}%
\pgfsetdash{}{0pt}%
\pgfsys@defobject{currentmarker}{\pgfqpoint{0.000000in}{-0.048611in}}{\pgfqpoint{0.000000in}{0.000000in}}{%
\pgfpathmoveto{\pgfqpoint{0.000000in}{0.000000in}}%
\pgfpathlineto{\pgfqpoint{0.000000in}{-0.048611in}}%
\pgfusepath{stroke,fill}%
}%
\begin{pgfscope}%
\pgfsys@transformshift{1.269830in}{0.598111in}%
\pgfsys@useobject{currentmarker}{}%
\end{pgfscope}%
\end{pgfscope}%
\begin{pgfscope}%
\definecolor{textcolor}{rgb}{0.000000,0.000000,0.000000}%
\pgfsetstrokecolor{textcolor}%
\pgfsetfillcolor{textcolor}%
\pgftext[x=1.269830in,y=0.500889in,,top]{\color{textcolor}\rmfamily\fontsize{14.000000}{16.800000}\selectfont \(\displaystyle {\ensuremath{-}6}\)}%
\end{pgfscope}%
\begin{pgfscope}%
\pgfsetbuttcap%
\pgfsetroundjoin%
\definecolor{currentfill}{rgb}{0.000000,0.000000,0.000000}%
\pgfsetfillcolor{currentfill}%
\pgfsetlinewidth{0.803000pt}%
\definecolor{currentstroke}{rgb}{0.000000,0.000000,0.000000}%
\pgfsetstrokecolor{currentstroke}%
\pgfsetdash{}{0pt}%
\pgfsys@defobject{currentmarker}{\pgfqpoint{0.000000in}{-0.048611in}}{\pgfqpoint{0.000000in}{0.000000in}}{%
\pgfpathmoveto{\pgfqpoint{0.000000in}{0.000000in}}%
\pgfpathlineto{\pgfqpoint{0.000000in}{-0.048611in}}%
\pgfusepath{stroke,fill}%
}%
\begin{pgfscope}%
\pgfsys@transformshift{1.870356in}{0.598111in}%
\pgfsys@useobject{currentmarker}{}%
\end{pgfscope}%
\end{pgfscope}%
\begin{pgfscope}%
\definecolor{textcolor}{rgb}{0.000000,0.000000,0.000000}%
\pgfsetstrokecolor{textcolor}%
\pgfsetfillcolor{textcolor}%
\pgftext[x=1.870356in,y=0.500889in,,top]{\color{textcolor}\rmfamily\fontsize{14.000000}{16.800000}\selectfont \(\displaystyle {\ensuremath{-}4}\)}%
\end{pgfscope}%
\begin{pgfscope}%
\pgfsetbuttcap%
\pgfsetroundjoin%
\definecolor{currentfill}{rgb}{0.000000,0.000000,0.000000}%
\pgfsetfillcolor{currentfill}%
\pgfsetlinewidth{0.803000pt}%
\definecolor{currentstroke}{rgb}{0.000000,0.000000,0.000000}%
\pgfsetstrokecolor{currentstroke}%
\pgfsetdash{}{0pt}%
\pgfsys@defobject{currentmarker}{\pgfqpoint{0.000000in}{-0.048611in}}{\pgfqpoint{0.000000in}{0.000000in}}{%
\pgfpathmoveto{\pgfqpoint{0.000000in}{0.000000in}}%
\pgfpathlineto{\pgfqpoint{0.000000in}{-0.048611in}}%
\pgfusepath{stroke,fill}%
}%
\begin{pgfscope}%
\pgfsys@transformshift{2.470881in}{0.598111in}%
\pgfsys@useobject{currentmarker}{}%
\end{pgfscope}%
\end{pgfscope}%
\begin{pgfscope}%
\definecolor{textcolor}{rgb}{0.000000,0.000000,0.000000}%
\pgfsetstrokecolor{textcolor}%
\pgfsetfillcolor{textcolor}%
\pgftext[x=2.470881in,y=0.500889in,,top]{\color{textcolor}\rmfamily\fontsize{14.000000}{16.800000}\selectfont \(\displaystyle {\ensuremath{-}2}\)}%
\end{pgfscope}%
\begin{pgfscope}%
\pgfsetbuttcap%
\pgfsetroundjoin%
\definecolor{currentfill}{rgb}{0.000000,0.000000,0.000000}%
\pgfsetfillcolor{currentfill}%
\pgfsetlinewidth{0.803000pt}%
\definecolor{currentstroke}{rgb}{0.000000,0.000000,0.000000}%
\pgfsetstrokecolor{currentstroke}%
\pgfsetdash{}{0pt}%
\pgfsys@defobject{currentmarker}{\pgfqpoint{0.000000in}{-0.048611in}}{\pgfqpoint{0.000000in}{0.000000in}}{%
\pgfpathmoveto{\pgfqpoint{0.000000in}{0.000000in}}%
\pgfpathlineto{\pgfqpoint{0.000000in}{-0.048611in}}%
\pgfusepath{stroke,fill}%
}%
\begin{pgfscope}%
\pgfsys@transformshift{3.071407in}{0.598111in}%
\pgfsys@useobject{currentmarker}{}%
\end{pgfscope}%
\end{pgfscope}%
\begin{pgfscope}%
\definecolor{textcolor}{rgb}{0.000000,0.000000,0.000000}%
\pgfsetstrokecolor{textcolor}%
\pgfsetfillcolor{textcolor}%
\pgftext[x=3.071407in,y=0.500889in,,top]{\color{textcolor}\rmfamily\fontsize{14.000000}{16.800000}\selectfont \(\displaystyle {0}\)}%
\end{pgfscope}%
\begin{pgfscope}%
\pgfsetbuttcap%
\pgfsetroundjoin%
\definecolor{currentfill}{rgb}{0.000000,0.000000,0.000000}%
\pgfsetfillcolor{currentfill}%
\pgfsetlinewidth{0.803000pt}%
\definecolor{currentstroke}{rgb}{0.000000,0.000000,0.000000}%
\pgfsetstrokecolor{currentstroke}%
\pgfsetdash{}{0pt}%
\pgfsys@defobject{currentmarker}{\pgfqpoint{0.000000in}{-0.048611in}}{\pgfqpoint{0.000000in}{0.000000in}}{%
\pgfpathmoveto{\pgfqpoint{0.000000in}{0.000000in}}%
\pgfpathlineto{\pgfqpoint{0.000000in}{-0.048611in}}%
\pgfusepath{stroke,fill}%
}%
\begin{pgfscope}%
\pgfsys@transformshift{3.671933in}{0.598111in}%
\pgfsys@useobject{currentmarker}{}%
\end{pgfscope}%
\end{pgfscope}%
\begin{pgfscope}%
\definecolor{textcolor}{rgb}{0.000000,0.000000,0.000000}%
\pgfsetstrokecolor{textcolor}%
\pgfsetfillcolor{textcolor}%
\pgftext[x=3.671933in,y=0.500889in,,top]{\color{textcolor}\rmfamily\fontsize{14.000000}{16.800000}\selectfont \(\displaystyle {2}\)}%
\end{pgfscope}%
\begin{pgfscope}%
\pgfsetbuttcap%
\pgfsetroundjoin%
\definecolor{currentfill}{rgb}{0.000000,0.000000,0.000000}%
\pgfsetfillcolor{currentfill}%
\pgfsetlinewidth{0.803000pt}%
\definecolor{currentstroke}{rgb}{0.000000,0.000000,0.000000}%
\pgfsetstrokecolor{currentstroke}%
\pgfsetdash{}{0pt}%
\pgfsys@defobject{currentmarker}{\pgfqpoint{0.000000in}{-0.048611in}}{\pgfqpoint{0.000000in}{0.000000in}}{%
\pgfpathmoveto{\pgfqpoint{0.000000in}{0.000000in}}%
\pgfpathlineto{\pgfqpoint{0.000000in}{-0.048611in}}%
\pgfusepath{stroke,fill}%
}%
\begin{pgfscope}%
\pgfsys@transformshift{4.272459in}{0.598111in}%
\pgfsys@useobject{currentmarker}{}%
\end{pgfscope}%
\end{pgfscope}%
\begin{pgfscope}%
\definecolor{textcolor}{rgb}{0.000000,0.000000,0.000000}%
\pgfsetstrokecolor{textcolor}%
\pgfsetfillcolor{textcolor}%
\pgftext[x=4.272459in,y=0.500889in,,top]{\color{textcolor}\rmfamily\fontsize{14.000000}{16.800000}\selectfont \(\displaystyle {4}\)}%
\end{pgfscope}%
\begin{pgfscope}%
\pgfsetbuttcap%
\pgfsetroundjoin%
\definecolor{currentfill}{rgb}{0.000000,0.000000,0.000000}%
\pgfsetfillcolor{currentfill}%
\pgfsetlinewidth{0.803000pt}%
\definecolor{currentstroke}{rgb}{0.000000,0.000000,0.000000}%
\pgfsetstrokecolor{currentstroke}%
\pgfsetdash{}{0pt}%
\pgfsys@defobject{currentmarker}{\pgfqpoint{0.000000in}{-0.048611in}}{\pgfqpoint{0.000000in}{0.000000in}}{%
\pgfpathmoveto{\pgfqpoint{0.000000in}{0.000000in}}%
\pgfpathlineto{\pgfqpoint{0.000000in}{-0.048611in}}%
\pgfusepath{stroke,fill}%
}%
\begin{pgfscope}%
\pgfsys@transformshift{4.872985in}{0.598111in}%
\pgfsys@useobject{currentmarker}{}%
\end{pgfscope}%
\end{pgfscope}%
\begin{pgfscope}%
\definecolor{textcolor}{rgb}{0.000000,0.000000,0.000000}%
\pgfsetstrokecolor{textcolor}%
\pgfsetfillcolor{textcolor}%
\pgftext[x=4.872985in,y=0.500889in,,top]{\color{textcolor}\rmfamily\fontsize{14.000000}{16.800000}\selectfont \(\displaystyle {6}\)}%
\end{pgfscope}%
\begin{pgfscope}%
\pgfsetbuttcap%
\pgfsetroundjoin%
\definecolor{currentfill}{rgb}{0.000000,0.000000,0.000000}%
\pgfsetfillcolor{currentfill}%
\pgfsetlinewidth{0.803000pt}%
\definecolor{currentstroke}{rgb}{0.000000,0.000000,0.000000}%
\pgfsetstrokecolor{currentstroke}%
\pgfsetdash{}{0pt}%
\pgfsys@defobject{currentmarker}{\pgfqpoint{0.000000in}{-0.048611in}}{\pgfqpoint{0.000000in}{0.000000in}}{%
\pgfpathmoveto{\pgfqpoint{0.000000in}{0.000000in}}%
\pgfpathlineto{\pgfqpoint{0.000000in}{-0.048611in}}%
\pgfusepath{stroke,fill}%
}%
\begin{pgfscope}%
\pgfsys@transformshift{5.473510in}{0.598111in}%
\pgfsys@useobject{currentmarker}{}%
\end{pgfscope}%
\end{pgfscope}%
\begin{pgfscope}%
\definecolor{textcolor}{rgb}{0.000000,0.000000,0.000000}%
\pgfsetstrokecolor{textcolor}%
\pgfsetfillcolor{textcolor}%
\pgftext[x=5.473510in,y=0.500889in,,top]{\color{textcolor}\rmfamily\fontsize{14.000000}{16.800000}\selectfont \(\displaystyle {8}\)}%
\end{pgfscope}%
\begin{pgfscope}%
\definecolor{textcolor}{rgb}{0.000000,0.000000,0.000000}%
\pgfsetstrokecolor{textcolor}%
\pgfsetfillcolor{textcolor}%
\pgftext[x=3.184006in,y=0.272667in,,top]{\color{textcolor}\rmfamily\fontsize{14.000000}{16.800000}\selectfont Index i}%
\end{pgfscope}%
\begin{pgfscope}%
\pgfsetbuttcap%
\pgfsetroundjoin%
\definecolor{currentfill}{rgb}{0.000000,0.000000,0.000000}%
\pgfsetfillcolor{currentfill}%
\pgfsetlinewidth{0.803000pt}%
\definecolor{currentstroke}{rgb}{0.000000,0.000000,0.000000}%
\pgfsetstrokecolor{currentstroke}%
\pgfsetdash{}{0pt}%
\pgfsys@defobject{currentmarker}{\pgfqpoint{-0.048611in}{0.000000in}}{\pgfqpoint{-0.000000in}{0.000000in}}{%
\pgfpathmoveto{\pgfqpoint{-0.000000in}{0.000000in}}%
\pgfpathlineto{\pgfqpoint{-0.048611in}{0.000000in}}%
\pgfusepath{stroke,fill}%
}%
\begin{pgfscope}%
\pgfsys@transformshift{0.669304in}{0.853866in}%
\pgfsys@useobject{currentmarker}{}%
\end{pgfscope}%
\end{pgfscope}%
\begin{pgfscope}%
\definecolor{textcolor}{rgb}{0.000000,0.000000,0.000000}%
\pgfsetstrokecolor{textcolor}%
\pgfsetfillcolor{textcolor}%
\pgftext[x=0.376251in, y=0.786394in, left, base]{\color{textcolor}\rmfamily\fontsize{14.000000}{16.800000}\selectfont \(\displaystyle {16}\)}%
\end{pgfscope}%
\begin{pgfscope}%
\pgfsetbuttcap%
\pgfsetroundjoin%
\definecolor{currentfill}{rgb}{0.000000,0.000000,0.000000}%
\pgfsetfillcolor{currentfill}%
\pgfsetlinewidth{0.803000pt}%
\definecolor{currentstroke}{rgb}{0.000000,0.000000,0.000000}%
\pgfsetstrokecolor{currentstroke}%
\pgfsetdash{}{0pt}%
\pgfsys@defobject{currentmarker}{\pgfqpoint{-0.048611in}{0.000000in}}{\pgfqpoint{-0.000000in}{0.000000in}}{%
\pgfpathmoveto{\pgfqpoint{-0.000000in}{0.000000in}}%
\pgfpathlineto{\pgfqpoint{-0.048611in}{0.000000in}}%
\pgfusepath{stroke,fill}%
}%
\begin{pgfscope}%
\pgfsys@transformshift{0.669304in}{1.469216in}%
\pgfsys@useobject{currentmarker}{}%
\end{pgfscope}%
\end{pgfscope}%
\begin{pgfscope}%
\definecolor{textcolor}{rgb}{0.000000,0.000000,0.000000}%
\pgfsetstrokecolor{textcolor}%
\pgfsetfillcolor{textcolor}%
\pgftext[x=0.376251in, y=1.401744in, left, base]{\color{textcolor}\rmfamily\fontsize{14.000000}{16.800000}\selectfont \(\displaystyle {18}\)}%
\end{pgfscope}%
\begin{pgfscope}%
\pgfsetbuttcap%
\pgfsetroundjoin%
\definecolor{currentfill}{rgb}{0.000000,0.000000,0.000000}%
\pgfsetfillcolor{currentfill}%
\pgfsetlinewidth{0.803000pt}%
\definecolor{currentstroke}{rgb}{0.000000,0.000000,0.000000}%
\pgfsetstrokecolor{currentstroke}%
\pgfsetdash{}{0pt}%
\pgfsys@defobject{currentmarker}{\pgfqpoint{-0.048611in}{0.000000in}}{\pgfqpoint{-0.000000in}{0.000000in}}{%
\pgfpathmoveto{\pgfqpoint{-0.000000in}{0.000000in}}%
\pgfpathlineto{\pgfqpoint{-0.048611in}{0.000000in}}%
\pgfusepath{stroke,fill}%
}%
\begin{pgfscope}%
\pgfsys@transformshift{0.669304in}{2.084566in}%
\pgfsys@useobject{currentmarker}{}%
\end{pgfscope}%
\end{pgfscope}%
\begin{pgfscope}%
\definecolor{textcolor}{rgb}{0.000000,0.000000,0.000000}%
\pgfsetstrokecolor{textcolor}%
\pgfsetfillcolor{textcolor}%
\pgftext[x=0.376251in, y=2.017094in, left, base]{\color{textcolor}\rmfamily\fontsize{14.000000}{16.800000}\selectfont \(\displaystyle {20}\)}%
\end{pgfscope}%
\begin{pgfscope}%
\pgfsetbuttcap%
\pgfsetroundjoin%
\definecolor{currentfill}{rgb}{0.000000,0.000000,0.000000}%
\pgfsetfillcolor{currentfill}%
\pgfsetlinewidth{0.803000pt}%
\definecolor{currentstroke}{rgb}{0.000000,0.000000,0.000000}%
\pgfsetstrokecolor{currentstroke}%
\pgfsetdash{}{0pt}%
\pgfsys@defobject{currentmarker}{\pgfqpoint{-0.048611in}{0.000000in}}{\pgfqpoint{-0.000000in}{0.000000in}}{%
\pgfpathmoveto{\pgfqpoint{-0.000000in}{0.000000in}}%
\pgfpathlineto{\pgfqpoint{-0.048611in}{0.000000in}}%
\pgfusepath{stroke,fill}%
}%
\begin{pgfscope}%
\pgfsys@transformshift{0.669304in}{2.699916in}%
\pgfsys@useobject{currentmarker}{}%
\end{pgfscope}%
\end{pgfscope}%
\begin{pgfscope}%
\definecolor{textcolor}{rgb}{0.000000,0.000000,0.000000}%
\pgfsetstrokecolor{textcolor}%
\pgfsetfillcolor{textcolor}%
\pgftext[x=0.376251in, y=2.632444in, left, base]{\color{textcolor}\rmfamily\fontsize{14.000000}{16.800000}\selectfont \(\displaystyle {22}\)}%
\end{pgfscope}%
\begin{pgfscope}%
\pgfsetbuttcap%
\pgfsetroundjoin%
\definecolor{currentfill}{rgb}{0.000000,0.000000,0.000000}%
\pgfsetfillcolor{currentfill}%
\pgfsetlinewidth{0.803000pt}%
\definecolor{currentstroke}{rgb}{0.000000,0.000000,0.000000}%
\pgfsetstrokecolor{currentstroke}%
\pgfsetdash{}{0pt}%
\pgfsys@defobject{currentmarker}{\pgfqpoint{-0.048611in}{0.000000in}}{\pgfqpoint{-0.000000in}{0.000000in}}{%
\pgfpathmoveto{\pgfqpoint{-0.000000in}{0.000000in}}%
\pgfpathlineto{\pgfqpoint{-0.048611in}{0.000000in}}%
\pgfusepath{stroke,fill}%
}%
\begin{pgfscope}%
\pgfsys@transformshift{0.669304in}{3.315266in}%
\pgfsys@useobject{currentmarker}{}%
\end{pgfscope}%
\end{pgfscope}%
\begin{pgfscope}%
\definecolor{textcolor}{rgb}{0.000000,0.000000,0.000000}%
\pgfsetstrokecolor{textcolor}%
\pgfsetfillcolor{textcolor}%
\pgftext[x=0.376251in, y=3.247794in, left, base]{\color{textcolor}\rmfamily\fontsize{14.000000}{16.800000}\selectfont \(\displaystyle {24}\)}%
\end{pgfscope}%
\begin{pgfscope}%
\definecolor{textcolor}{rgb}{0.000000,0.000000,0.000000}%
\pgfsetstrokecolor{textcolor}%
\pgfsetfillcolor{textcolor}%
\pgftext[x=0.320695in,y=2.142255in,,bottom,rotate=90.000000]{\color{textcolor}\rmfamily\fontsize{14.000000}{16.800000}\selectfont First order differences [cm\(\displaystyle ^{-1}\)]}%
\end{pgfscope}%
\begin{pgfscope}%
\pgfpathrectangle{\pgfqpoint{0.669304in}{0.598111in}}{\pgfqpoint{5.029404in}{3.088289in}}%
\pgfusepath{clip}%
\pgfsetbuttcap%
\pgfsetroundjoin%
\definecolor{currentfill}{rgb}{1.000000,0.000000,0.000000}%
\pgfsetfillcolor{currentfill}%
\pgfsetlinewidth{1.003750pt}%
\definecolor{currentstroke}{rgb}{1.000000,0.000000,0.000000}%
\pgfsetstrokecolor{currentstroke}%
\pgfsetdash{}{0pt}%
\pgfsys@defobject{currentmarker}{\pgfqpoint{-0.041667in}{-0.041667in}}{\pgfqpoint{0.041667in}{0.041667in}}{%
\pgfpathmoveto{\pgfqpoint{0.000000in}{-0.041667in}}%
\pgfpathcurveto{\pgfqpoint{0.011050in}{-0.041667in}}{\pgfqpoint{0.021649in}{-0.037276in}}{\pgfqpoint{0.029463in}{-0.029463in}}%
\pgfpathcurveto{\pgfqpoint{0.037276in}{-0.021649in}}{\pgfqpoint{0.041667in}{-0.011050in}}{\pgfqpoint{0.041667in}{0.000000in}}%
\pgfpathcurveto{\pgfqpoint{0.041667in}{0.011050in}}{\pgfqpoint{0.037276in}{0.021649in}}{\pgfqpoint{0.029463in}{0.029463in}}%
\pgfpathcurveto{\pgfqpoint{0.021649in}{0.037276in}}{\pgfqpoint{0.011050in}{0.041667in}}{\pgfqpoint{0.000000in}{0.041667in}}%
\pgfpathcurveto{\pgfqpoint{-0.011050in}{0.041667in}}{\pgfqpoint{-0.021649in}{0.037276in}}{\pgfqpoint{-0.029463in}{0.029463in}}%
\pgfpathcurveto{\pgfqpoint{-0.037276in}{0.021649in}}{\pgfqpoint{-0.041667in}{0.011050in}}{\pgfqpoint{-0.041667in}{0.000000in}}%
\pgfpathcurveto{\pgfqpoint{-0.041667in}{-0.011050in}}{\pgfqpoint{-0.037276in}{-0.021649in}}{\pgfqpoint{-0.029463in}{-0.029463in}}%
\pgfpathcurveto{\pgfqpoint{-0.021649in}{-0.037276in}}{\pgfqpoint{-0.011050in}{-0.041667in}}{\pgfqpoint{0.000000in}{-0.041667in}}%
\pgfpathlineto{\pgfqpoint{0.000000in}{-0.041667in}}%
\pgfpathclose%
\pgfusepath{stroke,fill}%
}%
\begin{pgfscope}%
\pgfsys@transformshift{0.969567in}{3.546023in}%
\pgfsys@useobject{currentmarker}{}%
\end{pgfscope}%
\begin{pgfscope}%
\pgfsys@transformshift{1.269830in}{3.392185in}%
\pgfsys@useobject{currentmarker}{}%
\end{pgfscope}%
\begin{pgfscope}%
\pgfsys@transformshift{1.570093in}{3.199888in}%
\pgfsys@useobject{currentmarker}{}%
\end{pgfscope}%
\begin{pgfscope}%
\pgfsys@transformshift{1.870356in}{3.084510in}%
\pgfsys@useobject{currentmarker}{}%
\end{pgfscope}%
\begin{pgfscope}%
\pgfsys@transformshift{2.170618in}{2.892213in}%
\pgfsys@useobject{currentmarker}{}%
\end{pgfscope}%
\begin{pgfscope}%
\pgfsys@transformshift{2.470881in}{2.699916in}%
\pgfsys@useobject{currentmarker}{}%
\end{pgfscope}%
\begin{pgfscope}%
\pgfsys@transformshift{2.771144in}{2.546079in}%
\pgfsys@useobject{currentmarker}{}%
\end{pgfscope}%
\begin{pgfscope}%
\pgfsys@transformshift{3.671933in}{1.969188in}%
\pgfsys@useobject{currentmarker}{}%
\end{pgfscope}%
\begin{pgfscope}%
\pgfsys@transformshift{3.972196in}{1.776891in}%
\pgfsys@useobject{currentmarker}{}%
\end{pgfscope}%
\begin{pgfscope}%
\pgfsys@transformshift{4.272459in}{1.584594in}%
\pgfsys@useobject{currentmarker}{}%
\end{pgfscope}%
\begin{pgfscope}%
\pgfsys@transformshift{4.572722in}{1.392297in}%
\pgfsys@useobject{currentmarker}{}%
\end{pgfscope}%
\begin{pgfscope}%
\pgfsys@transformshift{4.872985in}{1.161541in}%
\pgfsys@useobject{currentmarker}{}%
\end{pgfscope}%
\begin{pgfscope}%
\pgfsys@transformshift{5.173247in}{0.969244in}%
\pgfsys@useobject{currentmarker}{}%
\end{pgfscope}%
\begin{pgfscope}%
\pgfsys@transformshift{5.473510in}{0.738488in}%
\pgfsys@useobject{currentmarker}{}%
\end{pgfscope}%
\end{pgfscope}%
\begin{pgfscope}%
\pgfpathrectangle{\pgfqpoint{0.669304in}{0.598111in}}{\pgfqpoint{5.029404in}{3.088289in}}%
\pgfusepath{clip}%
\pgfsetrectcap%
\pgfsetroundjoin%
\pgfsetlinewidth{1.505625pt}%
\definecolor{currentstroke}{rgb}{0.000000,0.000000,0.000000}%
\pgfsetstrokecolor{currentstroke}%
\pgfsetdash{}{0pt}%
\pgfpathmoveto{\pgfqpoint{0.969567in}{3.544650in}}%
\pgfpathlineto{\pgfqpoint{1.269830in}{3.387034in}}%
\pgfpathlineto{\pgfqpoint{1.570093in}{3.225264in}}%
\pgfpathlineto{\pgfqpoint{1.870356in}{3.059342in}}%
\pgfpathlineto{\pgfqpoint{2.170618in}{2.889266in}}%
\pgfpathlineto{\pgfqpoint{2.470881in}{2.715036in}}%
\pgfpathlineto{\pgfqpoint{2.771144in}{2.536654in}}%
\pgfpathlineto{\pgfqpoint{3.671933in}{1.976585in}}%
\pgfpathlineto{\pgfqpoint{3.972196in}{1.781589in}}%
\pgfpathlineto{\pgfqpoint{4.272459in}{1.582440in}}%
\pgfpathlineto{\pgfqpoint{4.572722in}{1.379137in}}%
\pgfpathlineto{\pgfqpoint{4.872985in}{1.171681in}}%
\pgfpathlineto{\pgfqpoint{5.173247in}{0.960071in}}%
\pgfpathlineto{\pgfqpoint{5.473510in}{0.744308in}}%
\pgfusepath{stroke}%
\end{pgfscope}%
\begin{pgfscope}%
\pgfsetrectcap%
\pgfsetmiterjoin%
\pgfsetlinewidth{0.803000pt}%
\definecolor{currentstroke}{rgb}{0.000000,0.000000,0.000000}%
\pgfsetstrokecolor{currentstroke}%
\pgfsetdash{}{0pt}%
\pgfpathmoveto{\pgfqpoint{0.669304in}{0.598111in}}%
\pgfpathlineto{\pgfqpoint{0.669304in}{3.686399in}}%
\pgfusepath{stroke}%
\end{pgfscope}%
\begin{pgfscope}%
\pgfsetrectcap%
\pgfsetmiterjoin%
\pgfsetlinewidth{0.803000pt}%
\definecolor{currentstroke}{rgb}{0.000000,0.000000,0.000000}%
\pgfsetstrokecolor{currentstroke}%
\pgfsetdash{}{0pt}%
\pgfpathmoveto{\pgfqpoint{5.698708in}{0.598111in}}%
\pgfpathlineto{\pgfqpoint{5.698708in}{3.686399in}}%
\pgfusepath{stroke}%
\end{pgfscope}%
\begin{pgfscope}%
\pgfsetrectcap%
\pgfsetmiterjoin%
\pgfsetlinewidth{0.803000pt}%
\definecolor{currentstroke}{rgb}{0.000000,0.000000,0.000000}%
\pgfsetstrokecolor{currentstroke}%
\pgfsetdash{}{0pt}%
\pgfpathmoveto{\pgfqpoint{0.669304in}{0.598111in}}%
\pgfpathlineto{\pgfqpoint{5.698708in}{0.598111in}}%
\pgfusepath{stroke}%
\end{pgfscope}%
\begin{pgfscope}%
\pgfsetrectcap%
\pgfsetmiterjoin%
\pgfsetlinewidth{0.803000pt}%
\definecolor{currentstroke}{rgb}{0.000000,0.000000,0.000000}%
\pgfsetstrokecolor{currentstroke}%
\pgfsetdash{}{0pt}%
\pgfpathmoveto{\pgfqpoint{0.669304in}{3.686399in}}%
\pgfpathlineto{\pgfqpoint{5.698708in}{3.686399in}}%
\pgfusepath{stroke}%
\end{pgfscope}%
\begin{pgfscope}%
\pgfsetbuttcap%
\pgfsetmiterjoin%
\definecolor{currentfill}{rgb}{1.000000,1.000000,1.000000}%
\pgfsetfillcolor{currentfill}%
\pgfsetfillopacity{0.800000}%
\pgfsetlinewidth{1.003750pt}%
\definecolor{currentstroke}{rgb}{0.800000,0.800000,0.800000}%
\pgfsetstrokecolor{currentstroke}%
\pgfsetstrokeopacity{0.800000}%
\pgfsetdash{}{0pt}%
\pgfpathmoveto{\pgfqpoint{3.789846in}{2.987955in}}%
\pgfpathlineto{\pgfqpoint{5.562596in}{2.987955in}}%
\pgfpathquadraticcurveto{\pgfqpoint{5.601485in}{2.987955in}}{\pgfqpoint{5.601485in}{3.026844in}}%
\pgfpathlineto{\pgfqpoint{5.601485in}{3.550288in}}%
\pgfpathquadraticcurveto{\pgfqpoint{5.601485in}{3.589177in}}{\pgfqpoint{5.562596in}{3.589177in}}%
\pgfpathlineto{\pgfqpoint{3.789846in}{3.589177in}}%
\pgfpathquadraticcurveto{\pgfqpoint{3.750958in}{3.589177in}}{\pgfqpoint{3.750958in}{3.550288in}}%
\pgfpathlineto{\pgfqpoint{3.750958in}{3.026844in}}%
\pgfpathquadraticcurveto{\pgfqpoint{3.750958in}{2.987955in}}{\pgfqpoint{3.789846in}{2.987955in}}%
\pgfpathlineto{\pgfqpoint{3.789846in}{2.987955in}}%
\pgfpathclose%
\pgfusepath{stroke,fill}%
\end{pgfscope}%
\begin{pgfscope}%
\pgfsetbuttcap%
\pgfsetroundjoin%
\definecolor{currentfill}{rgb}{1.000000,0.000000,0.000000}%
\pgfsetfillcolor{currentfill}%
\pgfsetlinewidth{1.003750pt}%
\definecolor{currentstroke}{rgb}{1.000000,0.000000,0.000000}%
\pgfsetstrokecolor{currentstroke}%
\pgfsetdash{}{0pt}%
\pgfsys@defobject{currentmarker}{\pgfqpoint{-0.041667in}{-0.041667in}}{\pgfqpoint{0.041667in}{0.041667in}}{%
\pgfpathmoveto{\pgfqpoint{0.000000in}{-0.041667in}}%
\pgfpathcurveto{\pgfqpoint{0.011050in}{-0.041667in}}{\pgfqpoint{0.021649in}{-0.037276in}}{\pgfqpoint{0.029463in}{-0.029463in}}%
\pgfpathcurveto{\pgfqpoint{0.037276in}{-0.021649in}}{\pgfqpoint{0.041667in}{-0.011050in}}{\pgfqpoint{0.041667in}{0.000000in}}%
\pgfpathcurveto{\pgfqpoint{0.041667in}{0.011050in}}{\pgfqpoint{0.037276in}{0.021649in}}{\pgfqpoint{0.029463in}{0.029463in}}%
\pgfpathcurveto{\pgfqpoint{0.021649in}{0.037276in}}{\pgfqpoint{0.011050in}{0.041667in}}{\pgfqpoint{0.000000in}{0.041667in}}%
\pgfpathcurveto{\pgfqpoint{-0.011050in}{0.041667in}}{\pgfqpoint{-0.021649in}{0.037276in}}{\pgfqpoint{-0.029463in}{0.029463in}}%
\pgfpathcurveto{\pgfqpoint{-0.037276in}{0.021649in}}{\pgfqpoint{-0.041667in}{0.011050in}}{\pgfqpoint{-0.041667in}{0.000000in}}%
\pgfpathcurveto{\pgfqpoint{-0.041667in}{-0.011050in}}{\pgfqpoint{-0.037276in}{-0.021649in}}{\pgfqpoint{-0.029463in}{-0.029463in}}%
\pgfpathcurveto{\pgfqpoint{-0.021649in}{-0.037276in}}{\pgfqpoint{-0.011050in}{-0.041667in}}{\pgfqpoint{0.000000in}{-0.041667in}}%
\pgfpathlineto{\pgfqpoint{0.000000in}{-0.041667in}}%
\pgfpathclose%
\pgfusepath{stroke,fill}%
}%
\begin{pgfscope}%
\pgfsys@transformshift{4.023180in}{3.443344in}%
\pgfsys@useobject{currentmarker}{}%
\end{pgfscope}%
\end{pgfscope}%
\begin{pgfscope}%
\definecolor{textcolor}{rgb}{0.000000,0.000000,0.000000}%
\pgfsetstrokecolor{textcolor}%
\pgfsetfillcolor{textcolor}%
\pgftext[x=4.373180in,y=3.375288in,left,base]{\color{textcolor}\rmfamily\fontsize{14.000000}{16.800000}\selectfont Data}%
\end{pgfscope}%
\begin{pgfscope}%
\pgfsetrectcap%
\pgfsetroundjoin%
\pgfsetlinewidth{1.505625pt}%
\definecolor{currentstroke}{rgb}{0.000000,0.000000,0.000000}%
\pgfsetstrokecolor{currentstroke}%
\pgfsetdash{}{0pt}%
\pgfpathmoveto{\pgfqpoint{3.828735in}{3.171511in}}%
\pgfpathlineto{\pgfqpoint{4.023180in}{3.171511in}}%
\pgfpathlineto{\pgfqpoint{4.217624in}{3.171511in}}%
\pgfusepath{stroke}%
\end{pgfscope}%
\begin{pgfscope}%
\definecolor{textcolor}{rgb}{0.000000,0.000000,0.000000}%
\pgfsetstrokecolor{textcolor}%
\pgfsetfillcolor{textcolor}%
\pgftext[x=4.373180in,y=3.103455in,left,base]{\color{textcolor}\rmfamily\fontsize{14.000000}{16.800000}\selectfont Quadratic Fit}%
\end{pgfscope}%
\end{pgfpicture}%
\makeatother%
\endgroup%
}
		\caption{$^{35}$Cl}
		\label{fig:Quadratic_fit_Cl35}
	\end{subfigure}
	\hspace{0.5cm}
	\begin{subfigure}{0.45\textwidth}
		\centering
		\scalebox{0.50}{%% Creator: Matplotlib, PGF backend
%%
%% To include the figure in your LaTeX document, write
%%   \input{<filename>.pgf}
%%
%% Make sure the required packages are loaded in your preamble
%%   \usepackage{pgf}
%%
%% Also ensure that all the required font packages are loaded; for instance,
%% the lmodern package is sometimes necessary when using math font.
%%   \usepackage{lmodern}
%%
%% Figures using additional raster images can only be included by \input if
%% they are in the same directory as the main LaTeX file. For loading figures
%% from other directories you can use the `import` package
%%   \usepackage{import}
%%
%% and then include the figures with
%%   \import{<path to file>}{<filename>.pgf}
%%
%% Matplotlib used the following preamble
%%   
%%   \usepackage{fontspec}
%%   \makeatletter\@ifpackageloaded{underscore}{}{\usepackage[strings]{underscore}}\makeatother
%%
\begingroup%
\makeatletter%
\begin{pgfpicture}%
\pgfpathrectangle{\pgfpointorigin}{\pgfqpoint{5.798708in}{3.786399in}}%
\pgfusepath{use as bounding box, clip}%
\begin{pgfscope}%
\pgfsetbuttcap%
\pgfsetmiterjoin%
\definecolor{currentfill}{rgb}{1.000000,1.000000,1.000000}%
\pgfsetfillcolor{currentfill}%
\pgfsetlinewidth{0.000000pt}%
\definecolor{currentstroke}{rgb}{1.000000,1.000000,1.000000}%
\pgfsetstrokecolor{currentstroke}%
\pgfsetdash{}{0pt}%
\pgfpathmoveto{\pgfqpoint{0.000000in}{0.000000in}}%
\pgfpathlineto{\pgfqpoint{5.798708in}{0.000000in}}%
\pgfpathlineto{\pgfqpoint{5.798708in}{3.786399in}}%
\pgfpathlineto{\pgfqpoint{0.000000in}{3.786399in}}%
\pgfpathlineto{\pgfqpoint{0.000000in}{0.000000in}}%
\pgfpathclose%
\pgfusepath{fill}%
\end{pgfscope}%
\begin{pgfscope}%
\pgfsetbuttcap%
\pgfsetmiterjoin%
\definecolor{currentfill}{rgb}{1.000000,1.000000,1.000000}%
\pgfsetfillcolor{currentfill}%
\pgfsetlinewidth{0.000000pt}%
\definecolor{currentstroke}{rgb}{0.000000,0.000000,0.000000}%
\pgfsetstrokecolor{currentstroke}%
\pgfsetstrokeopacity{0.000000}%
\pgfsetdash{}{0pt}%
\pgfpathmoveto{\pgfqpoint{0.669304in}{0.598111in}}%
\pgfpathlineto{\pgfqpoint{5.698708in}{0.598111in}}%
\pgfpathlineto{\pgfqpoint{5.698708in}{3.686399in}}%
\pgfpathlineto{\pgfqpoint{0.669304in}{3.686399in}}%
\pgfpathlineto{\pgfqpoint{0.669304in}{0.598111in}}%
\pgfpathclose%
\pgfusepath{fill}%
\end{pgfscope}%
\begin{pgfscope}%
\pgfsetbuttcap%
\pgfsetroundjoin%
\definecolor{currentfill}{rgb}{0.000000,0.000000,0.000000}%
\pgfsetfillcolor{currentfill}%
\pgfsetlinewidth{0.803000pt}%
\definecolor{currentstroke}{rgb}{0.000000,0.000000,0.000000}%
\pgfsetstrokecolor{currentstroke}%
\pgfsetdash{}{0pt}%
\pgfsys@defobject{currentmarker}{\pgfqpoint{0.000000in}{-0.048611in}}{\pgfqpoint{0.000000in}{0.000000in}}{%
\pgfpathmoveto{\pgfqpoint{0.000000in}{0.000000in}}%
\pgfpathlineto{\pgfqpoint{0.000000in}{-0.048611in}}%
\pgfusepath{stroke,fill}%
}%
\begin{pgfscope}%
\pgfsys@transformshift{0.669304in}{0.598111in}%
\pgfsys@useobject{currentmarker}{}%
\end{pgfscope}%
\end{pgfscope}%
\begin{pgfscope}%
\definecolor{textcolor}{rgb}{0.000000,0.000000,0.000000}%
\pgfsetstrokecolor{textcolor}%
\pgfsetfillcolor{textcolor}%
\pgftext[x=0.669304in,y=0.500889in,,top]{\color{textcolor}\rmfamily\fontsize{14.000000}{16.800000}\selectfont \(\displaystyle {\ensuremath{-}8}\)}%
\end{pgfscope}%
\begin{pgfscope}%
\pgfsetbuttcap%
\pgfsetroundjoin%
\definecolor{currentfill}{rgb}{0.000000,0.000000,0.000000}%
\pgfsetfillcolor{currentfill}%
\pgfsetlinewidth{0.803000pt}%
\definecolor{currentstroke}{rgb}{0.000000,0.000000,0.000000}%
\pgfsetstrokecolor{currentstroke}%
\pgfsetdash{}{0pt}%
\pgfsys@defobject{currentmarker}{\pgfqpoint{0.000000in}{-0.048611in}}{\pgfqpoint{0.000000in}{0.000000in}}{%
\pgfpathmoveto{\pgfqpoint{0.000000in}{0.000000in}}%
\pgfpathlineto{\pgfqpoint{0.000000in}{-0.048611in}}%
\pgfusepath{stroke,fill}%
}%
\begin{pgfscope}%
\pgfsys@transformshift{1.269830in}{0.598111in}%
\pgfsys@useobject{currentmarker}{}%
\end{pgfscope}%
\end{pgfscope}%
\begin{pgfscope}%
\definecolor{textcolor}{rgb}{0.000000,0.000000,0.000000}%
\pgfsetstrokecolor{textcolor}%
\pgfsetfillcolor{textcolor}%
\pgftext[x=1.269830in,y=0.500889in,,top]{\color{textcolor}\rmfamily\fontsize{14.000000}{16.800000}\selectfont \(\displaystyle {\ensuremath{-}6}\)}%
\end{pgfscope}%
\begin{pgfscope}%
\pgfsetbuttcap%
\pgfsetroundjoin%
\definecolor{currentfill}{rgb}{0.000000,0.000000,0.000000}%
\pgfsetfillcolor{currentfill}%
\pgfsetlinewidth{0.803000pt}%
\definecolor{currentstroke}{rgb}{0.000000,0.000000,0.000000}%
\pgfsetstrokecolor{currentstroke}%
\pgfsetdash{}{0pt}%
\pgfsys@defobject{currentmarker}{\pgfqpoint{0.000000in}{-0.048611in}}{\pgfqpoint{0.000000in}{0.000000in}}{%
\pgfpathmoveto{\pgfqpoint{0.000000in}{0.000000in}}%
\pgfpathlineto{\pgfqpoint{0.000000in}{-0.048611in}}%
\pgfusepath{stroke,fill}%
}%
\begin{pgfscope}%
\pgfsys@transformshift{1.870356in}{0.598111in}%
\pgfsys@useobject{currentmarker}{}%
\end{pgfscope}%
\end{pgfscope}%
\begin{pgfscope}%
\definecolor{textcolor}{rgb}{0.000000,0.000000,0.000000}%
\pgfsetstrokecolor{textcolor}%
\pgfsetfillcolor{textcolor}%
\pgftext[x=1.870356in,y=0.500889in,,top]{\color{textcolor}\rmfamily\fontsize{14.000000}{16.800000}\selectfont \(\displaystyle {\ensuremath{-}4}\)}%
\end{pgfscope}%
\begin{pgfscope}%
\pgfsetbuttcap%
\pgfsetroundjoin%
\definecolor{currentfill}{rgb}{0.000000,0.000000,0.000000}%
\pgfsetfillcolor{currentfill}%
\pgfsetlinewidth{0.803000pt}%
\definecolor{currentstroke}{rgb}{0.000000,0.000000,0.000000}%
\pgfsetstrokecolor{currentstroke}%
\pgfsetdash{}{0pt}%
\pgfsys@defobject{currentmarker}{\pgfqpoint{0.000000in}{-0.048611in}}{\pgfqpoint{0.000000in}{0.000000in}}{%
\pgfpathmoveto{\pgfqpoint{0.000000in}{0.000000in}}%
\pgfpathlineto{\pgfqpoint{0.000000in}{-0.048611in}}%
\pgfusepath{stroke,fill}%
}%
\begin{pgfscope}%
\pgfsys@transformshift{2.470881in}{0.598111in}%
\pgfsys@useobject{currentmarker}{}%
\end{pgfscope}%
\end{pgfscope}%
\begin{pgfscope}%
\definecolor{textcolor}{rgb}{0.000000,0.000000,0.000000}%
\pgfsetstrokecolor{textcolor}%
\pgfsetfillcolor{textcolor}%
\pgftext[x=2.470881in,y=0.500889in,,top]{\color{textcolor}\rmfamily\fontsize{14.000000}{16.800000}\selectfont \(\displaystyle {\ensuremath{-}2}\)}%
\end{pgfscope}%
\begin{pgfscope}%
\pgfsetbuttcap%
\pgfsetroundjoin%
\definecolor{currentfill}{rgb}{0.000000,0.000000,0.000000}%
\pgfsetfillcolor{currentfill}%
\pgfsetlinewidth{0.803000pt}%
\definecolor{currentstroke}{rgb}{0.000000,0.000000,0.000000}%
\pgfsetstrokecolor{currentstroke}%
\pgfsetdash{}{0pt}%
\pgfsys@defobject{currentmarker}{\pgfqpoint{0.000000in}{-0.048611in}}{\pgfqpoint{0.000000in}{0.000000in}}{%
\pgfpathmoveto{\pgfqpoint{0.000000in}{0.000000in}}%
\pgfpathlineto{\pgfqpoint{0.000000in}{-0.048611in}}%
\pgfusepath{stroke,fill}%
}%
\begin{pgfscope}%
\pgfsys@transformshift{3.071407in}{0.598111in}%
\pgfsys@useobject{currentmarker}{}%
\end{pgfscope}%
\end{pgfscope}%
\begin{pgfscope}%
\definecolor{textcolor}{rgb}{0.000000,0.000000,0.000000}%
\pgfsetstrokecolor{textcolor}%
\pgfsetfillcolor{textcolor}%
\pgftext[x=3.071407in,y=0.500889in,,top]{\color{textcolor}\rmfamily\fontsize{14.000000}{16.800000}\selectfont \(\displaystyle {0}\)}%
\end{pgfscope}%
\begin{pgfscope}%
\pgfsetbuttcap%
\pgfsetroundjoin%
\definecolor{currentfill}{rgb}{0.000000,0.000000,0.000000}%
\pgfsetfillcolor{currentfill}%
\pgfsetlinewidth{0.803000pt}%
\definecolor{currentstroke}{rgb}{0.000000,0.000000,0.000000}%
\pgfsetstrokecolor{currentstroke}%
\pgfsetdash{}{0pt}%
\pgfsys@defobject{currentmarker}{\pgfqpoint{0.000000in}{-0.048611in}}{\pgfqpoint{0.000000in}{0.000000in}}{%
\pgfpathmoveto{\pgfqpoint{0.000000in}{0.000000in}}%
\pgfpathlineto{\pgfqpoint{0.000000in}{-0.048611in}}%
\pgfusepath{stroke,fill}%
}%
\begin{pgfscope}%
\pgfsys@transformshift{3.671933in}{0.598111in}%
\pgfsys@useobject{currentmarker}{}%
\end{pgfscope}%
\end{pgfscope}%
\begin{pgfscope}%
\definecolor{textcolor}{rgb}{0.000000,0.000000,0.000000}%
\pgfsetstrokecolor{textcolor}%
\pgfsetfillcolor{textcolor}%
\pgftext[x=3.671933in,y=0.500889in,,top]{\color{textcolor}\rmfamily\fontsize{14.000000}{16.800000}\selectfont \(\displaystyle {2}\)}%
\end{pgfscope}%
\begin{pgfscope}%
\pgfsetbuttcap%
\pgfsetroundjoin%
\definecolor{currentfill}{rgb}{0.000000,0.000000,0.000000}%
\pgfsetfillcolor{currentfill}%
\pgfsetlinewidth{0.803000pt}%
\definecolor{currentstroke}{rgb}{0.000000,0.000000,0.000000}%
\pgfsetstrokecolor{currentstroke}%
\pgfsetdash{}{0pt}%
\pgfsys@defobject{currentmarker}{\pgfqpoint{0.000000in}{-0.048611in}}{\pgfqpoint{0.000000in}{0.000000in}}{%
\pgfpathmoveto{\pgfqpoint{0.000000in}{0.000000in}}%
\pgfpathlineto{\pgfqpoint{0.000000in}{-0.048611in}}%
\pgfusepath{stroke,fill}%
}%
\begin{pgfscope}%
\pgfsys@transformshift{4.272459in}{0.598111in}%
\pgfsys@useobject{currentmarker}{}%
\end{pgfscope}%
\end{pgfscope}%
\begin{pgfscope}%
\definecolor{textcolor}{rgb}{0.000000,0.000000,0.000000}%
\pgfsetstrokecolor{textcolor}%
\pgfsetfillcolor{textcolor}%
\pgftext[x=4.272459in,y=0.500889in,,top]{\color{textcolor}\rmfamily\fontsize{14.000000}{16.800000}\selectfont \(\displaystyle {4}\)}%
\end{pgfscope}%
\begin{pgfscope}%
\pgfsetbuttcap%
\pgfsetroundjoin%
\definecolor{currentfill}{rgb}{0.000000,0.000000,0.000000}%
\pgfsetfillcolor{currentfill}%
\pgfsetlinewidth{0.803000pt}%
\definecolor{currentstroke}{rgb}{0.000000,0.000000,0.000000}%
\pgfsetstrokecolor{currentstroke}%
\pgfsetdash{}{0pt}%
\pgfsys@defobject{currentmarker}{\pgfqpoint{0.000000in}{-0.048611in}}{\pgfqpoint{0.000000in}{0.000000in}}{%
\pgfpathmoveto{\pgfqpoint{0.000000in}{0.000000in}}%
\pgfpathlineto{\pgfqpoint{0.000000in}{-0.048611in}}%
\pgfusepath{stroke,fill}%
}%
\begin{pgfscope}%
\pgfsys@transformshift{4.872985in}{0.598111in}%
\pgfsys@useobject{currentmarker}{}%
\end{pgfscope}%
\end{pgfscope}%
\begin{pgfscope}%
\definecolor{textcolor}{rgb}{0.000000,0.000000,0.000000}%
\pgfsetstrokecolor{textcolor}%
\pgfsetfillcolor{textcolor}%
\pgftext[x=4.872985in,y=0.500889in,,top]{\color{textcolor}\rmfamily\fontsize{14.000000}{16.800000}\selectfont \(\displaystyle {6}\)}%
\end{pgfscope}%
\begin{pgfscope}%
\pgfsetbuttcap%
\pgfsetroundjoin%
\definecolor{currentfill}{rgb}{0.000000,0.000000,0.000000}%
\pgfsetfillcolor{currentfill}%
\pgfsetlinewidth{0.803000pt}%
\definecolor{currentstroke}{rgb}{0.000000,0.000000,0.000000}%
\pgfsetstrokecolor{currentstroke}%
\pgfsetdash{}{0pt}%
\pgfsys@defobject{currentmarker}{\pgfqpoint{0.000000in}{-0.048611in}}{\pgfqpoint{0.000000in}{0.000000in}}{%
\pgfpathmoveto{\pgfqpoint{0.000000in}{0.000000in}}%
\pgfpathlineto{\pgfqpoint{0.000000in}{-0.048611in}}%
\pgfusepath{stroke,fill}%
}%
\begin{pgfscope}%
\pgfsys@transformshift{5.473510in}{0.598111in}%
\pgfsys@useobject{currentmarker}{}%
\end{pgfscope}%
\end{pgfscope}%
\begin{pgfscope}%
\definecolor{textcolor}{rgb}{0.000000,0.000000,0.000000}%
\pgfsetstrokecolor{textcolor}%
\pgfsetfillcolor{textcolor}%
\pgftext[x=5.473510in,y=0.500889in,,top]{\color{textcolor}\rmfamily\fontsize{14.000000}{16.800000}\selectfont \(\displaystyle {8}\)}%
\end{pgfscope}%
\begin{pgfscope}%
\definecolor{textcolor}{rgb}{0.000000,0.000000,0.000000}%
\pgfsetstrokecolor{textcolor}%
\pgfsetfillcolor{textcolor}%
\pgftext[x=3.184006in,y=0.272667in,,top]{\color{textcolor}\rmfamily\fontsize{14.000000}{16.800000}\selectfont Index i}%
\end{pgfscope}%
\begin{pgfscope}%
\pgfsetbuttcap%
\pgfsetroundjoin%
\definecolor{currentfill}{rgb}{0.000000,0.000000,0.000000}%
\pgfsetfillcolor{currentfill}%
\pgfsetlinewidth{0.803000pt}%
\definecolor{currentstroke}{rgb}{0.000000,0.000000,0.000000}%
\pgfsetstrokecolor{currentstroke}%
\pgfsetdash{}{0pt}%
\pgfsys@defobject{currentmarker}{\pgfqpoint{-0.048611in}{0.000000in}}{\pgfqpoint{-0.000000in}{0.000000in}}{%
\pgfpathmoveto{\pgfqpoint{-0.000000in}{0.000000in}}%
\pgfpathlineto{\pgfqpoint{-0.048611in}{0.000000in}}%
\pgfusepath{stroke,fill}%
}%
\begin{pgfscope}%
\pgfsys@transformshift{0.669304in}{0.831238in}%
\pgfsys@useobject{currentmarker}{}%
\end{pgfscope}%
\end{pgfscope}%
\begin{pgfscope}%
\definecolor{textcolor}{rgb}{0.000000,0.000000,0.000000}%
\pgfsetstrokecolor{textcolor}%
\pgfsetfillcolor{textcolor}%
\pgftext[x=0.376251in, y=0.763766in, left, base]{\color{textcolor}\rmfamily\fontsize{14.000000}{16.800000}\selectfont \(\displaystyle {16}\)}%
\end{pgfscope}%
\begin{pgfscope}%
\pgfsetbuttcap%
\pgfsetroundjoin%
\definecolor{currentfill}{rgb}{0.000000,0.000000,0.000000}%
\pgfsetfillcolor{currentfill}%
\pgfsetlinewidth{0.803000pt}%
\definecolor{currentstroke}{rgb}{0.000000,0.000000,0.000000}%
\pgfsetstrokecolor{currentstroke}%
\pgfsetdash{}{0pt}%
\pgfsys@defobject{currentmarker}{\pgfqpoint{-0.048611in}{0.000000in}}{\pgfqpoint{-0.000000in}{0.000000in}}{%
\pgfpathmoveto{\pgfqpoint{-0.000000in}{0.000000in}}%
\pgfpathlineto{\pgfqpoint{-0.048611in}{0.000000in}}%
\pgfusepath{stroke,fill}%
}%
\begin{pgfscope}%
\pgfsys@transformshift{0.669304in}{1.449858in}%
\pgfsys@useobject{currentmarker}{}%
\end{pgfscope}%
\end{pgfscope}%
\begin{pgfscope}%
\definecolor{textcolor}{rgb}{0.000000,0.000000,0.000000}%
\pgfsetstrokecolor{textcolor}%
\pgfsetfillcolor{textcolor}%
\pgftext[x=0.376251in, y=1.382386in, left, base]{\color{textcolor}\rmfamily\fontsize{14.000000}{16.800000}\selectfont \(\displaystyle {18}\)}%
\end{pgfscope}%
\begin{pgfscope}%
\pgfsetbuttcap%
\pgfsetroundjoin%
\definecolor{currentfill}{rgb}{0.000000,0.000000,0.000000}%
\pgfsetfillcolor{currentfill}%
\pgfsetlinewidth{0.803000pt}%
\definecolor{currentstroke}{rgb}{0.000000,0.000000,0.000000}%
\pgfsetstrokecolor{currentstroke}%
\pgfsetdash{}{0pt}%
\pgfsys@defobject{currentmarker}{\pgfqpoint{-0.048611in}{0.000000in}}{\pgfqpoint{-0.000000in}{0.000000in}}{%
\pgfpathmoveto{\pgfqpoint{-0.000000in}{0.000000in}}%
\pgfpathlineto{\pgfqpoint{-0.048611in}{0.000000in}}%
\pgfusepath{stroke,fill}%
}%
\begin{pgfscope}%
\pgfsys@transformshift{0.669304in}{2.068478in}%
\pgfsys@useobject{currentmarker}{}%
\end{pgfscope}%
\end{pgfscope}%
\begin{pgfscope}%
\definecolor{textcolor}{rgb}{0.000000,0.000000,0.000000}%
\pgfsetstrokecolor{textcolor}%
\pgfsetfillcolor{textcolor}%
\pgftext[x=0.376251in, y=2.001006in, left, base]{\color{textcolor}\rmfamily\fontsize{14.000000}{16.800000}\selectfont \(\displaystyle {20}\)}%
\end{pgfscope}%
\begin{pgfscope}%
\pgfsetbuttcap%
\pgfsetroundjoin%
\definecolor{currentfill}{rgb}{0.000000,0.000000,0.000000}%
\pgfsetfillcolor{currentfill}%
\pgfsetlinewidth{0.803000pt}%
\definecolor{currentstroke}{rgb}{0.000000,0.000000,0.000000}%
\pgfsetstrokecolor{currentstroke}%
\pgfsetdash{}{0pt}%
\pgfsys@defobject{currentmarker}{\pgfqpoint{-0.048611in}{0.000000in}}{\pgfqpoint{-0.000000in}{0.000000in}}{%
\pgfpathmoveto{\pgfqpoint{-0.000000in}{0.000000in}}%
\pgfpathlineto{\pgfqpoint{-0.048611in}{0.000000in}}%
\pgfusepath{stroke,fill}%
}%
\begin{pgfscope}%
\pgfsys@transformshift{0.669304in}{2.687098in}%
\pgfsys@useobject{currentmarker}{}%
\end{pgfscope}%
\end{pgfscope}%
\begin{pgfscope}%
\definecolor{textcolor}{rgb}{0.000000,0.000000,0.000000}%
\pgfsetstrokecolor{textcolor}%
\pgfsetfillcolor{textcolor}%
\pgftext[x=0.376251in, y=2.619626in, left, base]{\color{textcolor}\rmfamily\fontsize{14.000000}{16.800000}\selectfont \(\displaystyle {22}\)}%
\end{pgfscope}%
\begin{pgfscope}%
\pgfsetbuttcap%
\pgfsetroundjoin%
\definecolor{currentfill}{rgb}{0.000000,0.000000,0.000000}%
\pgfsetfillcolor{currentfill}%
\pgfsetlinewidth{0.803000pt}%
\definecolor{currentstroke}{rgb}{0.000000,0.000000,0.000000}%
\pgfsetstrokecolor{currentstroke}%
\pgfsetdash{}{0pt}%
\pgfsys@defobject{currentmarker}{\pgfqpoint{-0.048611in}{0.000000in}}{\pgfqpoint{-0.000000in}{0.000000in}}{%
\pgfpathmoveto{\pgfqpoint{-0.000000in}{0.000000in}}%
\pgfpathlineto{\pgfqpoint{-0.048611in}{0.000000in}}%
\pgfusepath{stroke,fill}%
}%
\begin{pgfscope}%
\pgfsys@transformshift{0.669304in}{3.305718in}%
\pgfsys@useobject{currentmarker}{}%
\end{pgfscope}%
\end{pgfscope}%
\begin{pgfscope}%
\definecolor{textcolor}{rgb}{0.000000,0.000000,0.000000}%
\pgfsetstrokecolor{textcolor}%
\pgfsetfillcolor{textcolor}%
\pgftext[x=0.376251in, y=3.238246in, left, base]{\color{textcolor}\rmfamily\fontsize{14.000000}{16.800000}\selectfont \(\displaystyle {24}\)}%
\end{pgfscope}%
\begin{pgfscope}%
\definecolor{textcolor}{rgb}{0.000000,0.000000,0.000000}%
\pgfsetstrokecolor{textcolor}%
\pgfsetfillcolor{textcolor}%
\pgftext[x=0.320695in,y=2.142255in,,bottom,rotate=90.000000]{\color{textcolor}\rmfamily\fontsize{14.000000}{16.800000}\selectfont First order differences [cm\(\displaystyle ^{-1}\)]}%
\end{pgfscope}%
\begin{pgfscope}%
\pgfpathrectangle{\pgfqpoint{0.669304in}{0.598111in}}{\pgfqpoint{5.029404in}{3.088289in}}%
\pgfusepath{clip}%
\pgfsetbuttcap%
\pgfsetroundjoin%
\definecolor{currentfill}{rgb}{1.000000,0.000000,0.000000}%
\pgfsetfillcolor{currentfill}%
\pgfsetlinewidth{1.003750pt}%
\definecolor{currentstroke}{rgb}{1.000000,0.000000,0.000000}%
\pgfsetstrokecolor{currentstroke}%
\pgfsetdash{}{0pt}%
\pgfsys@defobject{currentmarker}{\pgfqpoint{-0.041667in}{-0.041667in}}{\pgfqpoint{0.041667in}{0.041667in}}{%
\pgfpathmoveto{\pgfqpoint{0.000000in}{-0.041667in}}%
\pgfpathcurveto{\pgfqpoint{0.011050in}{-0.041667in}}{\pgfqpoint{0.021649in}{-0.037276in}}{\pgfqpoint{0.029463in}{-0.029463in}}%
\pgfpathcurveto{\pgfqpoint{0.037276in}{-0.021649in}}{\pgfqpoint{0.041667in}{-0.011050in}}{\pgfqpoint{0.041667in}{0.000000in}}%
\pgfpathcurveto{\pgfqpoint{0.041667in}{0.011050in}}{\pgfqpoint{0.037276in}{0.021649in}}{\pgfqpoint{0.029463in}{0.029463in}}%
\pgfpathcurveto{\pgfqpoint{0.021649in}{0.037276in}}{\pgfqpoint{0.011050in}{0.041667in}}{\pgfqpoint{0.000000in}{0.041667in}}%
\pgfpathcurveto{\pgfqpoint{-0.011050in}{0.041667in}}{\pgfqpoint{-0.021649in}{0.037276in}}{\pgfqpoint{-0.029463in}{0.029463in}}%
\pgfpathcurveto{\pgfqpoint{-0.037276in}{0.021649in}}{\pgfqpoint{-0.041667in}{0.011050in}}{\pgfqpoint{-0.041667in}{0.000000in}}%
\pgfpathcurveto{\pgfqpoint{-0.041667in}{-0.011050in}}{\pgfqpoint{-0.037276in}{-0.021649in}}{\pgfqpoint{-0.029463in}{-0.029463in}}%
\pgfpathcurveto{\pgfqpoint{-0.021649in}{-0.037276in}}{\pgfqpoint{-0.011050in}{-0.041667in}}{\pgfqpoint{0.000000in}{-0.041667in}}%
\pgfpathlineto{\pgfqpoint{0.000000in}{-0.041667in}}%
\pgfpathclose%
\pgfusepath{stroke,fill}%
}%
\begin{pgfscope}%
\pgfsys@transformshift{0.969567in}{3.537701in}%
\pgfsys@useobject{currentmarker}{}%
\end{pgfscope}%
\begin{pgfscope}%
\pgfsys@transformshift{1.269830in}{3.383046in}%
\pgfsys@useobject{currentmarker}{}%
\end{pgfscope}%
\begin{pgfscope}%
\pgfsys@transformshift{1.570093in}{3.228391in}%
\pgfsys@useobject{currentmarker}{}%
\end{pgfscope}%
\begin{pgfscope}%
\pgfsys@transformshift{1.870356in}{3.073736in}%
\pgfsys@useobject{currentmarker}{}%
\end{pgfscope}%
\begin{pgfscope}%
\pgfsys@transformshift{2.170618in}{2.880417in}%
\pgfsys@useobject{currentmarker}{}%
\end{pgfscope}%
\begin{pgfscope}%
\pgfsys@transformshift{2.470881in}{2.687098in}%
\pgfsys@useobject{currentmarker}{}%
\end{pgfscope}%
\begin{pgfscope}%
\pgfsys@transformshift{2.771144in}{2.532443in}%
\pgfsys@useobject{currentmarker}{}%
\end{pgfscope}%
\begin{pgfscope}%
\pgfsys@transformshift{3.671933in}{1.952487in}%
\pgfsys@useobject{currentmarker}{}%
\end{pgfscope}%
\begin{pgfscope}%
\pgfsys@transformshift{3.972196in}{1.797832in}%
\pgfsys@useobject{currentmarker}{}%
\end{pgfscope}%
\begin{pgfscope}%
\pgfsys@transformshift{4.272459in}{1.565850in}%
\pgfsys@useobject{currentmarker}{}%
\end{pgfscope}%
\begin{pgfscope}%
\pgfsys@transformshift{4.572722in}{1.372531in}%
\pgfsys@useobject{currentmarker}{}%
\end{pgfscope}%
\begin{pgfscope}%
\pgfsys@transformshift{4.872985in}{1.140548in}%
\pgfsys@useobject{currentmarker}{}%
\end{pgfscope}%
\begin{pgfscope}%
\pgfsys@transformshift{5.173247in}{0.947230in}%
\pgfsys@useobject{currentmarker}{}%
\end{pgfscope}%
\begin{pgfscope}%
\pgfsys@transformshift{5.473510in}{0.753911in}%
\pgfsys@useobject{currentmarker}{}%
\end{pgfscope}%
\end{pgfscope}%
\begin{pgfscope}%
\pgfpathrectangle{\pgfqpoint{0.669304in}{0.598111in}}{\pgfqpoint{5.029404in}{3.088289in}}%
\pgfusepath{clip}%
\pgfsetrectcap%
\pgfsetroundjoin%
\pgfsetlinewidth{1.505625pt}%
\definecolor{currentstroke}{rgb}{0.000000,0.000000,0.000000}%
\pgfsetstrokecolor{currentstroke}%
\pgfsetdash{}{0pt}%
\pgfpathmoveto{\pgfqpoint{0.969567in}{3.546023in}}%
\pgfpathlineto{\pgfqpoint{1.269830in}{3.385780in}}%
\pgfpathlineto{\pgfqpoint{1.570093in}{3.221691in}}%
\pgfpathlineto{\pgfqpoint{1.870356in}{3.053755in}}%
\pgfpathlineto{\pgfqpoint{2.170618in}{2.881973in}}%
\pgfpathlineto{\pgfqpoint{2.470881in}{2.706344in}}%
\pgfpathlineto{\pgfqpoint{2.771144in}{2.526868in}}%
\pgfpathlineto{\pgfqpoint{3.671933in}{1.965361in}}%
\pgfpathlineto{\pgfqpoint{3.972196in}{1.770499in}}%
\pgfpathlineto{\pgfqpoint{4.272459in}{1.571790in}}%
\pgfpathlineto{\pgfqpoint{4.572722in}{1.369234in}}%
\pgfpathlineto{\pgfqpoint{4.872985in}{1.162832in}}%
\pgfpathlineto{\pgfqpoint{5.173247in}{0.952583in}}%
\pgfpathlineto{\pgfqpoint{5.473510in}{0.738488in}}%
\pgfusepath{stroke}%
\end{pgfscope}%
\begin{pgfscope}%
\pgfsetrectcap%
\pgfsetmiterjoin%
\pgfsetlinewidth{0.803000pt}%
\definecolor{currentstroke}{rgb}{0.000000,0.000000,0.000000}%
\pgfsetstrokecolor{currentstroke}%
\pgfsetdash{}{0pt}%
\pgfpathmoveto{\pgfqpoint{0.669304in}{0.598111in}}%
\pgfpathlineto{\pgfqpoint{0.669304in}{3.686399in}}%
\pgfusepath{stroke}%
\end{pgfscope}%
\begin{pgfscope}%
\pgfsetrectcap%
\pgfsetmiterjoin%
\pgfsetlinewidth{0.803000pt}%
\definecolor{currentstroke}{rgb}{0.000000,0.000000,0.000000}%
\pgfsetstrokecolor{currentstroke}%
\pgfsetdash{}{0pt}%
\pgfpathmoveto{\pgfqpoint{5.698708in}{0.598111in}}%
\pgfpathlineto{\pgfqpoint{5.698708in}{3.686399in}}%
\pgfusepath{stroke}%
\end{pgfscope}%
\begin{pgfscope}%
\pgfsetrectcap%
\pgfsetmiterjoin%
\pgfsetlinewidth{0.803000pt}%
\definecolor{currentstroke}{rgb}{0.000000,0.000000,0.000000}%
\pgfsetstrokecolor{currentstroke}%
\pgfsetdash{}{0pt}%
\pgfpathmoveto{\pgfqpoint{0.669304in}{0.598111in}}%
\pgfpathlineto{\pgfqpoint{5.698708in}{0.598111in}}%
\pgfusepath{stroke}%
\end{pgfscope}%
\begin{pgfscope}%
\pgfsetrectcap%
\pgfsetmiterjoin%
\pgfsetlinewidth{0.803000pt}%
\definecolor{currentstroke}{rgb}{0.000000,0.000000,0.000000}%
\pgfsetstrokecolor{currentstroke}%
\pgfsetdash{}{0pt}%
\pgfpathmoveto{\pgfqpoint{0.669304in}{3.686399in}}%
\pgfpathlineto{\pgfqpoint{5.698708in}{3.686399in}}%
\pgfusepath{stroke}%
\end{pgfscope}%
\begin{pgfscope}%
\pgfsetbuttcap%
\pgfsetmiterjoin%
\definecolor{currentfill}{rgb}{1.000000,1.000000,1.000000}%
\pgfsetfillcolor{currentfill}%
\pgfsetfillopacity{0.800000}%
\pgfsetlinewidth{1.003750pt}%
\definecolor{currentstroke}{rgb}{0.800000,0.800000,0.800000}%
\pgfsetstrokecolor{currentstroke}%
\pgfsetstrokeopacity{0.800000}%
\pgfsetdash{}{0pt}%
\pgfpathmoveto{\pgfqpoint{3.789846in}{2.987955in}}%
\pgfpathlineto{\pgfqpoint{5.562596in}{2.987955in}}%
\pgfpathquadraticcurveto{\pgfqpoint{5.601485in}{2.987955in}}{\pgfqpoint{5.601485in}{3.026844in}}%
\pgfpathlineto{\pgfqpoint{5.601485in}{3.550288in}}%
\pgfpathquadraticcurveto{\pgfqpoint{5.601485in}{3.589177in}}{\pgfqpoint{5.562596in}{3.589177in}}%
\pgfpathlineto{\pgfqpoint{3.789846in}{3.589177in}}%
\pgfpathquadraticcurveto{\pgfqpoint{3.750958in}{3.589177in}}{\pgfqpoint{3.750958in}{3.550288in}}%
\pgfpathlineto{\pgfqpoint{3.750958in}{3.026844in}}%
\pgfpathquadraticcurveto{\pgfqpoint{3.750958in}{2.987955in}}{\pgfqpoint{3.789846in}{2.987955in}}%
\pgfpathlineto{\pgfqpoint{3.789846in}{2.987955in}}%
\pgfpathclose%
\pgfusepath{stroke,fill}%
\end{pgfscope}%
\begin{pgfscope}%
\pgfsetbuttcap%
\pgfsetroundjoin%
\definecolor{currentfill}{rgb}{1.000000,0.000000,0.000000}%
\pgfsetfillcolor{currentfill}%
\pgfsetlinewidth{1.003750pt}%
\definecolor{currentstroke}{rgb}{1.000000,0.000000,0.000000}%
\pgfsetstrokecolor{currentstroke}%
\pgfsetdash{}{0pt}%
\pgfsys@defobject{currentmarker}{\pgfqpoint{-0.041667in}{-0.041667in}}{\pgfqpoint{0.041667in}{0.041667in}}{%
\pgfpathmoveto{\pgfqpoint{0.000000in}{-0.041667in}}%
\pgfpathcurveto{\pgfqpoint{0.011050in}{-0.041667in}}{\pgfqpoint{0.021649in}{-0.037276in}}{\pgfqpoint{0.029463in}{-0.029463in}}%
\pgfpathcurveto{\pgfqpoint{0.037276in}{-0.021649in}}{\pgfqpoint{0.041667in}{-0.011050in}}{\pgfqpoint{0.041667in}{0.000000in}}%
\pgfpathcurveto{\pgfqpoint{0.041667in}{0.011050in}}{\pgfqpoint{0.037276in}{0.021649in}}{\pgfqpoint{0.029463in}{0.029463in}}%
\pgfpathcurveto{\pgfqpoint{0.021649in}{0.037276in}}{\pgfqpoint{0.011050in}{0.041667in}}{\pgfqpoint{0.000000in}{0.041667in}}%
\pgfpathcurveto{\pgfqpoint{-0.011050in}{0.041667in}}{\pgfqpoint{-0.021649in}{0.037276in}}{\pgfqpoint{-0.029463in}{0.029463in}}%
\pgfpathcurveto{\pgfqpoint{-0.037276in}{0.021649in}}{\pgfqpoint{-0.041667in}{0.011050in}}{\pgfqpoint{-0.041667in}{0.000000in}}%
\pgfpathcurveto{\pgfqpoint{-0.041667in}{-0.011050in}}{\pgfqpoint{-0.037276in}{-0.021649in}}{\pgfqpoint{-0.029463in}{-0.029463in}}%
\pgfpathcurveto{\pgfqpoint{-0.021649in}{-0.037276in}}{\pgfqpoint{-0.011050in}{-0.041667in}}{\pgfqpoint{0.000000in}{-0.041667in}}%
\pgfpathlineto{\pgfqpoint{0.000000in}{-0.041667in}}%
\pgfpathclose%
\pgfusepath{stroke,fill}%
}%
\begin{pgfscope}%
\pgfsys@transformshift{4.023180in}{3.443344in}%
\pgfsys@useobject{currentmarker}{}%
\end{pgfscope}%
\end{pgfscope}%
\begin{pgfscope}%
\definecolor{textcolor}{rgb}{0.000000,0.000000,0.000000}%
\pgfsetstrokecolor{textcolor}%
\pgfsetfillcolor{textcolor}%
\pgftext[x=4.373180in,y=3.375288in,left,base]{\color{textcolor}\rmfamily\fontsize{14.000000}{16.800000}\selectfont Data}%
\end{pgfscope}%
\begin{pgfscope}%
\pgfsetrectcap%
\pgfsetroundjoin%
\pgfsetlinewidth{1.505625pt}%
\definecolor{currentstroke}{rgb}{0.000000,0.000000,0.000000}%
\pgfsetstrokecolor{currentstroke}%
\pgfsetdash{}{0pt}%
\pgfpathmoveto{\pgfqpoint{3.828735in}{3.171511in}}%
\pgfpathlineto{\pgfqpoint{4.023180in}{3.171511in}}%
\pgfpathlineto{\pgfqpoint{4.217624in}{3.171511in}}%
\pgfusepath{stroke}%
\end{pgfscope}%
\begin{pgfscope}%
\definecolor{textcolor}{rgb}{0.000000,0.000000,0.000000}%
\pgfsetstrokecolor{textcolor}%
\pgfsetfillcolor{textcolor}%
\pgftext[x=4.373180in,y=3.103455in,left,base]{\color{textcolor}\rmfamily\fontsize{14.000000}{16.800000}\selectfont Quadratic Fit}%
\end{pgfscope}%
\end{pgfpicture}%
\makeatother%
\endgroup%
}
		\caption{$^{37}$Cl}
		\label{fig:Quadratic_fit_Cl37}
	\end{subfigure}
	\caption{Quadratic fits for isotopes of Chlorine}
\end{figure}

The resulting parameters are shown in Table \ref{table:parameters_HCl}.

\begin{table}[h!]
    \centering
    \[
    \begin{array}{|c|c|c|}
    \hline \text{Parameter} &^{35} \mathrm{Cl} & ^{37} \mathrm{Cl} \\
    \hline B_0\left[\mathrm{cm}^{-1}\right] & 10.143 \pm 0.012 & 10.135 \pm 0.011 \\
    \hline B_1\left[\mathrm{cm}^{-1}\right] & 10.445 \pm 0.011 & 10.439 \pm 0.011 \\
    \hline D\left[\mathrm{cm}^{-1}\right] & 0.00052 \pm 6.09 \times 10^{-5} &  0.00056 \pm 5.47 \times 10^{-5} \\
    \hline
    \end{array}
    \]
    \caption{Parameters for $^{35} \mathrm{Cl}$ and $^{37} \mathrm{Cl}$}
    \label{table:parameters_HCl}
\end{table}

The spectral splitting is found to be: 

\begin{table}[H]
    \centering
    \[
    \begin{array}{|c|c|}
    \hline \text{Index} & \text{Value} \\
    \hline -8 & 2.375 \\
    \hline -7 & 2.25 \\
    \hline -6 & 2.25 \\
    \hline -5 & 2.25 \\
    \hline -4 & 2.25 \\
    \hline -3 & 2.25 \\
    \hline -2 & 2.125 \\
    \hline -1 & 2.125 \\
    \hline 1 & 2.0 \\
    \hline 2 & 2.0 \\
    \hline 3 & 2.0 \\
    \hline 4 & 2.0 \\
    \hline 5 & 2.0 \\
    \hline 6 & 1.875 \\
    \hline 7 & 1.875 \\
    \hline 8 & 1.875 \\
    \hline
    \end{array}
    \]
    \caption{Values of the splitting between the two isotopes}
    \label{table:splitting_values}
\end{table}

To find the ratio between the two isotopes, we had to adjust the spectrum to eliminate other sources of absorption. The minima were masked and the remaining data was interpolated using the cubic spline method. The interpolated data was then renormalized using this cubic spline to obtain the intensity. This process is shown in Fig. \ref{fig:Interpolation}.

\begin{figure}[H]
	\centering
	\begin{subfigure}{0.45\textwidth}
		\centering
		\scalebox{0.50}{%% Creator: Matplotlib, PGF backend
%%
%% To include the figure in your LaTeX document, write
%%   \input{<filename>.pgf}
%%
%% Make sure the required packages are loaded in your preamble
%%   \usepackage{pgf}
%%
%% Also ensure that all the required font packages are loaded; for instance,
%% the lmodern package is sometimes necessary when using math font.
%%   \usepackage{lmodern}
%%
%% Figures using additional raster images can only be included by \input if
%% they are in the same directory as the main LaTeX file. For loading figures
%% from other directories you can use the `import` package
%%   \usepackage{import}
%%
%% and then include the figures with
%%   \import{<path to file>}{<filename>.pgf}
%%
%% Matplotlib used the following preamble
%%   
%%   \usepackage{fontspec}
%%   \makeatletter\@ifpackageloaded{underscore}{}{\usepackage[strings]{underscore}}\makeatother
%%
\begingroup%
\makeatletter%
\begin{pgfpicture}%
\pgfpathrectangle{\pgfpointorigin}{\pgfqpoint{5.980895in}{4.092528in}}%
\pgfusepath{use as bounding box, clip}%
\begin{pgfscope}%
\pgfsetbuttcap%
\pgfsetmiterjoin%
\definecolor{currentfill}{rgb}{1.000000,1.000000,1.000000}%
\pgfsetfillcolor{currentfill}%
\pgfsetlinewidth{0.000000pt}%
\definecolor{currentstroke}{rgb}{1.000000,1.000000,1.000000}%
\pgfsetstrokecolor{currentstroke}%
\pgfsetdash{}{0pt}%
\pgfpathmoveto{\pgfqpoint{0.000000in}{-0.000000in}}%
\pgfpathlineto{\pgfqpoint{5.980895in}{-0.000000in}}%
\pgfpathlineto{\pgfqpoint{5.980895in}{4.092528in}}%
\pgfpathlineto{\pgfqpoint{0.000000in}{4.092528in}}%
\pgfpathlineto{\pgfqpoint{0.000000in}{-0.000000in}}%
\pgfpathclose%
\pgfusepath{fill}%
\end{pgfscope}%
\begin{pgfscope}%
\pgfsetbuttcap%
\pgfsetmiterjoin%
\definecolor{currentfill}{rgb}{1.000000,1.000000,1.000000}%
\pgfsetfillcolor{currentfill}%
\pgfsetlinewidth{0.000000pt}%
\definecolor{currentstroke}{rgb}{0.000000,0.000000,0.000000}%
\pgfsetstrokecolor{currentstroke}%
\pgfsetstrokeopacity{0.000000}%
\pgfsetdash{}{0pt}%
\pgfpathmoveto{\pgfqpoint{0.795366in}{0.646140in}}%
\pgfpathlineto{\pgfqpoint{5.824769in}{0.646140in}}%
\pgfpathlineto{\pgfqpoint{5.824769in}{3.734428in}}%
\pgfpathlineto{\pgfqpoint{0.795366in}{3.734428in}}%
\pgfpathlineto{\pgfqpoint{0.795366in}{0.646140in}}%
\pgfpathclose%
\pgfusepath{fill}%
\end{pgfscope}%
\begin{pgfscope}%
\pgfsetbuttcap%
\pgfsetroundjoin%
\definecolor{currentfill}{rgb}{0.000000,0.000000,0.000000}%
\pgfsetfillcolor{currentfill}%
\pgfsetlinewidth{0.803000pt}%
\definecolor{currentstroke}{rgb}{0.000000,0.000000,0.000000}%
\pgfsetstrokecolor{currentstroke}%
\pgfsetdash{}{0pt}%
\pgfsys@defobject{currentmarker}{\pgfqpoint{0.000000in}{-0.048611in}}{\pgfqpoint{0.000000in}{0.000000in}}{%
\pgfpathmoveto{\pgfqpoint{0.000000in}{0.000000in}}%
\pgfpathlineto{\pgfqpoint{0.000000in}{-0.048611in}}%
\pgfusepath{stroke,fill}%
}%
\begin{pgfscope}%
\pgfsys@transformshift{0.795366in}{0.646140in}%
\pgfsys@useobject{currentmarker}{}%
\end{pgfscope}%
\end{pgfscope}%
\begin{pgfscope}%
\definecolor{textcolor}{rgb}{0.000000,0.000000,0.000000}%
\pgfsetstrokecolor{textcolor}%
\pgfsetfillcolor{textcolor}%
\pgftext[x=0.795366in,y=0.548917in,,top]{\color{textcolor}\rmfamily\fontsize{14.000000}{16.800000}\selectfont \(\displaystyle {2700}\)}%
\end{pgfscope}%
\begin{pgfscope}%
\pgfsetbuttcap%
\pgfsetroundjoin%
\definecolor{currentfill}{rgb}{0.000000,0.000000,0.000000}%
\pgfsetfillcolor{currentfill}%
\pgfsetlinewidth{0.803000pt}%
\definecolor{currentstroke}{rgb}{0.000000,0.000000,0.000000}%
\pgfsetstrokecolor{currentstroke}%
\pgfsetdash{}{0pt}%
\pgfsys@defobject{currentmarker}{\pgfqpoint{0.000000in}{-0.048611in}}{\pgfqpoint{0.000000in}{0.000000in}}{%
\pgfpathmoveto{\pgfqpoint{0.000000in}{0.000000in}}%
\pgfpathlineto{\pgfqpoint{0.000000in}{-0.048611in}}%
\pgfusepath{stroke,fill}%
}%
\begin{pgfscope}%
\pgfsys@transformshift{1.493894in}{0.646140in}%
\pgfsys@useobject{currentmarker}{}%
\end{pgfscope}%
\end{pgfscope}%
\begin{pgfscope}%
\definecolor{textcolor}{rgb}{0.000000,0.000000,0.000000}%
\pgfsetstrokecolor{textcolor}%
\pgfsetfillcolor{textcolor}%
\pgftext[x=1.493894in,y=0.548917in,,top]{\color{textcolor}\rmfamily\fontsize{14.000000}{16.800000}\selectfont \(\displaystyle {2725}\)}%
\end{pgfscope}%
\begin{pgfscope}%
\pgfsetbuttcap%
\pgfsetroundjoin%
\definecolor{currentfill}{rgb}{0.000000,0.000000,0.000000}%
\pgfsetfillcolor{currentfill}%
\pgfsetlinewidth{0.803000pt}%
\definecolor{currentstroke}{rgb}{0.000000,0.000000,0.000000}%
\pgfsetstrokecolor{currentstroke}%
\pgfsetdash{}{0pt}%
\pgfsys@defobject{currentmarker}{\pgfqpoint{0.000000in}{-0.048611in}}{\pgfqpoint{0.000000in}{0.000000in}}{%
\pgfpathmoveto{\pgfqpoint{0.000000in}{0.000000in}}%
\pgfpathlineto{\pgfqpoint{0.000000in}{-0.048611in}}%
\pgfusepath{stroke,fill}%
}%
\begin{pgfscope}%
\pgfsys@transformshift{2.192422in}{0.646140in}%
\pgfsys@useobject{currentmarker}{}%
\end{pgfscope}%
\end{pgfscope}%
\begin{pgfscope}%
\definecolor{textcolor}{rgb}{0.000000,0.000000,0.000000}%
\pgfsetstrokecolor{textcolor}%
\pgfsetfillcolor{textcolor}%
\pgftext[x=2.192422in,y=0.548917in,,top]{\color{textcolor}\rmfamily\fontsize{14.000000}{16.800000}\selectfont \(\displaystyle {2750}\)}%
\end{pgfscope}%
\begin{pgfscope}%
\pgfsetbuttcap%
\pgfsetroundjoin%
\definecolor{currentfill}{rgb}{0.000000,0.000000,0.000000}%
\pgfsetfillcolor{currentfill}%
\pgfsetlinewidth{0.803000pt}%
\definecolor{currentstroke}{rgb}{0.000000,0.000000,0.000000}%
\pgfsetstrokecolor{currentstroke}%
\pgfsetdash{}{0pt}%
\pgfsys@defobject{currentmarker}{\pgfqpoint{0.000000in}{-0.048611in}}{\pgfqpoint{0.000000in}{0.000000in}}{%
\pgfpathmoveto{\pgfqpoint{0.000000in}{0.000000in}}%
\pgfpathlineto{\pgfqpoint{0.000000in}{-0.048611in}}%
\pgfusepath{stroke,fill}%
}%
\begin{pgfscope}%
\pgfsys@transformshift{2.890951in}{0.646140in}%
\pgfsys@useobject{currentmarker}{}%
\end{pgfscope}%
\end{pgfscope}%
\begin{pgfscope}%
\definecolor{textcolor}{rgb}{0.000000,0.000000,0.000000}%
\pgfsetstrokecolor{textcolor}%
\pgfsetfillcolor{textcolor}%
\pgftext[x=2.890951in,y=0.548917in,,top]{\color{textcolor}\rmfamily\fontsize{14.000000}{16.800000}\selectfont \(\displaystyle {2775}\)}%
\end{pgfscope}%
\begin{pgfscope}%
\pgfsetbuttcap%
\pgfsetroundjoin%
\definecolor{currentfill}{rgb}{0.000000,0.000000,0.000000}%
\pgfsetfillcolor{currentfill}%
\pgfsetlinewidth{0.803000pt}%
\definecolor{currentstroke}{rgb}{0.000000,0.000000,0.000000}%
\pgfsetstrokecolor{currentstroke}%
\pgfsetdash{}{0pt}%
\pgfsys@defobject{currentmarker}{\pgfqpoint{0.000000in}{-0.048611in}}{\pgfqpoint{0.000000in}{0.000000in}}{%
\pgfpathmoveto{\pgfqpoint{0.000000in}{0.000000in}}%
\pgfpathlineto{\pgfqpoint{0.000000in}{-0.048611in}}%
\pgfusepath{stroke,fill}%
}%
\begin{pgfscope}%
\pgfsys@transformshift{3.589479in}{0.646140in}%
\pgfsys@useobject{currentmarker}{}%
\end{pgfscope}%
\end{pgfscope}%
\begin{pgfscope}%
\definecolor{textcolor}{rgb}{0.000000,0.000000,0.000000}%
\pgfsetstrokecolor{textcolor}%
\pgfsetfillcolor{textcolor}%
\pgftext[x=3.589479in,y=0.548917in,,top]{\color{textcolor}\rmfamily\fontsize{14.000000}{16.800000}\selectfont \(\displaystyle {2800}\)}%
\end{pgfscope}%
\begin{pgfscope}%
\pgfsetbuttcap%
\pgfsetroundjoin%
\definecolor{currentfill}{rgb}{0.000000,0.000000,0.000000}%
\pgfsetfillcolor{currentfill}%
\pgfsetlinewidth{0.803000pt}%
\definecolor{currentstroke}{rgb}{0.000000,0.000000,0.000000}%
\pgfsetstrokecolor{currentstroke}%
\pgfsetdash{}{0pt}%
\pgfsys@defobject{currentmarker}{\pgfqpoint{0.000000in}{-0.048611in}}{\pgfqpoint{0.000000in}{0.000000in}}{%
\pgfpathmoveto{\pgfqpoint{0.000000in}{0.000000in}}%
\pgfpathlineto{\pgfqpoint{0.000000in}{-0.048611in}}%
\pgfusepath{stroke,fill}%
}%
\begin{pgfscope}%
\pgfsys@transformshift{4.288007in}{0.646140in}%
\pgfsys@useobject{currentmarker}{}%
\end{pgfscope}%
\end{pgfscope}%
\begin{pgfscope}%
\definecolor{textcolor}{rgb}{0.000000,0.000000,0.000000}%
\pgfsetstrokecolor{textcolor}%
\pgfsetfillcolor{textcolor}%
\pgftext[x=4.288007in,y=0.548917in,,top]{\color{textcolor}\rmfamily\fontsize{14.000000}{16.800000}\selectfont \(\displaystyle {2825}\)}%
\end{pgfscope}%
\begin{pgfscope}%
\pgfsetbuttcap%
\pgfsetroundjoin%
\definecolor{currentfill}{rgb}{0.000000,0.000000,0.000000}%
\pgfsetfillcolor{currentfill}%
\pgfsetlinewidth{0.803000pt}%
\definecolor{currentstroke}{rgb}{0.000000,0.000000,0.000000}%
\pgfsetstrokecolor{currentstroke}%
\pgfsetdash{}{0pt}%
\pgfsys@defobject{currentmarker}{\pgfqpoint{0.000000in}{-0.048611in}}{\pgfqpoint{0.000000in}{0.000000in}}{%
\pgfpathmoveto{\pgfqpoint{0.000000in}{0.000000in}}%
\pgfpathlineto{\pgfqpoint{0.000000in}{-0.048611in}}%
\pgfusepath{stroke,fill}%
}%
\begin{pgfscope}%
\pgfsys@transformshift{4.986535in}{0.646140in}%
\pgfsys@useobject{currentmarker}{}%
\end{pgfscope}%
\end{pgfscope}%
\begin{pgfscope}%
\definecolor{textcolor}{rgb}{0.000000,0.000000,0.000000}%
\pgfsetstrokecolor{textcolor}%
\pgfsetfillcolor{textcolor}%
\pgftext[x=4.986535in,y=0.548917in,,top]{\color{textcolor}\rmfamily\fontsize{14.000000}{16.800000}\selectfont \(\displaystyle {2850}\)}%
\end{pgfscope}%
\begin{pgfscope}%
\pgfsetbuttcap%
\pgfsetroundjoin%
\definecolor{currentfill}{rgb}{0.000000,0.000000,0.000000}%
\pgfsetfillcolor{currentfill}%
\pgfsetlinewidth{0.803000pt}%
\definecolor{currentstroke}{rgb}{0.000000,0.000000,0.000000}%
\pgfsetstrokecolor{currentstroke}%
\pgfsetdash{}{0pt}%
\pgfsys@defobject{currentmarker}{\pgfqpoint{0.000000in}{-0.048611in}}{\pgfqpoint{0.000000in}{0.000000in}}{%
\pgfpathmoveto{\pgfqpoint{0.000000in}{0.000000in}}%
\pgfpathlineto{\pgfqpoint{0.000000in}{-0.048611in}}%
\pgfusepath{stroke,fill}%
}%
\begin{pgfscope}%
\pgfsys@transformshift{5.685064in}{0.646140in}%
\pgfsys@useobject{currentmarker}{}%
\end{pgfscope}%
\end{pgfscope}%
\begin{pgfscope}%
\definecolor{textcolor}{rgb}{0.000000,0.000000,0.000000}%
\pgfsetstrokecolor{textcolor}%
\pgfsetfillcolor{textcolor}%
\pgftext[x=5.685064in,y=0.548917in,,top]{\color{textcolor}\rmfamily\fontsize{14.000000}{16.800000}\selectfont \(\displaystyle {2875}\)}%
\end{pgfscope}%
\begin{pgfscope}%
\definecolor{textcolor}{rgb}{0.000000,0.000000,0.000000}%
\pgfsetstrokecolor{textcolor}%
\pgfsetfillcolor{textcolor}%
\pgftext[x=3.310068in,y=0.320695in,,top]{\color{textcolor}\rmfamily\fontsize{14.000000}{16.800000}\selectfont Wavenumber [cm\(\displaystyle ^{-1}\)]}%
\end{pgfscope}%
\begin{pgfscope}%
\pgfsetbuttcap%
\pgfsetroundjoin%
\definecolor{currentfill}{rgb}{0.000000,0.000000,0.000000}%
\pgfsetfillcolor{currentfill}%
\pgfsetlinewidth{0.803000pt}%
\definecolor{currentstroke}{rgb}{0.000000,0.000000,0.000000}%
\pgfsetstrokecolor{currentstroke}%
\pgfsetdash{}{0pt}%
\pgfsys@defobject{currentmarker}{\pgfqpoint{-0.048611in}{0.000000in}}{\pgfqpoint{-0.000000in}{0.000000in}}{%
\pgfpathmoveto{\pgfqpoint{-0.000000in}{0.000000in}}%
\pgfpathlineto{\pgfqpoint{-0.048611in}{0.000000in}}%
\pgfusepath{stroke,fill}%
}%
\begin{pgfscope}%
\pgfsys@transformshift{0.795366in}{0.646140in}%
\pgfsys@useobject{currentmarker}{}%
\end{pgfscope}%
\end{pgfscope}%
\begin{pgfscope}%
\definecolor{textcolor}{rgb}{0.000000,0.000000,0.000000}%
\pgfsetstrokecolor{textcolor}%
\pgfsetfillcolor{textcolor}%
\pgftext[x=0.350000in, y=0.578667in, left, base]{\color{textcolor}\rmfamily\fontsize{14.000000}{16.800000}\selectfont \(\displaystyle {0.40}\)}%
\end{pgfscope}%
\begin{pgfscope}%
\pgfsetbuttcap%
\pgfsetroundjoin%
\definecolor{currentfill}{rgb}{0.000000,0.000000,0.000000}%
\pgfsetfillcolor{currentfill}%
\pgfsetlinewidth{0.803000pt}%
\definecolor{currentstroke}{rgb}{0.000000,0.000000,0.000000}%
\pgfsetstrokecolor{currentstroke}%
\pgfsetdash{}{0pt}%
\pgfsys@defobject{currentmarker}{\pgfqpoint{-0.048611in}{0.000000in}}{\pgfqpoint{-0.000000in}{0.000000in}}{%
\pgfpathmoveto{\pgfqpoint{-0.000000in}{0.000000in}}%
\pgfpathlineto{\pgfqpoint{-0.048611in}{0.000000in}}%
\pgfusepath{stroke,fill}%
}%
\begin{pgfscope}%
\pgfsys@transformshift{0.795366in}{1.160854in}%
\pgfsys@useobject{currentmarker}{}%
\end{pgfscope}%
\end{pgfscope}%
\begin{pgfscope}%
\definecolor{textcolor}{rgb}{0.000000,0.000000,0.000000}%
\pgfsetstrokecolor{textcolor}%
\pgfsetfillcolor{textcolor}%
\pgftext[x=0.350000in, y=1.093382in, left, base]{\color{textcolor}\rmfamily\fontsize{14.000000}{16.800000}\selectfont \(\displaystyle {0.45}\)}%
\end{pgfscope}%
\begin{pgfscope}%
\pgfsetbuttcap%
\pgfsetroundjoin%
\definecolor{currentfill}{rgb}{0.000000,0.000000,0.000000}%
\pgfsetfillcolor{currentfill}%
\pgfsetlinewidth{0.803000pt}%
\definecolor{currentstroke}{rgb}{0.000000,0.000000,0.000000}%
\pgfsetstrokecolor{currentstroke}%
\pgfsetdash{}{0pt}%
\pgfsys@defobject{currentmarker}{\pgfqpoint{-0.048611in}{0.000000in}}{\pgfqpoint{-0.000000in}{0.000000in}}{%
\pgfpathmoveto{\pgfqpoint{-0.000000in}{0.000000in}}%
\pgfpathlineto{\pgfqpoint{-0.048611in}{0.000000in}}%
\pgfusepath{stroke,fill}%
}%
\begin{pgfscope}%
\pgfsys@transformshift{0.795366in}{1.675569in}%
\pgfsys@useobject{currentmarker}{}%
\end{pgfscope}%
\end{pgfscope}%
\begin{pgfscope}%
\definecolor{textcolor}{rgb}{0.000000,0.000000,0.000000}%
\pgfsetstrokecolor{textcolor}%
\pgfsetfillcolor{textcolor}%
\pgftext[x=0.350000in, y=1.608097in, left, base]{\color{textcolor}\rmfamily\fontsize{14.000000}{16.800000}\selectfont \(\displaystyle {0.50}\)}%
\end{pgfscope}%
\begin{pgfscope}%
\pgfsetbuttcap%
\pgfsetroundjoin%
\definecolor{currentfill}{rgb}{0.000000,0.000000,0.000000}%
\pgfsetfillcolor{currentfill}%
\pgfsetlinewidth{0.803000pt}%
\definecolor{currentstroke}{rgb}{0.000000,0.000000,0.000000}%
\pgfsetstrokecolor{currentstroke}%
\pgfsetdash{}{0pt}%
\pgfsys@defobject{currentmarker}{\pgfqpoint{-0.048611in}{0.000000in}}{\pgfqpoint{-0.000000in}{0.000000in}}{%
\pgfpathmoveto{\pgfqpoint{-0.000000in}{0.000000in}}%
\pgfpathlineto{\pgfqpoint{-0.048611in}{0.000000in}}%
\pgfusepath{stroke,fill}%
}%
\begin{pgfscope}%
\pgfsys@transformshift{0.795366in}{2.190284in}%
\pgfsys@useobject{currentmarker}{}%
\end{pgfscope}%
\end{pgfscope}%
\begin{pgfscope}%
\definecolor{textcolor}{rgb}{0.000000,0.000000,0.000000}%
\pgfsetstrokecolor{textcolor}%
\pgfsetfillcolor{textcolor}%
\pgftext[x=0.350000in, y=2.122812in, left, base]{\color{textcolor}\rmfamily\fontsize{14.000000}{16.800000}\selectfont \(\displaystyle {0.55}\)}%
\end{pgfscope}%
\begin{pgfscope}%
\pgfsetbuttcap%
\pgfsetroundjoin%
\definecolor{currentfill}{rgb}{0.000000,0.000000,0.000000}%
\pgfsetfillcolor{currentfill}%
\pgfsetlinewidth{0.803000pt}%
\definecolor{currentstroke}{rgb}{0.000000,0.000000,0.000000}%
\pgfsetstrokecolor{currentstroke}%
\pgfsetdash{}{0pt}%
\pgfsys@defobject{currentmarker}{\pgfqpoint{-0.048611in}{0.000000in}}{\pgfqpoint{-0.000000in}{0.000000in}}{%
\pgfpathmoveto{\pgfqpoint{-0.000000in}{0.000000in}}%
\pgfpathlineto{\pgfqpoint{-0.048611in}{0.000000in}}%
\pgfusepath{stroke,fill}%
}%
\begin{pgfscope}%
\pgfsys@transformshift{0.795366in}{2.704999in}%
\pgfsys@useobject{currentmarker}{}%
\end{pgfscope}%
\end{pgfscope}%
\begin{pgfscope}%
\definecolor{textcolor}{rgb}{0.000000,0.000000,0.000000}%
\pgfsetstrokecolor{textcolor}%
\pgfsetfillcolor{textcolor}%
\pgftext[x=0.350000in, y=2.637526in, left, base]{\color{textcolor}\rmfamily\fontsize{14.000000}{16.800000}\selectfont \(\displaystyle {0.60}\)}%
\end{pgfscope}%
\begin{pgfscope}%
\pgfsetbuttcap%
\pgfsetroundjoin%
\definecolor{currentfill}{rgb}{0.000000,0.000000,0.000000}%
\pgfsetfillcolor{currentfill}%
\pgfsetlinewidth{0.803000pt}%
\definecolor{currentstroke}{rgb}{0.000000,0.000000,0.000000}%
\pgfsetstrokecolor{currentstroke}%
\pgfsetdash{}{0pt}%
\pgfsys@defobject{currentmarker}{\pgfqpoint{-0.048611in}{0.000000in}}{\pgfqpoint{-0.000000in}{0.000000in}}{%
\pgfpathmoveto{\pgfqpoint{-0.000000in}{0.000000in}}%
\pgfpathlineto{\pgfqpoint{-0.048611in}{0.000000in}}%
\pgfusepath{stroke,fill}%
}%
\begin{pgfscope}%
\pgfsys@transformshift{0.795366in}{3.219713in}%
\pgfsys@useobject{currentmarker}{}%
\end{pgfscope}%
\end{pgfscope}%
\begin{pgfscope}%
\definecolor{textcolor}{rgb}{0.000000,0.000000,0.000000}%
\pgfsetstrokecolor{textcolor}%
\pgfsetfillcolor{textcolor}%
\pgftext[x=0.350000in, y=3.152241in, left, base]{\color{textcolor}\rmfamily\fontsize{14.000000}{16.800000}\selectfont \(\displaystyle {0.65}\)}%
\end{pgfscope}%
\begin{pgfscope}%
\pgfsetbuttcap%
\pgfsetroundjoin%
\definecolor{currentfill}{rgb}{0.000000,0.000000,0.000000}%
\pgfsetfillcolor{currentfill}%
\pgfsetlinewidth{0.803000pt}%
\definecolor{currentstroke}{rgb}{0.000000,0.000000,0.000000}%
\pgfsetstrokecolor{currentstroke}%
\pgfsetdash{}{0pt}%
\pgfsys@defobject{currentmarker}{\pgfqpoint{-0.048611in}{0.000000in}}{\pgfqpoint{-0.000000in}{0.000000in}}{%
\pgfpathmoveto{\pgfqpoint{-0.000000in}{0.000000in}}%
\pgfpathlineto{\pgfqpoint{-0.048611in}{0.000000in}}%
\pgfusepath{stroke,fill}%
}%
\begin{pgfscope}%
\pgfsys@transformshift{0.795366in}{3.734428in}%
\pgfsys@useobject{currentmarker}{}%
\end{pgfscope}%
\end{pgfscope}%
\begin{pgfscope}%
\definecolor{textcolor}{rgb}{0.000000,0.000000,0.000000}%
\pgfsetstrokecolor{textcolor}%
\pgfsetfillcolor{textcolor}%
\pgftext[x=0.350000in, y=3.666956in, left, base]{\color{textcolor}\rmfamily\fontsize{14.000000}{16.800000}\selectfont \(\displaystyle {0.70}\)}%
\end{pgfscope}%
\begin{pgfscope}%
\definecolor{textcolor}{rgb}{0.000000,0.000000,0.000000}%
\pgfsetstrokecolor{textcolor}%
\pgfsetfillcolor{textcolor}%
\pgftext[x=0.294444in,y=2.190284in,,bottom,rotate=90.000000]{\color{textcolor}\rmfamily\fontsize{14.000000}{16.800000}\selectfont Transmittance [\(\displaystyle \%\)]}%
\end{pgfscope}%
\begin{pgfscope}%
\pgfpathrectangle{\pgfqpoint{0.795366in}{0.646140in}}{\pgfqpoint{5.029404in}{3.088289in}}%
\pgfusepath{clip}%
\pgfsetrectcap%
\pgfsetroundjoin%
\pgfsetlinewidth{0.501875pt}%
\definecolor{currentstroke}{rgb}{0.000000,0.000000,0.000000}%
\pgfsetstrokecolor{currentstroke}%
\pgfsetdash{}{0pt}%
\pgfpathmoveto{\pgfqpoint{0.795366in}{1.436086in}}%
\pgfpathlineto{\pgfqpoint{0.798858in}{1.428093in}}%
\pgfpathlineto{\pgfqpoint{0.805844in}{1.400164in}}%
\pgfpathlineto{\pgfqpoint{0.809336in}{1.394628in}}%
\pgfpathlineto{\pgfqpoint{0.812829in}{1.396775in}}%
\pgfpathlineto{\pgfqpoint{0.819814in}{1.405312in}}%
\pgfpathlineto{\pgfqpoint{0.823307in}{1.406654in}}%
\pgfpathlineto{\pgfqpoint{0.826799in}{1.406136in}}%
\pgfpathlineto{\pgfqpoint{0.830292in}{1.403586in}}%
\pgfpathlineto{\pgfqpoint{0.837277in}{1.396681in}}%
\pgfpathlineto{\pgfqpoint{0.840770in}{1.397098in}}%
\pgfpathlineto{\pgfqpoint{0.844263in}{1.401315in}}%
\pgfpathlineto{\pgfqpoint{0.851248in}{1.418343in}}%
\pgfpathlineto{\pgfqpoint{0.858233in}{1.435758in}}%
\pgfpathlineto{\pgfqpoint{0.861726in}{1.438102in}}%
\pgfpathlineto{\pgfqpoint{0.868711in}{1.434384in}}%
\pgfpathlineto{\pgfqpoint{0.875696in}{1.428027in}}%
\pgfpathlineto{\pgfqpoint{0.886174in}{1.410160in}}%
\pgfpathlineto{\pgfqpoint{0.889667in}{1.402636in}}%
\pgfpathlineto{\pgfqpoint{0.893160in}{1.390766in}}%
\pgfpathlineto{\pgfqpoint{0.900145in}{1.358010in}}%
\pgfpathlineto{\pgfqpoint{0.903638in}{1.351459in}}%
\pgfpathlineto{\pgfqpoint{0.907130in}{1.361541in}}%
\pgfpathlineto{\pgfqpoint{0.917608in}{1.434011in}}%
\pgfpathlineto{\pgfqpoint{0.921101in}{1.441546in}}%
\pgfpathlineto{\pgfqpoint{0.924593in}{1.442735in}}%
\pgfpathlineto{\pgfqpoint{0.928086in}{1.446510in}}%
\pgfpathlineto{\pgfqpoint{0.935071in}{1.468831in}}%
\pgfpathlineto{\pgfqpoint{0.938564in}{1.468273in}}%
\pgfpathlineto{\pgfqpoint{0.942057in}{1.440455in}}%
\pgfpathlineto{\pgfqpoint{0.952535in}{1.264451in}}%
\pgfpathlineto{\pgfqpoint{0.956027in}{1.260974in}}%
\pgfpathlineto{\pgfqpoint{0.959520in}{1.299585in}}%
\pgfpathlineto{\pgfqpoint{0.966505in}{1.395935in}}%
\pgfpathlineto{\pgfqpoint{0.969998in}{1.416892in}}%
\pgfpathlineto{\pgfqpoint{0.987461in}{1.457834in}}%
\pgfpathlineto{\pgfqpoint{0.994446in}{1.461074in}}%
\pgfpathlineto{\pgfqpoint{0.997939in}{1.466420in}}%
\pgfpathlineto{\pgfqpoint{1.001432in}{1.479450in}}%
\pgfpathlineto{\pgfqpoint{1.008417in}{1.516587in}}%
\pgfpathlineto{\pgfqpoint{1.011909in}{1.529302in}}%
\pgfpathlineto{\pgfqpoint{1.015402in}{1.535804in}}%
\pgfpathlineto{\pgfqpoint{1.018895in}{1.538053in}}%
\pgfpathlineto{\pgfqpoint{1.022387in}{1.536754in}}%
\pgfpathlineto{\pgfqpoint{1.025880in}{1.531147in}}%
\pgfpathlineto{\pgfqpoint{1.036358in}{1.503135in}}%
\pgfpathlineto{\pgfqpoint{1.046836in}{1.489326in}}%
\pgfpathlineto{\pgfqpoint{1.053821in}{1.466264in}}%
\pgfpathlineto{\pgfqpoint{1.057314in}{1.464843in}}%
\pgfpathlineto{\pgfqpoint{1.060806in}{1.477657in}}%
\pgfpathlineto{\pgfqpoint{1.067792in}{1.515732in}}%
\pgfpathlineto{\pgfqpoint{1.071284in}{1.526019in}}%
\pgfpathlineto{\pgfqpoint{1.078270in}{1.537743in}}%
\pgfpathlineto{\pgfqpoint{1.081762in}{1.541261in}}%
\pgfpathlineto{\pgfqpoint{1.085255in}{1.541251in}}%
\pgfpathlineto{\pgfqpoint{1.088748in}{1.539614in}}%
\pgfpathlineto{\pgfqpoint{1.092240in}{1.539606in}}%
\pgfpathlineto{\pgfqpoint{1.099225in}{1.543790in}}%
\pgfpathlineto{\pgfqpoint{1.102718in}{1.542760in}}%
\pgfpathlineto{\pgfqpoint{1.106211in}{1.537160in}}%
\pgfpathlineto{\pgfqpoint{1.113196in}{1.519064in}}%
\pgfpathlineto{\pgfqpoint{1.116689in}{1.514757in}}%
\pgfpathlineto{\pgfqpoint{1.120181in}{1.515692in}}%
\pgfpathlineto{\pgfqpoint{1.130659in}{1.523499in}}%
\pgfpathlineto{\pgfqpoint{1.134152in}{1.530518in}}%
\pgfpathlineto{\pgfqpoint{1.151615in}{1.586924in}}%
\pgfpathlineto{\pgfqpoint{1.158600in}{1.598106in}}%
\pgfpathlineto{\pgfqpoint{1.162093in}{1.599319in}}%
\pgfpathlineto{\pgfqpoint{1.165586in}{1.596951in}}%
\pgfpathlineto{\pgfqpoint{1.172571in}{1.585500in}}%
\pgfpathlineto{\pgfqpoint{1.183049in}{1.565484in}}%
\pgfpathlineto{\pgfqpoint{1.186542in}{1.565319in}}%
\pgfpathlineto{\pgfqpoint{1.190034in}{1.571209in}}%
\pgfpathlineto{\pgfqpoint{1.197019in}{1.588563in}}%
\pgfpathlineto{\pgfqpoint{1.200512in}{1.591843in}}%
\pgfpathlineto{\pgfqpoint{1.204005in}{1.590067in}}%
\pgfpathlineto{\pgfqpoint{1.207497in}{1.586646in}}%
\pgfpathlineto{\pgfqpoint{1.210990in}{1.585510in}}%
\pgfpathlineto{\pgfqpoint{1.214483in}{1.586295in}}%
\pgfpathlineto{\pgfqpoint{1.217975in}{1.584711in}}%
\pgfpathlineto{\pgfqpoint{1.221468in}{1.579757in}}%
\pgfpathlineto{\pgfqpoint{1.224961in}{1.577899in}}%
\pgfpathlineto{\pgfqpoint{1.228453in}{1.585655in}}%
\pgfpathlineto{\pgfqpoint{1.235439in}{1.609223in}}%
\pgfpathlineto{\pgfqpoint{1.238931in}{1.610650in}}%
\pgfpathlineto{\pgfqpoint{1.242424in}{1.610086in}}%
\pgfpathlineto{\pgfqpoint{1.249409in}{1.616191in}}%
\pgfpathlineto{\pgfqpoint{1.252902in}{1.613891in}}%
\pgfpathlineto{\pgfqpoint{1.259887in}{1.602709in}}%
\pgfpathlineto{\pgfqpoint{1.266872in}{1.594781in}}%
\pgfpathlineto{\pgfqpoint{1.270365in}{1.590241in}}%
\pgfpathlineto{\pgfqpoint{1.273858in}{1.591593in}}%
\pgfpathlineto{\pgfqpoint{1.287828in}{1.623876in}}%
\pgfpathlineto{\pgfqpoint{1.294813in}{1.635307in}}%
\pgfpathlineto{\pgfqpoint{1.301799in}{1.652675in}}%
\pgfpathlineto{\pgfqpoint{1.308784in}{1.672423in}}%
\pgfpathlineto{\pgfqpoint{1.312277in}{1.676285in}}%
\pgfpathlineto{\pgfqpoint{1.315769in}{1.672487in}}%
\pgfpathlineto{\pgfqpoint{1.326247in}{1.639705in}}%
\pgfpathlineto{\pgfqpoint{1.329740in}{1.642379in}}%
\pgfpathlineto{\pgfqpoint{1.333232in}{1.650399in}}%
\pgfpathlineto{\pgfqpoint{1.336725in}{1.652448in}}%
\pgfpathlineto{\pgfqpoint{1.340218in}{1.643167in}}%
\pgfpathlineto{\pgfqpoint{1.343710in}{1.628882in}}%
\pgfpathlineto{\pgfqpoint{1.347203in}{1.620365in}}%
\pgfpathlineto{\pgfqpoint{1.350696in}{1.622885in}}%
\pgfpathlineto{\pgfqpoint{1.361174in}{1.658194in}}%
\pgfpathlineto{\pgfqpoint{1.364666in}{1.661623in}}%
\pgfpathlineto{\pgfqpoint{1.368159in}{1.661468in}}%
\pgfpathlineto{\pgfqpoint{1.371652in}{1.664648in}}%
\pgfpathlineto{\pgfqpoint{1.382129in}{1.693627in}}%
\pgfpathlineto{\pgfqpoint{1.389115in}{1.703856in}}%
\pgfpathlineto{\pgfqpoint{1.392607in}{1.705151in}}%
\pgfpathlineto{\pgfqpoint{1.396100in}{1.697446in}}%
\pgfpathlineto{\pgfqpoint{1.403085in}{1.670685in}}%
\pgfpathlineto{\pgfqpoint{1.406578in}{1.670669in}}%
\pgfpathlineto{\pgfqpoint{1.417056in}{1.700799in}}%
\pgfpathlineto{\pgfqpoint{1.420549in}{1.699600in}}%
\pgfpathlineto{\pgfqpoint{1.424041in}{1.689760in}}%
\pgfpathlineto{\pgfqpoint{1.431026in}{1.660867in}}%
\pgfpathlineto{\pgfqpoint{1.434519in}{1.657668in}}%
\pgfpathlineto{\pgfqpoint{1.438012in}{1.667684in}}%
\pgfpathlineto{\pgfqpoint{1.444997in}{1.702678in}}%
\pgfpathlineto{\pgfqpoint{1.448490in}{1.714106in}}%
\pgfpathlineto{\pgfqpoint{1.458968in}{1.735742in}}%
\pgfpathlineto{\pgfqpoint{1.462460in}{1.739441in}}%
\pgfpathlineto{\pgfqpoint{1.476431in}{1.735575in}}%
\pgfpathlineto{\pgfqpoint{1.479923in}{1.725260in}}%
\pgfpathlineto{\pgfqpoint{1.486909in}{1.692938in}}%
\pgfpathlineto{\pgfqpoint{1.490401in}{1.691125in}}%
\pgfpathlineto{\pgfqpoint{1.497387in}{1.706127in}}%
\pgfpathlineto{\pgfqpoint{1.500879in}{1.706709in}}%
\pgfpathlineto{\pgfqpoint{1.504372in}{1.704826in}}%
\pgfpathlineto{\pgfqpoint{1.507865in}{1.705805in}}%
\pgfpathlineto{\pgfqpoint{1.514850in}{1.714644in}}%
\pgfpathlineto{\pgfqpoint{1.518342in}{1.722092in}}%
\pgfpathlineto{\pgfqpoint{1.528820in}{1.759054in}}%
\pgfpathlineto{\pgfqpoint{1.532313in}{1.764128in}}%
\pgfpathlineto{\pgfqpoint{1.535806in}{1.765043in}}%
\pgfpathlineto{\pgfqpoint{1.542791in}{1.763068in}}%
\pgfpathlineto{\pgfqpoint{1.546284in}{1.763979in}}%
\pgfpathlineto{\pgfqpoint{1.556762in}{1.777859in}}%
\pgfpathlineto{\pgfqpoint{1.560254in}{1.772979in}}%
\pgfpathlineto{\pgfqpoint{1.567239in}{1.746940in}}%
\pgfpathlineto{\pgfqpoint{1.570732in}{1.734256in}}%
\pgfpathlineto{\pgfqpoint{1.574225in}{1.727179in}}%
\pgfpathlineto{\pgfqpoint{1.577717in}{1.725661in}}%
\pgfpathlineto{\pgfqpoint{1.581210in}{1.721649in}}%
\pgfpathlineto{\pgfqpoint{1.584703in}{1.702815in}}%
\pgfpathlineto{\pgfqpoint{1.591688in}{1.629008in}}%
\pgfpathlineto{\pgfqpoint{1.595181in}{1.618300in}}%
\pgfpathlineto{\pgfqpoint{1.598673in}{1.649293in}}%
\pgfpathlineto{\pgfqpoint{1.605659in}{1.767041in}}%
\pgfpathlineto{\pgfqpoint{1.609151in}{1.798204in}}%
\pgfpathlineto{\pgfqpoint{1.612644in}{1.802203in}}%
\pgfpathlineto{\pgfqpoint{1.616136in}{1.795619in}}%
\pgfpathlineto{\pgfqpoint{1.619629in}{1.792455in}}%
\pgfpathlineto{\pgfqpoint{1.623122in}{1.793604in}}%
\pgfpathlineto{\pgfqpoint{1.626614in}{1.791317in}}%
\pgfpathlineto{\pgfqpoint{1.630107in}{1.774686in}}%
\pgfpathlineto{\pgfqpoint{1.633600in}{1.728768in}}%
\pgfpathlineto{\pgfqpoint{1.637092in}{1.641102in}}%
\pgfpathlineto{\pgfqpoint{1.644078in}{1.424782in}}%
\pgfpathlineto{\pgfqpoint{1.647570in}{1.403198in}}%
\pgfpathlineto{\pgfqpoint{1.651063in}{1.476118in}}%
\pgfpathlineto{\pgfqpoint{1.658048in}{1.712679in}}%
\pgfpathlineto{\pgfqpoint{1.661541in}{1.777063in}}%
\pgfpathlineto{\pgfqpoint{1.665033in}{1.802584in}}%
\pgfpathlineto{\pgfqpoint{1.672019in}{1.819801in}}%
\pgfpathlineto{\pgfqpoint{1.682497in}{1.839545in}}%
\pgfpathlineto{\pgfqpoint{1.685989in}{1.838687in}}%
\pgfpathlineto{\pgfqpoint{1.692975in}{1.821618in}}%
\pgfpathlineto{\pgfqpoint{1.696467in}{1.818186in}}%
\pgfpathlineto{\pgfqpoint{1.703452in}{1.820797in}}%
\pgfpathlineto{\pgfqpoint{1.706945in}{1.817719in}}%
\pgfpathlineto{\pgfqpoint{1.713930in}{1.808671in}}%
\pgfpathlineto{\pgfqpoint{1.717423in}{1.809690in}}%
\pgfpathlineto{\pgfqpoint{1.724408in}{1.822364in}}%
\pgfpathlineto{\pgfqpoint{1.727901in}{1.822784in}}%
\pgfpathlineto{\pgfqpoint{1.731394in}{1.813621in}}%
\pgfpathlineto{\pgfqpoint{1.734886in}{1.799999in}}%
\pgfpathlineto{\pgfqpoint{1.738379in}{1.791756in}}%
\pgfpathlineto{\pgfqpoint{1.741872in}{1.794765in}}%
\pgfpathlineto{\pgfqpoint{1.748857in}{1.822311in}}%
\pgfpathlineto{\pgfqpoint{1.755842in}{1.848706in}}%
\pgfpathlineto{\pgfqpoint{1.762827in}{1.868543in}}%
\pgfpathlineto{\pgfqpoint{1.766320in}{1.873085in}}%
\pgfpathlineto{\pgfqpoint{1.769813in}{1.873524in}}%
\pgfpathlineto{\pgfqpoint{1.780291in}{1.869320in}}%
\pgfpathlineto{\pgfqpoint{1.790768in}{1.854739in}}%
\pgfpathlineto{\pgfqpoint{1.797754in}{1.854765in}}%
\pgfpathlineto{\pgfqpoint{1.804739in}{1.851721in}}%
\pgfpathlineto{\pgfqpoint{1.811724in}{1.853887in}}%
\pgfpathlineto{\pgfqpoint{1.815217in}{1.855994in}}%
\pgfpathlineto{\pgfqpoint{1.818710in}{1.860462in}}%
\pgfpathlineto{\pgfqpoint{1.839665in}{1.906259in}}%
\pgfpathlineto{\pgfqpoint{1.843158in}{1.908710in}}%
\pgfpathlineto{\pgfqpoint{1.846651in}{1.905212in}}%
\pgfpathlineto{\pgfqpoint{1.853636in}{1.892499in}}%
\pgfpathlineto{\pgfqpoint{1.857129in}{1.888685in}}%
\pgfpathlineto{\pgfqpoint{1.860621in}{1.887063in}}%
\pgfpathlineto{\pgfqpoint{1.867607in}{1.887565in}}%
\pgfpathlineto{\pgfqpoint{1.871099in}{1.890432in}}%
\pgfpathlineto{\pgfqpoint{1.878085in}{1.901304in}}%
\pgfpathlineto{\pgfqpoint{1.881577in}{1.903211in}}%
\pgfpathlineto{\pgfqpoint{1.888562in}{1.901128in}}%
\pgfpathlineto{\pgfqpoint{1.892055in}{1.904822in}}%
\pgfpathlineto{\pgfqpoint{1.899040in}{1.915510in}}%
\pgfpathlineto{\pgfqpoint{1.906026in}{1.921902in}}%
\pgfpathlineto{\pgfqpoint{1.909518in}{1.923852in}}%
\pgfpathlineto{\pgfqpoint{1.913011in}{1.922949in}}%
\pgfpathlineto{\pgfqpoint{1.919996in}{1.918438in}}%
\pgfpathlineto{\pgfqpoint{1.923489in}{1.916970in}}%
\pgfpathlineto{\pgfqpoint{1.926982in}{1.913662in}}%
\pgfpathlineto{\pgfqpoint{1.933967in}{1.904415in}}%
\pgfpathlineto{\pgfqpoint{1.937459in}{1.902563in}}%
\pgfpathlineto{\pgfqpoint{1.940952in}{1.902501in}}%
\pgfpathlineto{\pgfqpoint{1.944445in}{1.904067in}}%
\pgfpathlineto{\pgfqpoint{1.958415in}{1.917579in}}%
\pgfpathlineto{\pgfqpoint{1.961908in}{1.925782in}}%
\pgfpathlineto{\pgfqpoint{1.968893in}{1.951885in}}%
\pgfpathlineto{\pgfqpoint{1.972386in}{1.959425in}}%
\pgfpathlineto{\pgfqpoint{1.979371in}{1.964472in}}%
\pgfpathlineto{\pgfqpoint{1.982864in}{1.969463in}}%
\pgfpathlineto{\pgfqpoint{1.989849in}{1.984396in}}%
\pgfpathlineto{\pgfqpoint{1.993342in}{1.987280in}}%
\pgfpathlineto{\pgfqpoint{1.996834in}{1.983373in}}%
\pgfpathlineto{\pgfqpoint{2.014298in}{1.939908in}}%
\pgfpathlineto{\pgfqpoint{2.017790in}{1.929573in}}%
\pgfpathlineto{\pgfqpoint{2.021283in}{1.927758in}}%
\pgfpathlineto{\pgfqpoint{2.031761in}{1.947324in}}%
\pgfpathlineto{\pgfqpoint{2.035253in}{1.950937in}}%
\pgfpathlineto{\pgfqpoint{2.038746in}{1.958007in}}%
\pgfpathlineto{\pgfqpoint{2.045731in}{1.980951in}}%
\pgfpathlineto{\pgfqpoint{2.049224in}{1.986405in}}%
\pgfpathlineto{\pgfqpoint{2.052717in}{1.984153in}}%
\pgfpathlineto{\pgfqpoint{2.056209in}{1.978585in}}%
\pgfpathlineto{\pgfqpoint{2.059702in}{1.976531in}}%
\pgfpathlineto{\pgfqpoint{2.063195in}{1.980951in}}%
\pgfpathlineto{\pgfqpoint{2.066687in}{1.988333in}}%
\pgfpathlineto{\pgfqpoint{2.070180in}{1.992504in}}%
\pgfpathlineto{\pgfqpoint{2.073672in}{1.990472in}}%
\pgfpathlineto{\pgfqpoint{2.087643in}{1.966706in}}%
\pgfpathlineto{\pgfqpoint{2.094628in}{1.955851in}}%
\pgfpathlineto{\pgfqpoint{2.098121in}{1.953378in}}%
\pgfpathlineto{\pgfqpoint{2.101614in}{1.958659in}}%
\pgfpathlineto{\pgfqpoint{2.108599in}{1.992003in}}%
\pgfpathlineto{\pgfqpoint{2.112092in}{2.007503in}}%
\pgfpathlineto{\pgfqpoint{2.115584in}{2.014233in}}%
\pgfpathlineto{\pgfqpoint{2.119077in}{2.012711in}}%
\pgfpathlineto{\pgfqpoint{2.126062in}{2.003862in}}%
\pgfpathlineto{\pgfqpoint{2.129555in}{2.004199in}}%
\pgfpathlineto{\pgfqpoint{2.133047in}{2.008493in}}%
\pgfpathlineto{\pgfqpoint{2.140033in}{2.020264in}}%
\pgfpathlineto{\pgfqpoint{2.143525in}{2.023137in}}%
\pgfpathlineto{\pgfqpoint{2.147018in}{2.021884in}}%
\pgfpathlineto{\pgfqpoint{2.150511in}{2.015864in}}%
\pgfpathlineto{\pgfqpoint{2.160988in}{1.986246in}}%
\pgfpathlineto{\pgfqpoint{2.164481in}{1.981905in}}%
\pgfpathlineto{\pgfqpoint{2.167974in}{1.983538in}}%
\pgfpathlineto{\pgfqpoint{2.171466in}{1.991921in}}%
\pgfpathlineto{\pgfqpoint{2.178452in}{2.019720in}}%
\pgfpathlineto{\pgfqpoint{2.181944in}{2.028214in}}%
\pgfpathlineto{\pgfqpoint{2.185437in}{2.029036in}}%
\pgfpathlineto{\pgfqpoint{2.192422in}{2.023692in}}%
\pgfpathlineto{\pgfqpoint{2.195915in}{2.025415in}}%
\pgfpathlineto{\pgfqpoint{2.199408in}{2.031157in}}%
\pgfpathlineto{\pgfqpoint{2.202900in}{2.041323in}}%
\pgfpathlineto{\pgfqpoint{2.209885in}{2.068866in}}%
\pgfpathlineto{\pgfqpoint{2.213378in}{2.074101in}}%
\pgfpathlineto{\pgfqpoint{2.216871in}{2.068781in}}%
\pgfpathlineto{\pgfqpoint{2.220363in}{2.059378in}}%
\pgfpathlineto{\pgfqpoint{2.223856in}{2.054242in}}%
\pgfpathlineto{\pgfqpoint{2.230841in}{2.053093in}}%
\pgfpathlineto{\pgfqpoint{2.244812in}{2.029991in}}%
\pgfpathlineto{\pgfqpoint{2.255290in}{2.006335in}}%
\pgfpathlineto{\pgfqpoint{2.258782in}{1.984006in}}%
\pgfpathlineto{\pgfqpoint{2.262275in}{1.933474in}}%
\pgfpathlineto{\pgfqpoint{2.269260in}{1.801135in}}%
\pgfpathlineto{\pgfqpoint{2.272753in}{1.792400in}}%
\pgfpathlineto{\pgfqpoint{2.276246in}{1.845527in}}%
\pgfpathlineto{\pgfqpoint{2.283231in}{2.013334in}}%
\pgfpathlineto{\pgfqpoint{2.286724in}{2.060233in}}%
\pgfpathlineto{\pgfqpoint{2.290216in}{2.076762in}}%
\pgfpathlineto{\pgfqpoint{2.293709in}{2.079224in}}%
\pgfpathlineto{\pgfqpoint{2.297201in}{2.077690in}}%
\pgfpathlineto{\pgfqpoint{2.300694in}{2.072664in}}%
\pgfpathlineto{\pgfqpoint{2.304187in}{2.063884in}}%
\pgfpathlineto{\pgfqpoint{2.307679in}{2.048323in}}%
\pgfpathlineto{\pgfqpoint{2.311172in}{2.005281in}}%
\pgfpathlineto{\pgfqpoint{2.314665in}{1.900119in}}%
\pgfpathlineto{\pgfqpoint{2.321650in}{1.542913in}}%
\pgfpathlineto{\pgfqpoint{2.325143in}{1.452389in}}%
\pgfpathlineto{\pgfqpoint{2.328635in}{1.512848in}}%
\pgfpathlineto{\pgfqpoint{2.339113in}{1.996567in}}%
\pgfpathlineto{\pgfqpoint{2.342606in}{2.057097in}}%
\pgfpathlineto{\pgfqpoint{2.346098in}{2.081878in}}%
\pgfpathlineto{\pgfqpoint{2.349591in}{2.091215in}}%
\pgfpathlineto{\pgfqpoint{2.353084in}{2.089615in}}%
\pgfpathlineto{\pgfqpoint{2.356576in}{2.082748in}}%
\pgfpathlineto{\pgfqpoint{2.360069in}{2.080274in}}%
\pgfpathlineto{\pgfqpoint{2.367054in}{2.086443in}}%
\pgfpathlineto{\pgfqpoint{2.370547in}{2.081364in}}%
\pgfpathlineto{\pgfqpoint{2.374040in}{2.072432in}}%
\pgfpathlineto{\pgfqpoint{2.377532in}{2.067597in}}%
\pgfpathlineto{\pgfqpoint{2.381025in}{2.071038in}}%
\pgfpathlineto{\pgfqpoint{2.388010in}{2.086434in}}%
\pgfpathlineto{\pgfqpoint{2.391503in}{2.082663in}}%
\pgfpathlineto{\pgfqpoint{2.398488in}{2.054326in}}%
\pgfpathlineto{\pgfqpoint{2.401981in}{2.049403in}}%
\pgfpathlineto{\pgfqpoint{2.405473in}{2.055143in}}%
\pgfpathlineto{\pgfqpoint{2.415951in}{2.088525in}}%
\pgfpathlineto{\pgfqpoint{2.433415in}{2.167250in}}%
\pgfpathlineto{\pgfqpoint{2.436907in}{2.170272in}}%
\pgfpathlineto{\pgfqpoint{2.440400in}{2.163910in}}%
\pgfpathlineto{\pgfqpoint{2.454370in}{2.110709in}}%
\pgfpathlineto{\pgfqpoint{2.461356in}{2.109199in}}%
\pgfpathlineto{\pgfqpoint{2.468341in}{2.098635in}}%
\pgfpathlineto{\pgfqpoint{2.471834in}{2.097754in}}%
\pgfpathlineto{\pgfqpoint{2.478819in}{2.112175in}}%
\pgfpathlineto{\pgfqpoint{2.482311in}{2.113889in}}%
\pgfpathlineto{\pgfqpoint{2.485804in}{2.110796in}}%
\pgfpathlineto{\pgfqpoint{2.489297in}{2.111257in}}%
\pgfpathlineto{\pgfqpoint{2.492789in}{2.120545in}}%
\pgfpathlineto{\pgfqpoint{2.499775in}{2.145049in}}%
\pgfpathlineto{\pgfqpoint{2.503267in}{2.147069in}}%
\pgfpathlineto{\pgfqpoint{2.510253in}{2.141625in}}%
\pgfpathlineto{\pgfqpoint{2.517238in}{2.141150in}}%
\pgfpathlineto{\pgfqpoint{2.520731in}{2.136071in}}%
\pgfpathlineto{\pgfqpoint{2.527716in}{2.122413in}}%
\pgfpathlineto{\pgfqpoint{2.534701in}{2.113152in}}%
\pgfpathlineto{\pgfqpoint{2.541686in}{2.098713in}}%
\pgfpathlineto{\pgfqpoint{2.545179in}{2.098693in}}%
\pgfpathlineto{\pgfqpoint{2.552164in}{2.107132in}}%
\pgfpathlineto{\pgfqpoint{2.559150in}{2.104349in}}%
\pgfpathlineto{\pgfqpoint{2.562642in}{2.109134in}}%
\pgfpathlineto{\pgfqpoint{2.573120in}{2.145906in}}%
\pgfpathlineto{\pgfqpoint{2.587091in}{2.168072in}}%
\pgfpathlineto{\pgfqpoint{2.590583in}{2.167831in}}%
\pgfpathlineto{\pgfqpoint{2.594076in}{2.161638in}}%
\pgfpathlineto{\pgfqpoint{2.608047in}{2.121306in}}%
\pgfpathlineto{\pgfqpoint{2.611539in}{2.119192in}}%
\pgfpathlineto{\pgfqpoint{2.618525in}{2.124906in}}%
\pgfpathlineto{\pgfqpoint{2.622017in}{2.125270in}}%
\pgfpathlineto{\pgfqpoint{2.629002in}{2.120226in}}%
\pgfpathlineto{\pgfqpoint{2.632495in}{2.122328in}}%
\pgfpathlineto{\pgfqpoint{2.635988in}{2.131636in}}%
\pgfpathlineto{\pgfqpoint{2.646466in}{2.171645in}}%
\pgfpathlineto{\pgfqpoint{2.649958in}{2.174579in}}%
\pgfpathlineto{\pgfqpoint{2.653451in}{2.172456in}}%
\pgfpathlineto{\pgfqpoint{2.656944in}{2.168914in}}%
\pgfpathlineto{\pgfqpoint{2.660436in}{2.167916in}}%
\pgfpathlineto{\pgfqpoint{2.663929in}{2.171808in}}%
\pgfpathlineto{\pgfqpoint{2.670914in}{2.183424in}}%
\pgfpathlineto{\pgfqpoint{2.677899in}{2.185335in}}%
\pgfpathlineto{\pgfqpoint{2.681392in}{2.188806in}}%
\pgfpathlineto{\pgfqpoint{2.684885in}{2.189488in}}%
\pgfpathlineto{\pgfqpoint{2.688377in}{2.179910in}}%
\pgfpathlineto{\pgfqpoint{2.695363in}{2.146043in}}%
\pgfpathlineto{\pgfqpoint{2.698855in}{2.140506in}}%
\pgfpathlineto{\pgfqpoint{2.702348in}{2.143635in}}%
\pgfpathlineto{\pgfqpoint{2.705841in}{2.150678in}}%
\pgfpathlineto{\pgfqpoint{2.712826in}{2.173353in}}%
\pgfpathlineto{\pgfqpoint{2.719811in}{2.194682in}}%
\pgfpathlineto{\pgfqpoint{2.733782in}{2.219795in}}%
\pgfpathlineto{\pgfqpoint{2.737274in}{2.215279in}}%
\pgfpathlineto{\pgfqpoint{2.744260in}{2.197290in}}%
\pgfpathlineto{\pgfqpoint{2.747752in}{2.191983in}}%
\pgfpathlineto{\pgfqpoint{2.751245in}{2.190634in}}%
\pgfpathlineto{\pgfqpoint{2.758230in}{2.191271in}}%
\pgfpathlineto{\pgfqpoint{2.761723in}{2.188355in}}%
\pgfpathlineto{\pgfqpoint{2.768708in}{2.175996in}}%
\pgfpathlineto{\pgfqpoint{2.775693in}{2.162930in}}%
\pgfpathlineto{\pgfqpoint{2.779186in}{2.160578in}}%
\pgfpathlineto{\pgfqpoint{2.782679in}{2.163493in}}%
\pgfpathlineto{\pgfqpoint{2.793157in}{2.183231in}}%
\pgfpathlineto{\pgfqpoint{2.796649in}{2.183162in}}%
\pgfpathlineto{\pgfqpoint{2.800142in}{2.180159in}}%
\pgfpathlineto{\pgfqpoint{2.803635in}{2.179508in}}%
\pgfpathlineto{\pgfqpoint{2.807127in}{2.185943in}}%
\pgfpathlineto{\pgfqpoint{2.814112in}{2.211778in}}%
\pgfpathlineto{\pgfqpoint{2.817605in}{2.219017in}}%
\pgfpathlineto{\pgfqpoint{2.821098in}{2.218927in}}%
\pgfpathlineto{\pgfqpoint{2.835068in}{2.199477in}}%
\pgfpathlineto{\pgfqpoint{2.842054in}{2.196220in}}%
\pgfpathlineto{\pgfqpoint{2.849039in}{2.189663in}}%
\pgfpathlineto{\pgfqpoint{2.856024in}{2.186135in}}%
\pgfpathlineto{\pgfqpoint{2.863009in}{2.179928in}}%
\pgfpathlineto{\pgfqpoint{2.866502in}{2.179155in}}%
\pgfpathlineto{\pgfqpoint{2.869995in}{2.182328in}}%
\pgfpathlineto{\pgfqpoint{2.873487in}{2.191343in}}%
\pgfpathlineto{\pgfqpoint{2.880473in}{2.227889in}}%
\pgfpathlineto{\pgfqpoint{2.883965in}{2.246623in}}%
\pgfpathlineto{\pgfqpoint{2.887458in}{2.257723in}}%
\pgfpathlineto{\pgfqpoint{2.890951in}{2.258554in}}%
\pgfpathlineto{\pgfqpoint{2.894443in}{2.249637in}}%
\pgfpathlineto{\pgfqpoint{2.901428in}{2.222463in}}%
\pgfpathlineto{\pgfqpoint{2.904921in}{2.217074in}}%
\pgfpathlineto{\pgfqpoint{2.908414in}{2.216513in}}%
\pgfpathlineto{\pgfqpoint{2.911906in}{2.213555in}}%
\pgfpathlineto{\pgfqpoint{2.915399in}{2.202930in}}%
\pgfpathlineto{\pgfqpoint{2.918892in}{2.178354in}}%
\pgfpathlineto{\pgfqpoint{2.922384in}{2.125787in}}%
\pgfpathlineto{\pgfqpoint{2.932862in}{1.851615in}}%
\pgfpathlineto{\pgfqpoint{2.936355in}{1.860258in}}%
\pgfpathlineto{\pgfqpoint{2.939848in}{1.950535in}}%
\pgfpathlineto{\pgfqpoint{2.946833in}{2.169500in}}%
\pgfpathlineto{\pgfqpoint{2.950325in}{2.221460in}}%
\pgfpathlineto{\pgfqpoint{2.953818in}{2.240779in}}%
\pgfpathlineto{\pgfqpoint{2.957311in}{2.248370in}}%
\pgfpathlineto{\pgfqpoint{2.960803in}{2.252992in}}%
\pgfpathlineto{\pgfqpoint{2.964296in}{2.251851in}}%
\pgfpathlineto{\pgfqpoint{2.967789in}{2.240319in}}%
\pgfpathlineto{\pgfqpoint{2.971281in}{2.209328in}}%
\pgfpathlineto{\pgfqpoint{2.974774in}{2.131147in}}%
\pgfpathlineto{\pgfqpoint{2.978267in}{1.967072in}}%
\pgfpathlineto{\pgfqpoint{2.985252in}{1.467139in}}%
\pgfpathlineto{\pgfqpoint{2.988744in}{1.351365in}}%
\pgfpathlineto{\pgfqpoint{2.992237in}{1.444068in}}%
\pgfpathlineto{\pgfqpoint{3.002715in}{2.097549in}}%
\pgfpathlineto{\pgfqpoint{3.006208in}{2.165688in}}%
\pgfpathlineto{\pgfqpoint{3.013193in}{2.210142in}}%
\pgfpathlineto{\pgfqpoint{3.020178in}{2.239724in}}%
\pgfpathlineto{\pgfqpoint{3.023671in}{2.246219in}}%
\pgfpathlineto{\pgfqpoint{3.027164in}{2.244913in}}%
\pgfpathlineto{\pgfqpoint{3.030656in}{2.238097in}}%
\pgfpathlineto{\pgfqpoint{3.034149in}{2.235568in}}%
\pgfpathlineto{\pgfqpoint{3.037641in}{2.244124in}}%
\pgfpathlineto{\pgfqpoint{3.041134in}{2.257744in}}%
\pgfpathlineto{\pgfqpoint{3.044627in}{2.265993in}}%
\pgfpathlineto{\pgfqpoint{3.048119in}{2.267556in}}%
\pgfpathlineto{\pgfqpoint{3.051612in}{2.267270in}}%
\pgfpathlineto{\pgfqpoint{3.055105in}{2.264733in}}%
\pgfpathlineto{\pgfqpoint{3.058597in}{2.255377in}}%
\pgfpathlineto{\pgfqpoint{3.065583in}{2.228004in}}%
\pgfpathlineto{\pgfqpoint{3.069075in}{2.223033in}}%
\pgfpathlineto{\pgfqpoint{3.072568in}{2.224432in}}%
\pgfpathlineto{\pgfqpoint{3.083046in}{2.237222in}}%
\pgfpathlineto{\pgfqpoint{3.090031in}{2.240026in}}%
\pgfpathlineto{\pgfqpoint{3.107494in}{2.261306in}}%
\pgfpathlineto{\pgfqpoint{3.114480in}{2.276032in}}%
\pgfpathlineto{\pgfqpoint{3.121465in}{2.285197in}}%
\pgfpathlineto{\pgfqpoint{3.124958in}{2.284528in}}%
\pgfpathlineto{\pgfqpoint{3.135435in}{2.266586in}}%
\pgfpathlineto{\pgfqpoint{3.138928in}{2.264052in}}%
\pgfpathlineto{\pgfqpoint{3.142421in}{2.258744in}}%
\pgfpathlineto{\pgfqpoint{3.149406in}{2.240566in}}%
\pgfpathlineto{\pgfqpoint{3.152899in}{2.235206in}}%
\pgfpathlineto{\pgfqpoint{3.156391in}{2.235760in}}%
\pgfpathlineto{\pgfqpoint{3.159884in}{2.243739in}}%
\pgfpathlineto{\pgfqpoint{3.170362in}{2.282929in}}%
\pgfpathlineto{\pgfqpoint{3.173854in}{2.285282in}}%
\pgfpathlineto{\pgfqpoint{3.177347in}{2.284787in}}%
\pgfpathlineto{\pgfqpoint{3.180840in}{2.285851in}}%
\pgfpathlineto{\pgfqpoint{3.184332in}{2.289882in}}%
\pgfpathlineto{\pgfqpoint{3.191318in}{2.301883in}}%
\pgfpathlineto{\pgfqpoint{3.194810in}{2.303652in}}%
\pgfpathlineto{\pgfqpoint{3.198303in}{2.299104in}}%
\pgfpathlineto{\pgfqpoint{3.208781in}{2.269273in}}%
\pgfpathlineto{\pgfqpoint{3.219259in}{2.240877in}}%
\pgfpathlineto{\pgfqpoint{3.222751in}{2.237081in}}%
\pgfpathlineto{\pgfqpoint{3.226244in}{2.237490in}}%
\pgfpathlineto{\pgfqpoint{3.233229in}{2.243125in}}%
\pgfpathlineto{\pgfqpoint{3.236722in}{2.244678in}}%
\pgfpathlineto{\pgfqpoint{3.240215in}{2.248745in}}%
\pgfpathlineto{\pgfqpoint{3.243707in}{2.259556in}}%
\pgfpathlineto{\pgfqpoint{3.247200in}{2.274238in}}%
\pgfpathlineto{\pgfqpoint{3.250693in}{2.283389in}}%
\pgfpathlineto{\pgfqpoint{3.254185in}{2.282280in}}%
\pgfpathlineto{\pgfqpoint{3.257678in}{2.277856in}}%
\pgfpathlineto{\pgfqpoint{3.261171in}{2.280954in}}%
\pgfpathlineto{\pgfqpoint{3.271648in}{2.324876in}}%
\pgfpathlineto{\pgfqpoint{3.275141in}{2.328557in}}%
\pgfpathlineto{\pgfqpoint{3.278634in}{2.320708in}}%
\pgfpathlineto{\pgfqpoint{3.289112in}{2.273628in}}%
\pgfpathlineto{\pgfqpoint{3.292604in}{2.267601in}}%
\pgfpathlineto{\pgfqpoint{3.296097in}{2.264752in}}%
\pgfpathlineto{\pgfqpoint{3.299590in}{2.263737in}}%
\pgfpathlineto{\pgfqpoint{3.303082in}{2.265150in}}%
\pgfpathlineto{\pgfqpoint{3.310068in}{2.271162in}}%
\pgfpathlineto{\pgfqpoint{3.313560in}{2.275654in}}%
\pgfpathlineto{\pgfqpoint{3.317053in}{2.284198in}}%
\pgfpathlineto{\pgfqpoint{3.327531in}{2.316168in}}%
\pgfpathlineto{\pgfqpoint{3.334516in}{2.326402in}}%
\pgfpathlineto{\pgfqpoint{3.341501in}{2.330890in}}%
\pgfpathlineto{\pgfqpoint{3.344994in}{2.330478in}}%
\pgfpathlineto{\pgfqpoint{3.348487in}{2.325827in}}%
\pgfpathlineto{\pgfqpoint{3.355472in}{2.312021in}}%
\pgfpathlineto{\pgfqpoint{3.358964in}{2.313602in}}%
\pgfpathlineto{\pgfqpoint{3.362457in}{2.319364in}}%
\pgfpathlineto{\pgfqpoint{3.365950in}{2.320666in}}%
\pgfpathlineto{\pgfqpoint{3.369442in}{2.313035in}}%
\pgfpathlineto{\pgfqpoint{3.376428in}{2.290835in}}%
\pgfpathlineto{\pgfqpoint{3.379920in}{2.287018in}}%
\pgfpathlineto{\pgfqpoint{3.383413in}{2.286432in}}%
\pgfpathlineto{\pgfqpoint{3.386906in}{2.284181in}}%
\pgfpathlineto{\pgfqpoint{3.390398in}{2.279294in}}%
\pgfpathlineto{\pgfqpoint{3.393891in}{2.277190in}}%
\pgfpathlineto{\pgfqpoint{3.397384in}{2.284260in}}%
\pgfpathlineto{\pgfqpoint{3.404369in}{2.314343in}}%
\pgfpathlineto{\pgfqpoint{3.407861in}{2.321747in}}%
\pgfpathlineto{\pgfqpoint{3.411354in}{2.322679in}}%
\pgfpathlineto{\pgfqpoint{3.418339in}{2.319551in}}%
\pgfpathlineto{\pgfqpoint{3.425325in}{2.312303in}}%
\pgfpathlineto{\pgfqpoint{3.428817in}{2.307716in}}%
\pgfpathlineto{\pgfqpoint{3.439295in}{2.289318in}}%
\pgfpathlineto{\pgfqpoint{3.442788in}{2.292922in}}%
\pgfpathlineto{\pgfqpoint{3.449773in}{2.304528in}}%
\pgfpathlineto{\pgfqpoint{3.460251in}{2.314606in}}%
\pgfpathlineto{\pgfqpoint{3.463744in}{2.313590in}}%
\pgfpathlineto{\pgfqpoint{3.467236in}{2.311137in}}%
\pgfpathlineto{\pgfqpoint{3.470729in}{2.314392in}}%
\pgfpathlineto{\pgfqpoint{3.484700in}{2.357990in}}%
\pgfpathlineto{\pgfqpoint{3.488192in}{2.357811in}}%
\pgfpathlineto{\pgfqpoint{3.491685in}{2.350618in}}%
\pgfpathlineto{\pgfqpoint{3.498670in}{2.328405in}}%
\pgfpathlineto{\pgfqpoint{3.502163in}{2.320980in}}%
\pgfpathlineto{\pgfqpoint{3.505655in}{2.317935in}}%
\pgfpathlineto{\pgfqpoint{3.509148in}{2.319445in}}%
\pgfpathlineto{\pgfqpoint{3.512641in}{2.323220in}}%
\pgfpathlineto{\pgfqpoint{3.516133in}{2.324473in}}%
\pgfpathlineto{\pgfqpoint{3.523119in}{2.318078in}}%
\pgfpathlineto{\pgfqpoint{3.530104in}{2.318079in}}%
\pgfpathlineto{\pgfqpoint{3.540582in}{2.308396in}}%
\pgfpathlineto{\pgfqpoint{3.544074in}{2.311641in}}%
\pgfpathlineto{\pgfqpoint{3.547567in}{2.317720in}}%
\pgfpathlineto{\pgfqpoint{3.561538in}{2.350080in}}%
\pgfpathlineto{\pgfqpoint{3.565030in}{2.344394in}}%
\pgfpathlineto{\pgfqpoint{3.568523in}{2.307255in}}%
\pgfpathlineto{\pgfqpoint{3.572016in}{2.210668in}}%
\pgfpathlineto{\pgfqpoint{3.579001in}{1.917118in}}%
\pgfpathlineto{\pgfqpoint{3.582494in}{1.866335in}}%
\pgfpathlineto{\pgfqpoint{3.585986in}{1.935969in}}%
\pgfpathlineto{\pgfqpoint{3.592971in}{2.204759in}}%
\pgfpathlineto{\pgfqpoint{3.596464in}{2.279992in}}%
\pgfpathlineto{\pgfqpoint{3.599957in}{2.307491in}}%
\pgfpathlineto{\pgfqpoint{3.603449in}{2.313095in}}%
\pgfpathlineto{\pgfqpoint{3.606942in}{2.313254in}}%
\pgfpathlineto{\pgfqpoint{3.613927in}{2.310289in}}%
\pgfpathlineto{\pgfqpoint{3.617420in}{2.304460in}}%
\pgfpathlineto{\pgfqpoint{3.620913in}{2.269445in}}%
\pgfpathlineto{\pgfqpoint{3.624405in}{2.153071in}}%
\pgfpathlineto{\pgfqpoint{3.627898in}{1.915996in}}%
\pgfpathlineto{\pgfqpoint{3.634883in}{1.355346in}}%
\pgfpathlineto{\pgfqpoint{3.638376in}{1.325423in}}%
\pgfpathlineto{\pgfqpoint{3.641868in}{1.535987in}}%
\pgfpathlineto{\pgfqpoint{3.648854in}{2.128143in}}%
\pgfpathlineto{\pgfqpoint{3.652346in}{2.272683in}}%
\pgfpathlineto{\pgfqpoint{3.655839in}{2.323959in}}%
\pgfpathlineto{\pgfqpoint{3.659332in}{2.340036in}}%
\pgfpathlineto{\pgfqpoint{3.662824in}{2.348716in}}%
\pgfpathlineto{\pgfqpoint{3.666317in}{2.352099in}}%
\pgfpathlineto{\pgfqpoint{3.669810in}{2.348122in}}%
\pgfpathlineto{\pgfqpoint{3.673302in}{2.335797in}}%
\pgfpathlineto{\pgfqpoint{3.680287in}{2.301908in}}%
\pgfpathlineto{\pgfqpoint{3.683780in}{2.301546in}}%
\pgfpathlineto{\pgfqpoint{3.697751in}{2.357702in}}%
\pgfpathlineto{\pgfqpoint{3.704736in}{2.389801in}}%
\pgfpathlineto{\pgfqpoint{3.708229in}{2.393176in}}%
\pgfpathlineto{\pgfqpoint{3.718707in}{2.386930in}}%
\pgfpathlineto{\pgfqpoint{3.722199in}{2.378929in}}%
\pgfpathlineto{\pgfqpoint{3.729184in}{2.352879in}}%
\pgfpathlineto{\pgfqpoint{3.732677in}{2.347699in}}%
\pgfpathlineto{\pgfqpoint{3.736170in}{2.350950in}}%
\pgfpathlineto{\pgfqpoint{3.743155in}{2.363636in}}%
\pgfpathlineto{\pgfqpoint{3.746648in}{2.361573in}}%
\pgfpathlineto{\pgfqpoint{3.750140in}{2.354936in}}%
\pgfpathlineto{\pgfqpoint{3.753633in}{2.351197in}}%
\pgfpathlineto{\pgfqpoint{3.757126in}{2.354433in}}%
\pgfpathlineto{\pgfqpoint{3.771096in}{2.383928in}}%
\pgfpathlineto{\pgfqpoint{3.778081in}{2.405764in}}%
\pgfpathlineto{\pgfqpoint{3.781574in}{2.408140in}}%
\pgfpathlineto{\pgfqpoint{3.785067in}{2.404703in}}%
\pgfpathlineto{\pgfqpoint{3.788559in}{2.403956in}}%
\pgfpathlineto{\pgfqpoint{3.792052in}{2.412823in}}%
\pgfpathlineto{\pgfqpoint{3.795545in}{2.427977in}}%
\pgfpathlineto{\pgfqpoint{3.799037in}{2.437911in}}%
\pgfpathlineto{\pgfqpoint{3.802530in}{2.434150in}}%
\pgfpathlineto{\pgfqpoint{3.806023in}{2.418142in}}%
\pgfpathlineto{\pgfqpoint{3.816501in}{2.358203in}}%
\pgfpathlineto{\pgfqpoint{3.819993in}{2.348559in}}%
\pgfpathlineto{\pgfqpoint{3.823486in}{2.347872in}}%
\pgfpathlineto{\pgfqpoint{3.830471in}{2.355505in}}%
\pgfpathlineto{\pgfqpoint{3.833964in}{2.356962in}}%
\pgfpathlineto{\pgfqpoint{3.837456in}{2.359980in}}%
\pgfpathlineto{\pgfqpoint{3.840949in}{2.368694in}}%
\pgfpathlineto{\pgfqpoint{3.844442in}{2.384972in}}%
\pgfpathlineto{\pgfqpoint{3.851427in}{2.426535in}}%
\pgfpathlineto{\pgfqpoint{3.854920in}{2.434950in}}%
\pgfpathlineto{\pgfqpoint{3.858412in}{2.430035in}}%
\pgfpathlineto{\pgfqpoint{3.861905in}{2.420665in}}%
\pgfpathlineto{\pgfqpoint{3.865397in}{2.416726in}}%
\pgfpathlineto{\pgfqpoint{3.868890in}{2.418287in}}%
\pgfpathlineto{\pgfqpoint{3.872383in}{2.418624in}}%
\pgfpathlineto{\pgfqpoint{3.875875in}{2.415744in}}%
\pgfpathlineto{\pgfqpoint{3.879368in}{2.414917in}}%
\pgfpathlineto{\pgfqpoint{3.886353in}{2.424494in}}%
\pgfpathlineto{\pgfqpoint{3.889846in}{2.421960in}}%
\pgfpathlineto{\pgfqpoint{3.893339in}{2.409522in}}%
\pgfpathlineto{\pgfqpoint{3.900324in}{2.380831in}}%
\pgfpathlineto{\pgfqpoint{3.903817in}{2.378134in}}%
\pgfpathlineto{\pgfqpoint{3.910802in}{2.384596in}}%
\pgfpathlineto{\pgfqpoint{3.914294in}{2.383665in}}%
\pgfpathlineto{\pgfqpoint{3.917787in}{2.384308in}}%
\pgfpathlineto{\pgfqpoint{3.921280in}{2.393883in}}%
\pgfpathlineto{\pgfqpoint{3.931758in}{2.452650in}}%
\pgfpathlineto{\pgfqpoint{3.935250in}{2.457396in}}%
\pgfpathlineto{\pgfqpoint{3.938743in}{2.448269in}}%
\pgfpathlineto{\pgfqpoint{3.945728in}{2.412160in}}%
\pgfpathlineto{\pgfqpoint{3.949221in}{2.403991in}}%
\pgfpathlineto{\pgfqpoint{3.952714in}{2.406749in}}%
\pgfpathlineto{\pgfqpoint{3.959699in}{2.422547in}}%
\pgfpathlineto{\pgfqpoint{3.963191in}{2.425636in}}%
\pgfpathlineto{\pgfqpoint{3.966684in}{2.423027in}}%
\pgfpathlineto{\pgfqpoint{3.980655in}{2.393442in}}%
\pgfpathlineto{\pgfqpoint{3.984147in}{2.393102in}}%
\pgfpathlineto{\pgfqpoint{3.987640in}{2.397189in}}%
\pgfpathlineto{\pgfqpoint{3.991133in}{2.405389in}}%
\pgfpathlineto{\pgfqpoint{3.998118in}{2.432272in}}%
\pgfpathlineto{\pgfqpoint{4.005103in}{2.465311in}}%
\pgfpathlineto{\pgfqpoint{4.008596in}{2.471391in}}%
\pgfpathlineto{\pgfqpoint{4.012088in}{2.466724in}}%
\pgfpathlineto{\pgfqpoint{4.015581in}{2.457899in}}%
\pgfpathlineto{\pgfqpoint{4.019074in}{2.452773in}}%
\pgfpathlineto{\pgfqpoint{4.026059in}{2.453253in}}%
\pgfpathlineto{\pgfqpoint{4.029552in}{2.450010in}}%
\pgfpathlineto{\pgfqpoint{4.033044in}{2.441673in}}%
\pgfpathlineto{\pgfqpoint{4.043522in}{2.408735in}}%
\pgfpathlineto{\pgfqpoint{4.047015in}{2.404377in}}%
\pgfpathlineto{\pgfqpoint{4.050507in}{2.403713in}}%
\pgfpathlineto{\pgfqpoint{4.054000in}{2.406983in}}%
\pgfpathlineto{\pgfqpoint{4.060985in}{2.421795in}}%
\pgfpathlineto{\pgfqpoint{4.064478in}{2.423175in}}%
\pgfpathlineto{\pgfqpoint{4.067971in}{2.419182in}}%
\pgfpathlineto{\pgfqpoint{4.071463in}{2.419005in}}%
\pgfpathlineto{\pgfqpoint{4.074956in}{2.430012in}}%
\pgfpathlineto{\pgfqpoint{4.085434in}{2.486604in}}%
\pgfpathlineto{\pgfqpoint{4.088927in}{2.499343in}}%
\pgfpathlineto{\pgfqpoint{4.092419in}{2.506752in}}%
\pgfpathlineto{\pgfqpoint{4.095912in}{2.507475in}}%
\pgfpathlineto{\pgfqpoint{4.099404in}{2.502568in}}%
\pgfpathlineto{\pgfqpoint{4.102897in}{2.493600in}}%
\pgfpathlineto{\pgfqpoint{4.113375in}{2.453744in}}%
\pgfpathlineto{\pgfqpoint{4.116868in}{2.447446in}}%
\pgfpathlineto{\pgfqpoint{4.123853in}{2.439815in}}%
\pgfpathlineto{\pgfqpoint{4.130838in}{2.428867in}}%
\pgfpathlineto{\pgfqpoint{4.134331in}{2.425759in}}%
\pgfpathlineto{\pgfqpoint{4.137824in}{2.427093in}}%
\pgfpathlineto{\pgfqpoint{4.148301in}{2.449696in}}%
\pgfpathlineto{\pgfqpoint{4.155287in}{2.445417in}}%
\pgfpathlineto{\pgfqpoint{4.158779in}{2.450468in}}%
\pgfpathlineto{\pgfqpoint{4.165765in}{2.465891in}}%
\pgfpathlineto{\pgfqpoint{4.172750in}{2.477938in}}%
\pgfpathlineto{\pgfqpoint{4.176243in}{2.480518in}}%
\pgfpathlineto{\pgfqpoint{4.179735in}{2.476919in}}%
\pgfpathlineto{\pgfqpoint{4.190213in}{2.458050in}}%
\pgfpathlineto{\pgfqpoint{4.193706in}{2.440628in}}%
\pgfpathlineto{\pgfqpoint{4.197198in}{2.409446in}}%
\pgfpathlineto{\pgfqpoint{4.200691in}{2.347584in}}%
\pgfpathlineto{\pgfqpoint{4.204184in}{2.231548in}}%
\pgfpathlineto{\pgfqpoint{4.211169in}{1.940629in}}%
\pgfpathlineto{\pgfqpoint{4.214662in}{1.921625in}}%
\pgfpathlineto{\pgfqpoint{4.218154in}{2.034505in}}%
\pgfpathlineto{\pgfqpoint{4.225140in}{2.366626in}}%
\pgfpathlineto{\pgfqpoint{4.228632in}{2.451892in}}%
\pgfpathlineto{\pgfqpoint{4.232125in}{2.480342in}}%
\pgfpathlineto{\pgfqpoint{4.235617in}{2.484489in}}%
\pgfpathlineto{\pgfqpoint{4.239110in}{2.486402in}}%
\pgfpathlineto{\pgfqpoint{4.242603in}{2.490085in}}%
\pgfpathlineto{\pgfqpoint{4.246095in}{2.489182in}}%
\pgfpathlineto{\pgfqpoint{4.249588in}{2.474757in}}%
\pgfpathlineto{\pgfqpoint{4.253081in}{2.428287in}}%
\pgfpathlineto{\pgfqpoint{4.256573in}{2.306555in}}%
\pgfpathlineto{\pgfqpoint{4.260066in}{2.062051in}}%
\pgfpathlineto{\pgfqpoint{4.267051in}{1.394614in}}%
\pgfpathlineto{\pgfqpoint{4.270544in}{1.282857in}}%
\pgfpathlineto{\pgfqpoint{4.274037in}{1.449775in}}%
\pgfpathlineto{\pgfqpoint{4.281022in}{2.125126in}}%
\pgfpathlineto{\pgfqpoint{4.284514in}{2.337249in}}%
\pgfpathlineto{\pgfqpoint{4.288007in}{2.431702in}}%
\pgfpathlineto{\pgfqpoint{4.291500in}{2.467109in}}%
\pgfpathlineto{\pgfqpoint{4.298485in}{2.503481in}}%
\pgfpathlineto{\pgfqpoint{4.301978in}{2.517545in}}%
\pgfpathlineto{\pgfqpoint{4.305470in}{2.524158in}}%
\pgfpathlineto{\pgfqpoint{4.308963in}{2.519532in}}%
\pgfpathlineto{\pgfqpoint{4.312456in}{2.508487in}}%
\pgfpathlineto{\pgfqpoint{4.315948in}{2.502078in}}%
\pgfpathlineto{\pgfqpoint{4.319441in}{2.504402in}}%
\pgfpathlineto{\pgfqpoint{4.322934in}{2.508872in}}%
\pgfpathlineto{\pgfqpoint{4.326426in}{2.509634in}}%
\pgfpathlineto{\pgfqpoint{4.329919in}{2.508992in}}%
\pgfpathlineto{\pgfqpoint{4.333411in}{2.510837in}}%
\pgfpathlineto{\pgfqpoint{4.336904in}{2.513994in}}%
\pgfpathlineto{\pgfqpoint{4.340397in}{2.515488in}}%
\pgfpathlineto{\pgfqpoint{4.343889in}{2.514383in}}%
\pgfpathlineto{\pgfqpoint{4.347382in}{2.509163in}}%
\pgfpathlineto{\pgfqpoint{4.350875in}{2.496835in}}%
\pgfpathlineto{\pgfqpoint{4.357860in}{2.462580in}}%
\pgfpathlineto{\pgfqpoint{4.361353in}{2.457470in}}%
\pgfpathlineto{\pgfqpoint{4.364845in}{2.464976in}}%
\pgfpathlineto{\pgfqpoint{4.371830in}{2.500343in}}%
\pgfpathlineto{\pgfqpoint{4.378816in}{2.535464in}}%
\pgfpathlineto{\pgfqpoint{4.382308in}{2.542260in}}%
\pgfpathlineto{\pgfqpoint{4.385801in}{2.539329in}}%
\pgfpathlineto{\pgfqpoint{4.389294in}{2.532394in}}%
\pgfpathlineto{\pgfqpoint{4.392786in}{2.530022in}}%
\pgfpathlineto{\pgfqpoint{4.399772in}{2.539713in}}%
\pgfpathlineto{\pgfqpoint{4.403264in}{2.536801in}}%
\pgfpathlineto{\pgfqpoint{4.410250in}{2.520372in}}%
\pgfpathlineto{\pgfqpoint{4.417235in}{2.519626in}}%
\pgfpathlineto{\pgfqpoint{4.431205in}{2.505064in}}%
\pgfpathlineto{\pgfqpoint{4.438191in}{2.498117in}}%
\pgfpathlineto{\pgfqpoint{4.441683in}{2.499759in}}%
\pgfpathlineto{\pgfqpoint{4.455654in}{2.528050in}}%
\pgfpathlineto{\pgfqpoint{4.462639in}{2.555890in}}%
\pgfpathlineto{\pgfqpoint{4.466132in}{2.572611in}}%
\pgfpathlineto{\pgfqpoint{4.469624in}{2.583660in}}%
\pgfpathlineto{\pgfqpoint{4.473117in}{2.582201in}}%
\pgfpathlineto{\pgfqpoint{4.483595in}{2.544392in}}%
\pgfpathlineto{\pgfqpoint{4.487088in}{2.541685in}}%
\pgfpathlineto{\pgfqpoint{4.490580in}{2.536525in}}%
\pgfpathlineto{\pgfqpoint{4.501058in}{2.503797in}}%
\pgfpathlineto{\pgfqpoint{4.504551in}{2.500574in}}%
\pgfpathlineto{\pgfqpoint{4.508044in}{2.502487in}}%
\pgfpathlineto{\pgfqpoint{4.518521in}{2.522062in}}%
\pgfpathlineto{\pgfqpoint{4.522014in}{2.528660in}}%
\pgfpathlineto{\pgfqpoint{4.525507in}{2.538728in}}%
\pgfpathlineto{\pgfqpoint{4.539477in}{2.598387in}}%
\pgfpathlineto{\pgfqpoint{4.542970in}{2.596908in}}%
\pgfpathlineto{\pgfqpoint{4.556940in}{2.548519in}}%
\pgfpathlineto{\pgfqpoint{4.570911in}{2.532457in}}%
\pgfpathlineto{\pgfqpoint{4.574404in}{2.532013in}}%
\pgfpathlineto{\pgfqpoint{4.577896in}{2.532754in}}%
\pgfpathlineto{\pgfqpoint{4.581389in}{2.537064in}}%
\pgfpathlineto{\pgfqpoint{4.588374in}{2.558276in}}%
\pgfpathlineto{\pgfqpoint{4.595360in}{2.575546in}}%
\pgfpathlineto{\pgfqpoint{4.605837in}{2.591093in}}%
\pgfpathlineto{\pgfqpoint{4.612823in}{2.601138in}}%
\pgfpathlineto{\pgfqpoint{4.616315in}{2.602307in}}%
\pgfpathlineto{\pgfqpoint{4.619808in}{2.600498in}}%
\pgfpathlineto{\pgfqpoint{4.623301in}{2.594742in}}%
\pgfpathlineto{\pgfqpoint{4.633779in}{2.562002in}}%
\pgfpathlineto{\pgfqpoint{4.637271in}{2.559421in}}%
\pgfpathlineto{\pgfqpoint{4.640764in}{2.559960in}}%
\pgfpathlineto{\pgfqpoint{4.644257in}{2.557705in}}%
\pgfpathlineto{\pgfqpoint{4.654734in}{2.537731in}}%
\pgfpathlineto{\pgfqpoint{4.658227in}{2.538435in}}%
\pgfpathlineto{\pgfqpoint{4.661720in}{2.546861in}}%
\pgfpathlineto{\pgfqpoint{4.672198in}{2.592320in}}%
\pgfpathlineto{\pgfqpoint{4.675690in}{2.599707in}}%
\pgfpathlineto{\pgfqpoint{4.679183in}{2.601633in}}%
\pgfpathlineto{\pgfqpoint{4.682676in}{2.599323in}}%
\pgfpathlineto{\pgfqpoint{4.686168in}{2.594918in}}%
\pgfpathlineto{\pgfqpoint{4.689661in}{2.592349in}}%
\pgfpathlineto{\pgfqpoint{4.693154in}{2.595124in}}%
\pgfpathlineto{\pgfqpoint{4.703631in}{2.617555in}}%
\pgfpathlineto{\pgfqpoint{4.707124in}{2.615426in}}%
\pgfpathlineto{\pgfqpoint{4.710617in}{2.605551in}}%
\pgfpathlineto{\pgfqpoint{4.717602in}{2.578357in}}%
\pgfpathlineto{\pgfqpoint{4.721095in}{2.568666in}}%
\pgfpathlineto{\pgfqpoint{4.724587in}{2.562409in}}%
\pgfpathlineto{\pgfqpoint{4.728080in}{2.559626in}}%
\pgfpathlineto{\pgfqpoint{4.731573in}{2.560485in}}%
\pgfpathlineto{\pgfqpoint{4.735065in}{2.562627in}}%
\pgfpathlineto{\pgfqpoint{4.742050in}{2.561604in}}%
\pgfpathlineto{\pgfqpoint{4.745543in}{2.565185in}}%
\pgfpathlineto{\pgfqpoint{4.749036in}{2.576738in}}%
\pgfpathlineto{\pgfqpoint{4.756021in}{2.607792in}}%
\pgfpathlineto{\pgfqpoint{4.759514in}{2.618715in}}%
\pgfpathlineto{\pgfqpoint{4.763006in}{2.624560in}}%
\pgfpathlineto{\pgfqpoint{4.766499in}{2.623526in}}%
\pgfpathlineto{\pgfqpoint{4.769992in}{2.614817in}}%
\pgfpathlineto{\pgfqpoint{4.780470in}{2.573055in}}%
\pgfpathlineto{\pgfqpoint{4.783962in}{2.565735in}}%
\pgfpathlineto{\pgfqpoint{4.787455in}{2.564116in}}%
\pgfpathlineto{\pgfqpoint{4.790947in}{2.564349in}}%
\pgfpathlineto{\pgfqpoint{4.794440in}{2.562010in}}%
\pgfpathlineto{\pgfqpoint{4.797933in}{2.557724in}}%
\pgfpathlineto{\pgfqpoint{4.801425in}{2.555426in}}%
\pgfpathlineto{\pgfqpoint{4.808411in}{2.554925in}}%
\pgfpathlineto{\pgfqpoint{4.811903in}{2.546639in}}%
\pgfpathlineto{\pgfqpoint{4.815396in}{2.516614in}}%
\pgfpathlineto{\pgfqpoint{4.818889in}{2.443137in}}%
\pgfpathlineto{\pgfqpoint{4.825874in}{2.197803in}}%
\pgfpathlineto{\pgfqpoint{4.829367in}{2.144803in}}%
\pgfpathlineto{\pgfqpoint{4.832859in}{2.205514in}}%
\pgfpathlineto{\pgfqpoint{4.839844in}{2.488940in}}%
\pgfpathlineto{\pgfqpoint{4.843337in}{2.576094in}}%
\pgfpathlineto{\pgfqpoint{4.846830in}{2.603263in}}%
\pgfpathlineto{\pgfqpoint{4.850322in}{2.596975in}}%
\pgfpathlineto{\pgfqpoint{4.853815in}{2.583174in}}%
\pgfpathlineto{\pgfqpoint{4.857308in}{2.574148in}}%
\pgfpathlineto{\pgfqpoint{4.860800in}{2.569758in}}%
\pgfpathlineto{\pgfqpoint{4.864293in}{2.562717in}}%
\pgfpathlineto{\pgfqpoint{4.867786in}{2.540143in}}%
\pgfpathlineto{\pgfqpoint{4.871278in}{2.473552in}}%
\pgfpathlineto{\pgfqpoint{4.874771in}{2.314418in}}%
\pgfpathlineto{\pgfqpoint{4.885249in}{1.449451in}}%
\pgfpathlineto{\pgfqpoint{4.888741in}{1.455417in}}%
\pgfpathlineto{\pgfqpoint{4.892234in}{1.710322in}}%
\pgfpathlineto{\pgfqpoint{4.899219in}{2.339987in}}%
\pgfpathlineto{\pgfqpoint{4.902712in}{2.486330in}}%
\pgfpathlineto{\pgfqpoint{4.906205in}{2.543366in}}%
\pgfpathlineto{\pgfqpoint{4.913190in}{2.589385in}}%
\pgfpathlineto{\pgfqpoint{4.920175in}{2.620299in}}%
\pgfpathlineto{\pgfqpoint{4.923668in}{2.629176in}}%
\pgfpathlineto{\pgfqpoint{4.927160in}{2.624798in}}%
\pgfpathlineto{\pgfqpoint{4.941131in}{2.555035in}}%
\pgfpathlineto{\pgfqpoint{4.944624in}{2.549426in}}%
\pgfpathlineto{\pgfqpoint{4.948116in}{2.550460in}}%
\pgfpathlineto{\pgfqpoint{4.955102in}{2.561713in}}%
\pgfpathlineto{\pgfqpoint{4.958594in}{2.563497in}}%
\pgfpathlineto{\pgfqpoint{4.962087in}{2.563919in}}%
\pgfpathlineto{\pgfqpoint{4.965580in}{2.566340in}}%
\pgfpathlineto{\pgfqpoint{4.972565in}{2.578178in}}%
\pgfpathlineto{\pgfqpoint{4.976057in}{2.583958in}}%
\pgfpathlineto{\pgfqpoint{4.979550in}{2.586338in}}%
\pgfpathlineto{\pgfqpoint{4.983043in}{2.584875in}}%
\pgfpathlineto{\pgfqpoint{4.993521in}{2.576013in}}%
\pgfpathlineto{\pgfqpoint{4.997013in}{2.566982in}}%
\pgfpathlineto{\pgfqpoint{5.007491in}{2.530880in}}%
\pgfpathlineto{\pgfqpoint{5.010984in}{2.526407in}}%
\pgfpathlineto{\pgfqpoint{5.017969in}{2.523586in}}%
\pgfpathlineto{\pgfqpoint{5.021462in}{2.519275in}}%
\pgfpathlineto{\pgfqpoint{5.028447in}{2.507724in}}%
\pgfpathlineto{\pgfqpoint{5.031940in}{2.505342in}}%
\pgfpathlineto{\pgfqpoint{5.035432in}{2.505038in}}%
\pgfpathlineto{\pgfqpoint{5.038925in}{2.507986in}}%
\pgfpathlineto{\pgfqpoint{5.042418in}{2.517517in}}%
\pgfpathlineto{\pgfqpoint{5.049403in}{2.547863in}}%
\pgfpathlineto{\pgfqpoint{5.052896in}{2.556849in}}%
\pgfpathlineto{\pgfqpoint{5.059881in}{2.569650in}}%
\pgfpathlineto{\pgfqpoint{5.070359in}{2.599287in}}%
\pgfpathlineto{\pgfqpoint{5.073851in}{2.600631in}}%
\pgfpathlineto{\pgfqpoint{5.077344in}{2.591850in}}%
\pgfpathlineto{\pgfqpoint{5.084329in}{2.556934in}}%
\pgfpathlineto{\pgfqpoint{5.087822in}{2.548141in}}%
\pgfpathlineto{\pgfqpoint{5.091315in}{2.549901in}}%
\pgfpathlineto{\pgfqpoint{5.098300in}{2.569808in}}%
\pgfpathlineto{\pgfqpoint{5.101793in}{2.579400in}}%
\pgfpathlineto{\pgfqpoint{5.105285in}{2.582116in}}%
\pgfpathlineto{\pgfqpoint{5.112270in}{2.571723in}}%
\pgfpathlineto{\pgfqpoint{5.115763in}{2.573720in}}%
\pgfpathlineto{\pgfqpoint{5.122748in}{2.588527in}}%
\pgfpathlineto{\pgfqpoint{5.126241in}{2.589129in}}%
\pgfpathlineto{\pgfqpoint{5.133226in}{2.583227in}}%
\pgfpathlineto{\pgfqpoint{5.136719in}{2.577368in}}%
\pgfpathlineto{\pgfqpoint{5.143704in}{2.559992in}}%
\pgfpathlineto{\pgfqpoint{5.147197in}{2.560653in}}%
\pgfpathlineto{\pgfqpoint{5.150690in}{2.572044in}}%
\pgfpathlineto{\pgfqpoint{5.154182in}{2.587723in}}%
\pgfpathlineto{\pgfqpoint{5.157675in}{2.597117in}}%
\pgfpathlineto{\pgfqpoint{5.161167in}{2.594030in}}%
\pgfpathlineto{\pgfqpoint{5.168153in}{2.570870in}}%
\pgfpathlineto{\pgfqpoint{5.171645in}{2.566107in}}%
\pgfpathlineto{\pgfqpoint{5.175138in}{2.566014in}}%
\pgfpathlineto{\pgfqpoint{5.178631in}{2.567497in}}%
\pgfpathlineto{\pgfqpoint{5.182123in}{2.572051in}}%
\pgfpathlineto{\pgfqpoint{5.185616in}{2.582412in}}%
\pgfpathlineto{\pgfqpoint{5.192601in}{2.610226in}}%
\pgfpathlineto{\pgfqpoint{5.196094in}{2.619699in}}%
\pgfpathlineto{\pgfqpoint{5.199587in}{2.623686in}}%
\pgfpathlineto{\pgfqpoint{5.203079in}{2.620400in}}%
\pgfpathlineto{\pgfqpoint{5.210064in}{2.605483in}}%
\pgfpathlineto{\pgfqpoint{5.213557in}{2.612074in}}%
\pgfpathlineto{\pgfqpoint{5.227528in}{2.688028in}}%
\pgfpathlineto{\pgfqpoint{5.231020in}{2.692655in}}%
\pgfpathlineto{\pgfqpoint{5.234513in}{2.685451in}}%
\pgfpathlineto{\pgfqpoint{5.241498in}{2.640834in}}%
\pgfpathlineto{\pgfqpoint{5.248483in}{2.600587in}}%
\pgfpathlineto{\pgfqpoint{5.251976in}{2.591932in}}%
\pgfpathlineto{\pgfqpoint{5.255469in}{2.592513in}}%
\pgfpathlineto{\pgfqpoint{5.258961in}{2.601366in}}%
\pgfpathlineto{\pgfqpoint{5.265947in}{2.624024in}}%
\pgfpathlineto{\pgfqpoint{5.269439in}{2.629563in}}%
\pgfpathlineto{\pgfqpoint{5.272932in}{2.632821in}}%
\pgfpathlineto{\pgfqpoint{5.276425in}{2.638322in}}%
\pgfpathlineto{\pgfqpoint{5.283410in}{2.662658in}}%
\pgfpathlineto{\pgfqpoint{5.286903in}{2.674875in}}%
\pgfpathlineto{\pgfqpoint{5.290395in}{2.681692in}}%
\pgfpathlineto{\pgfqpoint{5.293888in}{2.683721in}}%
\pgfpathlineto{\pgfqpoint{5.297380in}{2.683490in}}%
\pgfpathlineto{\pgfqpoint{5.300873in}{2.681054in}}%
\pgfpathlineto{\pgfqpoint{5.304366in}{2.674577in}}%
\pgfpathlineto{\pgfqpoint{5.318336in}{2.636608in}}%
\pgfpathlineto{\pgfqpoint{5.321829in}{2.629041in}}%
\pgfpathlineto{\pgfqpoint{5.325322in}{2.627436in}}%
\pgfpathlineto{\pgfqpoint{5.328814in}{2.636225in}}%
\pgfpathlineto{\pgfqpoint{5.339292in}{2.684240in}}%
\pgfpathlineto{\pgfqpoint{5.346277in}{2.694296in}}%
\pgfpathlineto{\pgfqpoint{5.349770in}{2.706139in}}%
\pgfpathlineto{\pgfqpoint{5.356755in}{2.742104in}}%
\pgfpathlineto{\pgfqpoint{5.360248in}{2.743369in}}%
\pgfpathlineto{\pgfqpoint{5.367233in}{2.714323in}}%
\pgfpathlineto{\pgfqpoint{5.370726in}{2.709851in}}%
\pgfpathlineto{\pgfqpoint{5.374219in}{2.713488in}}%
\pgfpathlineto{\pgfqpoint{5.377711in}{2.714614in}}%
\pgfpathlineto{\pgfqpoint{5.384697in}{2.706567in}}%
\pgfpathlineto{\pgfqpoint{5.388189in}{2.709459in}}%
\pgfpathlineto{\pgfqpoint{5.391682in}{2.715498in}}%
\pgfpathlineto{\pgfqpoint{5.395174in}{2.719269in}}%
\pgfpathlineto{\pgfqpoint{5.402160in}{2.720302in}}%
\pgfpathlineto{\pgfqpoint{5.405652in}{2.719771in}}%
\pgfpathlineto{\pgfqpoint{5.409145in}{2.716772in}}%
\pgfpathlineto{\pgfqpoint{5.412638in}{2.706195in}}%
\pgfpathlineto{\pgfqpoint{5.416130in}{2.675394in}}%
\pgfpathlineto{\pgfqpoint{5.419623in}{2.610446in}}%
\pgfpathlineto{\pgfqpoint{5.426608in}{2.439051in}}%
\pgfpathlineto{\pgfqpoint{5.430101in}{2.422398in}}%
\pgfpathlineto{\pgfqpoint{5.433593in}{2.484602in}}%
\pgfpathlineto{\pgfqpoint{5.440579in}{2.696679in}}%
\pgfpathlineto{\pgfqpoint{5.444071in}{2.762463in}}%
\pgfpathlineto{\pgfqpoint{5.447564in}{2.788879in}}%
\pgfpathlineto{\pgfqpoint{5.451057in}{2.790079in}}%
\pgfpathlineto{\pgfqpoint{5.454549in}{2.779395in}}%
\pgfpathlineto{\pgfqpoint{5.461535in}{2.743081in}}%
\pgfpathlineto{\pgfqpoint{5.468520in}{2.698812in}}%
\pgfpathlineto{\pgfqpoint{5.472013in}{2.655080in}}%
\pgfpathlineto{\pgfqpoint{5.475505in}{2.548377in}}%
\pgfpathlineto{\pgfqpoint{5.485983in}{1.940877in}}%
\pgfpathlineto{\pgfqpoint{5.489476in}{1.953620in}}%
\pgfpathlineto{\pgfqpoint{5.492968in}{2.141148in}}%
\pgfpathlineto{\pgfqpoint{5.499954in}{2.587088in}}%
\pgfpathlineto{\pgfqpoint{5.503446in}{2.690148in}}%
\pgfpathlineto{\pgfqpoint{5.506939in}{2.732583in}}%
\pgfpathlineto{\pgfqpoint{5.513924in}{2.773692in}}%
\pgfpathlineto{\pgfqpoint{5.517417in}{2.787965in}}%
\pgfpathlineto{\pgfqpoint{5.520910in}{2.796245in}}%
\pgfpathlineto{\pgfqpoint{5.524402in}{2.797458in}}%
\pgfpathlineto{\pgfqpoint{5.534880in}{2.782988in}}%
\pgfpathlineto{\pgfqpoint{5.538373in}{2.782704in}}%
\pgfpathlineto{\pgfqpoint{5.541865in}{2.779350in}}%
\pgfpathlineto{\pgfqpoint{5.545358in}{2.773046in}}%
\pgfpathlineto{\pgfqpoint{5.548851in}{2.769340in}}%
\pgfpathlineto{\pgfqpoint{5.559329in}{2.772159in}}%
\pgfpathlineto{\pgfqpoint{5.569806in}{2.761828in}}%
\pgfpathlineto{\pgfqpoint{5.573299in}{2.763710in}}%
\pgfpathlineto{\pgfqpoint{5.576792in}{2.773661in}}%
\pgfpathlineto{\pgfqpoint{5.583777in}{2.805822in}}%
\pgfpathlineto{\pgfqpoint{5.587270in}{2.816186in}}%
\pgfpathlineto{\pgfqpoint{5.590762in}{2.819201in}}%
\pgfpathlineto{\pgfqpoint{5.597748in}{2.814664in}}%
\pgfpathlineto{\pgfqpoint{5.601240in}{2.818409in}}%
\pgfpathlineto{\pgfqpoint{5.604733in}{2.826627in}}%
\pgfpathlineto{\pgfqpoint{5.608226in}{2.828032in}}%
\pgfpathlineto{\pgfqpoint{5.611718in}{2.812922in}}%
\pgfpathlineto{\pgfqpoint{5.618703in}{2.764240in}}%
\pgfpathlineto{\pgfqpoint{5.622196in}{2.755683in}}%
\pgfpathlineto{\pgfqpoint{5.625689in}{2.756594in}}%
\pgfpathlineto{\pgfqpoint{5.629181in}{2.759957in}}%
\pgfpathlineto{\pgfqpoint{5.632674in}{2.765600in}}%
\pgfpathlineto{\pgfqpoint{5.643152in}{2.798454in}}%
\pgfpathlineto{\pgfqpoint{5.657123in}{2.817968in}}%
\pgfpathlineto{\pgfqpoint{5.660615in}{2.815867in}}%
\pgfpathlineto{\pgfqpoint{5.664108in}{2.809808in}}%
\pgfpathlineto{\pgfqpoint{5.667600in}{2.806186in}}%
\pgfpathlineto{\pgfqpoint{5.671093in}{2.809107in}}%
\pgfpathlineto{\pgfqpoint{5.674586in}{2.815907in}}%
\pgfpathlineto{\pgfqpoint{5.678078in}{2.820055in}}%
\pgfpathlineto{\pgfqpoint{5.681571in}{2.818039in}}%
\pgfpathlineto{\pgfqpoint{5.692049in}{2.799979in}}%
\pgfpathlineto{\pgfqpoint{5.695542in}{2.797498in}}%
\pgfpathlineto{\pgfqpoint{5.699034in}{2.799328in}}%
\pgfpathlineto{\pgfqpoint{5.702527in}{2.803909in}}%
\pgfpathlineto{\pgfqpoint{5.706020in}{2.805470in}}%
\pgfpathlineto{\pgfqpoint{5.709512in}{2.801391in}}%
\pgfpathlineto{\pgfqpoint{5.713005in}{2.799195in}}%
\pgfpathlineto{\pgfqpoint{5.716497in}{2.810551in}}%
\pgfpathlineto{\pgfqpoint{5.726975in}{2.883883in}}%
\pgfpathlineto{\pgfqpoint{5.730468in}{2.884644in}}%
\pgfpathlineto{\pgfqpoint{5.733961in}{2.873909in}}%
\pgfpathlineto{\pgfqpoint{5.744439in}{2.832860in}}%
\pgfpathlineto{\pgfqpoint{5.747931in}{2.827909in}}%
\pgfpathlineto{\pgfqpoint{5.754916in}{2.825364in}}%
\pgfpathlineto{\pgfqpoint{5.758409in}{2.819944in}}%
\pgfpathlineto{\pgfqpoint{5.765394in}{2.806765in}}%
\pgfpathlineto{\pgfqpoint{5.768887in}{2.807239in}}%
\pgfpathlineto{\pgfqpoint{5.779365in}{2.824323in}}%
\pgfpathlineto{\pgfqpoint{5.782858in}{2.825298in}}%
\pgfpathlineto{\pgfqpoint{5.786350in}{2.824008in}}%
\pgfpathlineto{\pgfqpoint{5.789843in}{2.820172in}}%
\pgfpathlineto{\pgfqpoint{5.796828in}{2.804697in}}%
\pgfpathlineto{\pgfqpoint{5.800321in}{2.801407in}}%
\pgfpathlineto{\pgfqpoint{5.803813in}{2.807141in}}%
\pgfpathlineto{\pgfqpoint{5.817784in}{2.857374in}}%
\pgfpathlineto{\pgfqpoint{5.821277in}{2.865554in}}%
\pgfpathlineto{\pgfqpoint{5.824769in}{2.867992in}}%
\pgfpathlineto{\pgfqpoint{5.824769in}{2.867992in}}%
\pgfusepath{stroke}%
\end{pgfscope}%
\begin{pgfscope}%
\pgfpathrectangle{\pgfqpoint{0.795366in}{0.646140in}}{\pgfqpoint{5.029404in}{3.088289in}}%
\pgfusepath{clip}%
\pgfsetrectcap%
\pgfsetroundjoin%
\pgfsetlinewidth{1.003750pt}%
\definecolor{currentstroke}{rgb}{0.000000,0.000000,1.000000}%
\pgfsetstrokecolor{currentstroke}%
\pgfsetdash{}{0pt}%
\pgfpathmoveto{\pgfqpoint{0.795366in}{1.339923in}}%
\pgfpathlineto{\pgfqpoint{0.886174in}{1.399001in}}%
\pgfpathlineto{\pgfqpoint{0.976983in}{1.455489in}}%
\pgfpathlineto{\pgfqpoint{1.071284in}{1.511493in}}%
\pgfpathlineto{\pgfqpoint{1.165586in}{1.564877in}}%
\pgfpathlineto{\pgfqpoint{1.259887in}{1.615729in}}%
\pgfpathlineto{\pgfqpoint{1.357681in}{1.665884in}}%
\pgfpathlineto{\pgfqpoint{1.455475in}{1.713510in}}%
\pgfpathlineto{\pgfqpoint{1.553269in}{1.758705in}}%
\pgfpathlineto{\pgfqpoint{1.654555in}{1.803056in}}%
\pgfpathlineto{\pgfqpoint{1.755842in}{1.845013in}}%
\pgfpathlineto{\pgfqpoint{1.860621in}{1.886015in}}%
\pgfpathlineto{\pgfqpoint{1.968893in}{1.925945in}}%
\pgfpathlineto{\pgfqpoint{2.077165in}{1.963527in}}%
\pgfpathlineto{\pgfqpoint{2.188930in}{2.000002in}}%
\pgfpathlineto{\pgfqpoint{2.304187in}{2.035301in}}%
\pgfpathlineto{\pgfqpoint{2.422937in}{2.069379in}}%
\pgfpathlineto{\pgfqpoint{2.548672in}{2.103122in}}%
\pgfpathlineto{\pgfqpoint{2.677899in}{2.135511in}}%
\pgfpathlineto{\pgfqpoint{2.814112in}{2.167382in}}%
\pgfpathlineto{\pgfqpoint{2.960803in}{2.199403in}}%
\pgfpathlineto{\pgfqpoint{3.117972in}{2.231429in}}%
\pgfpathlineto{\pgfqpoint{3.292604in}{2.264732in}}%
\pgfpathlineto{\pgfqpoint{3.498670in}{2.301678in}}%
\pgfpathlineto{\pgfqpoint{3.774589in}{2.348695in}}%
\pgfpathlineto{\pgfqpoint{4.298485in}{2.437623in}}%
\pgfpathlineto{\pgfqpoint{4.508044in}{2.475787in}}%
\pgfpathlineto{\pgfqpoint{4.682676in}{2.509790in}}%
\pgfpathlineto{\pgfqpoint{4.839844in}{2.542597in}}%
\pgfpathlineto{\pgfqpoint{4.986535in}{2.575475in}}%
\pgfpathlineto{\pgfqpoint{5.122748in}{2.608254in}}%
\pgfpathlineto{\pgfqpoint{5.251976in}{2.641602in}}%
\pgfpathlineto{\pgfqpoint{5.377711in}{2.676371in}}%
\pgfpathlineto{\pgfqpoint{5.496461in}{2.711502in}}%
\pgfpathlineto{\pgfqpoint{5.611718in}{2.747901in}}%
\pgfpathlineto{\pgfqpoint{5.723483in}{2.785515in}}%
\pgfpathlineto{\pgfqpoint{5.824769in}{2.821699in}}%
\pgfpathlineto{\pgfqpoint{5.824769in}{2.821699in}}%
\pgfusepath{stroke}%
\end{pgfscope}%
\begin{pgfscope}%
\pgfpathrectangle{\pgfqpoint{0.795366in}{0.646140in}}{\pgfqpoint{5.029404in}{3.088289in}}%
\pgfusepath{clip}%
\pgfsetbuttcap%
\pgfsetroundjoin%
\definecolor{currentfill}{rgb}{1.000000,0.000000,0.000000}%
\pgfsetfillcolor{currentfill}%
\pgfsetlinewidth{1.003750pt}%
\definecolor{currentstroke}{rgb}{1.000000,0.000000,0.000000}%
\pgfsetstrokecolor{currentstroke}%
\pgfsetdash{}{0pt}%
\pgfsys@defobject{currentmarker}{\pgfqpoint{-0.020833in}{-0.020833in}}{\pgfqpoint{0.020833in}{0.020833in}}{%
\pgfpathmoveto{\pgfqpoint{0.000000in}{-0.020833in}}%
\pgfpathcurveto{\pgfqpoint{0.005525in}{-0.020833in}}{\pgfqpoint{0.010825in}{-0.018638in}}{\pgfqpoint{0.014731in}{-0.014731in}}%
\pgfpathcurveto{\pgfqpoint{0.018638in}{-0.010825in}}{\pgfqpoint{0.020833in}{-0.005525in}}{\pgfqpoint{0.020833in}{0.000000in}}%
\pgfpathcurveto{\pgfqpoint{0.020833in}{0.005525in}}{\pgfqpoint{0.018638in}{0.010825in}}{\pgfqpoint{0.014731in}{0.014731in}}%
\pgfpathcurveto{\pgfqpoint{0.010825in}{0.018638in}}{\pgfqpoint{0.005525in}{0.020833in}}{\pgfqpoint{0.000000in}{0.020833in}}%
\pgfpathcurveto{\pgfqpoint{-0.005525in}{0.020833in}}{\pgfqpoint{-0.010825in}{0.018638in}}{\pgfqpoint{-0.014731in}{0.014731in}}%
\pgfpathcurveto{\pgfqpoint{-0.018638in}{0.010825in}}{\pgfqpoint{-0.020833in}{0.005525in}}{\pgfqpoint{-0.020833in}{0.000000in}}%
\pgfpathcurveto{\pgfqpoint{-0.020833in}{-0.005525in}}{\pgfqpoint{-0.018638in}{-0.010825in}}{\pgfqpoint{-0.014731in}{-0.014731in}}%
\pgfpathcurveto{\pgfqpoint{-0.010825in}{-0.018638in}}{\pgfqpoint{-0.005525in}{-0.020833in}}{\pgfqpoint{0.000000in}{-0.020833in}}%
\pgfpathlineto{\pgfqpoint{0.000000in}{-0.020833in}}%
\pgfpathclose%
\pgfusepath{stroke,fill}%
}%
\begin{pgfscope}%
\pgfsys@transformshift{5.485983in}{1.940877in}%
\pgfsys@useobject{currentmarker}{}%
\end{pgfscope}%
\begin{pgfscope}%
\pgfsys@transformshift{5.430101in}{2.422398in}%
\pgfsys@useobject{currentmarker}{}%
\end{pgfscope}%
\begin{pgfscope}%
\pgfsys@transformshift{4.885249in}{1.449451in}%
\pgfsys@useobject{currentmarker}{}%
\end{pgfscope}%
\begin{pgfscope}%
\pgfsys@transformshift{4.829367in}{2.144803in}%
\pgfsys@useobject{currentmarker}{}%
\end{pgfscope}%
\begin{pgfscope}%
\pgfsys@transformshift{4.270544in}{1.282857in}%
\pgfsys@useobject{currentmarker}{}%
\end{pgfscope}%
\begin{pgfscope}%
\pgfsys@transformshift{4.214662in}{1.921625in}%
\pgfsys@useobject{currentmarker}{}%
\end{pgfscope}%
\begin{pgfscope}%
\pgfsys@transformshift{3.638376in}{1.325423in}%
\pgfsys@useobject{currentmarker}{}%
\end{pgfscope}%
\begin{pgfscope}%
\pgfsys@transformshift{3.582494in}{1.866335in}%
\pgfsys@useobject{currentmarker}{}%
\end{pgfscope}%
\begin{pgfscope}%
\pgfsys@transformshift{2.988744in}{1.351365in}%
\pgfsys@useobject{currentmarker}{}%
\end{pgfscope}%
\begin{pgfscope}%
\pgfsys@transformshift{2.932862in}{1.851615in}%
\pgfsys@useobject{currentmarker}{}%
\end{pgfscope}%
\begin{pgfscope}%
\pgfsys@transformshift{2.325143in}{1.452389in}%
\pgfsys@useobject{currentmarker}{}%
\end{pgfscope}%
\begin{pgfscope}%
\pgfsys@transformshift{2.272753in}{1.792400in}%
\pgfsys@useobject{currentmarker}{}%
\end{pgfscope}%
\begin{pgfscope}%
\pgfsys@transformshift{1.647570in}{1.403198in}%
\pgfsys@useobject{currentmarker}{}%
\end{pgfscope}%
\begin{pgfscope}%
\pgfsys@transformshift{1.595181in}{1.618300in}%
\pgfsys@useobject{currentmarker}{}%
\end{pgfscope}%
\begin{pgfscope}%
\pgfsys@transformshift{0.956027in}{1.260974in}%
\pgfsys@useobject{currentmarker}{}%
\end{pgfscope}%
\begin{pgfscope}%
\pgfsys@transformshift{0.903638in}{1.351459in}%
\pgfsys@useobject{currentmarker}{}%
\end{pgfscope}%
\end{pgfscope}%
\begin{pgfscope}%
\pgfsetrectcap%
\pgfsetmiterjoin%
\pgfsetlinewidth{0.803000pt}%
\definecolor{currentstroke}{rgb}{0.000000,0.000000,0.000000}%
\pgfsetstrokecolor{currentstroke}%
\pgfsetdash{}{0pt}%
\pgfpathmoveto{\pgfqpoint{0.795366in}{0.646140in}}%
\pgfpathlineto{\pgfqpoint{0.795366in}{3.734428in}}%
\pgfusepath{stroke}%
\end{pgfscope}%
\begin{pgfscope}%
\pgfsetrectcap%
\pgfsetmiterjoin%
\pgfsetlinewidth{0.803000pt}%
\definecolor{currentstroke}{rgb}{0.000000,0.000000,0.000000}%
\pgfsetstrokecolor{currentstroke}%
\pgfsetdash{}{0pt}%
\pgfpathmoveto{\pgfqpoint{5.824769in}{0.646140in}}%
\pgfpathlineto{\pgfqpoint{5.824769in}{3.734428in}}%
\pgfusepath{stroke}%
\end{pgfscope}%
\begin{pgfscope}%
\pgfsetrectcap%
\pgfsetmiterjoin%
\pgfsetlinewidth{0.803000pt}%
\definecolor{currentstroke}{rgb}{0.000000,0.000000,0.000000}%
\pgfsetstrokecolor{currentstroke}%
\pgfsetdash{}{0pt}%
\pgfpathmoveto{\pgfqpoint{0.795366in}{0.646140in}}%
\pgfpathlineto{\pgfqpoint{5.824769in}{0.646140in}}%
\pgfusepath{stroke}%
\end{pgfscope}%
\begin{pgfscope}%
\pgfsetrectcap%
\pgfsetmiterjoin%
\pgfsetlinewidth{0.803000pt}%
\definecolor{currentstroke}{rgb}{0.000000,0.000000,0.000000}%
\pgfsetstrokecolor{currentstroke}%
\pgfsetdash{}{0pt}%
\pgfpathmoveto{\pgfqpoint{0.795366in}{3.734428in}}%
\pgfpathlineto{\pgfqpoint{5.824769in}{3.734428in}}%
\pgfusepath{stroke}%
\end{pgfscope}%
\begin{pgfscope}%
\definecolor{textcolor}{rgb}{0.000000,0.000000,0.000000}%
\pgfsetstrokecolor{textcolor}%
\pgfsetfillcolor{textcolor}%
\pgftext[x=5.485983in,y=1.734991in,,base]{\color{textcolor}\rmfamily\fontsize{8.000000}{9.600000}\selectfont P(1)}%
\end{pgfscope}%
\begin{pgfscope}%
\definecolor{textcolor}{rgb}{0.000000,0.000000,0.000000}%
\pgfsetstrokecolor{textcolor}%
\pgfsetfillcolor{textcolor}%
\pgftext[x=4.885249in,y=1.243565in,,base]{\color{textcolor}\rmfamily\fontsize{8.000000}{9.600000}\selectfont P(2)}%
\end{pgfscope}%
\begin{pgfscope}%
\definecolor{textcolor}{rgb}{0.000000,0.000000,0.000000}%
\pgfsetstrokecolor{textcolor}%
\pgfsetfillcolor{textcolor}%
\pgftext[x=4.270544in,y=1.076971in,,base]{\color{textcolor}\rmfamily\fontsize{8.000000}{9.600000}\selectfont P(3)}%
\end{pgfscope}%
\begin{pgfscope}%
\definecolor{textcolor}{rgb}{0.000000,0.000000,0.000000}%
\pgfsetstrokecolor{textcolor}%
\pgfsetfillcolor{textcolor}%
\pgftext[x=3.638376in,y=1.119538in,,base]{\color{textcolor}\rmfamily\fontsize{8.000000}{9.600000}\selectfont P(4)}%
\end{pgfscope}%
\begin{pgfscope}%
\definecolor{textcolor}{rgb}{0.000000,0.000000,0.000000}%
\pgfsetstrokecolor{textcolor}%
\pgfsetfillcolor{textcolor}%
\pgftext[x=2.988744in,y=1.145479in,,base]{\color{textcolor}\rmfamily\fontsize{8.000000}{9.600000}\selectfont P(5)}%
\end{pgfscope}%
\begin{pgfscope}%
\definecolor{textcolor}{rgb}{0.000000,0.000000,0.000000}%
\pgfsetstrokecolor{textcolor}%
\pgfsetfillcolor{textcolor}%
\pgftext[x=2.325143in,y=1.246503in,,base]{\color{textcolor}\rmfamily\fontsize{8.000000}{9.600000}\selectfont P(6)}%
\end{pgfscope}%
\begin{pgfscope}%
\definecolor{textcolor}{rgb}{0.000000,0.000000,0.000000}%
\pgfsetstrokecolor{textcolor}%
\pgfsetfillcolor{textcolor}%
\pgftext[x=1.647570in,y=1.197312in,,base]{\color{textcolor}\rmfamily\fontsize{8.000000}{9.600000}\selectfont P(7)}%
\end{pgfscope}%
\begin{pgfscope}%
\definecolor{textcolor}{rgb}{0.000000,0.000000,0.000000}%
\pgfsetstrokecolor{textcolor}%
\pgfsetfillcolor{textcolor}%
\pgftext[x=0.956027in,y=1.055088in,,base]{\color{textcolor}\rmfamily\fontsize{8.000000}{9.600000}\selectfont P(8)}%
\end{pgfscope}%
\begin{pgfscope}%
\definecolor{textcolor}{rgb}{0.000000,0.000000,0.000000}%
\pgfsetstrokecolor{textcolor}%
\pgfsetfillcolor{textcolor}%
\pgftext[x=3.310068in,y=3.817761in,,base]{\color{textcolor}\rmfamily\fontsize{16.800000}{20.160000}\selectfont Original Data and Interpolated Data (P-branch)}%
\end{pgfscope}%
\begin{pgfscope}%
\pgfsetbuttcap%
\pgfsetmiterjoin%
\definecolor{currentfill}{rgb}{1.000000,1.000000,1.000000}%
\pgfsetfillcolor{currentfill}%
\pgfsetfillopacity{0.800000}%
\pgfsetlinewidth{1.003750pt}%
\definecolor{currentstroke}{rgb}{0.800000,0.800000,0.800000}%
\pgfsetstrokecolor{currentstroke}%
\pgfsetstrokeopacity{0.800000}%
\pgfsetdash{}{0pt}%
\pgfpathmoveto{\pgfqpoint{0.931477in}{2.762790in}}%
\pgfpathlineto{\pgfqpoint{2.704421in}{2.762790in}}%
\pgfpathquadraticcurveto{\pgfqpoint{2.743310in}{2.762790in}}{\pgfqpoint{2.743310in}{2.801679in}}%
\pgfpathlineto{\pgfqpoint{2.743310in}{3.598317in}}%
\pgfpathquadraticcurveto{\pgfqpoint{2.743310in}{3.637206in}}{\pgfqpoint{2.704421in}{3.637206in}}%
\pgfpathlineto{\pgfqpoint{0.931477in}{3.637206in}}%
\pgfpathquadraticcurveto{\pgfqpoint{0.892588in}{3.637206in}}{\pgfqpoint{0.892588in}{3.598317in}}%
\pgfpathlineto{\pgfqpoint{0.892588in}{2.801679in}}%
\pgfpathquadraticcurveto{\pgfqpoint{0.892588in}{2.762790in}}{\pgfqpoint{0.931477in}{2.762790in}}%
\pgfpathlineto{\pgfqpoint{0.931477in}{2.762790in}}%
\pgfpathclose%
\pgfusepath{stroke,fill}%
\end{pgfscope}%
\begin{pgfscope}%
\pgfsetrectcap%
\pgfsetroundjoin%
\pgfsetlinewidth{0.501875pt}%
\definecolor{currentstroke}{rgb}{0.000000,0.000000,0.000000}%
\pgfsetstrokecolor{currentstroke}%
\pgfsetdash{}{0pt}%
\pgfpathmoveto{\pgfqpoint{0.970366in}{3.490595in}}%
\pgfpathlineto{\pgfqpoint{1.164810in}{3.490595in}}%
\pgfpathlineto{\pgfqpoint{1.359255in}{3.490595in}}%
\pgfusepath{stroke}%
\end{pgfscope}%
\begin{pgfscope}%
\definecolor{textcolor}{rgb}{0.000000,0.000000,0.000000}%
\pgfsetstrokecolor{textcolor}%
\pgfsetfillcolor{textcolor}%
\pgftext[x=1.514810in,y=3.422539in,left,base]{\color{textcolor}\rmfamily\fontsize{14.000000}{16.800000}\selectfont Original Data}%
\end{pgfscope}%
\begin{pgfscope}%
\pgfsetrectcap%
\pgfsetroundjoin%
\pgfsetlinewidth{1.003750pt}%
\definecolor{currentstroke}{rgb}{0.000000,0.000000,1.000000}%
\pgfsetstrokecolor{currentstroke}%
\pgfsetdash{}{0pt}%
\pgfpathmoveto{\pgfqpoint{0.970366in}{3.217401in}}%
\pgfpathlineto{\pgfqpoint{1.164810in}{3.217401in}}%
\pgfpathlineto{\pgfqpoint{1.359255in}{3.217401in}}%
\pgfusepath{stroke}%
\end{pgfscope}%
\begin{pgfscope}%
\definecolor{textcolor}{rgb}{0.000000,0.000000,0.000000}%
\pgfsetstrokecolor{textcolor}%
\pgfsetfillcolor{textcolor}%
\pgftext[x=1.514810in,y=3.149345in,left,base]{\color{textcolor}\rmfamily\fontsize{14.000000}{16.800000}\selectfont Interpolation}%
\end{pgfscope}%
\begin{pgfscope}%
\pgfsetbuttcap%
\pgfsetroundjoin%
\definecolor{currentfill}{rgb}{1.000000,0.000000,0.000000}%
\pgfsetfillcolor{currentfill}%
\pgfsetlinewidth{1.003750pt}%
\definecolor{currentstroke}{rgb}{1.000000,0.000000,0.000000}%
\pgfsetstrokecolor{currentstroke}%
\pgfsetdash{}{0pt}%
\pgfsys@defobject{currentmarker}{\pgfqpoint{-0.020833in}{-0.020833in}}{\pgfqpoint{0.020833in}{0.020833in}}{%
\pgfpathmoveto{\pgfqpoint{0.000000in}{-0.020833in}}%
\pgfpathcurveto{\pgfqpoint{0.005525in}{-0.020833in}}{\pgfqpoint{0.010825in}{-0.018638in}}{\pgfqpoint{0.014731in}{-0.014731in}}%
\pgfpathcurveto{\pgfqpoint{0.018638in}{-0.010825in}}{\pgfqpoint{0.020833in}{-0.005525in}}{\pgfqpoint{0.020833in}{0.000000in}}%
\pgfpathcurveto{\pgfqpoint{0.020833in}{0.005525in}}{\pgfqpoint{0.018638in}{0.010825in}}{\pgfqpoint{0.014731in}{0.014731in}}%
\pgfpathcurveto{\pgfqpoint{0.010825in}{0.018638in}}{\pgfqpoint{0.005525in}{0.020833in}}{\pgfqpoint{0.000000in}{0.020833in}}%
\pgfpathcurveto{\pgfqpoint{-0.005525in}{0.020833in}}{\pgfqpoint{-0.010825in}{0.018638in}}{\pgfqpoint{-0.014731in}{0.014731in}}%
\pgfpathcurveto{\pgfqpoint{-0.018638in}{0.010825in}}{\pgfqpoint{-0.020833in}{0.005525in}}{\pgfqpoint{-0.020833in}{0.000000in}}%
\pgfpathcurveto{\pgfqpoint{-0.020833in}{-0.005525in}}{\pgfqpoint{-0.018638in}{-0.010825in}}{\pgfqpoint{-0.014731in}{-0.014731in}}%
\pgfpathcurveto{\pgfqpoint{-0.010825in}{-0.018638in}}{\pgfqpoint{-0.005525in}{-0.020833in}}{\pgfqpoint{0.000000in}{-0.020833in}}%
\pgfpathlineto{\pgfqpoint{0.000000in}{-0.020833in}}%
\pgfpathclose%
\pgfusepath{stroke,fill}%
}%
\begin{pgfscope}%
\pgfsys@transformshift{1.164810in}{2.946345in}%
\pgfsys@useobject{currentmarker}{}%
\end{pgfscope}%
\end{pgfscope}%
\begin{pgfscope}%
\definecolor{textcolor}{rgb}{0.000000,0.000000,0.000000}%
\pgfsetstrokecolor{textcolor}%
\pgfsetfillcolor{textcolor}%
\pgftext[x=1.514810in,y=2.878290in,left,base]{\color{textcolor}\rmfamily\fontsize{14.000000}{16.800000}\selectfont Dips}%
\end{pgfscope}%
\end{pgfpicture}%
\makeatother%
\endgroup%
}
		\label{fig:original_interpolated_data_p_branch}
	\end{subfigure}
	\hspace{0.5cm}
	\begin{subfigure}{0.45\textwidth}
		\centering
		\scalebox{0.50}{%% Creator: Matplotlib, PGF backend
%%
%% To include the figure in your LaTeX document, write
%%   \input{<filename>.pgf}
%%
%% Make sure the required packages are loaded in your preamble
%%   \usepackage{pgf}
%%
%% Also ensure that all the required font packages are loaded; for instance,
%% the lmodern package is sometimes necessary when using math font.
%%   \usepackage{lmodern}
%%
%% Figures using additional raster images can only be included by \input if
%% they are in the same directory as the main LaTeX file. For loading figures
%% from other directories you can use the `import` package
%%   \usepackage{import}
%%
%% and then include the figures with
%%   \import{<path to file>}{<filename>.pgf}
%%
%% Matplotlib used the following preamble
%%   
%%   \usepackage{fontspec}
%%   \makeatletter\@ifpackageloaded{underscore}{}{\usepackage[strings]{underscore}}\makeatother
%%
\begingroup%
\makeatletter%
\begin{pgfpicture}%
\pgfpathrectangle{\pgfpointorigin}{\pgfqpoint{5.959117in}{4.092528in}}%
\pgfusepath{use as bounding box, clip}%
\begin{pgfscope}%
\pgfsetbuttcap%
\pgfsetmiterjoin%
\definecolor{currentfill}{rgb}{1.000000,1.000000,1.000000}%
\pgfsetfillcolor{currentfill}%
\pgfsetlinewidth{0.000000pt}%
\definecolor{currentstroke}{rgb}{1.000000,1.000000,1.000000}%
\pgfsetstrokecolor{currentstroke}%
\pgfsetdash{}{0pt}%
\pgfpathmoveto{\pgfqpoint{0.000000in}{-0.000000in}}%
\pgfpathlineto{\pgfqpoint{5.959117in}{-0.000000in}}%
\pgfpathlineto{\pgfqpoint{5.959117in}{4.092528in}}%
\pgfpathlineto{\pgfqpoint{0.000000in}{4.092528in}}%
\pgfpathlineto{\pgfqpoint{0.000000in}{-0.000000in}}%
\pgfpathclose%
\pgfusepath{fill}%
\end{pgfscope}%
\begin{pgfscope}%
\pgfsetbuttcap%
\pgfsetmiterjoin%
\definecolor{currentfill}{rgb}{1.000000,1.000000,1.000000}%
\pgfsetfillcolor{currentfill}%
\pgfsetlinewidth{0.000000pt}%
\definecolor{currentstroke}{rgb}{0.000000,0.000000,0.000000}%
\pgfsetstrokecolor{currentstroke}%
\pgfsetstrokeopacity{0.000000}%
\pgfsetdash{}{0pt}%
\pgfpathmoveto{\pgfqpoint{0.773588in}{0.646140in}}%
\pgfpathlineto{\pgfqpoint{5.802992in}{0.646140in}}%
\pgfpathlineto{\pgfqpoint{5.802992in}{3.734428in}}%
\pgfpathlineto{\pgfqpoint{0.773588in}{3.734428in}}%
\pgfpathlineto{\pgfqpoint{0.773588in}{0.646140in}}%
\pgfpathclose%
\pgfusepath{fill}%
\end{pgfscope}%
\begin{pgfscope}%
\pgfsetbuttcap%
\pgfsetroundjoin%
\definecolor{currentfill}{rgb}{0.000000,0.000000,0.000000}%
\pgfsetfillcolor{currentfill}%
\pgfsetlinewidth{0.803000pt}%
\definecolor{currentstroke}{rgb}{0.000000,0.000000,0.000000}%
\pgfsetstrokecolor{currentstroke}%
\pgfsetdash{}{0pt}%
\pgfsys@defobject{currentmarker}{\pgfqpoint{0.000000in}{-0.048611in}}{\pgfqpoint{0.000000in}{0.000000in}}{%
\pgfpathmoveto{\pgfqpoint{0.000000in}{0.000000in}}%
\pgfpathlineto{\pgfqpoint{0.000000in}{-0.048611in}}%
\pgfusepath{stroke,fill}%
}%
\begin{pgfscope}%
\pgfsys@transformshift{0.773588in}{0.646140in}%
\pgfsys@useobject{currentmarker}{}%
\end{pgfscope}%
\end{pgfscope}%
\begin{pgfscope}%
\definecolor{textcolor}{rgb}{0.000000,0.000000,0.000000}%
\pgfsetstrokecolor{textcolor}%
\pgfsetfillcolor{textcolor}%
\pgftext[x=0.773588in,y=0.548917in,,top]{\color{textcolor}\rmfamily\fontsize{14.000000}{16.800000}\selectfont \(\displaystyle {2700}\)}%
\end{pgfscope}%
\begin{pgfscope}%
\pgfsetbuttcap%
\pgfsetroundjoin%
\definecolor{currentfill}{rgb}{0.000000,0.000000,0.000000}%
\pgfsetfillcolor{currentfill}%
\pgfsetlinewidth{0.803000pt}%
\definecolor{currentstroke}{rgb}{0.000000,0.000000,0.000000}%
\pgfsetstrokecolor{currentstroke}%
\pgfsetdash{}{0pt}%
\pgfsys@defobject{currentmarker}{\pgfqpoint{0.000000in}{-0.048611in}}{\pgfqpoint{0.000000in}{0.000000in}}{%
\pgfpathmoveto{\pgfqpoint{0.000000in}{0.000000in}}%
\pgfpathlineto{\pgfqpoint{0.000000in}{-0.048611in}}%
\pgfusepath{stroke,fill}%
}%
\begin{pgfscope}%
\pgfsys@transformshift{1.472116in}{0.646140in}%
\pgfsys@useobject{currentmarker}{}%
\end{pgfscope}%
\end{pgfscope}%
\begin{pgfscope}%
\definecolor{textcolor}{rgb}{0.000000,0.000000,0.000000}%
\pgfsetstrokecolor{textcolor}%
\pgfsetfillcolor{textcolor}%
\pgftext[x=1.472116in,y=0.548917in,,top]{\color{textcolor}\rmfamily\fontsize{14.000000}{16.800000}\selectfont \(\displaystyle {2725}\)}%
\end{pgfscope}%
\begin{pgfscope}%
\pgfsetbuttcap%
\pgfsetroundjoin%
\definecolor{currentfill}{rgb}{0.000000,0.000000,0.000000}%
\pgfsetfillcolor{currentfill}%
\pgfsetlinewidth{0.803000pt}%
\definecolor{currentstroke}{rgb}{0.000000,0.000000,0.000000}%
\pgfsetstrokecolor{currentstroke}%
\pgfsetdash{}{0pt}%
\pgfsys@defobject{currentmarker}{\pgfqpoint{0.000000in}{-0.048611in}}{\pgfqpoint{0.000000in}{0.000000in}}{%
\pgfpathmoveto{\pgfqpoint{0.000000in}{0.000000in}}%
\pgfpathlineto{\pgfqpoint{0.000000in}{-0.048611in}}%
\pgfusepath{stroke,fill}%
}%
\begin{pgfscope}%
\pgfsys@transformshift{2.170645in}{0.646140in}%
\pgfsys@useobject{currentmarker}{}%
\end{pgfscope}%
\end{pgfscope}%
\begin{pgfscope}%
\definecolor{textcolor}{rgb}{0.000000,0.000000,0.000000}%
\pgfsetstrokecolor{textcolor}%
\pgfsetfillcolor{textcolor}%
\pgftext[x=2.170645in,y=0.548917in,,top]{\color{textcolor}\rmfamily\fontsize{14.000000}{16.800000}\selectfont \(\displaystyle {2750}\)}%
\end{pgfscope}%
\begin{pgfscope}%
\pgfsetbuttcap%
\pgfsetroundjoin%
\definecolor{currentfill}{rgb}{0.000000,0.000000,0.000000}%
\pgfsetfillcolor{currentfill}%
\pgfsetlinewidth{0.803000pt}%
\definecolor{currentstroke}{rgb}{0.000000,0.000000,0.000000}%
\pgfsetstrokecolor{currentstroke}%
\pgfsetdash{}{0pt}%
\pgfsys@defobject{currentmarker}{\pgfqpoint{0.000000in}{-0.048611in}}{\pgfqpoint{0.000000in}{0.000000in}}{%
\pgfpathmoveto{\pgfqpoint{0.000000in}{0.000000in}}%
\pgfpathlineto{\pgfqpoint{0.000000in}{-0.048611in}}%
\pgfusepath{stroke,fill}%
}%
\begin{pgfscope}%
\pgfsys@transformshift{2.869173in}{0.646140in}%
\pgfsys@useobject{currentmarker}{}%
\end{pgfscope}%
\end{pgfscope}%
\begin{pgfscope}%
\definecolor{textcolor}{rgb}{0.000000,0.000000,0.000000}%
\pgfsetstrokecolor{textcolor}%
\pgfsetfillcolor{textcolor}%
\pgftext[x=2.869173in,y=0.548917in,,top]{\color{textcolor}\rmfamily\fontsize{14.000000}{16.800000}\selectfont \(\displaystyle {2775}\)}%
\end{pgfscope}%
\begin{pgfscope}%
\pgfsetbuttcap%
\pgfsetroundjoin%
\definecolor{currentfill}{rgb}{0.000000,0.000000,0.000000}%
\pgfsetfillcolor{currentfill}%
\pgfsetlinewidth{0.803000pt}%
\definecolor{currentstroke}{rgb}{0.000000,0.000000,0.000000}%
\pgfsetstrokecolor{currentstroke}%
\pgfsetdash{}{0pt}%
\pgfsys@defobject{currentmarker}{\pgfqpoint{0.000000in}{-0.048611in}}{\pgfqpoint{0.000000in}{0.000000in}}{%
\pgfpathmoveto{\pgfqpoint{0.000000in}{0.000000in}}%
\pgfpathlineto{\pgfqpoint{0.000000in}{-0.048611in}}%
\pgfusepath{stroke,fill}%
}%
\begin{pgfscope}%
\pgfsys@transformshift{3.567701in}{0.646140in}%
\pgfsys@useobject{currentmarker}{}%
\end{pgfscope}%
\end{pgfscope}%
\begin{pgfscope}%
\definecolor{textcolor}{rgb}{0.000000,0.000000,0.000000}%
\pgfsetstrokecolor{textcolor}%
\pgfsetfillcolor{textcolor}%
\pgftext[x=3.567701in,y=0.548917in,,top]{\color{textcolor}\rmfamily\fontsize{14.000000}{16.800000}\selectfont \(\displaystyle {2800}\)}%
\end{pgfscope}%
\begin{pgfscope}%
\pgfsetbuttcap%
\pgfsetroundjoin%
\definecolor{currentfill}{rgb}{0.000000,0.000000,0.000000}%
\pgfsetfillcolor{currentfill}%
\pgfsetlinewidth{0.803000pt}%
\definecolor{currentstroke}{rgb}{0.000000,0.000000,0.000000}%
\pgfsetstrokecolor{currentstroke}%
\pgfsetdash{}{0pt}%
\pgfsys@defobject{currentmarker}{\pgfqpoint{0.000000in}{-0.048611in}}{\pgfqpoint{0.000000in}{0.000000in}}{%
\pgfpathmoveto{\pgfqpoint{0.000000in}{0.000000in}}%
\pgfpathlineto{\pgfqpoint{0.000000in}{-0.048611in}}%
\pgfusepath{stroke,fill}%
}%
\begin{pgfscope}%
\pgfsys@transformshift{4.266229in}{0.646140in}%
\pgfsys@useobject{currentmarker}{}%
\end{pgfscope}%
\end{pgfscope}%
\begin{pgfscope}%
\definecolor{textcolor}{rgb}{0.000000,0.000000,0.000000}%
\pgfsetstrokecolor{textcolor}%
\pgfsetfillcolor{textcolor}%
\pgftext[x=4.266229in,y=0.548917in,,top]{\color{textcolor}\rmfamily\fontsize{14.000000}{16.800000}\selectfont \(\displaystyle {2825}\)}%
\end{pgfscope}%
\begin{pgfscope}%
\pgfsetbuttcap%
\pgfsetroundjoin%
\definecolor{currentfill}{rgb}{0.000000,0.000000,0.000000}%
\pgfsetfillcolor{currentfill}%
\pgfsetlinewidth{0.803000pt}%
\definecolor{currentstroke}{rgb}{0.000000,0.000000,0.000000}%
\pgfsetstrokecolor{currentstroke}%
\pgfsetdash{}{0pt}%
\pgfsys@defobject{currentmarker}{\pgfqpoint{0.000000in}{-0.048611in}}{\pgfqpoint{0.000000in}{0.000000in}}{%
\pgfpathmoveto{\pgfqpoint{0.000000in}{0.000000in}}%
\pgfpathlineto{\pgfqpoint{0.000000in}{-0.048611in}}%
\pgfusepath{stroke,fill}%
}%
\begin{pgfscope}%
\pgfsys@transformshift{4.964758in}{0.646140in}%
\pgfsys@useobject{currentmarker}{}%
\end{pgfscope}%
\end{pgfscope}%
\begin{pgfscope}%
\definecolor{textcolor}{rgb}{0.000000,0.000000,0.000000}%
\pgfsetstrokecolor{textcolor}%
\pgfsetfillcolor{textcolor}%
\pgftext[x=4.964758in,y=0.548917in,,top]{\color{textcolor}\rmfamily\fontsize{14.000000}{16.800000}\selectfont \(\displaystyle {2850}\)}%
\end{pgfscope}%
\begin{pgfscope}%
\pgfsetbuttcap%
\pgfsetroundjoin%
\definecolor{currentfill}{rgb}{0.000000,0.000000,0.000000}%
\pgfsetfillcolor{currentfill}%
\pgfsetlinewidth{0.803000pt}%
\definecolor{currentstroke}{rgb}{0.000000,0.000000,0.000000}%
\pgfsetstrokecolor{currentstroke}%
\pgfsetdash{}{0pt}%
\pgfsys@defobject{currentmarker}{\pgfqpoint{0.000000in}{-0.048611in}}{\pgfqpoint{0.000000in}{0.000000in}}{%
\pgfpathmoveto{\pgfqpoint{0.000000in}{0.000000in}}%
\pgfpathlineto{\pgfqpoint{0.000000in}{-0.048611in}}%
\pgfusepath{stroke,fill}%
}%
\begin{pgfscope}%
\pgfsys@transformshift{5.663286in}{0.646140in}%
\pgfsys@useobject{currentmarker}{}%
\end{pgfscope}%
\end{pgfscope}%
\begin{pgfscope}%
\definecolor{textcolor}{rgb}{0.000000,0.000000,0.000000}%
\pgfsetstrokecolor{textcolor}%
\pgfsetfillcolor{textcolor}%
\pgftext[x=5.663286in,y=0.548917in,,top]{\color{textcolor}\rmfamily\fontsize{14.000000}{16.800000}\selectfont \(\displaystyle {2875}\)}%
\end{pgfscope}%
\begin{pgfscope}%
\definecolor{textcolor}{rgb}{0.000000,0.000000,0.000000}%
\pgfsetstrokecolor{textcolor}%
\pgfsetfillcolor{textcolor}%
\pgftext[x=3.288290in,y=0.320695in,,top]{\color{textcolor}\rmfamily\fontsize{14.000000}{16.800000}\selectfont Wavenumber [cm\(\displaystyle ^{-1}\)]}%
\end{pgfscope}%
\begin{pgfscope}%
\pgfsetbuttcap%
\pgfsetroundjoin%
\definecolor{currentfill}{rgb}{0.000000,0.000000,0.000000}%
\pgfsetfillcolor{currentfill}%
\pgfsetlinewidth{0.803000pt}%
\definecolor{currentstroke}{rgb}{0.000000,0.000000,0.000000}%
\pgfsetstrokecolor{currentstroke}%
\pgfsetdash{}{0pt}%
\pgfsys@defobject{currentmarker}{\pgfqpoint{-0.048611in}{0.000000in}}{\pgfqpoint{-0.000000in}{0.000000in}}{%
\pgfpathmoveto{\pgfqpoint{-0.000000in}{0.000000in}}%
\pgfpathlineto{\pgfqpoint{-0.048611in}{0.000000in}}%
\pgfusepath{stroke,fill}%
}%
\begin{pgfscope}%
\pgfsys@transformshift{0.773588in}{0.646140in}%
\pgfsys@useobject{currentmarker}{}%
\end{pgfscope}%
\end{pgfscope}%
\begin{pgfscope}%
\definecolor{textcolor}{rgb}{0.000000,0.000000,0.000000}%
\pgfsetstrokecolor{textcolor}%
\pgfsetfillcolor{textcolor}%
\pgftext[x=0.328222in, y=0.578667in, left, base]{\color{textcolor}\rmfamily\fontsize{14.000000}{16.800000}\selectfont \(\displaystyle {0.75}\)}%
\end{pgfscope}%
\begin{pgfscope}%
\pgfsetbuttcap%
\pgfsetroundjoin%
\definecolor{currentfill}{rgb}{0.000000,0.000000,0.000000}%
\pgfsetfillcolor{currentfill}%
\pgfsetlinewidth{0.803000pt}%
\definecolor{currentstroke}{rgb}{0.000000,0.000000,0.000000}%
\pgfsetstrokecolor{currentstroke}%
\pgfsetdash{}{0pt}%
\pgfsys@defobject{currentmarker}{\pgfqpoint{-0.048611in}{0.000000in}}{\pgfqpoint{-0.000000in}{0.000000in}}{%
\pgfpathmoveto{\pgfqpoint{-0.000000in}{0.000000in}}%
\pgfpathlineto{\pgfqpoint{-0.048611in}{0.000000in}}%
\pgfusepath{stroke,fill}%
}%
\begin{pgfscope}%
\pgfsys@transformshift{0.773588in}{1.160854in}%
\pgfsys@useobject{currentmarker}{}%
\end{pgfscope}%
\end{pgfscope}%
\begin{pgfscope}%
\definecolor{textcolor}{rgb}{0.000000,0.000000,0.000000}%
\pgfsetstrokecolor{textcolor}%
\pgfsetfillcolor{textcolor}%
\pgftext[x=0.328222in, y=1.093382in, left, base]{\color{textcolor}\rmfamily\fontsize{14.000000}{16.800000}\selectfont \(\displaystyle {0.80}\)}%
\end{pgfscope}%
\begin{pgfscope}%
\pgfsetbuttcap%
\pgfsetroundjoin%
\definecolor{currentfill}{rgb}{0.000000,0.000000,0.000000}%
\pgfsetfillcolor{currentfill}%
\pgfsetlinewidth{0.803000pt}%
\definecolor{currentstroke}{rgb}{0.000000,0.000000,0.000000}%
\pgfsetstrokecolor{currentstroke}%
\pgfsetdash{}{0pt}%
\pgfsys@defobject{currentmarker}{\pgfqpoint{-0.048611in}{0.000000in}}{\pgfqpoint{-0.000000in}{0.000000in}}{%
\pgfpathmoveto{\pgfqpoint{-0.000000in}{0.000000in}}%
\pgfpathlineto{\pgfqpoint{-0.048611in}{0.000000in}}%
\pgfusepath{stroke,fill}%
}%
\begin{pgfscope}%
\pgfsys@transformshift{0.773588in}{1.675569in}%
\pgfsys@useobject{currentmarker}{}%
\end{pgfscope}%
\end{pgfscope}%
\begin{pgfscope}%
\definecolor{textcolor}{rgb}{0.000000,0.000000,0.000000}%
\pgfsetstrokecolor{textcolor}%
\pgfsetfillcolor{textcolor}%
\pgftext[x=0.328222in, y=1.608097in, left, base]{\color{textcolor}\rmfamily\fontsize{14.000000}{16.800000}\selectfont \(\displaystyle {0.85}\)}%
\end{pgfscope}%
\begin{pgfscope}%
\pgfsetbuttcap%
\pgfsetroundjoin%
\definecolor{currentfill}{rgb}{0.000000,0.000000,0.000000}%
\pgfsetfillcolor{currentfill}%
\pgfsetlinewidth{0.803000pt}%
\definecolor{currentstroke}{rgb}{0.000000,0.000000,0.000000}%
\pgfsetstrokecolor{currentstroke}%
\pgfsetdash{}{0pt}%
\pgfsys@defobject{currentmarker}{\pgfqpoint{-0.048611in}{0.000000in}}{\pgfqpoint{-0.000000in}{0.000000in}}{%
\pgfpathmoveto{\pgfqpoint{-0.000000in}{0.000000in}}%
\pgfpathlineto{\pgfqpoint{-0.048611in}{0.000000in}}%
\pgfusepath{stroke,fill}%
}%
\begin{pgfscope}%
\pgfsys@transformshift{0.773588in}{2.190284in}%
\pgfsys@useobject{currentmarker}{}%
\end{pgfscope}%
\end{pgfscope}%
\begin{pgfscope}%
\definecolor{textcolor}{rgb}{0.000000,0.000000,0.000000}%
\pgfsetstrokecolor{textcolor}%
\pgfsetfillcolor{textcolor}%
\pgftext[x=0.328222in, y=2.122812in, left, base]{\color{textcolor}\rmfamily\fontsize{14.000000}{16.800000}\selectfont \(\displaystyle {0.90}\)}%
\end{pgfscope}%
\begin{pgfscope}%
\pgfsetbuttcap%
\pgfsetroundjoin%
\definecolor{currentfill}{rgb}{0.000000,0.000000,0.000000}%
\pgfsetfillcolor{currentfill}%
\pgfsetlinewidth{0.803000pt}%
\definecolor{currentstroke}{rgb}{0.000000,0.000000,0.000000}%
\pgfsetstrokecolor{currentstroke}%
\pgfsetdash{}{0pt}%
\pgfsys@defobject{currentmarker}{\pgfqpoint{-0.048611in}{0.000000in}}{\pgfqpoint{-0.000000in}{0.000000in}}{%
\pgfpathmoveto{\pgfqpoint{-0.000000in}{0.000000in}}%
\pgfpathlineto{\pgfqpoint{-0.048611in}{0.000000in}}%
\pgfusepath{stroke,fill}%
}%
\begin{pgfscope}%
\pgfsys@transformshift{0.773588in}{2.704999in}%
\pgfsys@useobject{currentmarker}{}%
\end{pgfscope}%
\end{pgfscope}%
\begin{pgfscope}%
\definecolor{textcolor}{rgb}{0.000000,0.000000,0.000000}%
\pgfsetstrokecolor{textcolor}%
\pgfsetfillcolor{textcolor}%
\pgftext[x=0.328222in, y=2.637526in, left, base]{\color{textcolor}\rmfamily\fontsize{14.000000}{16.800000}\selectfont \(\displaystyle {0.95}\)}%
\end{pgfscope}%
\begin{pgfscope}%
\pgfsetbuttcap%
\pgfsetroundjoin%
\definecolor{currentfill}{rgb}{0.000000,0.000000,0.000000}%
\pgfsetfillcolor{currentfill}%
\pgfsetlinewidth{0.803000pt}%
\definecolor{currentstroke}{rgb}{0.000000,0.000000,0.000000}%
\pgfsetstrokecolor{currentstroke}%
\pgfsetdash{}{0pt}%
\pgfsys@defobject{currentmarker}{\pgfqpoint{-0.048611in}{0.000000in}}{\pgfqpoint{-0.000000in}{0.000000in}}{%
\pgfpathmoveto{\pgfqpoint{-0.000000in}{0.000000in}}%
\pgfpathlineto{\pgfqpoint{-0.048611in}{0.000000in}}%
\pgfusepath{stroke,fill}%
}%
\begin{pgfscope}%
\pgfsys@transformshift{0.773588in}{3.219713in}%
\pgfsys@useobject{currentmarker}{}%
\end{pgfscope}%
\end{pgfscope}%
\begin{pgfscope}%
\definecolor{textcolor}{rgb}{0.000000,0.000000,0.000000}%
\pgfsetstrokecolor{textcolor}%
\pgfsetfillcolor{textcolor}%
\pgftext[x=0.328222in, y=3.152241in, left, base]{\color{textcolor}\rmfamily\fontsize{14.000000}{16.800000}\selectfont \(\displaystyle {1.00}\)}%
\end{pgfscope}%
\begin{pgfscope}%
\pgfsetbuttcap%
\pgfsetroundjoin%
\definecolor{currentfill}{rgb}{0.000000,0.000000,0.000000}%
\pgfsetfillcolor{currentfill}%
\pgfsetlinewidth{0.803000pt}%
\definecolor{currentstroke}{rgb}{0.000000,0.000000,0.000000}%
\pgfsetstrokecolor{currentstroke}%
\pgfsetdash{}{0pt}%
\pgfsys@defobject{currentmarker}{\pgfqpoint{-0.048611in}{0.000000in}}{\pgfqpoint{-0.000000in}{0.000000in}}{%
\pgfpathmoveto{\pgfqpoint{-0.000000in}{0.000000in}}%
\pgfpathlineto{\pgfqpoint{-0.048611in}{0.000000in}}%
\pgfusepath{stroke,fill}%
}%
\begin{pgfscope}%
\pgfsys@transformshift{0.773588in}{3.734428in}%
\pgfsys@useobject{currentmarker}{}%
\end{pgfscope}%
\end{pgfscope}%
\begin{pgfscope}%
\definecolor{textcolor}{rgb}{0.000000,0.000000,0.000000}%
\pgfsetstrokecolor{textcolor}%
\pgfsetfillcolor{textcolor}%
\pgftext[x=0.328222in, y=3.666956in, left, base]{\color{textcolor}\rmfamily\fontsize{14.000000}{16.800000}\selectfont \(\displaystyle {1.05}\)}%
\end{pgfscope}%
\begin{pgfscope}%
\definecolor{textcolor}{rgb}{0.000000,0.000000,0.000000}%
\pgfsetstrokecolor{textcolor}%
\pgfsetfillcolor{textcolor}%
\pgftext[x=0.272667in,y=2.190284in,,bottom,rotate=90.000000]{\color{textcolor}\rmfamily\fontsize{14.000000}{16.800000}\selectfont Intensity}%
\end{pgfscope}%
\begin{pgfscope}%
\pgfpathrectangle{\pgfqpoint{0.773588in}{0.646140in}}{\pgfqpoint{5.029404in}{3.088289in}}%
\pgfusepath{clip}%
\pgfsetrectcap%
\pgfsetroundjoin%
\pgfsetlinewidth{0.501875pt}%
\definecolor{currentstroke}{rgb}{0.000000,0.000000,0.000000}%
\pgfsetstrokecolor{currentstroke}%
\pgfsetdash{}{0pt}%
\pgfpathmoveto{\pgfqpoint{0.773588in}{3.425456in}}%
\pgfpathlineto{\pgfqpoint{0.777081in}{3.403299in}}%
\pgfpathlineto{\pgfqpoint{0.784066in}{3.333562in}}%
\pgfpathlineto{\pgfqpoint{0.787559in}{3.316755in}}%
\pgfpathlineto{\pgfqpoint{0.791051in}{3.316372in}}%
\pgfpathlineto{\pgfqpoint{0.794544in}{3.321992in}}%
\pgfpathlineto{\pgfqpoint{0.798037in}{3.324676in}}%
\pgfpathlineto{\pgfqpoint{0.801529in}{3.322599in}}%
\pgfpathlineto{\pgfqpoint{0.805022in}{3.316569in}}%
\pgfpathlineto{\pgfqpoint{0.812007in}{3.292897in}}%
\pgfpathlineto{\pgfqpoint{0.815500in}{3.281755in}}%
\pgfpathlineto{\pgfqpoint{0.818992in}{3.277778in}}%
\pgfpathlineto{\pgfqpoint{0.822485in}{3.281888in}}%
\pgfpathlineto{\pgfqpoint{0.825978in}{3.292653in}}%
\pgfpathlineto{\pgfqpoint{0.836456in}{3.335657in}}%
\pgfpathlineto{\pgfqpoint{0.839948in}{3.335801in}}%
\pgfpathlineto{\pgfqpoint{0.846934in}{3.318305in}}%
\pgfpathlineto{\pgfqpoint{0.853919in}{3.295290in}}%
\pgfpathlineto{\pgfqpoint{0.864397in}{3.243299in}}%
\pgfpathlineto{\pgfqpoint{0.867889in}{3.222702in}}%
\pgfpathlineto{\pgfqpoint{0.871382in}{3.192957in}}%
\pgfpathlineto{\pgfqpoint{0.878367in}{3.114547in}}%
\pgfpathlineto{\pgfqpoint{0.881860in}{3.096131in}}%
\pgfpathlineto{\pgfqpoint{0.885353in}{3.112798in}}%
\pgfpathlineto{\pgfqpoint{0.895831in}{3.251642in}}%
\pgfpathlineto{\pgfqpoint{0.899323in}{3.262882in}}%
\pgfpathlineto{\pgfqpoint{0.902816in}{3.260775in}}%
\pgfpathlineto{\pgfqpoint{0.906308in}{3.264113in}}%
\pgfpathlineto{\pgfqpoint{0.913294in}{3.301827in}}%
\pgfpathlineto{\pgfqpoint{0.916786in}{3.296071in}}%
\pgfpathlineto{\pgfqpoint{0.920279in}{3.233123in}}%
\pgfpathlineto{\pgfqpoint{0.930757in}{2.850703in}}%
\pgfpathlineto{\pgfqpoint{0.934250in}{2.839081in}}%
\pgfpathlineto{\pgfqpoint{0.937742in}{2.915603in}}%
\pgfpathlineto{\pgfqpoint{0.944727in}{3.108488in}}%
\pgfpathlineto{\pgfqpoint{0.948220in}{3.147899in}}%
\pgfpathlineto{\pgfqpoint{0.958698in}{3.185973in}}%
\pgfpathlineto{\pgfqpoint{0.962191in}{3.202429in}}%
\pgfpathlineto{\pgfqpoint{0.965683in}{3.211347in}}%
\pgfpathlineto{\pgfqpoint{0.969176in}{3.211405in}}%
\pgfpathlineto{\pgfqpoint{0.972669in}{3.209314in}}%
\pgfpathlineto{\pgfqpoint{0.976161in}{3.216077in}}%
\pgfpathlineto{\pgfqpoint{0.979654in}{3.238846in}}%
\pgfpathlineto{\pgfqpoint{0.986639in}{3.307406in}}%
\pgfpathlineto{\pgfqpoint{0.990132in}{3.329478in}}%
\pgfpathlineto{\pgfqpoint{0.993624in}{3.338616in}}%
\pgfpathlineto{\pgfqpoint{0.997117in}{3.338915in}}%
\pgfpathlineto{\pgfqpoint{1.000610in}{3.331849in}}%
\pgfpathlineto{\pgfqpoint{1.004102in}{3.315848in}}%
\pgfpathlineto{\pgfqpoint{1.014580in}{3.244759in}}%
\pgfpathlineto{\pgfqpoint{1.025058in}{3.203341in}}%
\pgfpathlineto{\pgfqpoint{1.032044in}{3.147141in}}%
\pgfpathlineto{\pgfqpoint{1.035536in}{3.140006in}}%
\pgfpathlineto{\pgfqpoint{1.039029in}{3.162327in}}%
\pgfpathlineto{\pgfqpoint{1.046014in}{3.232665in}}%
\pgfpathlineto{\pgfqpoint{1.049507in}{3.249722in}}%
\pgfpathlineto{\pgfqpoint{1.056492in}{3.265553in}}%
\pgfpathlineto{\pgfqpoint{1.059985in}{3.268633in}}%
\pgfpathlineto{\pgfqpoint{1.063477in}{3.264442in}}%
\pgfpathlineto{\pgfqpoint{1.066970in}{3.256908in}}%
\pgfpathlineto{\pgfqpoint{1.070463in}{3.252743in}}%
\pgfpathlineto{\pgfqpoint{1.077448in}{3.253095in}}%
\pgfpathlineto{\pgfqpoint{1.080940in}{3.246855in}}%
\pgfpathlineto{\pgfqpoint{1.084433in}{3.231225in}}%
\pgfpathlineto{\pgfqpoint{1.091418in}{3.185846in}}%
\pgfpathlineto{\pgfqpoint{1.094911in}{3.172938in}}%
\pgfpathlineto{\pgfqpoint{1.098404in}{3.170815in}}%
\pgfpathlineto{\pgfqpoint{1.101896in}{3.172608in}}%
\pgfpathlineto{\pgfqpoint{1.105389in}{3.172865in}}%
\pgfpathlineto{\pgfqpoint{1.108882in}{3.174764in}}%
\pgfpathlineto{\pgfqpoint{1.112374in}{3.185154in}}%
\pgfpathlineto{\pgfqpoint{1.129837in}{3.280706in}}%
\pgfpathlineto{\pgfqpoint{1.133330in}{3.290442in}}%
\pgfpathlineto{\pgfqpoint{1.136823in}{3.295594in}}%
\pgfpathlineto{\pgfqpoint{1.140315in}{3.294088in}}%
\pgfpathlineto{\pgfqpoint{1.143808in}{3.285273in}}%
\pgfpathlineto{\pgfqpoint{1.150793in}{3.253966in}}%
\pgfpathlineto{\pgfqpoint{1.161271in}{3.201356in}}%
\pgfpathlineto{\pgfqpoint{1.164764in}{3.197131in}}%
\pgfpathlineto{\pgfqpoint{1.168257in}{3.205259in}}%
\pgfpathlineto{\pgfqpoint{1.175242in}{3.232869in}}%
\pgfpathlineto{\pgfqpoint{1.178734in}{3.235682in}}%
\pgfpathlineto{\pgfqpoint{1.182227in}{3.228206in}}%
\pgfpathlineto{\pgfqpoint{1.185720in}{3.217397in}}%
\pgfpathlineto{\pgfqpoint{1.189212in}{3.211249in}}%
\pgfpathlineto{\pgfqpoint{1.192705in}{3.209020in}}%
\pgfpathlineto{\pgfqpoint{1.196198in}{3.201984in}}%
\pgfpathlineto{\pgfqpoint{1.199690in}{3.188112in}}%
\pgfpathlineto{\pgfqpoint{1.203183in}{3.180547in}}%
\pgfpathlineto{\pgfqpoint{1.206676in}{3.192512in}}%
\pgfpathlineto{\pgfqpoint{1.210168in}{3.216936in}}%
\pgfpathlineto{\pgfqpoint{1.213661in}{3.232779in}}%
\pgfpathlineto{\pgfqpoint{1.217154in}{3.231896in}}%
\pgfpathlineto{\pgfqpoint{1.220646in}{3.226987in}}%
\pgfpathlineto{\pgfqpoint{1.227631in}{3.231848in}}%
\pgfpathlineto{\pgfqpoint{1.231124in}{3.223448in}}%
\pgfpathlineto{\pgfqpoint{1.238109in}{3.193368in}}%
\pgfpathlineto{\pgfqpoint{1.245095in}{3.169938in}}%
\pgfpathlineto{\pgfqpoint{1.248587in}{3.157081in}}%
\pgfpathlineto{\pgfqpoint{1.252080in}{3.156147in}}%
\pgfpathlineto{\pgfqpoint{1.262558in}{3.200402in}}%
\pgfpathlineto{\pgfqpoint{1.269543in}{3.213215in}}%
\pgfpathlineto{\pgfqpoint{1.276528in}{3.234586in}}%
\pgfpathlineto{\pgfqpoint{1.287006in}{3.282724in}}%
\pgfpathlineto{\pgfqpoint{1.290499in}{3.286873in}}%
\pgfpathlineto{\pgfqpoint{1.293992in}{3.275614in}}%
\pgfpathlineto{\pgfqpoint{1.304470in}{3.198935in}}%
\pgfpathlineto{\pgfqpoint{1.307962in}{3.200751in}}%
\pgfpathlineto{\pgfqpoint{1.311455in}{3.213312in}}%
\pgfpathlineto{\pgfqpoint{1.314947in}{3.213881in}}%
\pgfpathlineto{\pgfqpoint{1.318440in}{3.191716in}}%
\pgfpathlineto{\pgfqpoint{1.321933in}{3.159530in}}%
\pgfpathlineto{\pgfqpoint{1.325425in}{3.138943in}}%
\pgfpathlineto{\pgfqpoint{1.328918in}{3.140507in}}%
\pgfpathlineto{\pgfqpoint{1.339396in}{3.200817in}}%
\pgfpathlineto{\pgfqpoint{1.342889in}{3.204204in}}%
\pgfpathlineto{\pgfqpoint{1.346381in}{3.200420in}}%
\pgfpathlineto{\pgfqpoint{1.349874in}{3.203320in}}%
\pgfpathlineto{\pgfqpoint{1.360352in}{3.250904in}}%
\pgfpathlineto{\pgfqpoint{1.367337in}{3.264448in}}%
\pgfpathlineto{\pgfqpoint{1.370830in}{3.263593in}}%
\pgfpathlineto{\pgfqpoint{1.374322in}{3.244779in}}%
\pgfpathlineto{\pgfqpoint{1.381308in}{3.184560in}}%
\pgfpathlineto{\pgfqpoint{1.384800in}{3.181145in}}%
\pgfpathlineto{\pgfqpoint{1.395278in}{3.231079in}}%
\pgfpathlineto{\pgfqpoint{1.398771in}{3.225320in}}%
\pgfpathlineto{\pgfqpoint{1.402264in}{3.202367in}}%
\pgfpathlineto{\pgfqpoint{1.409249in}{3.138193in}}%
\pgfpathlineto{\pgfqpoint{1.412741in}{3.128520in}}%
\pgfpathlineto{\pgfqpoint{1.416234in}{3.145138in}}%
\pgfpathlineto{\pgfqpoint{1.423219in}{3.208093in}}%
\pgfpathlineto{\pgfqpoint{1.426712in}{3.227493in}}%
\pgfpathlineto{\pgfqpoint{1.437190in}{3.260553in}}%
\pgfpathlineto{\pgfqpoint{1.440683in}{3.264601in}}%
\pgfpathlineto{\pgfqpoint{1.451160in}{3.250581in}}%
\pgfpathlineto{\pgfqpoint{1.454653in}{3.243847in}}%
\pgfpathlineto{\pgfqpoint{1.458146in}{3.220163in}}%
\pgfpathlineto{\pgfqpoint{1.465131in}{3.149712in}}%
\pgfpathlineto{\pgfqpoint{1.468624in}{3.142924in}}%
\pgfpathlineto{\pgfqpoint{1.475609in}{3.166223in}}%
\pgfpathlineto{\pgfqpoint{1.479102in}{3.164192in}}%
\pgfpathlineto{\pgfqpoint{1.482594in}{3.157292in}}%
\pgfpathlineto{\pgfqpoint{1.486087in}{3.156061in}}%
\pgfpathlineto{\pgfqpoint{1.489580in}{3.160558in}}%
\pgfpathlineto{\pgfqpoint{1.493072in}{3.167205in}}%
\pgfpathlineto{\pgfqpoint{1.496565in}{3.178759in}}%
\pgfpathlineto{\pgfqpoint{1.507043in}{3.242248in}}%
\pgfpathlineto{\pgfqpoint{1.510535in}{3.249110in}}%
\pgfpathlineto{\pgfqpoint{1.514028in}{3.247773in}}%
\pgfpathlineto{\pgfqpoint{1.521013in}{3.237624in}}%
\pgfpathlineto{\pgfqpoint{1.524506in}{3.236304in}}%
\pgfpathlineto{\pgfqpoint{1.527999in}{3.242439in}}%
\pgfpathlineto{\pgfqpoint{1.531491in}{3.252033in}}%
\pgfpathlineto{\pgfqpoint{1.534984in}{3.254311in}}%
\pgfpathlineto{\pgfqpoint{1.538477in}{3.241619in}}%
\pgfpathlineto{\pgfqpoint{1.552447in}{3.139350in}}%
\pgfpathlineto{\pgfqpoint{1.555940in}{3.133344in}}%
\pgfpathlineto{\pgfqpoint{1.559432in}{3.122449in}}%
\pgfpathlineto{\pgfqpoint{1.562925in}{3.082471in}}%
\pgfpathlineto{\pgfqpoint{1.569910in}{2.931707in}}%
\pgfpathlineto{\pgfqpoint{1.573403in}{2.907773in}}%
\pgfpathlineto{\pgfqpoint{1.576896in}{2.965621in}}%
\pgfpathlineto{\pgfqpoint{1.583881in}{3.190499in}}%
\pgfpathlineto{\pgfqpoint{1.587374in}{3.248564in}}%
\pgfpathlineto{\pgfqpoint{1.590866in}{3.253404in}}%
\pgfpathlineto{\pgfqpoint{1.594359in}{3.237527in}}%
\pgfpathlineto{\pgfqpoint{1.597851in}{3.228358in}}%
\pgfpathlineto{\pgfqpoint{1.601344in}{3.227640in}}%
\pgfpathlineto{\pgfqpoint{1.604837in}{3.220208in}}%
\pgfpathlineto{\pgfqpoint{1.608329in}{3.184733in}}%
\pgfpathlineto{\pgfqpoint{1.611822in}{3.092031in}}%
\pgfpathlineto{\pgfqpoint{1.615315in}{2.917791in}}%
\pgfpathlineto{\pgfqpoint{1.622300in}{2.489567in}}%
\pgfpathlineto{\pgfqpoint{1.625793in}{2.444707in}}%
\pgfpathlineto{\pgfqpoint{1.629285in}{2.584368in}}%
\pgfpathlineto{\pgfqpoint{1.636270in}{3.040480in}}%
\pgfpathlineto{\pgfqpoint{1.639763in}{3.163222in}}%
\pgfpathlineto{\pgfqpoint{1.643256in}{3.210116in}}%
\pgfpathlineto{\pgfqpoint{1.650241in}{3.237923in}}%
\pgfpathlineto{\pgfqpoint{1.660719in}{3.267774in}}%
\pgfpathlineto{\pgfqpoint{1.664212in}{3.263242in}}%
\pgfpathlineto{\pgfqpoint{1.671197in}{3.224327in}}%
\pgfpathlineto{\pgfqpoint{1.674690in}{3.214820in}}%
\pgfpathlineto{\pgfqpoint{1.678182in}{3.215195in}}%
\pgfpathlineto{\pgfqpoint{1.681675in}{3.214258in}}%
\pgfpathlineto{\pgfqpoint{1.685167in}{3.205466in}}%
\pgfpathlineto{\pgfqpoint{1.692153in}{3.182294in}}%
\pgfpathlineto{\pgfqpoint{1.695645in}{3.181490in}}%
\pgfpathlineto{\pgfqpoint{1.702631in}{3.200543in}}%
\pgfpathlineto{\pgfqpoint{1.706123in}{3.198589in}}%
\pgfpathlineto{\pgfqpoint{1.709616in}{3.178049in}}%
\pgfpathlineto{\pgfqpoint{1.713109in}{3.148881in}}%
\pgfpathlineto{\pgfqpoint{1.716601in}{3.130160in}}%
\pgfpathlineto{\pgfqpoint{1.720094in}{3.133267in}}%
\pgfpathlineto{\pgfqpoint{1.727079in}{3.181195in}}%
\pgfpathlineto{\pgfqpoint{1.734064in}{3.226863in}}%
\pgfpathlineto{\pgfqpoint{1.741050in}{3.259814in}}%
\pgfpathlineto{\pgfqpoint{1.744542in}{3.265883in}}%
\pgfpathlineto{\pgfqpoint{1.748035in}{3.264017in}}%
\pgfpathlineto{\pgfqpoint{1.755020in}{3.254908in}}%
\pgfpathlineto{\pgfqpoint{1.758513in}{3.247788in}}%
\pgfpathlineto{\pgfqpoint{1.768991in}{3.211589in}}%
\pgfpathlineto{\pgfqpoint{1.775976in}{3.206327in}}%
\pgfpathlineto{\pgfqpoint{1.782961in}{3.195168in}}%
\pgfpathlineto{\pgfqpoint{1.786454in}{3.193669in}}%
\pgfpathlineto{\pgfqpoint{1.793439in}{3.195528in}}%
\pgfpathlineto{\pgfqpoint{1.796932in}{3.201525in}}%
\pgfpathlineto{\pgfqpoint{1.817888in}{3.274083in}}%
\pgfpathlineto{\pgfqpoint{1.821380in}{3.276204in}}%
\pgfpathlineto{\pgfqpoint{1.824873in}{3.266888in}}%
\pgfpathlineto{\pgfqpoint{1.831858in}{3.237292in}}%
\pgfpathlineto{\pgfqpoint{1.835351in}{3.227398in}}%
\pgfpathlineto{\pgfqpoint{1.838844in}{3.221727in}}%
\pgfpathlineto{\pgfqpoint{1.842336in}{3.218620in}}%
\pgfpathlineto{\pgfqpoint{1.845829in}{3.217601in}}%
\pgfpathlineto{\pgfqpoint{1.849322in}{3.220571in}}%
\pgfpathlineto{\pgfqpoint{1.856307in}{3.236380in}}%
\pgfpathlineto{\pgfqpoint{1.859800in}{3.237514in}}%
\pgfpathlineto{\pgfqpoint{1.866785in}{3.228489in}}%
\pgfpathlineto{\pgfqpoint{1.870277in}{3.233066in}}%
\pgfpathlineto{\pgfqpoint{1.877263in}{3.248552in}}%
\pgfpathlineto{\pgfqpoint{1.884248in}{3.255813in}}%
\pgfpathlineto{\pgfqpoint{1.887741in}{3.257063in}}%
\pgfpathlineto{\pgfqpoint{1.891233in}{3.252855in}}%
\pgfpathlineto{\pgfqpoint{1.898219in}{3.239283in}}%
\pgfpathlineto{\pgfqpoint{1.901711in}{3.234015in}}%
\pgfpathlineto{\pgfqpoint{1.905204in}{3.225237in}}%
\pgfpathlineto{\pgfqpoint{1.912189in}{3.202676in}}%
\pgfpathlineto{\pgfqpoint{1.915682in}{3.196707in}}%
\pgfpathlineto{\pgfqpoint{1.919174in}{3.194167in}}%
\pgfpathlineto{\pgfqpoint{1.922667in}{3.194741in}}%
\pgfpathlineto{\pgfqpoint{1.936638in}{3.210921in}}%
\pgfpathlineto{\pgfqpoint{1.940130in}{3.224181in}}%
\pgfpathlineto{\pgfqpoint{1.947116in}{3.269186in}}%
\pgfpathlineto{\pgfqpoint{1.950608in}{3.281172in}}%
\pgfpathlineto{\pgfqpoint{1.957593in}{3.286020in}}%
\pgfpathlineto{\pgfqpoint{1.961086in}{3.293149in}}%
\pgfpathlineto{\pgfqpoint{1.968071in}{3.316844in}}%
\pgfpathlineto{\pgfqpoint{1.971564in}{3.319964in}}%
\pgfpathlineto{\pgfqpoint{1.975057in}{3.310160in}}%
\pgfpathlineto{\pgfqpoint{1.992520in}{3.215768in}}%
\pgfpathlineto{\pgfqpoint{1.996013in}{3.193806in}}%
\pgfpathlineto{\pgfqpoint{1.999505in}{3.188052in}}%
\pgfpathlineto{\pgfqpoint{2.009983in}{3.218346in}}%
\pgfpathlineto{\pgfqpoint{2.013476in}{3.222919in}}%
\pgfpathlineto{\pgfqpoint{2.016968in}{3.234058in}}%
\pgfpathlineto{\pgfqpoint{2.023954in}{3.273042in}}%
\pgfpathlineto{\pgfqpoint{2.027446in}{3.281109in}}%
\pgfpathlineto{\pgfqpoint{2.030939in}{3.274561in}}%
\pgfpathlineto{\pgfqpoint{2.034432in}{3.261730in}}%
\pgfpathlineto{\pgfqpoint{2.037924in}{3.255572in}}%
\pgfpathlineto{\pgfqpoint{2.041417in}{3.261694in}}%
\pgfpathlineto{\pgfqpoint{2.044910in}{3.273431in}}%
\pgfpathlineto{\pgfqpoint{2.048402in}{3.279085in}}%
\pgfpathlineto{\pgfqpoint{2.051895in}{3.272988in}}%
\pgfpathlineto{\pgfqpoint{2.065865in}{3.219072in}}%
\pgfpathlineto{\pgfqpoint{2.072851in}{3.194119in}}%
\pgfpathlineto{\pgfqpoint{2.076343in}{3.187247in}}%
\pgfpathlineto{\pgfqpoint{2.079836in}{3.195045in}}%
\pgfpathlineto{\pgfqpoint{2.083329in}{3.220119in}}%
\pgfpathlineto{\pgfqpoint{2.090314in}{3.280818in}}%
\pgfpathlineto{\pgfqpoint{2.093807in}{3.291346in}}%
\pgfpathlineto{\pgfqpoint{2.097299in}{3.286283in}}%
\pgfpathlineto{\pgfqpoint{2.104284in}{3.265214in}}%
\pgfpathlineto{\pgfqpoint{2.107777in}{3.263683in}}%
\pgfpathlineto{\pgfqpoint{2.111270in}{3.269627in}}%
\pgfpathlineto{\pgfqpoint{2.118255in}{3.287523in}}%
\pgfpathlineto{\pgfqpoint{2.121748in}{3.290790in}}%
\pgfpathlineto{\pgfqpoint{2.125240in}{3.286277in}}%
\pgfpathlineto{\pgfqpoint{2.128733in}{3.272782in}}%
\pgfpathlineto{\pgfqpoint{2.139211in}{3.210572in}}%
\pgfpathlineto{\pgfqpoint{2.142703in}{3.200284in}}%
\pgfpathlineto{\pgfqpoint{2.146196in}{3.201257in}}%
\pgfpathlineto{\pgfqpoint{2.149689in}{3.214946in}}%
\pgfpathlineto{\pgfqpoint{2.156674in}{3.263090in}}%
\pgfpathlineto{\pgfqpoint{2.160167in}{3.276979in}}%
\pgfpathlineto{\pgfqpoint{2.163659in}{3.276430in}}%
\pgfpathlineto{\pgfqpoint{2.170645in}{3.262199in}}%
\pgfpathlineto{\pgfqpoint{2.174137in}{3.263361in}}%
\pgfpathlineto{\pgfqpoint{2.177630in}{3.272082in}}%
\pgfpathlineto{\pgfqpoint{2.181123in}{3.289121in}}%
\pgfpathlineto{\pgfqpoint{2.188108in}{3.336738in}}%
\pgfpathlineto{\pgfqpoint{2.191600in}{3.344502in}}%
\pgfpathlineto{\pgfqpoint{2.195093in}{3.332440in}}%
\pgfpathlineto{\pgfqpoint{2.198586in}{3.312719in}}%
\pgfpathlineto{\pgfqpoint{2.202078in}{3.301021in}}%
\pgfpathlineto{\pgfqpoint{2.205571in}{3.298798in}}%
\pgfpathlineto{\pgfqpoint{2.209064in}{3.294773in}}%
\pgfpathlineto{\pgfqpoint{2.223034in}{3.243327in}}%
\pgfpathlineto{\pgfqpoint{2.233512in}{3.192972in}}%
\pgfpathlineto{\pgfqpoint{2.237005in}{3.149140in}}%
\pgfpathlineto{\pgfqpoint{2.240497in}{3.052487in}}%
\pgfpathlineto{\pgfqpoint{2.247483in}{2.800729in}}%
\pgfpathlineto{\pgfqpoint{2.250975in}{2.782476in}}%
\pgfpathlineto{\pgfqpoint{2.254468in}{2.880054in}}%
\pgfpathlineto{\pgfqpoint{2.261453in}{3.190300in}}%
\pgfpathlineto{\pgfqpoint{2.264946in}{3.276101in}}%
\pgfpathlineto{\pgfqpoint{2.268439in}{3.305058in}}%
\pgfpathlineto{\pgfqpoint{2.271931in}{3.307696in}}%
\pgfpathlineto{\pgfqpoint{2.275424in}{3.302864in}}%
\pgfpathlineto{\pgfqpoint{2.278917in}{3.291510in}}%
\pgfpathlineto{\pgfqpoint{2.282409in}{3.273146in}}%
\pgfpathlineto{\pgfqpoint{2.285902in}{3.242119in}}%
\pgfpathlineto{\pgfqpoint{2.289394in}{3.159754in}}%
\pgfpathlineto{\pgfqpoint{2.292887in}{2.961365in}}%
\pgfpathlineto{\pgfqpoint{2.299872in}{2.290503in}}%
\pgfpathlineto{\pgfqpoint{2.303365in}{2.119732in}}%
\pgfpathlineto{\pgfqpoint{2.306858in}{2.230898in}}%
\pgfpathlineto{\pgfqpoint{2.317336in}{3.128304in}}%
\pgfpathlineto{\pgfqpoint{2.320828in}{3.239347in}}%
\pgfpathlineto{\pgfqpoint{2.324321in}{3.283676in}}%
\pgfpathlineto{\pgfqpoint{2.327813in}{3.299192in}}%
\pgfpathlineto{\pgfqpoint{2.331306in}{3.294313in}}%
\pgfpathlineto{\pgfqpoint{2.334799in}{3.279621in}}%
\pgfpathlineto{\pgfqpoint{2.338291in}{3.273126in}}%
\pgfpathlineto{\pgfqpoint{2.341784in}{3.278406in}}%
\pgfpathlineto{\pgfqpoint{2.345277in}{3.280866in}}%
\pgfpathlineto{\pgfqpoint{2.348769in}{3.269533in}}%
\pgfpathlineto{\pgfqpoint{2.352262in}{3.251030in}}%
\pgfpathlineto{\pgfqpoint{2.355755in}{3.240169in}}%
\pgfpathlineto{\pgfqpoint{2.359247in}{3.244720in}}%
\pgfpathlineto{\pgfqpoint{2.366233in}{3.269679in}}%
\pgfpathlineto{\pgfqpoint{2.369725in}{3.260816in}}%
\pgfpathlineto{\pgfqpoint{2.376710in}{3.204426in}}%
\pgfpathlineto{\pgfqpoint{2.380203in}{3.193446in}}%
\pgfpathlineto{\pgfqpoint{2.383696in}{3.202302in}}%
\pgfpathlineto{\pgfqpoint{2.394174in}{3.258908in}}%
\pgfpathlineto{\pgfqpoint{2.411637in}{3.396061in}}%
\pgfpathlineto{\pgfqpoint{2.415130in}{3.399852in}}%
\pgfpathlineto{\pgfqpoint{2.418622in}{3.386227in}}%
\pgfpathlineto{\pgfqpoint{2.432593in}{3.280326in}}%
\pgfpathlineto{\pgfqpoint{2.439578in}{3.273978in}}%
\pgfpathlineto{\pgfqpoint{2.446563in}{3.250862in}}%
\pgfpathlineto{\pgfqpoint{2.450056in}{3.247472in}}%
\pgfpathlineto{\pgfqpoint{2.457041in}{3.270689in}}%
\pgfpathlineto{\pgfqpoint{2.460534in}{3.272116in}}%
\pgfpathlineto{\pgfqpoint{2.464026in}{3.264640in}}%
\pgfpathlineto{\pgfqpoint{2.467519in}{3.263753in}}%
\pgfpathlineto{\pgfqpoint{2.471012in}{3.279212in}}%
\pgfpathlineto{\pgfqpoint{2.477997in}{3.321098in}}%
\pgfpathlineto{\pgfqpoint{2.481490in}{3.323100in}}%
\pgfpathlineto{\pgfqpoint{2.488475in}{3.309565in}}%
\pgfpathlineto{\pgfqpoint{2.491968in}{3.308354in}}%
\pgfpathlineto{\pgfqpoint{2.495460in}{3.305244in}}%
\pgfpathlineto{\pgfqpoint{2.498953in}{3.294136in}}%
\pgfpathlineto{\pgfqpoint{2.505938in}{3.265473in}}%
\pgfpathlineto{\pgfqpoint{2.512923in}{3.244966in}}%
\pgfpathlineto{\pgfqpoint{2.519909in}{3.214921in}}%
\pgfpathlineto{\pgfqpoint{2.523401in}{3.213207in}}%
\pgfpathlineto{\pgfqpoint{2.526894in}{3.220642in}}%
\pgfpathlineto{\pgfqpoint{2.530387in}{3.225446in}}%
\pgfpathlineto{\pgfqpoint{2.537372in}{3.216976in}}%
\pgfpathlineto{\pgfqpoint{2.540865in}{3.224146in}}%
\pgfpathlineto{\pgfqpoint{2.551343in}{3.287010in}}%
\pgfpathlineto{\pgfqpoint{2.565313in}{3.321261in}}%
\pgfpathlineto{\pgfqpoint{2.568806in}{3.319165in}}%
\pgfpathlineto{\pgfqpoint{2.572298in}{3.306106in}}%
\pgfpathlineto{\pgfqpoint{2.586269in}{3.225288in}}%
\pgfpathlineto{\pgfqpoint{2.589762in}{3.219781in}}%
\pgfpathlineto{\pgfqpoint{2.596747in}{3.227081in}}%
\pgfpathlineto{\pgfqpoint{2.600240in}{3.226145in}}%
\pgfpathlineto{\pgfqpoint{2.607225in}{3.213663in}}%
\pgfpathlineto{\pgfqpoint{2.610717in}{3.215935in}}%
\pgfpathlineto{\pgfqpoint{2.614210in}{3.231465in}}%
\pgfpathlineto{\pgfqpoint{2.621195in}{3.283621in}}%
\pgfpathlineto{\pgfqpoint{2.624688in}{3.300257in}}%
\pgfpathlineto{\pgfqpoint{2.628181in}{3.304059in}}%
\pgfpathlineto{\pgfqpoint{2.631673in}{3.298568in}}%
\pgfpathlineto{\pgfqpoint{2.635166in}{3.290475in}}%
\pgfpathlineto{\pgfqpoint{2.638659in}{3.287059in}}%
\pgfpathlineto{\pgfqpoint{2.642151in}{3.292632in}}%
\pgfpathlineto{\pgfqpoint{2.645644in}{3.303595in}}%
\pgfpathlineto{\pgfqpoint{2.649136in}{3.310820in}}%
\pgfpathlineto{\pgfqpoint{2.656122in}{3.311189in}}%
\pgfpathlineto{\pgfqpoint{2.659614in}{3.315995in}}%
\pgfpathlineto{\pgfqpoint{2.663107in}{3.315684in}}%
\pgfpathlineto{\pgfqpoint{2.666600in}{3.296546in}}%
\pgfpathlineto{\pgfqpoint{2.673585in}{3.231311in}}%
\pgfpathlineto{\pgfqpoint{2.677078in}{3.219616in}}%
\pgfpathlineto{\pgfqpoint{2.680570in}{3.223821in}}%
\pgfpathlineto{\pgfqpoint{2.684063in}{3.235205in}}%
\pgfpathlineto{\pgfqpoint{2.691048in}{3.273717in}}%
\pgfpathlineto{\pgfqpoint{2.698033in}{3.309749in}}%
\pgfpathlineto{\pgfqpoint{2.705019in}{3.333015in}}%
\pgfpathlineto{\pgfqpoint{2.708511in}{3.345808in}}%
\pgfpathlineto{\pgfqpoint{2.712004in}{3.349649in}}%
\pgfpathlineto{\pgfqpoint{2.715497in}{3.339855in}}%
\pgfpathlineto{\pgfqpoint{2.722482in}{3.303889in}}%
\pgfpathlineto{\pgfqpoint{2.725975in}{3.292667in}}%
\pgfpathlineto{\pgfqpoint{2.729467in}{3.288694in}}%
\pgfpathlineto{\pgfqpoint{2.732960in}{3.288558in}}%
\pgfpathlineto{\pgfqpoint{2.736453in}{3.286862in}}%
\pgfpathlineto{\pgfqpoint{2.739945in}{3.280034in}}%
\pgfpathlineto{\pgfqpoint{2.746930in}{3.254453in}}%
\pgfpathlineto{\pgfqpoint{2.753916in}{3.227605in}}%
\pgfpathlineto{\pgfqpoint{2.757408in}{3.221834in}}%
\pgfpathlineto{\pgfqpoint{2.760901in}{3.225696in}}%
\pgfpathlineto{\pgfqpoint{2.771379in}{3.257372in}}%
\pgfpathlineto{\pgfqpoint{2.774872in}{3.255785in}}%
\pgfpathlineto{\pgfqpoint{2.778364in}{3.248840in}}%
\pgfpathlineto{\pgfqpoint{2.781857in}{3.246197in}}%
\pgfpathlineto{\pgfqpoint{2.785350in}{3.256496in}}%
\pgfpathlineto{\pgfqpoint{2.792335in}{3.300761in}}%
\pgfpathlineto{\pgfqpoint{2.795827in}{3.312523in}}%
\pgfpathlineto{\pgfqpoint{2.799320in}{3.310908in}}%
\pgfpathlineto{\pgfqpoint{2.813291in}{3.269659in}}%
\pgfpathlineto{\pgfqpoint{2.816783in}{3.266239in}}%
\pgfpathlineto{\pgfqpoint{2.820276in}{3.260860in}}%
\pgfpathlineto{\pgfqpoint{2.827261in}{3.246060in}}%
\pgfpathlineto{\pgfqpoint{2.834246in}{3.236798in}}%
\pgfpathlineto{\pgfqpoint{2.841232in}{3.222670in}}%
\pgfpathlineto{\pgfqpoint{2.844724in}{3.219858in}}%
\pgfpathlineto{\pgfqpoint{2.848217in}{3.224235in}}%
\pgfpathlineto{\pgfqpoint{2.851710in}{3.239256in}}%
\pgfpathlineto{\pgfqpoint{2.858695in}{3.303006in}}%
\pgfpathlineto{\pgfqpoint{2.862188in}{3.335708in}}%
\pgfpathlineto{\pgfqpoint{2.865680in}{3.354510in}}%
\pgfpathlineto{\pgfqpoint{2.869173in}{3.354620in}}%
\pgfpathlineto{\pgfqpoint{2.872666in}{3.336990in}}%
\pgfpathlineto{\pgfqpoint{2.879651in}{3.284761in}}%
\pgfpathlineto{\pgfqpoint{2.883143in}{3.273574in}}%
\pgfpathlineto{\pgfqpoint{2.886636in}{3.271172in}}%
\pgfpathlineto{\pgfqpoint{2.890129in}{3.264415in}}%
\pgfpathlineto{\pgfqpoint{2.893621in}{3.243721in}}%
\pgfpathlineto{\pgfqpoint{2.897114in}{3.197668in}}%
\pgfpathlineto{\pgfqpoint{2.900607in}{3.100748in}}%
\pgfpathlineto{\pgfqpoint{2.911085in}{2.598504in}}%
\pgfpathlineto{\pgfqpoint{2.914577in}{2.612934in}}%
\pgfpathlineto{\pgfqpoint{2.918070in}{2.775667in}}%
\pgfpathlineto{\pgfqpoint{2.925055in}{3.170771in}}%
\pgfpathlineto{\pgfqpoint{2.928548in}{3.263790in}}%
\pgfpathlineto{\pgfqpoint{2.932040in}{3.297521in}}%
\pgfpathlineto{\pgfqpoint{2.935533in}{3.309952in}}%
\pgfpathlineto{\pgfqpoint{2.939026in}{3.316992in}}%
\pgfpathlineto{\pgfqpoint{2.942518in}{3.313572in}}%
\pgfpathlineto{\pgfqpoint{2.946011in}{3.291298in}}%
\pgfpathlineto{\pgfqpoint{2.949504in}{3.233723in}}%
\pgfpathlineto{\pgfqpoint{2.952996in}{3.090547in}}%
\pgfpathlineto{\pgfqpoint{2.956489in}{2.791591in}}%
\pgfpathlineto{\pgfqpoint{2.963474in}{1.882368in}}%
\pgfpathlineto{\pgfqpoint{2.966967in}{1.671276in}}%
\pgfpathlineto{\pgfqpoint{2.970460in}{1.838242in}}%
\pgfpathlineto{\pgfqpoint{2.980937in}{3.019239in}}%
\pgfpathlineto{\pgfqpoint{2.984430in}{3.141438in}}%
\pgfpathlineto{\pgfqpoint{2.991415in}{3.219389in}}%
\pgfpathlineto{\pgfqpoint{2.998401in}{3.270369in}}%
\pgfpathlineto{\pgfqpoint{3.001893in}{3.280828in}}%
\pgfpathlineto{\pgfqpoint{3.005386in}{3.277158in}}%
\pgfpathlineto{\pgfqpoint{3.008879in}{3.263516in}}%
\pgfpathlineto{\pgfqpoint{3.012371in}{3.257640in}}%
\pgfpathlineto{\pgfqpoint{3.015864in}{3.271833in}}%
\pgfpathlineto{\pgfqpoint{3.019356in}{3.295192in}}%
\pgfpathlineto{\pgfqpoint{3.022849in}{3.308825in}}%
\pgfpathlineto{\pgfqpoint{3.026342in}{3.310360in}}%
\pgfpathlineto{\pgfqpoint{3.029834in}{3.308548in}}%
\pgfpathlineto{\pgfqpoint{3.033327in}{3.302670in}}%
\pgfpathlineto{\pgfqpoint{3.036820in}{3.284458in}}%
\pgfpathlineto{\pgfqpoint{3.043805in}{3.232395in}}%
\pgfpathlineto{\pgfqpoint{3.047298in}{3.222133in}}%
\pgfpathlineto{\pgfqpoint{3.050790in}{3.223394in}}%
\pgfpathlineto{\pgfqpoint{3.061268in}{3.242713in}}%
\pgfpathlineto{\pgfqpoint{3.068253in}{3.245253in}}%
\pgfpathlineto{\pgfqpoint{3.082224in}{3.267860in}}%
\pgfpathlineto{\pgfqpoint{3.089209in}{3.290148in}}%
\pgfpathlineto{\pgfqpoint{3.096195in}{3.310168in}}%
\pgfpathlineto{\pgfqpoint{3.099687in}{3.315515in}}%
\pgfpathlineto{\pgfqpoint{3.103180in}{3.313055in}}%
\pgfpathlineto{\pgfqpoint{3.113658in}{3.276947in}}%
\pgfpathlineto{\pgfqpoint{3.117150in}{3.271138in}}%
\pgfpathlineto{\pgfqpoint{3.120643in}{3.260329in}}%
\pgfpathlineto{\pgfqpoint{3.127628in}{3.225087in}}%
\pgfpathlineto{\pgfqpoint{3.131121in}{3.214198in}}%
\pgfpathlineto{\pgfqpoint{3.134614in}{3.213975in}}%
\pgfpathlineto{\pgfqpoint{3.138106in}{3.227136in}}%
\pgfpathlineto{\pgfqpoint{3.148584in}{3.294093in}}%
\pgfpathlineto{\pgfqpoint{3.152077in}{3.297110in}}%
\pgfpathlineto{\pgfqpoint{3.155569in}{3.294996in}}%
\pgfpathlineto{\pgfqpoint{3.159062in}{3.295695in}}%
\pgfpathlineto{\pgfqpoint{3.162555in}{3.301737in}}%
\pgfpathlineto{\pgfqpoint{3.169540in}{3.320913in}}%
\pgfpathlineto{\pgfqpoint{3.173033in}{3.322882in}}%
\pgfpathlineto{\pgfqpoint{3.176525in}{3.313481in}}%
\pgfpathlineto{\pgfqpoint{3.187003in}{3.256171in}}%
\pgfpathlineto{\pgfqpoint{3.197481in}{3.201496in}}%
\pgfpathlineto{\pgfqpoint{3.200974in}{3.193480in}}%
\pgfpathlineto{\pgfqpoint{3.204466in}{3.193030in}}%
\pgfpathlineto{\pgfqpoint{3.211452in}{3.200796in}}%
\pgfpathlineto{\pgfqpoint{3.214944in}{3.202407in}}%
\pgfpathlineto{\pgfqpoint{3.218437in}{3.208538in}}%
\pgfpathlineto{\pgfqpoint{3.221930in}{3.226791in}}%
\pgfpathlineto{\pgfqpoint{3.225422in}{3.251997in}}%
\pgfpathlineto{\pgfqpoint{3.228915in}{3.267262in}}%
\pgfpathlineto{\pgfqpoint{3.232408in}{3.264086in}}%
\pgfpathlineto{\pgfqpoint{3.235900in}{3.254960in}}%
\pgfpathlineto{\pgfqpoint{3.239393in}{3.259349in}}%
\pgfpathlineto{\pgfqpoint{3.249871in}{3.334700in}}%
\pgfpathlineto{\pgfqpoint{3.253363in}{3.340130in}}%
\pgfpathlineto{\pgfqpoint{3.256856in}{3.324859in}}%
\pgfpathlineto{\pgfqpoint{3.267334in}{3.236840in}}%
\pgfpathlineto{\pgfqpoint{3.270827in}{3.224862in}}%
\pgfpathlineto{\pgfqpoint{3.274319in}{3.218592in}}%
\pgfpathlineto{\pgfqpoint{3.277812in}{3.215615in}}%
\pgfpathlineto{\pgfqpoint{3.281305in}{3.216995in}}%
\pgfpathlineto{\pgfqpoint{3.288290in}{3.225473in}}%
\pgfpathlineto{\pgfqpoint{3.291783in}{3.232379in}}%
\pgfpathlineto{\pgfqpoint{3.295275in}{3.246549in}}%
\pgfpathlineto{\pgfqpoint{3.302260in}{3.286136in}}%
\pgfpathlineto{\pgfqpoint{3.309246in}{3.310040in}}%
\pgfpathlineto{\pgfqpoint{3.312738in}{3.316447in}}%
\pgfpathlineto{\pgfqpoint{3.316231in}{3.320323in}}%
\pgfpathlineto{\pgfqpoint{3.319724in}{3.322189in}}%
\pgfpathlineto{\pgfqpoint{3.323216in}{3.320299in}}%
\pgfpathlineto{\pgfqpoint{3.326709in}{3.310820in}}%
\pgfpathlineto{\pgfqpoint{3.333694in}{3.283804in}}%
\pgfpathlineto{\pgfqpoint{3.337187in}{3.285497in}}%
\pgfpathlineto{\pgfqpoint{3.340679in}{3.294674in}}%
\pgfpathlineto{\pgfqpoint{3.344172in}{3.295868in}}%
\pgfpathlineto{\pgfqpoint{3.347665in}{3.281069in}}%
\pgfpathlineto{\pgfqpoint{3.354650in}{3.239068in}}%
\pgfpathlineto{\pgfqpoint{3.358143in}{3.231110in}}%
\pgfpathlineto{\pgfqpoint{3.361635in}{3.228936in}}%
\pgfpathlineto{\pgfqpoint{3.365128in}{3.223785in}}%
\pgfpathlineto{\pgfqpoint{3.368621in}{3.213921in}}%
\pgfpathlineto{\pgfqpoint{3.372113in}{3.209037in}}%
\pgfpathlineto{\pgfqpoint{3.375606in}{3.220564in}}%
\pgfpathlineto{\pgfqpoint{3.382591in}{3.272126in}}%
\pgfpathlineto{\pgfqpoint{3.386084in}{3.284245in}}%
\pgfpathlineto{\pgfqpoint{3.389576in}{3.284790in}}%
\pgfpathlineto{\pgfqpoint{3.396562in}{3.276959in}}%
\pgfpathlineto{\pgfqpoint{3.403547in}{3.261774in}}%
\pgfpathlineto{\pgfqpoint{3.407040in}{3.252464in}}%
\pgfpathlineto{\pgfqpoint{3.417518in}{3.216271in}}%
\pgfpathlineto{\pgfqpoint{3.421010in}{3.221605in}}%
\pgfpathlineto{\pgfqpoint{3.424503in}{3.232823in}}%
\pgfpathlineto{\pgfqpoint{3.427996in}{3.240126in}}%
\pgfpathlineto{\pgfqpoint{3.431488in}{3.244237in}}%
\pgfpathlineto{\pgfqpoint{3.438473in}{3.254814in}}%
\pgfpathlineto{\pgfqpoint{3.441966in}{3.251898in}}%
\pgfpathlineto{\pgfqpoint{3.445459in}{3.246420in}}%
\pgfpathlineto{\pgfqpoint{3.448951in}{3.251131in}}%
\pgfpathlineto{\pgfqpoint{3.455937in}{3.294343in}}%
\pgfpathlineto{\pgfqpoint{3.459429in}{3.314088in}}%
\pgfpathlineto{\pgfqpoint{3.462922in}{3.324521in}}%
\pgfpathlineto{\pgfqpoint{3.466415in}{3.323100in}}%
\pgfpathlineto{\pgfqpoint{3.469907in}{3.309172in}}%
\pgfpathlineto{\pgfqpoint{3.476893in}{3.267370in}}%
\pgfpathlineto{\pgfqpoint{3.480385in}{3.253040in}}%
\pgfpathlineto{\pgfqpoint{3.483878in}{3.246524in}}%
\pgfpathlineto{\pgfqpoint{3.487370in}{3.248128in}}%
\pgfpathlineto{\pgfqpoint{3.490863in}{3.253770in}}%
\pgfpathlineto{\pgfqpoint{3.494356in}{3.254917in}}%
\pgfpathlineto{\pgfqpoint{3.501341in}{3.241354in}}%
\pgfpathlineto{\pgfqpoint{3.508326in}{3.239192in}}%
\pgfpathlineto{\pgfqpoint{3.511819in}{3.232028in}}%
\pgfpathlineto{\pgfqpoint{3.515312in}{3.222215in}}%
\pgfpathlineto{\pgfqpoint{3.518804in}{3.218711in}}%
\pgfpathlineto{\pgfqpoint{3.522297in}{3.223416in}}%
\pgfpathlineto{\pgfqpoint{3.525789in}{3.233163in}}%
\pgfpathlineto{\pgfqpoint{3.539760in}{3.286463in}}%
\pgfpathlineto{\pgfqpoint{3.543253in}{3.275267in}}%
\pgfpathlineto{\pgfqpoint{3.546745in}{3.208107in}}%
\pgfpathlineto{\pgfqpoint{3.550238in}{3.035194in}}%
\pgfpathlineto{\pgfqpoint{3.557223in}{2.510926in}}%
\pgfpathlineto{\pgfqpoint{3.560716in}{2.419610in}}%
\pgfpathlineto{\pgfqpoint{3.564209in}{2.542468in}}%
\pgfpathlineto{\pgfqpoint{3.571194in}{3.018410in}}%
\pgfpathlineto{\pgfqpoint{3.574686in}{3.151123in}}%
\pgfpathlineto{\pgfqpoint{3.578179in}{3.198954in}}%
\pgfpathlineto{\pgfqpoint{3.581672in}{3.207856in}}%
\pgfpathlineto{\pgfqpoint{3.585164in}{3.207078in}}%
\pgfpathlineto{\pgfqpoint{3.592150in}{3.199693in}}%
\pgfpathlineto{\pgfqpoint{3.595642in}{3.188279in}}%
\pgfpathlineto{\pgfqpoint{3.599135in}{3.125017in}}%
\pgfpathlineto{\pgfqpoint{3.602628in}{2.917238in}}%
\pgfpathlineto{\pgfqpoint{3.606120in}{2.495107in}}%
\pgfpathlineto{\pgfqpoint{3.613106in}{1.497493in}}%
\pgfpathlineto{\pgfqpoint{3.616598in}{1.443481in}}%
\pgfpathlineto{\pgfqpoint{3.620091in}{1.816470in}}%
\pgfpathlineto{\pgfqpoint{3.627076in}{2.865822in}}%
\pgfpathlineto{\pgfqpoint{3.630569in}{3.121361in}}%
\pgfpathlineto{\pgfqpoint{3.634061in}{3.211324in}}%
\pgfpathlineto{\pgfqpoint{3.637554in}{3.238805in}}%
\pgfpathlineto{\pgfqpoint{3.641047in}{3.253154in}}%
\pgfpathlineto{\pgfqpoint{3.644539in}{3.258104in}}%
\pgfpathlineto{\pgfqpoint{3.648032in}{3.249995in}}%
\pgfpathlineto{\pgfqpoint{3.651525in}{3.227079in}}%
\pgfpathlineto{\pgfqpoint{3.658510in}{3.164879in}}%
\pgfpathlineto{\pgfqpoint{3.662002in}{3.163197in}}%
\pgfpathlineto{\pgfqpoint{3.675973in}{3.258581in}}%
\pgfpathlineto{\pgfqpoint{3.682958in}{3.313373in}}%
\pgfpathlineto{\pgfqpoint{3.686451in}{3.318300in}}%
\pgfpathlineto{\pgfqpoint{3.693436in}{3.308852in}}%
\pgfpathlineto{\pgfqpoint{3.696929in}{3.304079in}}%
\pgfpathlineto{\pgfqpoint{3.700422in}{3.288855in}}%
\pgfpathlineto{\pgfqpoint{3.707407in}{3.240625in}}%
\pgfpathlineto{\pgfqpoint{3.710899in}{3.230409in}}%
\pgfpathlineto{\pgfqpoint{3.714392in}{3.235126in}}%
\pgfpathlineto{\pgfqpoint{3.717885in}{3.247567in}}%
\pgfpathlineto{\pgfqpoint{3.721377in}{3.255502in}}%
\pgfpathlineto{\pgfqpoint{3.724870in}{3.250808in}}%
\pgfpathlineto{\pgfqpoint{3.728363in}{3.238020in}}%
\pgfpathlineto{\pgfqpoint{3.731855in}{3.230362in}}%
\pgfpathlineto{\pgfqpoint{3.735348in}{3.235051in}}%
\pgfpathlineto{\pgfqpoint{3.749319in}{3.283071in}}%
\pgfpathlineto{\pgfqpoint{3.756304in}{3.319606in}}%
\pgfpathlineto{\pgfqpoint{3.759796in}{3.322762in}}%
\pgfpathlineto{\pgfqpoint{3.763289in}{3.315640in}}%
\pgfpathlineto{\pgfqpoint{3.766782in}{3.313275in}}%
\pgfpathlineto{\pgfqpoint{3.770274in}{3.327906in}}%
\pgfpathlineto{\pgfqpoint{3.773767in}{3.353648in}}%
\pgfpathlineto{\pgfqpoint{3.777260in}{3.370159in}}%
\pgfpathlineto{\pgfqpoint{3.780752in}{3.362464in}}%
\pgfpathlineto{\pgfqpoint{3.784245in}{3.333128in}}%
\pgfpathlineto{\pgfqpoint{3.794723in}{3.224111in}}%
\pgfpathlineto{\pgfqpoint{3.798216in}{3.206043in}}%
\pgfpathlineto{\pgfqpoint{3.801708in}{3.203800in}}%
\pgfpathlineto{\pgfqpoint{3.808693in}{3.215219in}}%
\pgfpathlineto{\pgfqpoint{3.812186in}{3.216760in}}%
\pgfpathlineto{\pgfqpoint{3.815679in}{3.221058in}}%
\pgfpathlineto{\pgfqpoint{3.819171in}{3.235411in}}%
\pgfpathlineto{\pgfqpoint{3.822664in}{3.263112in}}%
\pgfpathlineto{\pgfqpoint{3.829649in}{3.334393in}}%
\pgfpathlineto{\pgfqpoint{3.833142in}{3.348199in}}%
\pgfpathlineto{\pgfqpoint{3.836635in}{3.338484in}}%
\pgfpathlineto{\pgfqpoint{3.840127in}{3.320911in}}%
\pgfpathlineto{\pgfqpoint{3.843620in}{3.312922in}}%
\pgfpathlineto{\pgfqpoint{3.847112in}{3.314637in}}%
\pgfpathlineto{\pgfqpoint{3.850605in}{3.314193in}}%
\pgfpathlineto{\pgfqpoint{3.854098in}{3.308075in}}%
\pgfpathlineto{\pgfqpoint{3.857590in}{3.305579in}}%
\pgfpathlineto{\pgfqpoint{3.864576in}{3.320386in}}%
\pgfpathlineto{\pgfqpoint{3.868068in}{3.314882in}}%
\pgfpathlineto{\pgfqpoint{3.871561in}{3.291918in}}%
\pgfpathlineto{\pgfqpoint{3.878546in}{3.239283in}}%
\pgfpathlineto{\pgfqpoint{3.882039in}{3.233499in}}%
\pgfpathlineto{\pgfqpoint{3.889024in}{3.242824in}}%
\pgfpathlineto{\pgfqpoint{3.892517in}{3.240153in}}%
\pgfpathlineto{\pgfqpoint{3.896009in}{3.240254in}}%
\pgfpathlineto{\pgfqpoint{3.899502in}{3.256088in}}%
\pgfpathlineto{\pgfqpoint{3.909980in}{3.356464in}}%
\pgfpathlineto{\pgfqpoint{3.913473in}{3.363777in}}%
\pgfpathlineto{\pgfqpoint{3.916965in}{3.346667in}}%
\pgfpathlineto{\pgfqpoint{3.923951in}{3.281030in}}%
\pgfpathlineto{\pgfqpoint{3.927443in}{3.265619in}}%
\pgfpathlineto{\pgfqpoint{3.930936in}{3.269438in}}%
\pgfpathlineto{\pgfqpoint{3.937921in}{3.295164in}}%
\pgfpathlineto{\pgfqpoint{3.941414in}{3.299562in}}%
\pgfpathlineto{\pgfqpoint{3.944906in}{3.293933in}}%
\pgfpathlineto{\pgfqpoint{3.958877in}{3.237767in}}%
\pgfpathlineto{\pgfqpoint{3.962370in}{3.236138in}}%
\pgfpathlineto{\pgfqpoint{3.965862in}{3.242290in}}%
\pgfpathlineto{\pgfqpoint{3.969355in}{3.255671in}}%
\pgfpathlineto{\pgfqpoint{3.976340in}{3.300847in}}%
\pgfpathlineto{\pgfqpoint{3.983326in}{3.356820in}}%
\pgfpathlineto{\pgfqpoint{3.986818in}{3.366456in}}%
\pgfpathlineto{\pgfqpoint{3.990311in}{3.357212in}}%
\pgfpathlineto{\pgfqpoint{3.993803in}{3.340665in}}%
\pgfpathlineto{\pgfqpoint{3.997296in}{3.330617in}}%
\pgfpathlineto{\pgfqpoint{4.004281in}{3.329374in}}%
\pgfpathlineto{\pgfqpoint{4.007774in}{3.322636in}}%
\pgfpathlineto{\pgfqpoint{4.011267in}{3.306956in}}%
\pgfpathlineto{\pgfqpoint{4.021745in}{3.246019in}}%
\pgfpathlineto{\pgfqpoint{4.025237in}{3.237335in}}%
\pgfpathlineto{\pgfqpoint{4.028730in}{3.235134in}}%
\pgfpathlineto{\pgfqpoint{4.032222in}{3.239836in}}%
\pgfpathlineto{\pgfqpoint{4.039208in}{3.263745in}}%
\pgfpathlineto{\pgfqpoint{4.042700in}{3.265125in}}%
\pgfpathlineto{\pgfqpoint{4.046193in}{3.257081in}}%
\pgfpathlineto{\pgfqpoint{4.049686in}{3.255731in}}%
\pgfpathlineto{\pgfqpoint{4.053178in}{3.273993in}}%
\pgfpathlineto{\pgfqpoint{4.063656in}{3.370069in}}%
\pgfpathlineto{\pgfqpoint{4.067149in}{3.391347in}}%
\pgfpathlineto{\pgfqpoint{4.070642in}{3.403276in}}%
\pgfpathlineto{\pgfqpoint{4.074134in}{3.403485in}}%
\pgfpathlineto{\pgfqpoint{4.077627in}{3.393830in}}%
\pgfpathlineto{\pgfqpoint{4.081119in}{3.377059in}}%
\pgfpathlineto{\pgfqpoint{4.091597in}{3.304080in}}%
\pgfpathlineto{\pgfqpoint{4.095090in}{3.292000in}}%
\pgfpathlineto{\pgfqpoint{4.102075in}{3.276539in}}%
\pgfpathlineto{\pgfqpoint{4.109061in}{3.255276in}}%
\pgfpathlineto{\pgfqpoint{4.112553in}{3.248788in}}%
\pgfpathlineto{\pgfqpoint{4.116046in}{3.250076in}}%
\pgfpathlineto{\pgfqpoint{4.119539in}{3.262480in}}%
\pgfpathlineto{\pgfqpoint{4.123031in}{3.279296in}}%
\pgfpathlineto{\pgfqpoint{4.126524in}{3.286483in}}%
\pgfpathlineto{\pgfqpoint{4.133509in}{3.276891in}}%
\pgfpathlineto{\pgfqpoint{4.137002in}{3.284675in}}%
\pgfpathlineto{\pgfqpoint{4.143987in}{3.309538in}}%
\pgfpathlineto{\pgfqpoint{4.150972in}{3.328487in}}%
\pgfpathlineto{\pgfqpoint{4.154465in}{3.331937in}}%
\pgfpathlineto{\pgfqpoint{4.157958in}{3.324583in}}%
\pgfpathlineto{\pgfqpoint{4.168436in}{3.288420in}}%
\pgfpathlineto{\pgfqpoint{4.171928in}{3.256916in}}%
\pgfpathlineto{\pgfqpoint{4.175421in}{3.201374in}}%
\pgfpathlineto{\pgfqpoint{4.178913in}{3.092242in}}%
\pgfpathlineto{\pgfqpoint{4.182406in}{2.888491in}}%
\pgfpathlineto{\pgfqpoint{4.189391in}{2.378331in}}%
\pgfpathlineto{\pgfqpoint{4.192884in}{2.344172in}}%
\pgfpathlineto{\pgfqpoint{4.196377in}{2.540319in}}%
\pgfpathlineto{\pgfqpoint{4.203362in}{3.118184in}}%
\pgfpathlineto{\pgfqpoint{4.206855in}{3.265983in}}%
\pgfpathlineto{\pgfqpoint{4.210347in}{3.314580in}}%
\pgfpathlineto{\pgfqpoint{4.213840in}{3.320747in}}%
\pgfpathlineto{\pgfqpoint{4.217332in}{3.323013in}}%
\pgfpathlineto{\pgfqpoint{4.220825in}{3.328368in}}%
\pgfpathlineto{\pgfqpoint{4.224318in}{3.325719in}}%
\pgfpathlineto{\pgfqpoint{4.227810in}{3.299478in}}%
\pgfpathlineto{\pgfqpoint{4.231303in}{3.217341in}}%
\pgfpathlineto{\pgfqpoint{4.234796in}{3.003945in}}%
\pgfpathlineto{\pgfqpoint{4.238288in}{2.576469in}}%
\pgfpathlineto{\pgfqpoint{4.245274in}{1.410651in}}%
\pgfpathlineto{\pgfqpoint{4.248766in}{1.214919in}}%
\pgfpathlineto{\pgfqpoint{4.252259in}{1.505056in}}%
\pgfpathlineto{\pgfqpoint{4.259244in}{2.680400in}}%
\pgfpathlineto{\pgfqpoint{4.262737in}{3.049075in}}%
\pgfpathlineto{\pgfqpoint{4.266229in}{3.212616in}}%
\pgfpathlineto{\pgfqpoint{4.269722in}{3.273238in}}%
\pgfpathlineto{\pgfqpoint{4.276707in}{3.334443in}}%
\pgfpathlineto{\pgfqpoint{4.280200in}{3.357853in}}%
\pgfpathlineto{\pgfqpoint{4.283693in}{3.368280in}}%
\pgfpathlineto{\pgfqpoint{4.287185in}{3.359132in}}%
\pgfpathlineto{\pgfqpoint{4.290678in}{3.338807in}}%
\pgfpathlineto{\pgfqpoint{4.294171in}{3.326557in}}%
\pgfpathlineto{\pgfqpoint{4.297663in}{3.329511in}}%
\pgfpathlineto{\pgfqpoint{4.301156in}{3.336202in}}%
\pgfpathlineto{\pgfqpoint{4.304649in}{3.336433in}}%
\pgfpathlineto{\pgfqpoint{4.308141in}{3.334222in}}%
\pgfpathlineto{\pgfqpoint{4.311634in}{3.336337in}}%
\pgfpathlineto{\pgfqpoint{4.315126in}{3.340735in}}%
\pgfpathlineto{\pgfqpoint{4.318619in}{3.342236in}}%
\pgfpathlineto{\pgfqpoint{4.322112in}{3.339215in}}%
\pgfpathlineto{\pgfqpoint{4.325604in}{3.329035in}}%
\pgfpathlineto{\pgfqpoint{4.329097in}{3.306493in}}%
\pgfpathlineto{\pgfqpoint{4.336082in}{3.244729in}}%
\pgfpathlineto{\pgfqpoint{4.339575in}{3.234751in}}%
\pgfpathlineto{\pgfqpoint{4.343068in}{3.246708in}}%
\pgfpathlineto{\pgfqpoint{4.350053in}{3.305991in}}%
\pgfpathlineto{\pgfqpoint{4.357038in}{3.364817in}}%
\pgfpathlineto{\pgfqpoint{4.360531in}{3.375516in}}%
\pgfpathlineto{\pgfqpoint{4.364023in}{3.369309in}}%
\pgfpathlineto{\pgfqpoint{4.367516in}{3.356145in}}%
\pgfpathlineto{\pgfqpoint{4.371009in}{3.350912in}}%
\pgfpathlineto{\pgfqpoint{4.377994in}{3.365512in}}%
\pgfpathlineto{\pgfqpoint{4.381487in}{3.359336in}}%
\pgfpathlineto{\pgfqpoint{4.388472in}{3.328574in}}%
\pgfpathlineto{\pgfqpoint{4.391965in}{3.326003in}}%
\pgfpathlineto{\pgfqpoint{4.395457in}{3.325043in}}%
\pgfpathlineto{\pgfqpoint{4.398950in}{3.316931in}}%
\pgfpathlineto{\pgfqpoint{4.402442in}{3.305975in}}%
\pgfpathlineto{\pgfqpoint{4.416413in}{3.280999in}}%
\pgfpathlineto{\pgfqpoint{4.419906in}{3.282725in}}%
\pgfpathlineto{\pgfqpoint{4.430384in}{3.314628in}}%
\pgfpathlineto{\pgfqpoint{4.433876in}{3.327278in}}%
\pgfpathlineto{\pgfqpoint{4.437369in}{3.346883in}}%
\pgfpathlineto{\pgfqpoint{4.447847in}{3.419112in}}%
\pgfpathlineto{\pgfqpoint{4.451339in}{3.415438in}}%
\pgfpathlineto{\pgfqpoint{4.461817in}{3.346496in}}%
\pgfpathlineto{\pgfqpoint{4.465310in}{3.340663in}}%
\pgfpathlineto{\pgfqpoint{4.468803in}{3.330582in}}%
\pgfpathlineto{\pgfqpoint{4.479281in}{3.270481in}}%
\pgfpathlineto{\pgfqpoint{4.482773in}{3.263759in}}%
\pgfpathlineto{\pgfqpoint{4.486266in}{3.265927in}}%
\pgfpathlineto{\pgfqpoint{4.493251in}{3.286405in}}%
\pgfpathlineto{\pgfqpoint{4.500236in}{3.306627in}}%
\pgfpathlineto{\pgfqpoint{4.503729in}{3.322890in}}%
\pgfpathlineto{\pgfqpoint{4.510714in}{3.377162in}}%
\pgfpathlineto{\pgfqpoint{4.514207in}{3.404906in}}%
\pgfpathlineto{\pgfqpoint{4.517700in}{3.421413in}}%
\pgfpathlineto{\pgfqpoint{4.521192in}{3.417683in}}%
\pgfpathlineto{\pgfqpoint{4.535163in}{3.329350in}}%
\pgfpathlineto{\pgfqpoint{4.549133in}{3.296917in}}%
\pgfpathlineto{\pgfqpoint{4.552626in}{3.294978in}}%
\pgfpathlineto{\pgfqpoint{4.556119in}{3.295083in}}%
\pgfpathlineto{\pgfqpoint{4.559611in}{3.301351in}}%
\pgfpathlineto{\pgfqpoint{4.566597in}{3.335617in}}%
\pgfpathlineto{\pgfqpoint{4.570089in}{3.352279in}}%
\pgfpathlineto{\pgfqpoint{4.573582in}{3.363056in}}%
\pgfpathlineto{\pgfqpoint{4.584060in}{3.386308in}}%
\pgfpathlineto{\pgfqpoint{4.587552in}{3.395449in}}%
\pgfpathlineto{\pgfqpoint{4.591045in}{3.401241in}}%
\pgfpathlineto{\pgfqpoint{4.594538in}{3.402056in}}%
\pgfpathlineto{\pgfqpoint{4.598030in}{3.397732in}}%
\pgfpathlineto{\pgfqpoint{4.601523in}{3.386601in}}%
\pgfpathlineto{\pgfqpoint{4.612001in}{3.326543in}}%
\pgfpathlineto{\pgfqpoint{4.615494in}{3.320891in}}%
\pgfpathlineto{\pgfqpoint{4.618986in}{3.320616in}}%
\pgfpathlineto{\pgfqpoint{4.622479in}{3.315525in}}%
\pgfpathlineto{\pgfqpoint{4.632957in}{3.277495in}}%
\pgfpathlineto{\pgfqpoint{4.636449in}{3.277500in}}%
\pgfpathlineto{\pgfqpoint{4.639942in}{3.290804in}}%
\pgfpathlineto{\pgfqpoint{4.650420in}{3.365432in}}%
\pgfpathlineto{\pgfqpoint{4.653913in}{3.376923in}}%
\pgfpathlineto{\pgfqpoint{4.657405in}{3.379010in}}%
\pgfpathlineto{\pgfqpoint{4.660898in}{3.373805in}}%
\pgfpathlineto{\pgfqpoint{4.664391in}{3.364994in}}%
\pgfpathlineto{\pgfqpoint{4.667883in}{3.359342in}}%
\pgfpathlineto{\pgfqpoint{4.671376in}{3.362883in}}%
\pgfpathlineto{\pgfqpoint{4.681854in}{3.397754in}}%
\pgfpathlineto{\pgfqpoint{4.685346in}{3.392849in}}%
\pgfpathlineto{\pgfqpoint{4.688839in}{3.374625in}}%
\pgfpathlineto{\pgfqpoint{4.695824in}{3.325387in}}%
\pgfpathlineto{\pgfqpoint{4.699317in}{3.307488in}}%
\pgfpathlineto{\pgfqpoint{4.702810in}{3.295492in}}%
\pgfpathlineto{\pgfqpoint{4.706302in}{3.289469in}}%
\pgfpathlineto{\pgfqpoint{4.709795in}{3.289700in}}%
\pgfpathlineto{\pgfqpoint{4.713288in}{3.292136in}}%
\pgfpathlineto{\pgfqpoint{4.716780in}{3.290945in}}%
\pgfpathlineto{\pgfqpoint{4.720273in}{3.287884in}}%
\pgfpathlineto{\pgfqpoint{4.723765in}{3.292784in}}%
\pgfpathlineto{\pgfqpoint{4.727258in}{3.311370in}}%
\pgfpathlineto{\pgfqpoint{4.734243in}{3.362168in}}%
\pgfpathlineto{\pgfqpoint{4.737736in}{3.379650in}}%
\pgfpathlineto{\pgfqpoint{4.741229in}{3.388413in}}%
\pgfpathlineto{\pgfqpoint{4.744721in}{3.385365in}}%
\pgfpathlineto{\pgfqpoint{4.748214in}{3.369147in}}%
\pgfpathlineto{\pgfqpoint{4.758692in}{3.293685in}}%
\pgfpathlineto{\pgfqpoint{4.762185in}{3.279859in}}%
\pgfpathlineto{\pgfqpoint{4.765677in}{3.275813in}}%
\pgfpathlineto{\pgfqpoint{4.769170in}{3.274940in}}%
\pgfpathlineto{\pgfqpoint{4.772662in}{3.269656in}}%
\pgfpathlineto{\pgfqpoint{4.776155in}{3.261033in}}%
\pgfpathlineto{\pgfqpoint{4.779648in}{3.255819in}}%
\pgfpathlineto{\pgfqpoint{4.783140in}{3.255066in}}%
\pgfpathlineto{\pgfqpoint{4.786633in}{3.252405in}}%
\pgfpathlineto{\pgfqpoint{4.790126in}{3.236926in}}%
\pgfpathlineto{\pgfqpoint{4.793618in}{3.184208in}}%
\pgfpathlineto{\pgfqpoint{4.797111in}{3.057069in}}%
\pgfpathlineto{\pgfqpoint{4.804096in}{2.634401in}}%
\pgfpathlineto{\pgfqpoint{4.807589in}{2.542436in}}%
\pgfpathlineto{\pgfqpoint{4.811082in}{2.645176in}}%
\pgfpathlineto{\pgfqpoint{4.818067in}{3.127871in}}%
\pgfpathlineto{\pgfqpoint{4.821559in}{3.275748in}}%
\pgfpathlineto{\pgfqpoint{4.825052in}{3.320937in}}%
\pgfpathlineto{\pgfqpoint{4.828545in}{3.308868in}}%
\pgfpathlineto{\pgfqpoint{4.832037in}{3.283946in}}%
\pgfpathlineto{\pgfqpoint{4.835530in}{3.267196in}}%
\pgfpathlineto{\pgfqpoint{4.839023in}{3.258378in}}%
\pgfpathlineto{\pgfqpoint{4.842515in}{3.245026in}}%
\pgfpathlineto{\pgfqpoint{4.846008in}{3.205115in}}%
\pgfpathlineto{\pgfqpoint{4.849501in}{3.089953in}}%
\pgfpathlineto{\pgfqpoint{4.852993in}{2.816617in}}%
\pgfpathlineto{\pgfqpoint{4.863471in}{1.334725in}}%
\pgfpathlineto{\pgfqpoint{4.866964in}{1.343840in}}%
\pgfpathlineto{\pgfqpoint{4.870456in}{1.778242in}}%
\pgfpathlineto{\pgfqpoint{4.877442in}{2.851414in}}%
\pgfpathlineto{\pgfqpoint{4.880934in}{3.100051in}}%
\pgfpathlineto{\pgfqpoint{4.884427in}{3.196126in}}%
\pgfpathlineto{\pgfqpoint{4.891412in}{3.272025in}}%
\pgfpathlineto{\pgfqpoint{4.898398in}{3.322094in}}%
\pgfpathlineto{\pgfqpoint{4.901890in}{3.335888in}}%
\pgfpathlineto{\pgfqpoint{4.905383in}{3.327059in}}%
\pgfpathlineto{\pgfqpoint{4.919353in}{3.202644in}}%
\pgfpathlineto{\pgfqpoint{4.922846in}{3.191728in}}%
\pgfpathlineto{\pgfqpoint{4.926339in}{3.192139in}}%
\pgfpathlineto{\pgfqpoint{4.933324in}{3.208607in}}%
\pgfpathlineto{\pgfqpoint{4.936817in}{3.210286in}}%
\pgfpathlineto{\pgfqpoint{4.940309in}{3.209643in}}%
\pgfpathlineto{\pgfqpoint{4.943802in}{3.212401in}}%
\pgfpathlineto{\pgfqpoint{4.950787in}{3.229826in}}%
\pgfpathlineto{\pgfqpoint{4.954280in}{3.238296in}}%
\pgfpathlineto{\pgfqpoint{4.957772in}{3.240970in}}%
\pgfpathlineto{\pgfqpoint{4.961265in}{3.237098in}}%
\pgfpathlineto{\pgfqpoint{4.971743in}{3.217863in}}%
\pgfpathlineto{\pgfqpoint{4.975236in}{3.201108in}}%
\pgfpathlineto{\pgfqpoint{4.985714in}{3.135531in}}%
\pgfpathlineto{\pgfqpoint{4.989206in}{3.126540in}}%
\pgfpathlineto{\pgfqpoint{4.996192in}{3.118969in}}%
\pgfpathlineto{\pgfqpoint{4.999684in}{3.110251in}}%
\pgfpathlineto{\pgfqpoint{5.006669in}{3.087837in}}%
\pgfpathlineto{\pgfqpoint{5.010162in}{3.082397in}}%
\pgfpathlineto{\pgfqpoint{5.013655in}{3.080486in}}%
\pgfpathlineto{\pgfqpoint{5.017147in}{3.084100in}}%
\pgfpathlineto{\pgfqpoint{5.020640in}{3.098891in}}%
\pgfpathlineto{\pgfqpoint{5.027625in}{3.147617in}}%
\pgfpathlineto{\pgfqpoint{5.031118in}{3.161461in}}%
\pgfpathlineto{\pgfqpoint{5.034611in}{3.169259in}}%
\pgfpathlineto{\pgfqpoint{5.038103in}{3.180352in}}%
\pgfpathlineto{\pgfqpoint{5.048581in}{3.226356in}}%
\pgfpathlineto{\pgfqpoint{5.052074in}{3.227196in}}%
\pgfpathlineto{\pgfqpoint{5.055566in}{3.210859in}}%
\pgfpathlineto{\pgfqpoint{5.062552in}{3.148762in}}%
\pgfpathlineto{\pgfqpoint{5.066044in}{3.132413in}}%
\pgfpathlineto{\pgfqpoint{5.069537in}{3.133960in}}%
\pgfpathlineto{\pgfqpoint{5.073030in}{3.147110in}}%
\pgfpathlineto{\pgfqpoint{5.080015in}{3.179620in}}%
\pgfpathlineto{\pgfqpoint{5.083508in}{3.182768in}}%
\pgfpathlineto{\pgfqpoint{5.090493in}{3.162245in}}%
\pgfpathlineto{\pgfqpoint{5.093985in}{3.164168in}}%
\pgfpathlineto{\pgfqpoint{5.100971in}{3.186312in}}%
\pgfpathlineto{\pgfqpoint{5.104463in}{3.185861in}}%
\pgfpathlineto{\pgfqpoint{5.111449in}{3.172924in}}%
\pgfpathlineto{\pgfqpoint{5.114941in}{3.161533in}}%
\pgfpathlineto{\pgfqpoint{5.121927in}{3.129181in}}%
\pgfpathlineto{\pgfqpoint{5.125419in}{3.128823in}}%
\pgfpathlineto{\pgfqpoint{5.128912in}{3.146607in}}%
\pgfpathlineto{\pgfqpoint{5.132405in}{3.171635in}}%
\pgfpathlineto{\pgfqpoint{5.135897in}{3.186027in}}%
\pgfpathlineto{\pgfqpoint{5.139390in}{3.179314in}}%
\pgfpathlineto{\pgfqpoint{5.146375in}{3.137176in}}%
\pgfpathlineto{\pgfqpoint{5.149868in}{3.127631in}}%
\pgfpathlineto{\pgfqpoint{5.153360in}{3.125976in}}%
\pgfpathlineto{\pgfqpoint{5.156853in}{3.126981in}}%
\pgfpathlineto{\pgfqpoint{5.160346in}{3.133171in}}%
\pgfpathlineto{\pgfqpoint{5.163838in}{3.149161in}}%
\pgfpathlineto{\pgfqpoint{5.170824in}{3.193095in}}%
\pgfpathlineto{\pgfqpoint{5.174316in}{3.207563in}}%
\pgfpathlineto{\pgfqpoint{5.177809in}{3.212765in}}%
\pgfpathlineto{\pgfqpoint{5.181302in}{3.205689in}}%
\pgfpathlineto{\pgfqpoint{5.188287in}{3.177456in}}%
\pgfpathlineto{\pgfqpoint{5.191779in}{3.187039in}}%
\pgfpathlineto{\pgfqpoint{5.205750in}{3.308912in}}%
\pgfpathlineto{\pgfqpoint{5.209243in}{3.315145in}}%
\pgfpathlineto{\pgfqpoint{5.212735in}{3.301435in}}%
\pgfpathlineto{\pgfqpoint{5.219721in}{3.223121in}}%
\pgfpathlineto{\pgfqpoint{5.226706in}{3.152205in}}%
\pgfpathlineto{\pgfqpoint{5.230198in}{3.136071in}}%
\pgfpathlineto{\pgfqpoint{5.233691in}{3.135490in}}%
\pgfpathlineto{\pgfqpoint{5.237184in}{3.148832in}}%
\pgfpathlineto{\pgfqpoint{5.244169in}{3.183827in}}%
\pgfpathlineto{\pgfqpoint{5.247662in}{3.191570in}}%
\pgfpathlineto{\pgfqpoint{5.251154in}{3.195469in}}%
\pgfpathlineto{\pgfqpoint{5.254647in}{3.203137in}}%
\pgfpathlineto{\pgfqpoint{5.258140in}{3.219194in}}%
\pgfpathlineto{\pgfqpoint{5.265125in}{3.259820in}}%
\pgfpathlineto{\pgfqpoint{5.268618in}{3.269675in}}%
\pgfpathlineto{\pgfqpoint{5.272110in}{3.271475in}}%
\pgfpathlineto{\pgfqpoint{5.275603in}{3.269472in}}%
\pgfpathlineto{\pgfqpoint{5.279095in}{3.263762in}}%
\pgfpathlineto{\pgfqpoint{5.282588in}{3.251263in}}%
\pgfpathlineto{\pgfqpoint{5.296559in}{3.181021in}}%
\pgfpathlineto{\pgfqpoint{5.300051in}{3.166698in}}%
\pgfpathlineto{\pgfqpoint{5.303544in}{3.162384in}}%
\pgfpathlineto{\pgfqpoint{5.307037in}{3.175512in}}%
\pgfpathlineto{\pgfqpoint{5.317515in}{3.251162in}}%
\pgfpathlineto{\pgfqpoint{5.324500in}{3.264727in}}%
\pgfpathlineto{\pgfqpoint{5.327992in}{3.282926in}}%
\pgfpathlineto{\pgfqpoint{5.334978in}{3.339879in}}%
\pgfpathlineto{\pgfqpoint{5.338470in}{3.340323in}}%
\pgfpathlineto{\pgfqpoint{5.345456in}{3.288301in}}%
\pgfpathlineto{\pgfqpoint{5.348948in}{3.279133in}}%
\pgfpathlineto{\pgfqpoint{5.352441in}{3.283545in}}%
\pgfpathlineto{\pgfqpoint{5.355934in}{3.283749in}}%
\pgfpathlineto{\pgfqpoint{5.362919in}{3.266905in}}%
\pgfpathlineto{\pgfqpoint{5.366412in}{3.270056in}}%
\pgfpathlineto{\pgfqpoint{5.369904in}{3.278469in}}%
\pgfpathlineto{\pgfqpoint{5.373397in}{3.283082in}}%
\pgfpathlineto{\pgfqpoint{5.376889in}{3.282911in}}%
\pgfpathlineto{\pgfqpoint{5.383875in}{3.278812in}}%
\pgfpathlineto{\pgfqpoint{5.387367in}{3.272089in}}%
\pgfpathlineto{\pgfqpoint{5.390860in}{3.252697in}}%
\pgfpathlineto{\pgfqpoint{5.394353in}{3.199505in}}%
\pgfpathlineto{\pgfqpoint{5.397845in}{3.089264in}}%
\pgfpathlineto{\pgfqpoint{5.404831in}{2.799555in}}%
\pgfpathlineto{\pgfqpoint{5.408323in}{2.770092in}}%
\pgfpathlineto{\pgfqpoint{5.411816in}{2.872329in}}%
\pgfpathlineto{\pgfqpoint{5.418801in}{3.223049in}}%
\pgfpathlineto{\pgfqpoint{5.422294in}{3.331124in}}%
\pgfpathlineto{\pgfqpoint{5.425786in}{3.373459in}}%
\pgfpathlineto{\pgfqpoint{5.429279in}{3.373699in}}%
\pgfpathlineto{\pgfqpoint{5.432772in}{3.354111in}}%
\pgfpathlineto{\pgfqpoint{5.439757in}{3.290016in}}%
\pgfpathlineto{\pgfqpoint{5.446742in}{3.212688in}}%
\pgfpathlineto{\pgfqpoint{5.450235in}{3.138037in}}%
\pgfpathlineto{\pgfqpoint{5.453728in}{2.958459in}}%
\pgfpathlineto{\pgfqpoint{5.464205in}{1.941346in}}%
\pgfpathlineto{\pgfqpoint{5.467698in}{1.961020in}}%
\pgfpathlineto{\pgfqpoint{5.471191in}{2.271735in}}%
\pgfpathlineto{\pgfqpoint{5.478176in}{3.010830in}}%
\pgfpathlineto{\pgfqpoint{5.481669in}{3.180610in}}%
\pgfpathlineto{\pgfqpoint{5.485161in}{3.249447in}}%
\pgfpathlineto{\pgfqpoint{5.492147in}{3.314239in}}%
\pgfpathlineto{\pgfqpoint{5.495639in}{3.336166in}}%
\pgfpathlineto{\pgfqpoint{5.499132in}{3.348116in}}%
\pgfpathlineto{\pgfqpoint{5.502625in}{3.348309in}}%
\pgfpathlineto{\pgfqpoint{5.513102in}{3.318773in}}%
\pgfpathlineto{\pgfqpoint{5.516595in}{3.316470in}}%
\pgfpathlineto{\pgfqpoint{5.520088in}{3.309063in}}%
\pgfpathlineto{\pgfqpoint{5.523580in}{3.296756in}}%
\pgfpathlineto{\pgfqpoint{5.527073in}{3.288765in}}%
\pgfpathlineto{\pgfqpoint{5.530566in}{3.288805in}}%
\pgfpathlineto{\pgfqpoint{5.534058in}{3.290591in}}%
\pgfpathlineto{\pgfqpoint{5.537551in}{3.287915in}}%
\pgfpathlineto{\pgfqpoint{5.548029in}{3.265223in}}%
\pgfpathlineto{\pgfqpoint{5.551521in}{3.266489in}}%
\pgfpathlineto{\pgfqpoint{5.555014in}{3.281130in}}%
\pgfpathlineto{\pgfqpoint{5.561999in}{3.330708in}}%
\pgfpathlineto{\pgfqpoint{5.565492in}{3.346003in}}%
\pgfpathlineto{\pgfqpoint{5.568985in}{3.349113in}}%
\pgfpathlineto{\pgfqpoint{5.575970in}{3.337814in}}%
\pgfpathlineto{\pgfqpoint{5.579463in}{3.342122in}}%
\pgfpathlineto{\pgfqpoint{5.582955in}{3.353831in}}%
\pgfpathlineto{\pgfqpoint{5.586448in}{3.354252in}}%
\pgfpathlineto{\pgfqpoint{5.589941in}{3.327334in}}%
\pgfpathlineto{\pgfqpoint{5.596926in}{3.242970in}}%
\pgfpathlineto{\pgfqpoint{5.600418in}{3.226918in}}%
\pgfpathlineto{\pgfqpoint{5.603911in}{3.226526in}}%
\pgfpathlineto{\pgfqpoint{5.607404in}{3.230185in}}%
\pgfpathlineto{\pgfqpoint{5.610896in}{3.237610in}}%
\pgfpathlineto{\pgfqpoint{5.621374in}{3.286160in}}%
\pgfpathlineto{\pgfqpoint{5.635345in}{3.310644in}}%
\pgfpathlineto{\pgfqpoint{5.638838in}{3.305227in}}%
\pgfpathlineto{\pgfqpoint{5.642330in}{3.293277in}}%
\pgfpathlineto{\pgfqpoint{5.645823in}{3.285348in}}%
\pgfpathlineto{\pgfqpoint{5.649315in}{3.288214in}}%
\pgfpathlineto{\pgfqpoint{5.652808in}{3.297474in}}%
\pgfpathlineto{\pgfqpoint{5.656301in}{3.302354in}}%
\pgfpathlineto{\pgfqpoint{5.659793in}{3.297062in}}%
\pgfpathlineto{\pgfqpoint{5.670271in}{3.261381in}}%
\pgfpathlineto{\pgfqpoint{5.673764in}{3.255318in}}%
\pgfpathlineto{\pgfqpoint{5.677257in}{3.256358in}}%
\pgfpathlineto{\pgfqpoint{5.680749in}{3.261924in}}%
\pgfpathlineto{\pgfqpoint{5.684242in}{3.262512in}}%
\pgfpathlineto{\pgfqpoint{5.687735in}{3.253808in}}%
\pgfpathlineto{\pgfqpoint{5.691227in}{3.248204in}}%
\pgfpathlineto{\pgfqpoint{5.694720in}{3.264903in}}%
\pgfpathlineto{\pgfqpoint{5.705198in}{3.379522in}}%
\pgfpathlineto{\pgfqpoint{5.708690in}{3.378743in}}%
\pgfpathlineto{\pgfqpoint{5.712183in}{3.359057in}}%
\pgfpathlineto{\pgfqpoint{5.722661in}{3.285487in}}%
\pgfpathlineto{\pgfqpoint{5.726154in}{3.275323in}}%
\pgfpathlineto{\pgfqpoint{5.733139in}{3.267075in}}%
\pgfpathlineto{\pgfqpoint{5.736631in}{3.256138in}}%
\pgfpathlineto{\pgfqpoint{5.743617in}{3.230424in}}%
\pgfpathlineto{\pgfqpoint{5.747109in}{3.229161in}}%
\pgfpathlineto{\pgfqpoint{5.757587in}{3.251036in}}%
\pgfpathlineto{\pgfqpoint{5.761080in}{3.250574in}}%
\pgfpathlineto{\pgfqpoint{5.764573in}{3.246396in}}%
\pgfpathlineto{\pgfqpoint{5.768065in}{3.238043in}}%
\pgfpathlineto{\pgfqpoint{5.775051in}{3.208549in}}%
\pgfpathlineto{\pgfqpoint{5.778543in}{3.201091in}}%
\pgfpathlineto{\pgfqpoint{5.782036in}{3.208409in}}%
\pgfpathlineto{\pgfqpoint{5.796006in}{3.282286in}}%
\pgfpathlineto{\pgfqpoint{5.799499in}{3.293563in}}%
\pgfpathlineto{\pgfqpoint{5.802992in}{3.295437in}}%
\pgfpathlineto{\pgfqpoint{5.802992in}{3.295437in}}%
\pgfusepath{stroke}%
\end{pgfscope}%
\begin{pgfscope}%
\pgfpathrectangle{\pgfqpoint{0.773588in}{0.646140in}}{\pgfqpoint{5.029404in}{3.088289in}}%
\pgfusepath{clip}%
\pgfsetbuttcap%
\pgfsetroundjoin%
\definecolor{currentfill}{rgb}{1.000000,0.000000,0.000000}%
\pgfsetfillcolor{currentfill}%
\pgfsetlinewidth{1.003750pt}%
\definecolor{currentstroke}{rgb}{1.000000,0.000000,0.000000}%
\pgfsetstrokecolor{currentstroke}%
\pgfsetdash{}{0pt}%
\pgfsys@defobject{currentmarker}{\pgfqpoint{-0.017361in}{-0.017361in}}{\pgfqpoint{0.017361in}{0.017361in}}{%
\pgfpathmoveto{\pgfqpoint{0.000000in}{-0.017361in}}%
\pgfpathcurveto{\pgfqpoint{0.004604in}{-0.017361in}}{\pgfqpoint{0.009020in}{-0.015532in}}{\pgfqpoint{0.012276in}{-0.012276in}}%
\pgfpathcurveto{\pgfqpoint{0.015532in}{-0.009020in}}{\pgfqpoint{0.017361in}{-0.004604in}}{\pgfqpoint{0.017361in}{0.000000in}}%
\pgfpathcurveto{\pgfqpoint{0.017361in}{0.004604in}}{\pgfqpoint{0.015532in}{0.009020in}}{\pgfqpoint{0.012276in}{0.012276in}}%
\pgfpathcurveto{\pgfqpoint{0.009020in}{0.015532in}}{\pgfqpoint{0.004604in}{0.017361in}}{\pgfqpoint{0.000000in}{0.017361in}}%
\pgfpathcurveto{\pgfqpoint{-0.004604in}{0.017361in}}{\pgfqpoint{-0.009020in}{0.015532in}}{\pgfqpoint{-0.012276in}{0.012276in}}%
\pgfpathcurveto{\pgfqpoint{-0.015532in}{0.009020in}}{\pgfqpoint{-0.017361in}{0.004604in}}{\pgfqpoint{-0.017361in}{0.000000in}}%
\pgfpathcurveto{\pgfqpoint{-0.017361in}{-0.004604in}}{\pgfqpoint{-0.015532in}{-0.009020in}}{\pgfqpoint{-0.012276in}{-0.012276in}}%
\pgfpathcurveto{\pgfqpoint{-0.009020in}{-0.015532in}}{\pgfqpoint{-0.004604in}{-0.017361in}}{\pgfqpoint{0.000000in}{-0.017361in}}%
\pgfpathlineto{\pgfqpoint{0.000000in}{-0.017361in}}%
\pgfpathclose%
\pgfusepath{stroke,fill}%
}%
\begin{pgfscope}%
\pgfsys@transformshift{5.464205in}{1.941346in}%
\pgfsys@useobject{currentmarker}{}%
\end{pgfscope}%
\begin{pgfscope}%
\pgfsys@transformshift{5.408323in}{2.770092in}%
\pgfsys@useobject{currentmarker}{}%
\end{pgfscope}%
\begin{pgfscope}%
\pgfsys@transformshift{4.863471in}{1.334725in}%
\pgfsys@useobject{currentmarker}{}%
\end{pgfscope}%
\begin{pgfscope}%
\pgfsys@transformshift{4.807589in}{2.542436in}%
\pgfsys@useobject{currentmarker}{}%
\end{pgfscope}%
\begin{pgfscope}%
\pgfsys@transformshift{4.248766in}{1.214919in}%
\pgfsys@useobject{currentmarker}{}%
\end{pgfscope}%
\begin{pgfscope}%
\pgfsys@transformshift{4.192884in}{2.344172in}%
\pgfsys@useobject{currentmarker}{}%
\end{pgfscope}%
\begin{pgfscope}%
\pgfsys@transformshift{3.616598in}{1.443481in}%
\pgfsys@useobject{currentmarker}{}%
\end{pgfscope}%
\begin{pgfscope}%
\pgfsys@transformshift{3.560716in}{2.419610in}%
\pgfsys@useobject{currentmarker}{}%
\end{pgfscope}%
\begin{pgfscope}%
\pgfsys@transformshift{2.966967in}{1.671276in}%
\pgfsys@useobject{currentmarker}{}%
\end{pgfscope}%
\begin{pgfscope}%
\pgfsys@transformshift{2.911085in}{2.598504in}%
\pgfsys@useobject{currentmarker}{}%
\end{pgfscope}%
\begin{pgfscope}%
\pgfsys@transformshift{2.303365in}{2.119732in}%
\pgfsys@useobject{currentmarker}{}%
\end{pgfscope}%
\begin{pgfscope}%
\pgfsys@transformshift{2.250975in}{2.782476in}%
\pgfsys@useobject{currentmarker}{}%
\end{pgfscope}%
\begin{pgfscope}%
\pgfsys@transformshift{1.625793in}{2.444707in}%
\pgfsys@useobject{currentmarker}{}%
\end{pgfscope}%
\begin{pgfscope}%
\pgfsys@transformshift{1.573403in}{2.907773in}%
\pgfsys@useobject{currentmarker}{}%
\end{pgfscope}%
\begin{pgfscope}%
\pgfsys@transformshift{0.934250in}{2.839081in}%
\pgfsys@useobject{currentmarker}{}%
\end{pgfscope}%
\begin{pgfscope}%
\pgfsys@transformshift{0.881860in}{3.096131in}%
\pgfsys@useobject{currentmarker}{}%
\end{pgfscope}%
\end{pgfscope}%
\begin{pgfscope}%
\pgfsetrectcap%
\pgfsetmiterjoin%
\pgfsetlinewidth{0.803000pt}%
\definecolor{currentstroke}{rgb}{0.000000,0.000000,0.000000}%
\pgfsetstrokecolor{currentstroke}%
\pgfsetdash{}{0pt}%
\pgfpathmoveto{\pgfqpoint{0.773588in}{0.646140in}}%
\pgfpathlineto{\pgfqpoint{0.773588in}{3.734428in}}%
\pgfusepath{stroke}%
\end{pgfscope}%
\begin{pgfscope}%
\pgfsetrectcap%
\pgfsetmiterjoin%
\pgfsetlinewidth{0.803000pt}%
\definecolor{currentstroke}{rgb}{0.000000,0.000000,0.000000}%
\pgfsetstrokecolor{currentstroke}%
\pgfsetdash{}{0pt}%
\pgfpathmoveto{\pgfqpoint{5.802992in}{0.646140in}}%
\pgfpathlineto{\pgfqpoint{5.802992in}{3.734428in}}%
\pgfusepath{stroke}%
\end{pgfscope}%
\begin{pgfscope}%
\pgfsetrectcap%
\pgfsetmiterjoin%
\pgfsetlinewidth{0.803000pt}%
\definecolor{currentstroke}{rgb}{0.000000,0.000000,0.000000}%
\pgfsetstrokecolor{currentstroke}%
\pgfsetdash{}{0pt}%
\pgfpathmoveto{\pgfqpoint{0.773588in}{0.646140in}}%
\pgfpathlineto{\pgfqpoint{5.802992in}{0.646140in}}%
\pgfusepath{stroke}%
\end{pgfscope}%
\begin{pgfscope}%
\pgfsetrectcap%
\pgfsetmiterjoin%
\pgfsetlinewidth{0.803000pt}%
\definecolor{currentstroke}{rgb}{0.000000,0.000000,0.000000}%
\pgfsetstrokecolor{currentstroke}%
\pgfsetdash{}{0pt}%
\pgfpathmoveto{\pgfqpoint{0.773588in}{3.734428in}}%
\pgfpathlineto{\pgfqpoint{5.802992in}{3.734428in}}%
\pgfusepath{stroke}%
\end{pgfscope}%
\begin{pgfscope}%
\definecolor{textcolor}{rgb}{0.000000,0.000000,0.000000}%
\pgfsetstrokecolor{textcolor}%
\pgfsetfillcolor{textcolor}%
\pgftext[x=5.464205in,y=1.735460in,,base]{\color{textcolor}\rmfamily\fontsize{8.000000}{9.600000}\selectfont P(1)}%
\end{pgfscope}%
\begin{pgfscope}%
\definecolor{textcolor}{rgb}{0.000000,0.000000,0.000000}%
\pgfsetstrokecolor{textcolor}%
\pgfsetfillcolor{textcolor}%
\pgftext[x=4.863471in,y=1.128839in,,base]{\color{textcolor}\rmfamily\fontsize{8.000000}{9.600000}\selectfont P(2)}%
\end{pgfscope}%
\begin{pgfscope}%
\definecolor{textcolor}{rgb}{0.000000,0.000000,0.000000}%
\pgfsetstrokecolor{textcolor}%
\pgfsetfillcolor{textcolor}%
\pgftext[x=4.248766in,y=1.009033in,,base]{\color{textcolor}\rmfamily\fontsize{8.000000}{9.600000}\selectfont P(3)}%
\end{pgfscope}%
\begin{pgfscope}%
\definecolor{textcolor}{rgb}{0.000000,0.000000,0.000000}%
\pgfsetstrokecolor{textcolor}%
\pgfsetfillcolor{textcolor}%
\pgftext[x=3.616598in,y=1.237595in,,base]{\color{textcolor}\rmfamily\fontsize{8.000000}{9.600000}\selectfont P(4)}%
\end{pgfscope}%
\begin{pgfscope}%
\definecolor{textcolor}{rgb}{0.000000,0.000000,0.000000}%
\pgfsetstrokecolor{textcolor}%
\pgfsetfillcolor{textcolor}%
\pgftext[x=2.966967in,y=1.465390in,,base]{\color{textcolor}\rmfamily\fontsize{8.000000}{9.600000}\selectfont P(5)}%
\end{pgfscope}%
\begin{pgfscope}%
\definecolor{textcolor}{rgb}{0.000000,0.000000,0.000000}%
\pgfsetstrokecolor{textcolor}%
\pgfsetfillcolor{textcolor}%
\pgftext[x=2.303365in,y=1.913846in,,base]{\color{textcolor}\rmfamily\fontsize{8.000000}{9.600000}\selectfont P(6)}%
\end{pgfscope}%
\begin{pgfscope}%
\definecolor{textcolor}{rgb}{0.000000,0.000000,0.000000}%
\pgfsetstrokecolor{textcolor}%
\pgfsetfillcolor{textcolor}%
\pgftext[x=1.625793in,y=2.238821in,,base]{\color{textcolor}\rmfamily\fontsize{8.000000}{9.600000}\selectfont P(7)}%
\end{pgfscope}%
\begin{pgfscope}%
\definecolor{textcolor}{rgb}{0.000000,0.000000,0.000000}%
\pgfsetstrokecolor{textcolor}%
\pgfsetfillcolor{textcolor}%
\pgftext[x=0.934250in,y=2.633195in,,base]{\color{textcolor}\rmfamily\fontsize{8.000000}{9.600000}\selectfont P(8)}%
\end{pgfscope}%
\begin{pgfscope}%
\definecolor{textcolor}{rgb}{0.000000,0.000000,0.000000}%
\pgfsetstrokecolor{textcolor}%
\pgfsetfillcolor{textcolor}%
\pgftext[x=3.288290in,y=3.817761in,,base]{\color{textcolor}\rmfamily\fontsize{16.800000}{20.160000}\selectfont Intensity (P-branch)}%
\end{pgfscope}%
\begin{pgfscope}%
\pgfsetbuttcap%
\pgfsetmiterjoin%
\definecolor{currentfill}{rgb}{1.000000,1.000000,1.000000}%
\pgfsetfillcolor{currentfill}%
\pgfsetfillopacity{0.800000}%
\pgfsetlinewidth{1.003750pt}%
\definecolor{currentstroke}{rgb}{0.800000,0.800000,0.800000}%
\pgfsetstrokecolor{currentstroke}%
\pgfsetstrokeopacity{0.800000}%
\pgfsetdash{}{0pt}%
\pgfpathmoveto{\pgfqpoint{0.909699in}{0.743362in}}%
\pgfpathlineto{\pgfqpoint{1.911088in}{0.743362in}}%
\pgfpathquadraticcurveto{\pgfqpoint{1.949977in}{0.743362in}}{\pgfqpoint{1.949977in}{0.782251in}}%
\pgfpathlineto{\pgfqpoint{1.949977in}{1.033862in}}%
\pgfpathquadraticcurveto{\pgfqpoint{1.949977in}{1.072751in}}{\pgfqpoint{1.911088in}{1.072751in}}%
\pgfpathlineto{\pgfqpoint{0.909699in}{1.072751in}}%
\pgfpathquadraticcurveto{\pgfqpoint{0.870810in}{1.072751in}}{\pgfqpoint{0.870810in}{1.033862in}}%
\pgfpathlineto{\pgfqpoint{0.870810in}{0.782251in}}%
\pgfpathquadraticcurveto{\pgfqpoint{0.870810in}{0.743362in}}{\pgfqpoint{0.909699in}{0.743362in}}%
\pgfpathlineto{\pgfqpoint{0.909699in}{0.743362in}}%
\pgfpathclose%
\pgfusepath{stroke,fill}%
\end{pgfscope}%
\begin{pgfscope}%
\pgfsetbuttcap%
\pgfsetroundjoin%
\definecolor{currentfill}{rgb}{1.000000,0.000000,0.000000}%
\pgfsetfillcolor{currentfill}%
\pgfsetlinewidth{1.003750pt}%
\definecolor{currentstroke}{rgb}{1.000000,0.000000,0.000000}%
\pgfsetstrokecolor{currentstroke}%
\pgfsetdash{}{0pt}%
\pgfsys@defobject{currentmarker}{\pgfqpoint{-0.017361in}{-0.017361in}}{\pgfqpoint{0.017361in}{0.017361in}}{%
\pgfpathmoveto{\pgfqpoint{0.000000in}{-0.017361in}}%
\pgfpathcurveto{\pgfqpoint{0.004604in}{-0.017361in}}{\pgfqpoint{0.009020in}{-0.015532in}}{\pgfqpoint{0.012276in}{-0.012276in}}%
\pgfpathcurveto{\pgfqpoint{0.015532in}{-0.009020in}}{\pgfqpoint{0.017361in}{-0.004604in}}{\pgfqpoint{0.017361in}{0.000000in}}%
\pgfpathcurveto{\pgfqpoint{0.017361in}{0.004604in}}{\pgfqpoint{0.015532in}{0.009020in}}{\pgfqpoint{0.012276in}{0.012276in}}%
\pgfpathcurveto{\pgfqpoint{0.009020in}{0.015532in}}{\pgfqpoint{0.004604in}{0.017361in}}{\pgfqpoint{0.000000in}{0.017361in}}%
\pgfpathcurveto{\pgfqpoint{-0.004604in}{0.017361in}}{\pgfqpoint{-0.009020in}{0.015532in}}{\pgfqpoint{-0.012276in}{0.012276in}}%
\pgfpathcurveto{\pgfqpoint{-0.015532in}{0.009020in}}{\pgfqpoint{-0.017361in}{0.004604in}}{\pgfqpoint{-0.017361in}{0.000000in}}%
\pgfpathcurveto{\pgfqpoint{-0.017361in}{-0.004604in}}{\pgfqpoint{-0.015532in}{-0.009020in}}{\pgfqpoint{-0.012276in}{-0.012276in}}%
\pgfpathcurveto{\pgfqpoint{-0.009020in}{-0.015532in}}{\pgfqpoint{-0.004604in}{-0.017361in}}{\pgfqpoint{0.000000in}{-0.017361in}}%
\pgfpathlineto{\pgfqpoint{0.000000in}{-0.017361in}}%
\pgfpathclose%
\pgfusepath{stroke,fill}%
}%
\begin{pgfscope}%
\pgfsys@transformshift{1.143033in}{0.926917in}%
\pgfsys@useobject{currentmarker}{}%
\end{pgfscope}%
\end{pgfscope}%
\begin{pgfscope}%
\definecolor{textcolor}{rgb}{0.000000,0.000000,0.000000}%
\pgfsetstrokecolor{textcolor}%
\pgfsetfillcolor{textcolor}%
\pgftext[x=1.493032in,y=0.858862in,left,base]{\color{textcolor}\rmfamily\fontsize{14.000000}{16.800000}\selectfont Dips}%
\end{pgfscope}%
\end{pgfpicture}%
\makeatother%
\endgroup%
}
		\label{fig:intensity_p_branch}
	\end{subfigure}
	\caption{Interpolating the data}
	\label{fig:Interpolation}
\end{figure}

Using Lambert-Beer's law \ref{eq:Lambert-Beer}, we find out that $^{35} \mathrm{Cl}$ makes up about 68\% of the atoms with the remaining 32\% being $^{37} \mathrm{Cl}$, since these are the only stable isotopes, which is a good approximation to the actual value of 75\%.

\subsection{Task 6}

Using the same interpolated data, the intensities of the minima was ploted against the rotational quantum number $J$. A curve fit was carried out using Lambert-Beers law \ref{eq:Lambert-Beer} for each isotope. The results are shown in Fig. \ref{fig:Task6}.



\begin{figure}[H]
	\centering
	\begin{subfigure}{0.45\textwidth}
		\centering
		\scalebox{0.50}{\input{Figures/Cl35_intensities.pgf}}
		\caption{$^{35}$Cl}
		\label{fig:Cl35_intensities}
	\end{subfigure}	
	\hspace{0.5cm}
	\begin{subfigure}{0.45\textwidth}
		\centering
		\scalebox{0.50}{\input{Figures/Cl37_intensities.pgf}}
		\caption{$^{37}$Cl}
		\label{fig:Cl37_intensities}
	\end{subfigure}	
	\label{fig:Task6}
	\caption{Fitting the intensities to determine the temperature}
\end{figure}

The temperatures obtained from the fits are: 

\begin{equation}
	T_{^{35} \mathrm{Cl}} = 356.95 \pm 8.46 \ \text{K}
	T_{^{37} \mathrm{Cl}} = 372.34 \pm 23.59 \ \text{K}
\end{equation}

Which is 20\% and 25\% off from room temperature of 293.15 K for $^{35} \mathrm{Cl}$ and $^{37} \mathrm{Cl}$, respectively.



























\pagebreak{}

\section{Conclusion}

\pagebreak{}

\begin{appendices}

\section{Task 1 Tables}


\begin{table}[h!]
	\centering
	\begin{tabular}{|lll|}
	\hline
	\multicolumn{3}{|c|}{Polystyrene}                                       \\ \hline
	\multicolumn{1}{|c|}{Theoretical} & \multicolumn{1}{c|}{Experimental} & \multicolumn{1}{c|}{$\%$ Change} \\ \hline
	\multicolumn{1}{|l|}{699.45}  & \multicolumn{1}{l|}{699.125}  & -0.0465 \\ \hline
	\multicolumn{1}{|l|}{756.58}  & \multicolumn{1}{l|}{756.625}  & 0.00595 \\ \hline
	\multicolumn{1}{|l|}{842.00}  & \multicolumn{1}{l|}{841.375}  & -0.0742 \\ \hline
	\multicolumn{1}{|l|}{906.80}  & \multicolumn{1}{l|}{907.75}   & 0.105   \\ \hline
	\multicolumn{1}{|l|}{965.70}  & \multicolumn{1}{l|}{965.875}  & 0.0181  \\ \hline
	\multicolumn{1}{|l|}{1028.30} & \multicolumn{1}{l|}{1028.875} & 0.0559  \\ \hline
	\multicolumn{1}{|l|}{1069.10} & \multicolumn{1}{l|}{1069.375} & 0.0257  \\ \hline
	\multicolumn{1}{|l|}{1154.60} & \multicolumn{1}{l|}{1155.75}  & 0.0996  \\ \hline
	\multicolumn{1}{|l|}{1368.50} & \multicolumn{1}{l|}{1367.75}  & -0.0548 \\ \hline
	\multicolumn{1}{|l|}{1449.70} & \multicolumn{1}{l|}{1449.375} & -0.0224 \\ \hline
	\multicolumn{1}{|l|}{1542.20} & \multicolumn{1}{l|}{1541.75}  & -0.0292 \\ \hline
	\multicolumn{1}{|l|}{1583.10} & \multicolumn{1}{l|}{1584.125} & 0.0647  \\ \hline
	\multicolumn{1}{|l|}{1601.40} & \multicolumn{1}{l|}{1602.75}  & 0.0843  \\ \hline
	\multicolumn{1}{|l|}{1803.80} & \multicolumn{1}{l|}{1804.25}  & 0.0249  \\ \hline
	\multicolumn{1}{|l|}{1945.20} & \multicolumn{1}{l|}{1944.5}   & -0.0360 \\ \hline
	\multicolumn{1}{|l|}{2850.00} & \multicolumn{1}{l|}{2853.00}  & 0.105   \\ \hline
	\multicolumn{1}{|l|}{2920.90} & \multicolumn{1}{l|}{2921.25}  & 0.0120  \\ \hline
	\multicolumn{1}{|l|}{3001.40} & \multicolumn{1}{l|}{3003.875} & 0.0825  \\ \hline
	\multicolumn{1}{|l|}{3026.40} & \multicolumn{1}{l|}{3027.125} & 0.0240  \\ \hline
	\multicolumn{1}{|l|}{3060.00} & \multicolumn{1}{l|}{3063.00}  & 0.0980  \\ \hline
	\multicolumn{1}{|l|}{3082.20} & \multicolumn{1}{l|}{3084.875} & 0.0868  \\ \hline
	\end{tabular}
	\caption{}
	\label{tab:polystyrene}
\end{table}

\begin{table}[h!]
	\centering
	\begin{tabular}{|ccc|}
	\hline
	\multicolumn{3}{|c|}{Water Vapour}                                                  \\ \hline
	\multicolumn{1}{|c|}{Theoretical} & \multicolumn{1}{c|}{Experimental} & $\%$ Change \\ \hline
	\multicolumn{1}{|c|}{1416.08}     & \multicolumn{1}{c|}{1417.50}      & 0.100       \\ \hline
	\multicolumn{1}{|c|}{1601.20}     & \multicolumn{1}{c|}{1602.875}     & 0.105       \\ \hline
	\multicolumn{1}{|c|}{1792.65}     & \multicolumn{1}{c|}{1792.75}      & 0.006       \\ \hline
	\multicolumn{1}{|c|}{1799.61}     & \multicolumn{1}{c|}{1801.375}     & 0.098       \\ \hline
	\multicolumn{1}{|c|}{1802.47}     & \multicolumn{1}{c|}{1803.125}     & 0.036       \\ \hline
	\multicolumn{1}{|c|}{1807.7}      & \multicolumn{1}{c|}{1806.875}     & -0.046      \\ \hline
	\multicolumn{1}{|c|}{1810.62}     & \multicolumn{1}{c|}{1809.50}      & -0.062      \\ \hline
	\multicolumn{1}{|c|}{2018.32}     & \multicolumn{1}{c|}{2018.75}      & 0.021       \\ \hline
	\multicolumn{1}{|c|}{2041.3}      & \multicolumn{1}{c|}{2043.375}     & 0.102       \\ \hline
	\end{tabular}
	\caption{}
	\label{tab:watervapour}
\end{table}


\end{appendices}

\pagebreak{}

%\bibliographystyle{ieeetr} 
%\bibliography{} 

\end{document}